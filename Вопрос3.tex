\textsl{\textbf{3. 	Должны ли были сработать в автомобиле INFINITI QX80 г.р.з. О376ХР123, VIN JN1JANZ62U0100644 элементы пассивной безопасности в условиях ДТП 26.05.2021 г. и были ли достигнуты условия для их срабатывания?}}

Для открытия передних подушек безопасностей, автомобиль должен совершить жёсткое, блокирующее контактирование (с другим автомобилем или препятствием), при этом после столкновения он не должен двигаться вперёд по ходу своего движения (в данной случае пороговое значение замедления не будет достигнуто). Передние подушки безопасности практически в 100\% случаях должны раскрыться, если автомобиль после столкновения «отскочил» назад (на 180 градусов от направления своего движения) и у него значительно деформировались передние лонжероны (или рама).

При рассмотрении вопросов № 1 и № 2, было определено, что в результате ДТП 26.05.2021 г. такие элементы пассивной безопасности исследуемого автомобиля, не входящие в систему SRS, как   бампер передний, капот, усилитель и ударопоглотитель переднего бампера, левый передний краш-бокс усилителя бампера переднего и другие конструктивные элементы  передней зоны программируемой деформации кузова автомобиля были смяты по  заданным участкам контролируемого сминания  при фронтальном столкновении, поглотив часть кинетической энергии удара, тем самым  снизив силу воздействия на водителя.   Так как  при ДТП водитель не получил серьезных травм и увечий, то есть все основания полагать, что  элементы пассивной безопасности автомобиля сработали надлежащим образом. 

 
В то же время, в данном ДТП подушки безопасности, натяжители ремней безопасности так же входящие в систему SRS не сработали.


На основании изложенного эксперты приходят к обобщению, что в контексте поставленного вопроса целью  исследования  является  определение причины несрабатывания подушек безопасности в автомобиле Infiniti QX80 2019 года при столкновении с бескапотным грузовиком КамАЗ-5490-S5.


Подушки безопасности являются дополнительной защитой и разработаны для оптимальной работы в сочетании с ремнями безопасности. Как было установлено выше  по совокупности признаков,  в момент ДТП водитель автмобиля INFINITI QX80 не был пристегнут ремнями безопасности.

 В автомобилях Infiniti QX80 подушки безопасности, \cite{патент:US8801033B2} срабатывают при определенных условиях, которые зависят от типа датчиков, уровня удара, направления столкновения и ряда других факторов. Прямой зависимости между пристегнутыми ремнями безопасности и срабатыванием подушек безопасности нет, однако алгоритм активации подушек безопасности учитывает, были или небыли  прстегнуты водитель и пассжир ремнями безопасности.
 
  Основные условия срабатывания подушек безопасности включают:

1. \textbf{Сила удара}: Подушки безопасности срабатывают при ударах определенной силы, когда ускорение или замедление автомобиля превышает заданный порог. Системы определяют, достаточно ли серьезен удар для активации подушек.  Фронтальные подушки безопасности, как правило, предназначены для срабатывания при «умеренных или серьезных» фронтальных или почти фронтальных столкновениях, которые определяются как столкновения, эквивалентные удару о твердый неподвижный барьер на скорости от 13 км/ч до 22 км/ч или выше, что примерно  эквивалентно удару о припаркованный автомобиль аналогичного размера на скорости от 25,5 км/ч до 45 км/ч, \url{https://www.nhtsa.gov/vehicle-safety/air-bags}.

2. \textbf{Направление удара}:
- Фронтальные подушки безопасности срабатывают при фронтальных столкновениях средней и высокой тяжести. Обычно они активируются при столкновениях спереди под углом от 10 до 30 градусов, если сила удара соответствует пороговому значению.

- Боковые подушки безопасности активируются при боковых ударах и защищают грудную клетку и голову пассажира.

- Шторки безопасности срабатывают при ударах сбоку и помогают защитить голову пассажиров от боковых стекол и внешних объектов.

3. \textbf{Скорость движения}: Подушки могут не сработать на низких скоростях, так как система оценивает, достаточно ли опасна ситуация для водителя и пассажиров.

4.\textbf{ Расположение пассажиров и ремни безопасности}: Система безопасности отслеживает, пристегнуты ли ремни безопасности, а также вес и положение пассажиров. Например, если на переднем пассажирском сидении находится ребенок (это определяется датчиками веса), фронтальная подушка может быть отключена для безопасности ребенка. 

5. \textbf{Особые ситуации}: Подушки не сработают при перевороте автомобиля, если нет сильного фронтального или бокового удара. Также они не сработают при заднем столкновении, так как подушки предназначены для защиты от фронтальных и боковых ударов.


Административный материал, составленный сотрудниками ГАИ по факту  ДТП позволяет определить только направление удара, является мало информативным и не позволяет  определить физические характеристики  столкновения автомобилей (силы удара, скорости, замедления, энергии и т.д). В таком случае,  разрешение вопроса исследования  возможно на основании анализа  деформации транспортных средств, полученных в ДТП. При этом результаты исследования будут иметь вероятностный характер.  


Для расчета энергии удара по деформации автомобиля применим   принципы механики и физики. В частности, энергия удара (кинетическая энергия) преобразуется в работу по деформации, когда автомобиль сталкивается с препятствием или другим автомобилем.

%Основные шаги для расчета:
%
%1. \textbf{Определение кинетической энергии до столкновения}:
Кинетическая энергия \( E_k \) автомобиля до столкновения рассчитывается по формуле:
\[
E_k = \frac{1}{2} m v^2
\]
где:

\begin{itemize}
	\item \( m \) — масса автомобиля (в килограммах),
	\item \( v \) — скорость автомобиля до столкновения (в метрах в секунду).
\end{itemize}


 Из закона сохранения энергии сдедует,что кинетическая энергия автомобиля до столкновения равна энергии деформации:
{\large 	\[
	\frac{1}{2} m v^2 = E_d.
	\]}

Отсюда скорость \( v \):
{\large 	\[
	v = \sqrt{\frac{2 E_d}{m}}.
	\]}
где:

${E_d}$ - энергия, затраченная на деформацию

Для реального ДТП, зная    массы автомобилей, их взаимное положение в момент столкновения и их
остаточные скорости после удара, технически возможно установить скорости    этих автомобилей в момент удара исходя из затрат энергии на деформацию их    конструкций.  

Общепризнанной методикой, позволяющей точно определить ключевые параметры столкновения автомобилей, включая скорость автомобилей, энергию столкновения и механику удара является методика \textbf{CRASH3},  
основанная на анализе лабораторных испытаний автомобилей на удар.

	Методика {CRASH3} (Calspan Reconstruction of Accident Speeds on the Highway, версия 3), \cite{crash3} — это инструмент для реконструкции дорожно-транспортных происшествий, разработанный с целью определения скоростей транспортных средств в момент столкновения, параметров столкновения, перемещения после удара и оценки энергии деформации.  Методика  используется экспертами для получения количественных выводов о механике аварий на основе данных о повреждениях и условиях происшествия.  В 2004 году в п.5.2.3 методических рекомендаций по проведению независимой технической экспертизы транспортного средства при ОСАГО № 001МР/СЭ (утв. ГНИИ АТ Минтранса РФ, РФЦСЭ МЮ РФ, ЭКЦ МВД РФ, НПСО «Объединение транспортных экспертов») для  определения причин повреждений объектов экспертизы рекомендовалось использовать справочники с  результатами краш-тестов конкретных марок (моделей, модификаций) транспортных  средств, и, в частности результаты краш-тестов NHTSA, имеющиеся в свободном доступе в виде отчетов и в цифровом виде, вместе с программами для обработки данных.  В Российской Федерации проведение краш-тестов осуществляется в соответствии с ГОСТ Р 41.94-99 (Правила
	ЕЭК ООН № 94) "Единообразные предписания, касающиеся официального утверждения транспортных средств в отношении защиты водителя и пассажиров в случае лобового столкновения" и ГОСТ Р 41.95-99 (Правила ЕЭК ООН № 95) "Единообразные предписания, касающиеся официального утверждения транспортных средств в отношении защиты водителя и пассажиров в случае бокового столкновения". 
	
	
	
%	Используются методика CRASH3 для расчёта энергии удара, закон сохранения импульса, данные диагностики через CONSULT-III, визуальный анализ повреждений по фотографиям, а также учёт увеличения массы обоих автомобилей. Анализируется применимость данных краш-теста Nissan Armada SV 4WD 2018 года и стандарта PART 563---Event Data Recorders.
	
   
Соглано методики CRASH3 для расчета величины затрат энергии на деформацию автомобиля при   фронтальном столкновении требуется знать три коэффициента жесткости его передней  части: \textbf{\textit{ A}},  \textbf{\textit{B}} и \textbf{\textit{G}}.

Коэффициент \textit{A} (Н/м)
 -- удельное сопротивление конструкции до начала пластической деформации.    
 
 Коэффициент \textit{B} (Н/м2) -- удельная сила, необходимая для развития пластической деформации.
 
  Коэффициент \textit{G} (производный) - параметр, зависящий от \textit{A} и \textit{B}.
  
  
  Расчет коэффициентов
	
	Коэффициент \textbf{A} для разных значений скорости нулевой деформации $b_0$:
	
	{\large   \[A = \frac{mb_0b_1}{3.6^2L} \]}
	
	Коэффициент  \textit{B}::
	
	
	{\large    \begin{equation}
			B = \frac{mb_1}{3.6^2L}
			\label{A}
		\end{equation}
	}
	
	
	
	
	Величина коэффициента жесткости G автомобиля  для
	разных значений скорости нулевой деформации $ b_0 $составляет
	
	
	{\large \[G = \frac{A^2}{2B} \]}
	
	Если деформация \textit{С} есть средняя деформация автомобиля:
	
	{\large \[ C = C_{i}+(C_{i+1}-C_{i})\frac{l}{L_{i}}, 0 \leq l \leq L_{i}, \eqno(11) \]}
	
	тогда затраты энергии на дефрмацию на \textit{i}-ом участке деформированной поверхности автомобиля составляют:
	
	{\large \[ E_{i} = \int_{0}^{L_{i}}\left( AC + \frac{BC^2}{2} + G\right) dl = L_{i}\left( \frac{A}{2}(C_{i}+C_{i+1})+\frac{B}{6}(C^{2}_{i}+C_{i}C_{i+1}+C^{2}_{i+1})+G\right), \eqno(12)  \]
	}
	
Суммарные затраты энергии определяются сложением $E_{i}$ для всех участков с учётом аддитивности энергии.

Необходимые для   расчета значения жесткости конструкций    автомобилей возможно установить на основании экспериментальных данных  краш-тестов. 

Публично доступные отчеты по краш-тестам автомобилей размещяются, например,   на сайтах NHTSA (Национального управления безопасности дорожного движения США).


% TODO: \usepackage{graphicx} required
\begin{figure}[H]
	\centering
	\includegraphics[width=0.9\linewidth]{../images/foto/rating1}
	\caption{{\small Рейтинг автомобиля Infiniti QX80 2019 года выпуска, официально опубликованный на сайте NHTSA, \url{https://www.nhtsa.gov/vehicle/2019/INFINITI/QX80/SUV/AWD}. Технический отчет v10562R001.pdf, 28601.392 kB (M20195203 2019 Nissan Armada SUV NCAP Final Report.pdf)}}
	\label{fig:rating1}
\end{figure}




%INFINITI QX80  крупный люксовый внедорожник,  в основном ориентированный на рынки США, Ближнего Востока и России. Однако официальных результатов краш-тестов от ведущих международных организаций, таких как Euro NCAP, IIHS (Insurance Institute for Highway Safety) или NHTSA (National Highway Traffic Safety Administration), для этой модели нет.
%
%Причины отсутствия официальных рейтингов:
%
%Рыночная специфика: QX80 не продается в Европе, поэтому Euro NCAP его не тестирует.
%
%Американский рынок: В США QX80 позиционируется как нишевая модель, и IIHS/NHTSA также не публиковали данных о его краш-тестах (на момент 2023 года).
%
%Российские тесты: В России краш-тесты проводятся редко и не имеют международного признания. Для QX80 таких данных нет.
%
%	
%	
%Публично доступные отчеты по краш-тестам автомобилей размещяются  на сайтах NHTSA (Национального управления безопасности дорожного движения США), \url{https://www.nhtsa.gov/},  IIHS (Insurance Institute for Highway Safety (Институт страхования и безопасности на дорогах). Американская некоммерческая организация, занимающаяся исследованиями в области безопасности дорожного движения), \url{https://www.iihs.org/ratings}, 
%Европейская программа оценки новых автомобилей (Euro NCAP): Оценила безопасность Infiniti QX80 для европейского рынка, \url{https://www.euroncap.com}, Официальный сайт Австралийской программы оценки новых автомобилей (ANCAP), \url{Официальный сайт Австралийской программы оценки новых автомобилей (ANCAP)}.
	
.

Для автомобиля Infiniti QX80 NHTSA при оценке безопасности автомобиля использует данные полноприводно версии Nissan Armada.  Nissan и Infiniti — дочерние компании одного концерна (Nissan Motor Corporation).     Infiniti QX80 и Nissan Armada построены на одной платформе (Nissan F-Alpha) и имеют: 

идентичную раму (лестничная конструкция);

идентичные кузова и силовые элементы;

одинаковые зоны деформации;

схожие системы пассивной безопасности (подушки, ремни, усиление стоек).


Различаются Infiniti QX80 и Nissan Armada  материалами отделки салона, дополнительными опциями, влюяющими на комфорт, внешним дизайном.

Краш-тесты, проводимые NHTSA для базовой полноприводной модели Armada  полностью применимы к люксовой версии INFINITI QX80.

	База 	данных 	краш-тестов 	транспортных средств (Vehicle Crash Test Database) NHTSA находится  по адресу в интернете: \url{ https://www.nhtsa.gov/research-data/research-testing-databases#/vehicle}. Эта база данных  содержит технические 	данные, полученные в ходе различных видов испытаний, в том числе 	по программе оценки новых автомобилей NCAP. Доступ к базе данных 	свободный, регистрация на сайте не требуется. 
%
%Кроме краш-тестов по программе NCAP, база данных содержит краш-тесты на боковые удары одного транспортного средства в другое.

 Также 	на сайте NHTSA имеется программное обеспечение  для инженерного анализа, использующее локальные файлы UDS для испытаний на удар. Эти параметры включают в себя графики сигналов 	лабораторных акселерометров, устанавливаемых на транспортных средствах 	и манекенах водителя и пассажиров, и графики сигналов тензодатчиков в ячейках жестких неподвижных барьеров. %Загрузку файлов указанных 	сигналов на локальный компьютер можно произвести в ряде форматов 	NHTSA. 	%Краш-тесты можно выбрать из базы данных следующим образом: 	• посмотреть последние 10 краш-тестов (Browse the latest tests);
%	• выбрать краш-тесты по параметрам (Query by test parameters). Выбор 	включает в себя интервал номеров краш-тестов (Test No) или его ссылочный 	номер (Test Reference No), тип теста (Test Type) из выпадающего списка,
%	название контракта или исследования (Contract or Study Title), лабораторию- 	исполнителя краш-теста из выпадающего списка (Test Performer), интервал
%	угла удара (Impact Angle) в транспортное средство в градусах, конфигурацию 	краш-теста (Test Configuration) из выпадающего списка, интервал величины 	смещения (Offset Distance) в мм, интервал скорости удара (Closing Speed)
%	в км/ч, 	порядок 	сортировки 	(Order 	By)  найденных
%	краш-тестов 	из выпадающего 	списка 	(по умолчанию — 	по номеру) 	по возрастанию 	(Ascending) или убыванию (Descending), число записей на странице (Records 	Per Page) в списке найденных краш-тестов. 
%	

%База данных краш-тестов NHTSA может быть экспортирована целиком любым пользователем. Для помощи пользователям доступны описания форматов экспортируемых файлов. Файлы сжаты zip-архиватором, поля данных разделены символом «|». 	Для извлечения файлов и их импорта в другие программы потребуется 	вспомогательное 	программное
%обеспечение. Такие программы, как, например, Microsoft Access и Excel, поддерживают механизм импорта файлов ASCII и позволяют пользователю выбирать разделитель полей
%данных. Для установки программного обеспечения NHTSA
%со страницы  в интернете \url{https://www.nhtsa.gov/databases-and-software/signal-analysis-software-windows} надо скачать пакет установщика VSR\_RDDBSOFTWARE 	и запустить его на своем компьютере. В результате будут установлены 	15 программ, среди которых 
Для настоящего исследования требуется применить две програмны:

· LoadCellAnalysis — анализ сигналов в ячейках барьера,

· SignalBrowser — вычисления и анализ сигналов в краш-тесте.




	


%	\input{сравнение nissan infiniti}







\textbf{Особенности частичного перекрытия (40\%) :}\\
При частичном ударе нагрузка распределяется неравномерно.
Основная деформация будет сосредоточена в зоне контакта.
Энергопоглощающие элементы могут работать менее эффективно.	


Кабина КАМАЗ-5490 S5 спроектирована на базе кабины Mercedes-Benz Actros третьего поколения. Габариты кабины КАМАЗ-5490 S5 практически не отличаются от прототипа Mercedes-Benz Actros. Основные изменения касаются внешнего дизайна и адаптации под российские условия, но они минимально влияют на общие размеры. Каркас кабины КАМАЗ-5490 S5 сохраняет базовую конструкцию Mercedes-Benz Actros, но имеет усиления, адаптацию материалов и измененные настройки для работы в российских условиях. Эти изменения минимально влияют на базовую геометрию, но делают конструкцию более прочной и ремонтопригодной. Номер лицензионного соглашения на использование конструкции кабины Mercedes-Benz Actros для КАМАЗ-5490 S5 не является публичной информацией.


Спроси

Объяснить








Результаты проверки истории автомобиля на предмет предыдущих аварий и замены подушек исключают события, связанные с активацией системы SRS с начала эксплуатации автомобиля.	
\textbf{\textsl{	Вывод по третьему вопросу}}
\textbf{Скорость удара составила примерно 45-55 км/ч , что находится на границе порога срабатывания подушек безопасности для данного типа столкновения. Невысвобождение подушек является результатом комбинации факторов: скорости удара, типа столкновения и состояния ремней безопасности.}
	