%	
%\sloppy  % если не заработает, удалить
%
\pagestyle{plain}

\pagestyle{fancy}
\frenchspacing 
\fancyhf{}
\fancyhead{}
\fancyfoot{} 
%
%
%\fancyhead[LE]{ }
%\fancyhead{}
%\fancyhead[LO, RE]{\footnotesize \textcolor{ForestGreen} {Акт экспертного исследования № \NomerDoc\, от \окончено}} 
%
%
%\fancyhead[LE,RO]{\thepage} номер страницы слева сверху на четных и справа на нечетных
%\fancyhead[CO]{текст-центр-нечетные}%текст-центр-нечетные
\fancyhead[RE]{ } %текст-слева-нечетные
\fancyhead{}
%\fancyhead[CE]{текст-центр-четные} %текст-центр-четные
\fancyhead[RO]{\small {Заключение эксперта \NomerDoc}} %текст-справа-четные
\fancyhead[RE]{\small \NomerDoc} %текст-справа-четные
%

\fancyfoot[R]{\textcolor{black}{ \textit{{\small }} \rule{4cm}{0.1 mm} \hfill \rule{4cm}{0.1 mm}}}

\fancyfoot[CE]{\thepage}% номер страницы слева снизу на четных и справа на нечетных
\fancyfoot[CO]{\thepage}
%\fancyfoot[RO]{}%текст-центр-нечетные
%\fancyfoot[LO]{текст2-слева-четные} %текст2-слева-нечетные
%\fancyfoot[CE]{\thepage} %текст-центр-четные
%\fancyfoot[CO]{\thepage} %текст-справа-четные
%
\renewcommand{\headrulewidth}{0 mm}% толщина отделяющей полоски сверху
%%  Сделаем полоску цветной (красной)
\renewcommand{\headrule}{\hbox to\headwidth{%
		\color{red}\leaders\hrule height \headrulewidth\hfill}}

\renewcommand{\footrulewidth}{0 mm}% толщина отделяющей полоски снизу
\futurelet\TMPfootrule\def\footrule{{\color{black}\TMPfootrule}}
%
%L, C, R — лево, центр и право.
%E, O — четные и нечетные страницы.