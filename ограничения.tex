Выводы настоящего исследования достоверны при соблюдении следующих условий:

1. Идентификационный номер автомобиля, содержащий информацию, необходимую для идентификации ТС, а также данные, содржащиеся в свидетельстве о регстации автомобиля, являются достоверными (определение подлинности номеров и технических документов является прерогативой криминалистической экспертизы).

2.Исходные данные о механизме происшествия и полученных повреждениях, отраженные в материалах дела и используемые в настоящем заключении являются объективными.

3. На момент повреждения ТС было комплектно, на нем отсутсвовали замененные или поврежденные составные части, влияющие на результат исследования (в предоставленных для исследования документах информация, позволяющая судить об обратном, отсутствует.)

4. Размер расходов на восстановительный ремонт определяется на дату
дорожно-транспортного происшествия с учетом условий и границ региональных
товарных рынков (экономических регионов) материалов и запасных частей,
соответствующих месту дорожно-транспортного происшествия.

При иных условиях выводы настоящего заключения могут изменится.