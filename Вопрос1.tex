\textbf{1. \textsl{Какими элементами пассивной безопасности должен быть оснащён автомобиль INFINITI QX80 г.р.з. О376ХР123, VIN JN1JANZ62U0100644 согласно спецификации производителя, присутствует ли на данном автомобиле система пассивной безопасности?}}
\par
Приказом Федерального агентства по техническому регулированию и метрологии от 19 июля 2018 г. № 420-ст с 1 сентября 2018 года на территории Российской Федерации отменены национальные стандарты РФ ГОСТ Р группы 41, утратившие свою актуальность в связи с прямым применением с 28 декабря 2000 года Правил ООН, принятых в соответствии с международным Женевским Соглашением 1958 года.
\par
Технический регламент ТР ТС 018/2011 «О безопасности колесных транспортных средств» устанавливает требования к активной, пассивной, экологической безопасности и защите от несанкционированного доступа. Обязателен для всех автомобилей, ввозимых или производимых в России и странах ЕАЭС.
\par
Согласно Технического регламента таможенного союза ТР ТС 018/2011 «О безопасности колесных транспортных средств», \cite[п. 6]{0182011:tr},  «безопасность транспортного средства - состояние, характеризуемое совокупностью параметров конструкции и технического состояния транспортного средства, обеспечивающих недопустимость или минимизацию риска причинения вреда жизни или здоровью граждан, имуществу физических и юридических лиц, государственному или муниципальному имуществу, окружающей среде».  

Система пассивной безопасности автомобиля — это комплекс элементов и технологий, предназначенных для минимизации травм и защиты жизни водителя, пассажиров, а иногда и пешеходов в случае аварии. В отличие от активной безопасности (которая предотвращает ДТП, например, ABS или ESP), пассивная безопасность активируется во время или сразу после столкновения. 


Требования к пассивной безопасности: защита  при ДТП.

%Подушки безопасности (водительские, передние, боковые, шторки).
%
%Ремни безопасности с преднатяжителями и ограничителями нагрузки.
%
%Конструкция кузова с зонами программируемой  деформации.
%
%Активные подголовники для защиты от травм шеи.
%
%Краш-тесты:
%
%Автомобили должны проходить испытания по методикам, приближенным к Euro NCAP (например, фронтальный удар на 64 км/ч, боковой удар).

К элементам пассивной безопасности, в общем случае, относятся:\\
\textsl{{\textbf{1. Ремни безопасности}}}

 Основным удерживающим  защитным устройством, предназначенным для удержания водителя и пассажиров в креслах при резком торможении, столкновении или перевороте автомобиля являются трехточечные ремни безопасноти.
 
Функция: Удерживают пассажиров на местах, предотвращая удары о элементы салона или вылет из автомобиля.\\
Дополнения:\\
Преднатяжители: Мгновенно натягивают ремень при ударе, устраняя провисание.\\
Ограничители нагрузки: Ослабляют натяжение при критическом давлении, чтобы не повредить грудную клетку.\\
\textsl{{\textbf{2. Подушки безопасности (Airbags)}}}\\
Дополнительным средством защиты являются подушки безопасности, которые дополняют ремни безопасности, создавая мягкий барьер и уменьшая риск травм головы и верхней части тела.  

Подушки безопасности дополняют ремни безопасности, увеличивая общую безопасность, но не могут полностью заменить их. Поэтому рекомендуется всегда использовать ремни безопасности, а подушки безопасности рассматривать как дополнительный уровень защиты.

Типы:\\
Фронтальные (для водителя и переднего пассажира).\\
Боковые (защита таза и грудной клетки).\\
Шторки безопасности (предотвращают травмы головы при боковом ударе).\\
Коленные (защищают ноги водителя).\\
Центральная (между передними сиденьями, снижает травмы при боковых столкновениях).\\
Как работают: Срабатывают за 20–30 миллисекунд после удара, смягчая контакт с твердыми поверхностями.\\
\textsl{{\textbf{3. Конструкция кузова}}}\\
Зоны програмируемой деформации: специальные участки кузова, которые поглощают энергию удара, деформируясь в запрограммированных направлениях. Это снижает силу удара, передающуюся на пассажиров\\
Жесткий каркас салона: Изготавливается из высокопрочной стали, чтобы сохранить «жилое пространство» для пассажиров даже при сильном ударе.\\
Усиленные стойки и двери: Защищают от проникновения посторонних объектов в салон.\\
\textsl{{\textbf{4. Активные подголовники}}}\\
При ударе сзади активные подголовники автоматически перемещаются вперед и вверх, чтобы минимизировать риск травм шеи (так называемого "хлыстового удара").\\
\textsl{{\textbf{5. Стекла  триплекс}}}\\
Особенность: При разрушении рассыпаются на мелкие неострые осколки, уменьшая риск порезов.\\
\textsl{{\textbf{6. Системы защиты пешеходов}}}\\
Приподнимающийся капот: Создает буферную зону между двигателем и капотом при наезде на пешехода.\\
Мягкие бампера: Поглощают удар, снижая травмы ног.\\
\textsl{{\textbf{7. Детские удерживающие системы}}}\\
Автокресла: Фиксируют ребенка в безопасном положении.\\
Крепление ISOFIX: Жестко пристегивает кресло к кузову, исключая смещение.\\
\textsl{{\textbf{8. Система экстренного вызова (например, ГЛОНАС)}}}\\
Функция: Автоматически отправляет координаты ДТП и сигнал SOS в службы спасения через GPS/GSM.\\
\textsl{{\textbf{9. Защита топливной системы}}}\\
Автоматическое отключение топливного насоса при ударе, чтобы предотвратить утечку и возгорание.\\
\textsl{{\textbf{10.Система защиты при опрокидывании
Шторки безопасности и усиленная конструкция крыши помогают защитить пассажиров при опрокидывании автомобиля.}}}\\
\textsl{{\textbf{11. Травмобезопасные элементы салона}}}\\
Скрытые крепления руля и педалей: Уменьшают риск проколов и переломов.\\
Мягкие панели на торпедо и дверях: Снижают травмы при контакте.\\


Техническая документация автомобиля  INFINITI QX80 г.р.з. О376ХР123, VIN JN1JANZ62U0100644 указывает на оснащение автомобиля большим количеством элементов пассивной безопасности. Так, Одобрение типа транмортного средства № TC RU E-CH.MT02.00303, выданное органом по сертификации	“САТР-ФОНД”  в разделе "Оборудование транспортного средства " содержт запись, прямо указывающую на оснащение автомобилей INFINITI QX 80 с VIN JN1JANZ62U??????? элементами пассивной безопасности (выделены курсивом): 
 "омыватель фар, магнитола, проигрыватель компакт-дисков, отопитель, бортовой компьюте,\textsl{ подушки безопасности}, подогрев сидений первого и второго ряда, кондиционер, электронная система контроля устойчивости, система мониторинга давления воздуха в шинах, по заказу: навигационная система, DVD проигрыватель, люк в крыше, датчики парковки, \textit{устройство вызова экстренных оперативных служб}". 
 
 «Руководство по эксплуатации автомобиля INFINITI QX80» в разделе 1 «Безопасность» содержит перечисление  элементов пассивной безопасности, которыми должен быть оснащён исследуемый автомобиль: \textsl{сиденья водителя и пассажиров; подголовники сидений водителя и пассажиров; ремни безопасности водителя и пассажиров; преднатяжители ремней безопасности водителя и переднего пассажира; детские удерживающие системы; фронтальные подушки безопасности водителя и переднего пассажира; боковые подушки безопасности водителя и переднего пассажира; головные подушки безопасности (шторки) левая и правая; система «ЭРА-ГЛОНАС».} 
 
 % TODO: \usepackage{graphicx} required
 \begin{figure}[H]
 	\centering
 	\includegraphics[width=0.8\linewidth]{"../images/foto/расположение подушек"}
 	\caption{Система SRS исследуемого автомобиля (см. РЭ INFINITI QX80 на русском языке с.1-38, дополни-тельные материалы)}
 	\label{SRS}
 \end{figure}
 
Ведущий онлайн-каталог электронных запчастей (EPC) Microcat EPC (Nissan Microcat EPC Online Parts Catalog, \url{https://www.infomedia.com.au/parts/electronic-parts-catalogue/}), интегрирированный с системами управления дилерской деятельностью (DMS) и использующися автопроизводителями  Nissan INFINITI для предоставления информации о подлинных запчастях и аксессуарах для их автомобилей по всему миру, содержит  полный перечень оригинальных запчастей для  исследуемого автомобиля INFINITI QX80, VIN JN1JANZ62U0100644, включая все элементы пассивной безопасности, указанные изготовителем данного автомобиля.

Фронтальный краш-тест, проведенный  NHTSA присвоила QX80 рейтинг в три звезды из пяти в категории фронтального столкновения, что указывает на умеренный уровень защиты от травм 4. Боковой краш-тест: В категории бокового удара автомобиль получил пять звезд, что свидетельствует о высоком уровне безопасности в этом типе столкновений 5. Тест на опрокидывание: в этой категории QX80 также получил три звезды, что указывает на вероятность опрокидывания в 23.50\%, \url{https://www.motorbiscuit.com/the-infiniti-qx80-fares-poorly-in-this-crucial-safety-area/}.

Все вышеперечисленное указывает на то, что согласно спецификации производителя автомобиль INFINITI QX80 2019 годв выпуска должен быть оснащен различными элементами пассивной безопасности.

Натурное исследовании автомобиля INFINITI QX80 г.р.з. О376ХР123, VIN JN1JANZ62U0100644 проведенное  с применением  диагностического оборудования  Consalt III и Launch 431  показало, что система SRS Airbag  
 (раздел «Self-Diagnostic Results» (Результаты самодиагностики) для проверки ошибок (DTC — Diagnostic Trouble Codes)) ошибок не имеет (код NO DTC).   В разделе «Data Monitor» (Режим данных) была проверена активность датчиков. Сотояние датчиков удара (Impact Sensors),  подушек безопасности (Airbag Status) и натяжителей ремней безопасности  отображаются как «Normal»,  что одноверенно указываетна на наличие этих компонентов в автомобиле и на их исправное состояние. При демонтаже подушки безопасности водителя подтверждена конструкция двухступенчатой подушки безопасности. По данным изготовителя, двухступенчатая система позволяет подушке безопасности адаптироваться к различным условиям аварии, обеспечивая оптимальный уровень защиты в зависимости от силы удара, веса пассажира и использования ремня безопасности. В зависимости от серьезности аварии подушка может срабатывать в два этапа.  Первый этап: при незначительном столкновении подушка срабатывает частично, чтобы смягчить удар и минимизировать травмы.  Второй этап: в случае более серьезного столкновения подушка полностью надувается, обеспечивая максимальную защиту водителя. 
 
 По совокупности результатов проведенного исследования, экспертами установлено, что исследуемый автомобиль   INFINITI QX80 2019 года, согласно спецификации производителя,   оснащен следующим комплексом систем пассивной безопасности, направленных на защиту водителя и пассажиров в случае аварии:
 

 
\textbf{ \textbf{1. Подушки безопасности (Airbags)}}

Автомобиль оснащен усовершенствованной двухступенчатой  системой  фронтальных подушек для водителя и переднего пассажира подушек  (AABS) с датчиками ремней безопасности и классификации пассажиров.

 
 Боковые подушки безопасности (передние), встроенные в спинки сидений.
 
 Шторки безопасности (защита головы), охватывающие все три ряда сидений.
 
 Коленная подушка безопасности для водителя (защита ног).
 
 
Процесс развертывания начинается через 0,01 секунды после обнаружения аварии, подушка полностью надувается через 0,04 секунды и начинает сдуваться через 0,1 секунды, как указано в руководствах по безопасности NISSAN и INFINITI, \url{https://www.nhtsa.gov/vehicle-safety/air-bags}.
 
 
\textbf{\textbf{ 2. Ремни безопасности}\\}
 Трехточечные инерционные ремни для всех пассажиров.
 
 Пиропатронные натяжители ремней (Seatbelt Pretensioners) и электромеханическими преднатяжителями ремней безопасности для водителя и переднего пассажира:
 
 Автоматически натягивают ремни при аварии.
 
 Ограничители нагрузки ремней (Load Limiters):
 
 Снижают давление ремня на грудную клетку при резком рывке.
 
\textbf{\textbf{ 3. Конструкция кузова}\\}
 Усиленный каркас кузова (High-Strength Steel):
 
 Зоны программируемой деформации для поглощения энергии удара.
 
 Защита при боковом ударе:
 
 Усиленные стойки и пороги.
 
 Система защиты педалей (при фронтальном ударе педали смещаются, снижая риск травм ног).
 
\textbf{\textbf{ 4. Защита детей}\\}
 Крепления ISOFIX для детских кресел (второй ряд сидений).
 
 Замки ремней безопасности с защитой от детей.
 
\textbf{ \textbf{5. Дополнительные системы}\\}
 Активные подголовники (Active Head Restraints):
 
 Снижают риск травм шеи при ударе сзади.
 
 Система фиксации сидений (при аварии предотвращает смещение кресел).
 
 Защита при опрокидывании (усиленная крыша и каркас).
 
\textbf{\textbf{ 6. Особенности для российского рынка}:\\}
 адаптация к холодному климату (надежность работы датчиков и систем).
 
 
  \par
 В Infiniti QX80 2019 года, как и в подавлябщем большинстве совпеменных автомобилей  ремни безопасности являются  элементом системы безопасности,  обеспечивающем основную защиту и эффективны в различных типах аварий. Подушки безопасности дополняют ремни, увеличивая общую безопасность, но не могут полностью заменить их. Поэтому рекомендуется всегда использовать ремни безопасности, а подушки безопасности рассматривать как дополнительный уровень защиты.
 
 \textbf{{Согласно данных изготовителя автомобиля, 
 		эффективность подушек зависит от правильного использования ремней безопасности и положения сидений. Например, водитель должен находиться на расстоянии не менее 25 см от руля}}.\\
 
 \vspace{3mm}
 
\textbf{ {Таким образом, на  данном автомобиле присутствует система пассивной безопасности, соответствующая спецификации производителя автомобиля}}.\\






 


%
%
%
%\par
%Система пассивной безопасности Infiniti QX80 2019 года это система, направленая на защиту водителя и пассажиров в случае аварии.   В систему входят:
%
%
%
%{ Ремни безопасности:}
%
%\begin{itemize}
%	\item Ремни с преднатяжителями: автоматически натягиваются при резком торможении или ударе.
%	\item Ограничители нагрузки: регулируют усилие натяжения ремня, снижая риск травм грудной клетки.
%	\item Ремни для всех пассажиров, включая третий ряд.
%\end{itemize}
%
%
%{ Конструкция кузова:}
%
%\begin{itemize}
%	\item Зоны программируемой деформации: поглощают энергию удара, минимизируя деформацию салона.
%	\item Усиленный каркас салона (из высокопрочной стали) для защиты от сдавливания.
%\end{itemize}
%
%{ Активные подголовники (для передних сидений):}
%
%\begin{itemize}
%	\item Система защиты от хлыстовых травм (Whiplash Protection): подголовники автоматически смещаются вперед при ударе сзади, снижая риск травм шеи.
%\end{itemize}
%
%{ Система фиксации детских кресел:}
%
%\begin{itemize}
%	\item Крепления LATCH (Lower Anchors and Tethers for Children) для безопасной установки детских автокресел.
%\end{itemize}
%
%{ Защита при опрокидывании:}
%
%\begin{itemize}
%	\item Усиленные стойки крыши и система стабилизации кузова.
%	\item Шторки безопасности активируются при риске переворота.
%\end{itemize}
%
%{ Дополнительные элементы:}
%
%\begin{itemize}
%	\item Аварийный размыкатель аккумулятора: отключает питание при серьезном ДТП, снижая риск возгорания.
%	\item Структура педального узла: предотвращает смещение педалей в салон при фронтальном ударе.
%\end{itemize}
%\par
%{Infiniti QX80 2019 соответствует современным требованиям безопасности, включая:}
%
%\begin{itemize}
%	\item Сертификацию NHTSA (Национальное управление безопасностью движения на трассах США).
%	\item Технологии, одобренные IIHS (Страховой институт дорожной безопасности).
%\end{itemize}
%
%\par
%В Infiniti QX80 2019 года подушки безопасности являются ключевым элементом пассивной безопасности. 
%
%{Типы и расположение подушек безопасности}
%
%\begin{itemize}
%	\item В автомобиле установлено 8 подушек безопасности:
%	\item 2 фронтальные подушки (водитель и передний пассажир).
%	\item 2 боковые подушки (в передних сиденьях, защита грудной клетки).
%	\item 4 шторки безопасности (передние и задние боковые окна, защита головы для всех трех рядов).
%\end{itemize}
%
%{Особенности фронтальных подушек}
%
%\begin{itemize}
%	\item Двухступенчатое срабатывание:
% Система определяет силу столкновения и регулирует скорость надувания подушек:
%	\item При легком ударе подушки надуваются медленнее, чтобы снизить риск травм.
% При сильном ударе — мгновенно.
%\item Оптимизированная форма:
%Подушки водителя и пассажира имеют разную конструкцию, учитывая расстояние до руля и панели приборов.
%\end{itemize}
%
%{Боковые подушки (передние сиденья)}
%
%\begin{itemize}
%	\item Интегрированы в боковины сидений, а не в дверь, что улучшает защиту при боковом ударе.
%	\item Защита грудной клетки и таза:
% Снижают риск травм от контакта с дверью или предметами вне автомобиля.
%\end{itemize}
%
%{Шторки безопасности}
%
%Расположение: Верхние части боковых стоек (A, B, C, D-стоек).
%
%\begin{itemize}
%	\item Защита при перевороте и боковых ударах:
% Шторки остаются надутыми несколько секунд, чтобы предотвратить множественные удары головы.
%\item Покрытие всех трех рядов:
% Даже пассажиры третьего ряда защищены от контакта с окнами или стойками.
%\end{itemize}
%
%{Система управления подушками (ACM — Airbag Control Module)}
%
%\begin{itemize}
%	\item Датчики удара:
% Расположены в передней части, дверях и центральной стойке. Определяют направление, силу и тип удара (фронтальный, боковой, переворот).
%	\item Алгоритмы адаптации:
%Например, при столкновении на низкой скорости подушки могут не сработать, если система считает, что ремней безопасности достаточно.
%\end{itemize}
%
%{Взаимодействие с другими системами}
%
%\begin{itemize}
%	\item Ремни безопасности с преднатяжителями:
% Подушки срабатывают синхронно с натяжением ремней, фиксируя тело в оптимальном положении.
%	\item Отключение передней пассажирской подушки:
% Если на сиденье установлено детское кресло (об этом сигнализирует датчик веса), подушка автоматически деактивируется.
%\end{itemize}
%
%{Технологии материалов}
%
%\begin{itemize}
%	\item Вентиляционные отверстия в подушках:
% Позволяют контролировать скорость сдувания, смягчая удар.
%\item Термостойкая ткань:
% Снижает риск ожогов при контакте с горячим газом (используемым для надува).
%\end{itemize}
%
%{ Краш-тесты и эффективность}
%
%\begin{itemize}
%	\item Оценка NHTSA:
% Infiniti QX80 2019 получил 4 из 5 звезд за защиту при фронтальном и боковом ударах.
%	\item Защита головы и шеи:
% Шторки и активные подголовники снижают риск травм шеи (хлыстовых) на 30–40\%.
%\end{itemize}
%
%{Обслуживание и предупреждения}
%
%\begin{itemize}
%	\item Индикатор на приборной панели:
%Если горит лампа «Airbag», это указывает на неисправность системы (требуется диагностика).
%	\item Замена после срабатывания:
%Подушки и датчики должны быть заменены у официального дилера.
%	\item Ремонт только у профессионалов:
%Cамостоятельное вмешательство может привести к несанкционированному срабатыванию.
%\end{itemize}
%
%\textbf{\textit{Важно!
%Эффективность подушек зависит от правильного использования ремней безопасности и положения сидений. Например, водитель должен находиться на расстоянии не менее 25 см от руля}}.
%
%
%
