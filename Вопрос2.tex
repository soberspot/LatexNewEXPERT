\textbf{\textsl{2.	Находилась ли система пассивной безопасности в автомобиле INFINITI QX80 г.р.з. О376ХР123, VIN JN1JANZ62U0100644 в исправном состоянии на момент ДТП 26.05.2021 г. и находится ли она в исправном состоянии в настоящий момент?}}
\par

Согласно ГОСТ Р 27.102-2021. Надежность в технике. Надежность объекта. Термины и определения» исправное состояние (исправность) это такое состояние объекта, в котором все параметры объекта соответствуют всем требованиям, установленным в документации на этот объект. Сллтветственно, неисправное состояние (неисправность): состояние объекта, в котором хотя бы один параметр объекта не соответствует хотя бы одному из требований, установленных в документации на этот объект,  \cite[п. 12, п.13]{271022021:gost}. 

 Технический регламент таможенного союза ТР ТС 018/2011 «О безопасности колесных транспортных средств», Приложение № 8 «Требование к транспортным средствам, находящимся в эксплуатации» содержит перечень требований к элементам пассивной безопасности. Методы проверки элементов пассивной безопасности определены ГОСТ 33997-2016. «Колёсные транспортные средства. Требования к безопасности в эксплуатации и методы проверки» \cite{33997:gost}. 


Неисправность элементов пассивной безопасности  может быть вызвана повреждением:\\
-	элементов пассивной безопасности, входящих и не входящих в систему SRS;\\
-	электрической цепи, предназначенной для управления элементами пассивной безопасности, входящих в систему SRS.\\

В ходе проведения настоящей экспертизы для определения наличия (отсутствия) неисправностей элементов пассивной безопасности экспертами  были выполнены:\\
- исследование элементов пассивной безопансоти органолептическими методами;\\
- тест системы SRS методом самодиагностики автомобиля;\\
- компьютерная диагностика электронных систем автомобиля, включая тестирование системы SRS.\\

Органолептическими методами было определено, что на момент производства экспертизы:\\

На всех сиденьях: на ленте ремней отсутствуют потертости, разрывы, трещины; катушки  плавно  втягивают ремни; при резком рывке ремни  блокируются; язычки в замок вставляются с слышимым четким щелчком; при нажатии на кнопки разблокировки язычки  легко извлекаются;
замки и ремни надежно зафиксированы на кузове/сиденьях, без люфтов; коррозия, деформация в зоне установки ремней отсутствует.

Сиденья передние и задние - смонтированы штатно,  чехлы на всех сиденьях отсутствуют,  обивка сидений без следов потертостей, швы обивки без разрывов и повреждений. Каркас сидений без деформаций, крепление сидений к полу без люфта, болты сидений визуально затянуты. Подголовники сидений находятся в исходном положении, регулировка работает плавно, люфт подголовников отсутствует. Зона установки сидений без видимых деформаций и иных повреждений.


Визуальным осмотром подушек безопасности установлено, что 
целостность модулей подушек безопасности (передние, боковые, шторки, коленные) не нарушена,  передние подушки (в руле и панели приборов) наличие трещин, деформаций или следов срабатывания не имеют. 
Боковые подушки (в спинках передних сидений) и шторки (под потолочной обивкой) скрыты обшивкой,  целостность обшивки не нарушена. Повреждения коленных подушек (под панелью в области ног водителя и переднего пассажира) отсутствуют.   Облицовка салона без видимых признаков демонтажа и повреждений. Монтажное положение облицовочных элементов салона соответствует заводской установке. 

Подушка безопасности водителя скрыта под панелью приборов. Облицовка панели приборов на стороне водителя не повреждена, видимые признаки демонтажа панели приборов отсутствуют.

Подголовники передних сидений без видимых механических повреждений, ощутима податливость под лёгким давлением на подголовники, что указывает на  подвижность подголовников и отсутствие повреждений механизма выдвижения.


Визуальным осмотром кузова автомобиля на предмет деформации зон поглощения энергии установлено, что каркас салона автомобиля без признаков нарушения герметрии, пространственный карас рамы автомобиля нарушений плоскотности не имеет. Присутствуют  повреждения переднего бампера, усилителя и ударопоглотителя переднего бамепра, капота, левого перенего брызговика,  верхнего усилителя  переднего левого брызговика, верхнего усилителя арки колеса передней правой, панели рамки радиатора,  левого переднего краш-бокса.    

Датчик удара  фронтальных подушек безопасности на момент осмотра демонтирован с посадочного места,  датчик находится в электрической цепи автомобиля и штатным соединением  подключен к блоку подушек безопасности,  повреждения  жгута проводов датчика удара, разъемных соединений визуально отсутствуют.


Автомобиль INFINITI Qx80 оснащен развитой системой самодиагностики электронных компонентов, в том числе системой самодиагностической проверки системы подушек безопасности. Согласно «Руководству по эксплуатации автомобиля INFINITI QX80», стр. 5-12, при включении зажигания автомобиль запускает тест самодиагностики системы SRS. Индикатор SRS на приборной панели  работает согласно описанию руководства по эксплуатации автомобиля. При включении зажигания индикатор  загорается примерно на 7 секунд и гаснет; постоянное свечение или мигание или отсутствие свечения при вкючении зажигания, указывающее на неисправность системы отсутствует.

Согласно   теста самодиагностики,  система SRS исправна.

Далее, с использованием диагностического оборудования  СТОА «Финик-Авто»   сканеров CONSULT III PLUS V.80.21 и LAUNCH X431 была проведена компьютерная диагностика всех электронных систем автомобиля.

В результате диагностики электронных систем автомобиля установлено, что коды ошибок, указывающие на неисправности датчиков системы SRS, обрывы электрических цепей подушек безопасности, преднатяжителей ремней безопасности отсутствуют.


В заключении была произведена проверка функциональности компонентов системы подушек безопасности.

Фронтальный датчик подушек безопасности был отключен путем разъедитения электрического разъема и с помощью диагностического сканера произведена диагностика системы SRS. Коды неисправностей, соответствующие повреждению датчика зафиксированы.  Так же было произведено отключение модуля подушки безопасности водителя путем разъединения разъемов  модуля подушки безопасности и произведена диагностика системы SRS. По окончанию теста  зафиксированы коды неисправностей, соответствующие неисправностям фронтальной подушки безопасности водителя.  Следовательно, электрические и электронные компоненты системы подушек безопасности функционируют исправно.


\begin{itemize}
	\item B1031 – Цепь индикатора отключения подушки безопасности пассажира
	\item B1032 – Цепь индикатора включения подушки безопасности пассажира
	\item B1033 – Высокое сопротивление модуля подушки безопасности пассажира
	\item B1034 – Низкое сопротивление модуля подушки безопасности пассажира
	\item B1049 — высокое сопротивление модуля подушки безопасности водителя
	\item B1050 – Низкое сопротивление модуля подушки безопасности водителя
	\item B1053 — Разрыв цепи модуля подушки безопасности водителя
	\item B1054 — Замыкание на массу модуля подушки безопасности водителя
	\item B1055 – Модуль подушки безопасности водителя замыкается на батарею
	\item B1057 — Разрыв цепи модуля подушки безопасности пассажира
	\item B1058 – Модуль подушки безопасности пассажира замыкается на массу
	\item B1059 – Модуль подушки безопасности пассажира замыкается на батарею
\end{itemize}

Имитация условий срабатывания путем симуляции ускорения/замедления с помощью диагностического сканера   и проверка соответствующей реакции системы SRS  (например, фиксации преднатяжителей или индикатора) в настоящем исследовании не проводилась, так как экспертам было отказано судом в разрушающих методах исследования, а имитация срабатывания подушек безопасности на автомобиле после ДТП не может являеться 100\% безопасной.


Таким образом, в результате проведенного исследования экспертами установлено, что на момент ДТП 26.05.2021 г. система пассивной безопасности находилась в исправном техническом состоянии.  На момент производства экспертизы электронные и электрические элементы системы пассивной безопасности находятся в исправном техническом состоянии, элементы пассивной безопасности кузова автомобиля в фрональной левой части автомобиля в составе: бампера переднего, ударопоглотителя бампера переднего, усилителя бампера переднего, левого краш-бокса, левого переднего брызговика, капота деформированы, следовательно элементы пассивной безопасности кузова транспортного средства на момент настоящего исследования находятся в неисправном техническом состоянии. 

