\begin{center}
	\begin{tabulary}{\textwidth}{LCL}
\hline 
\textbf{Наименование детали}      &   & \textbf{Ремонтное воздействие}\\
\hline    
Турбина левая     &   &    Заменить\\
Блок ДВС    &   &    Отремонтировать гильзованием, заменой колец и прокладок \\
Головка блока цилиндров & & Восстановить седла клапанов пятого цилиндра \\
Впускной тракт & &    Разобрать, прочистить\\
Интеркулер   & &     прочистить\\
 

	\end{tabulary}  
\end{center}



\begin{table}[H]
	{\small \begin{tabular}{c|c|c}
			\hline
			\textbf{n/n } & {\textbf{Момент затяжки болта ГБЦ, Нм}} & \textbf{Ремонттное воздействие} \\
			\hline
			\Rownum & 19.6-23.52 & Временно затянуть \\
			\hline
			\Rownum & 68.7 & Затянуть \\
			\hline
			\Rownum & 0,0 & Ослабить \\
			\hline
			\Rownum & 19.6-23.52 & Временно затянуть \\
			\hline
			%   5 & 68.7 & Затянуть \\
			%  \hline
			\Rownum & $90^{\circ}$$  \approx $ $92^{\circ}$ & Довернуть на угол \\
			\hline
	\end{tabular}}
	\caption{Схема затяжки болтов ГБЦ по данным изготовителя}
	\label{схемазатяжки}
\end{table}\setcounter{rownum}{0}



\begin{table}[h]
	\caption{Таблица зарегистрированных ошибок.}
	\label{table:ошибки}
	\begin{tabular}{c|m{45mm}|m{35mm}|m{63mm}}\hline
		\textbf{  n/n} & \textbf{Код ошибки} & \textbf{Повторяемость} & \textbf{Описание} \\
		\hline 
		\Rownum & P0700 & Фиксируется постоянно & TCU Signal Fault. Система управления трансмиссией (запрос MIL), подсистема электронный блок управления трансмиссией (TCU). Ошибка системы управления коробкой передач \\
		\hline
		\Rownum & P1124 & Фиксируется спорадически & Accelerator Pedal Sensor MalfunctionStuck . Ошибка сенсора педали акселератора.\\ \hline
	\end{tabular}
\end{table}\setcounter{rownum}{0}



\begin{longtable}{|p{5cm}|p{5cm}|p{5cm}|}
	\caption[]{\footnotesize {Таблица сравнения аналогов}} \label{tab:сравнитьаналог}\\ 
	\hline
		\rowcolor[gray]{0.95}
	Компонент сравнения & Исследуемый автомобиль& Аналог (1)  \\ \hline \endhead % повторение заголовка 
	Объем двигателя  &4.0 & 4.0 \\ \hline
	%	\rowcolor[HTML]{EFEFEF} 
	Модификация  &C 63 4.0 7G-Tronic (510 л.с.)  & C 63 4.0 7G-Tronic (510 л.с.) \\ \hline
	Регион  & Краснодар  & Новороссийск\\ \hline
	\rowcolor[HTML]{ FAEBD7} 
	Пробег & 12582   & 33000\\ \hline
	Год выпуска  & 2016  & 2016 \\ \hline
	\rowcolor[HTML]{ FAEBD7} 
	Комплектация  & Расширенный набор опций  & Base \\ \hline
	\rowcolor[HTML]{ FAEBD7} 
	Тип кузова  & Купе  & Седан, 4дв. \\ \hline
	Цена  & --  & 3400000 \\ \hline
	%%% ..............
\end{longtable}




  \begin{longtable}{|p{5cm}|p{5cm}|p{5cm}|}
	\caption[]{\footnotesize {Таблица сравнения аналогов}} \label{tab:12}\\ 
	\hline
	\rowcolor[HTML]{EFEFEF} 
	Компонент сравнения & Исследуемый автомобиль& Аналог (1)  \\ \hline \endhead % повторение заголовка 
	Объем двигателя  &4.0 & 4.0 \\ \hline
	Модификация  &C 63 4.0 7G-Tronic (510 л.с.)  & C 63 4.0 7G-Tronic (510 л.с.)\\ \hline
	Регион  & Краснодар  & Краснодар\\ \hline
	Пробег, км & 12582  & 0\\ \hline
	Год выпуска  & 2016  & 2016 \\ \hline
	Комплектация  & Расширенный набор опций  & Расширенный набор опций \\ \hline
	Тип кузова  & Купе  & Купе \\ \hline
	Цена  &  -- & 8\,461\,554 \\ \hline
	%%% ..............
\end{longtable}






{\footnotesize \
	\begin{longtable}{|p{25mm}|p{50mm}|p{50mm}|p{25mm}|}
		\caption{\footnotesize {Сводная сравнительная таблица поврежденных деталей}}
		\label{tab:sravnenie}\\ \hline
		\textbf{Наименование детали} & \textbf{Заключение эксперта} & \textbf{Акт осмотра ООО "Фаворит"} & \textbf{Дилер Мерседес} \\ \hline \endhead
		
		
\end{longtable}}



{\footnotesize \
\begin{longtable}[h]{c|m{25mm}|m{15mm}|m{15mm}|m{15mm}|m{30mm}|m{35mm}}
	\caption{Таблица истории транспортного средства}
	\label{table:история}\\ \hline
			\textit{\textbf{n/n}} 
			&\textit{\textbf{Причина обращения}} 
			&\textit{\textbf{Дата}} 
			&\textit{\textbf{Пробег}}
			&\textit{\textbf{Пробег от преды- дущего события}} 
	    	&\textit{\textbf{Документ}} 
		    &\textit{\textbf{Примечание}} \\
		\hline \endhead
		\Rownum & P0700 & P0700 & P0700 & P0700 & Фиксируется постоянно & TCU Signal Fault. Система управления трансмиссией (запрос MIL), подсистема электронный блок управления трансмиссией (TCU). Ошибка системы управления коробкой передач \\
		\hline
		\Rownum & P1124 & P0700 & P0700 & P0700 & Фиксируется спорадически & Accelerator Pedal Sensor MalfunctionStuck . Ошибка сенсора педали акселератора.\\ \hline
	\end{longtable}}\setcounter{rownum}{0}





















\begin{table}[t]
	\tabcolsep=0pt%
	\TBL{\caption{Tables which are too long to fit,
			should be written using the ``table*'' environment as~shown~here\label{tab2}}}
	{\begin{fntable}
			\begin{tabular*}{\textwidth}{@{\extracolsep{\fill}}lcccccc@{}}\toprule%
				& \multicolumn{3}{@{}c@{}}{\TCH{Element 1}}& \multicolumn{3}{@{}c@{}}{\TCH{Element 2\smash{\footnotemark[1]}}}
				\\\cmidrule{2-4}\cmidrule{5-7}%
				\TCH{Projectile} & \TCH{Energy} & \TCH{$\sigma_{\mathit{calc}}$} & \TCH{$\sigma_{\mathit{expt}}$} &
				\TCH{Energy} & \TCH{$\sigma_{\mathit{calc}}$} & \TCH{$\sigma_{\mathit{expt}}$} \\\midrule
				\TCH{Element 3}&990 A &1168 &$1547\pm12$ &780 A &1166 &$1239\pm100$\\
				{\TCH{Element 4}}&500 A &961 &$\hphantom{0}922\pm10$ &900 A &1268 &$1092\pm40\hphantom{0}$\\
				\botrule
			\end{tabular*}%
			\footnotetext[]{{Note:} This is an example of table footnote this is an example of table footnote this is an example of table footnote this is an example of~table footnote this is an example of table footnote}
			\footnotetext[1]{This is an example of table footnote}%
	\end{fntable}}
\end{table}
