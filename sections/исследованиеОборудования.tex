\subsection{Описание объекта исследования}

\paragraph{Область применения}

\paragraph{Технические данные}

\paragraph{Гарантийные обязательства}

\paragraph{История ремонта и сервисного обслуживания}

\subsection{Данные осмотра оборудования}

\фотомасштаб{example-image}{}{170mm}

\фотомасштаб{example-image}{}{170mm}

\subsection{Данные осмотра оборудования № 1}


\paragraph{При  осмотре установлено:}

Один

\дварядом{example-image}{}{example-image}{}


\subsubsection{Станок для резки шин Г - 650 № 01.02.08.2019}

Станок предназначен для удаления посадочного кольца шины.
С помощью гидравлического оборудования происходит вытягивание металлической бортовой проволоки.

\фотомасштаб{example-image}{example-image}{145mm}
\vspace{10mm}
\фотомасштаб{example-image}{example-image}{145mm}



\paragraph{При  осмотре установлено:}

........................

\дварядом{example-image}{В гидравлическую систему включен радиатор охлаждения, отсутствующий в штатном исполнении линии}{example-image}{}
\дварядом{example-image}{Крышка заливной горловины сломана}{example-image}{}


Ссылка на техническую документацию:
- Быстро изнашиваемые детали. Руководство по эксплуатации паспорт.
Гидравлический станок «КВ-700» редакция 1.

Внешнее состояние оборудования удовлетворительное.

Фактическое состояние на момент настоящего исследования:

\rule{145mm}{0.1 mm}

\subsubsection{Станок для резки шин Г - 650 № 01.03.08.2019}

Станок предназначен для разрезания подготовленных шин на
 части.

\фотомасштаб{example-image}{Маркировочная табличка станка для резки шин Г - 650}{145mm}
\vspace{10mm}
\фотомасштаб{example-image}{Станок для резки шин Г - 650 № 01.03.08.2019}{145mm}




\paragraph{При  осмотре установлено:}

........................

\дварядом{example-image}{}{example-image}{}

Ссылка на техническую документацию:
- Быстро изнашиваемые детали. Руководство по эксплуатации паспорт.
Гидравлический станок «КВ-700» редакция 1.

Внешнее состояние оборудования удовлетворительное.

Фактическое состояние на момент настоящего исследования:

\rule{145mm}{0.1 mm}

\subsubsection{Шкаф автоматики управления}



\paragraph{Оценка технического состояния ШВ 700}

Оценка технического состояния производится согласно  технической документации:
- п. 3.2.1 Обслуживание, Руководство по эксплуатации Паспорт Станок первичного измельчения ШВ-700 редакция 3.

\paragraph{Оценка технического состояния ШН 700}
Оценка технического состояния производится согласно  технической документации:
п. 3.2.1 Обслуживание, Руководство по эксплуатации Паспорт Станок первичного измельчения ШН-700 редакция 3.

Результаты соответствия оборудования ШВ 700 и ШН 700  технической документации, установленные по методике \cite{ecogold:2021}  приведены ниже в таблице \ref{tab:bolts}. 

%\begin{itemize}
%
%	\item 
%1. Толщина решетки должна быть не менее 10 мм. Фактическая толщина:
%
% \rule{15mm}{0.1 mm}
% 	\item 
%2. Очистить решётку от металлического корда.
%	\item 
%3. Скорость вращения валов редукторов Ц2У-400НМ-50-12У1 должна составлять около 17 об./мин. Скорость вращения обоих редукторов должна быть одинаковая.
%
%Фактическая скорость:  \rule{45mm}{0.1 mm}
%
%	\item 
%4. Целостность и состояние цепей:  \rule{100mm}{0.1 mm}
%	\item 
%5. Целостность и натяжения ремней:  \rule{95mm}{0.1 mm}
%	\item 
%6. Износ втулок:  \rule{50mm}{0.1 mm}
%	\item 
%7. Оценка максимального износа фрез. Износом считается возникновение скругления режущей кромки зуба к основным плоскостям фрезы должно быть не более 6 мм.
%
%Фактический износ:   \rule{15mm}{0.1 mm}
%	\item 
%Толщина фрез должна быть не менее 47 mm. Фактическая толщина:  \rule{25mm}{0.1 mm}
%	\item 
%Глубина от кромки зуба до цилиндрической поверхности межфрезного диска должна составлять не менее 105 mm.  Фактическая глубина:  \rule{25mm}{0.1 mm} 
%
%	\item 
%	Расстояние от внешней поверхности решетки до зуба фрезы должен составлять не более 40 мм.  Фактическое расстояние:  \rule{25mm}{0.1 mm}  
%
%\end{itemize}

%
%1. Измерить толщину решетки, должна быть не менее 10 мм.
%2. Очистить решетку от металлического корда.
%3. Очистить пространство между подшипниками и броней.
%4. Проверить скорость вращения вала редуктора Ц2У-400НМ-50-12У1 должна составить около 17 об./мин. Скорость вращения обоих редукторов должна быть одинаковая.
%5. Проверить целостность и состояние цепей.
%6. Проверить целостность и натяжения ремней.
%7. Проверить отсутствие износа втулок.
%8. Оценить максимальный износ фрез. Износом считается возникновение скругления режущей кромки зуба к основным плоскостям фрезы должно быть не более 5 мм.
%9. Толщины фрезы должна состоять не меньше 18.7мм.
%10. На каждом валу следует измерить глубину от кромки зуба до цилиндрической поверхности каждого межфрезного диска, она должна составлять не менее 77 мм.
%11. Измерить расстояние от внешней поверхности решетки до зуба фрезы, должен составлять от 8 до 12 мм.

\begin{table}[H]
	\caption{\label{tab:bolts} Таблица соответствия шредера требованиям технической документации}
	\centering
	\begin{tabular}{p{68mm}|p{21mm}|p{15mm}|p{21mm}|p{15mm}}
		\toprule
		\multirow{2}{*}{Шредер} & \multicolumn{3}{c}{ ШВ 700 |} & ШН 700\\
		\cmidrule{2-4} \cmidrule{5-5} \\
		{} & Допуск  & Факт & Допуск  & Факт \\
		\midrule
		Толщина решётки & min 10mm  &  & min 10mm  &  \\
		Скорость вращения валов редуктора & 17 об/мин &  & 17 об/мин &  \\
		Целостность и состояние ремней & Да/Нет &  & Да/Нет &  \\
		Целостность и состояние цепей & Да/Нет &  & Да/Нет &  \\
		Износ втулок & Да/Нет &  & Да/Нет &  \\
		Износ скругления режущей кромки зуба фрезы & max 6mm &  & max 5mm &  \\
		Толщина фрезы & min 47mm &  & min 18.7mm &  \\
		Глубина от кромки зуба до цилиндрической поверхности межфрезерного диска  & min 105mm &  & min 77mm &  \\
		Расстояние от внешней поверхности решетки до зуба фрезы & max 40mm &  & 8-12mm &  \\
		\bottomrule
	\end{tabular}
\end{table}


\subsubsection{Магнитный сепаратор ленточный № 01.06.08.2019}

Назначение оборудования -  удаление стальных частиц металлического корда

\фотомасштаб{example-image}{}{145mm}
\vspace{10mm}
\фотомасштаб{example-image}{}{145mm}



\дварядом{example-image}{}{example-image}{}

Материал поступает на магнитный конвейер для отделения металлического корда от резиновой крошки. Работа на станке осуществляется следующим образом: крошка с металлокордом подается в бункер, и далее поступает на ленточный конвейер с магнитным приводным барабаном. Находящиеся в потоке материала (резиновой крошки) магнитовосприимчивые включения под воздействием создаваемого барабаном магнитного поля притягиваются к нему и удерживаются на поверхности огибающей его конвейерной ленты, перемещающей включения в зону разгрузки. Устанавливаемая под магнитным валом пластина - делитель используется для отделения потока немагнитной составляющей материала от потока включений с малой магнитной восприимчивостью, изменяющих траекторию движения под воздействием мощного магнитного поля. На данном этапе происходит удаление 96-98\% металлических включений.



	
	
После измельчения материал через патрубок выгрузки подается в устройство очистки крошки от текстильного корда.



на вибросито-2, где происходит окончательное отделение крошки от текстильного корда и осуществляется рассев крошки на фракции, которые самотеком поступают в бункера-накопители готового продукта или расфасовывается в мешки.
Текстильный корд отсасывается с вибросита и по воздуховодам попадает в пылевой циклон, где текстиль отделяется от воздуха, происходит его коагуляция и текстиль отсеивается в мягкие контейнеры типа "big-bag".


\subsection{Данные осмотр оборудования № 2}
Для корректной оценки производительности линии 
 произведена проверка технического состоянию оборудования.
\subsubsection{Гидравлический станок «КВ-700»}

Ссылка на техническую документацию:
- Быстро изнашиваемые детали. Руководство по эксплуатации паспорт.
Гидравлический станок «КВ-700» редакция 1.

Внешнее состояние оборудования удовлетворительное.

Фактическое состояние на момент настоящего исследования:

\rule{145mm}{0.1 mm}



\paragraph{Анализ результатов исследования}


% TODO: 
\повопросу{Какие изменения были внесены покупателем в оборудование - линия по переработке Ecogold 700 после его ввода в эксплуатацию; повлияли ли произведенные покупателем изменения оборудование - линия по переработке Ecogold 700 на его работоспособность и иные характеристики? Была ли необходимость, рациональность в «несении изменений в Оборудование со стороны покупателя?}


..........................

 В ответ на ваш Исх. №67/25.1 от 01.12.2021 г. направляю перечень изменений, внесенных в конструкцию линии по переработке шин EcoGold- 700:
\begin{itemize}
\item	
 замена подшипника на ведомом валу нижнего шредера Машины первичного измельчения «Шредер», дата выполнения работ 04.03.2020г., согласно претензионного письма, б/н от 18.05.2020г. Согласно чертежей на оборудование должен стоять подшипник 3624, в условии поставки ООО «ЭкоГолдСтандарт» устанавливает подшипники производителя SKF, если установлен подшипник иного типоразмера, то возможно появление несоосности, что в последствии приведет к поломке (разбиванию посадочных мест), для определения типоразмера подшипника, необходимо выполнить демонтаж вала, демонтаж подшипников. Провести измерение габаритных размеров, согласно ГОСТ 5721-75;

 
\item	
 замена фрез и межфрезных дисков на ведущем валу нижнего шредера Машины первичного измельчения «Шредер» дата выполнения работ, ориентировочно 27.08.2020г., согласно письма Исх. № 137 от 25.08.2020г.;
\item	
 укрепление рамы Машины первичного измельчения «Шредер» под двигателем, дата выполнения работ ориентировочно 16.11.2019г., согласно претензионного письма Исх. № 112 от 11.03.2021 г.;
 
\item	
 замена подшипника на валу внешнего ротора Дробилке роторной, дата выполнения работ не известна. Согласно чертежей на оборудование должны стоять подшипники 22216 (производителя SKF), 22316 (производителя SKF), 53516, 53616, если установлены подшипники иного типоразмера, то возможно появление несоосности, что в последствии возможно приведет к поломке (разбиванию посадочных мест), для определения типоразмера подшипника, необходимо выполнить демонтаж привода дробилки, демонтаж подшипника. Провести измерение габаритных размеров, на соответствие. С видео­инструкцией по замене подшипников можно ознакомиться по ссылке https://youtu.be/ACQFxbhSH-4;
\item
 замена ведомого вала Машины первичного измельчения «Шредер», дата выполнения работ не известна, предположительно ведомый вал изготовлен сторонней организацией с отклонениями от чертежей. Для определения того, что вал изготовлен в соответствии с чертежами, необходимо выполнить демонтаж вала, демонтаж фрез и межфрезных дисков, демонтаж подшипников. Провести измерение всех размеров согласно чертежу и выполнить химический анализа стали на соответствие чертежу. Чертежи могут быть предоставлены по запросу, при необходимости;
\item	
 замена быстроходного вала редуктора верхнего шредера Машины первичного измельчения «Шредер», согласно ходатайству от ООО «Икара» от 27.10.2020 г. по настоящему судебному делу, дата выполнения работ не известна; Требуется проверить и определить, что передаточное отношение соответствует 50, заложенного конструктивно. Для получения передаточного числа редуктора требуется вручную прокрутить редуктор и посчитать количество оборотов. Количество оборотов кардана (большее число) поделить на количество оборотов, сделанных колесом (меньшее число). Полученное 	число и есть передаточное число редуктора. Для надежности, рекомендуем проверить все 3 редуктора;
\item	
 доподлинно известно, что внутренний ротор Дробилки роторной произведен сторонней организацией с возможными отклонениями от чертежей и производственной технологии. Гак как длительное время нс заказывался в ООО «ЭкоГолдСтандарт», а исходя из периодичности его плановой замены был заказан у сторонней организации. Необходимо выполнить проверку внутреннего ротора на соответствие чертежам. Чертежи могут быть предоставлены но запросу, при необходимости;
\item	
 были зафиксированы неоднократно пожары, полагаем из-за ненадлежащей эксплуатации линии, к примеру: 17.08.2021 - возгорание верхнего шредера.
\end{itemize}




% TODO: 
\повопросу{Соответствует ли действительности показания/значения/режимы шкафа? Являются ли зафиксированные шкафом внештатные ситуации, подтверждением ненадлежащей эксплуатации оборудования? Вносились ли изменения в настройки шкафа либо в его конструкцию?}

....................


% TODO: 
\повопросу{Есть ли износ (фактический износ) оборудование - линия по переработке Ecogold 700? Какова стоимость данного износа оборудование - линия по переработке Ecogold 700?}

...................................

% TODO: 
\повопросу{Соответствуют ли технические характеристики и производительность оборудования - линии по переработке Ecogold 700 - договору купли-продажи № 67551 от 26 марта 2019 года, а также технической и эксплуатационной документации, переданной покупателю в момент поставки оборудования? Какова реальная производительность оборудования в КГ/Ч?}

Заявленная производительность оборудования линии, согласно в методики \cite{ecogold:2021} приведена ниже в таблице на рис. \ref{} 

\фотомасштаб{example-image}{}{145mm}

Cогласно п. 8 Порядок запуска и эксплуатации линии, Паспорт и руководство по эксплуатации Линии по переработке шин Ecogold-700 редакция 1 замер производительности необходимо начинать не менее чем через 30-60 минут работы линии в зависимости от температуры в помещении.

%На момент исследования замер производительности начинался через 60 минут после пуска линии т.к. температура в помещении составляла менее 15\circ.  .

Экспериментальные результаты замера производительности, выполненные при производстве настоящей экспертизы представлены ниже:

\фотомасштаб{example-image}{}{145mm}


......................

Запуск проверки технических характеристик и производительности линии по переработке шин EcoGold 700.

На первом этапе технологического процесса поступающие со склада шины подаются на участок подготовки шин, где они очищаются от посторонних включений и взвешиваются. Общий вес шин должен составлять 700 кг. После этого шины подаются в блок предварительного измельчения.

Грузовая покрышка рабочим вручную или с помощью подъемного механизма устанавливается на гидравлический станок КВ-700 для вытягивания бортовой проволоки. Этот процесс увеличивает срок службы шредера. Металлический корд составляет до 30 \% массы перерабатываемой покрышки. Извлеченный металл складируется для прессовки и сдачи на металлолом.

Резиновая составляющая шины поступает на резку. Гидравлические станок «Гильотина» разрезает покрышки на части. При этом шины уменьшаются в объеме минимум в 5-7 раз.

Далее подготовленные фрагменты шин по транспортеру подаются на второй этап производства.
Здесь происходит последовательное измельчение кусков шин в резиновую крошку, удаление текстильного и металлического корда, разделение крошки на фракции.

По загрузочному конвейеру легковые автомобильные шины и «чипсы» поступают в машину первичного измельчения - «Шредер», предназначен для получения резиновых «чипсов» с размерами 12х12 мм. На первом этапе производится измельчение кусков шин верхним шредером до размера «чипсы» 70х70 мм. Далее «чипсы» автоматически попадают в нижний шредер, где окончательно измельчаются до размера 12х12 мм.

Затем материал поступает на магнитный конвейер для отделения металлического корда от резиновой крошки. Работа на станке осуществляется следующим образом: крошка с металлокордом подается в бункер, и далее поступает на ленточный конвейер с магнитным приводным барабаном. Находящиеся в потоке материала (резиновой крошки) магнитовосприимчивые включения под воздействием создаваемого барабаном магнитного поля притягиваются к нему и удерживаются на поверхности огибающей его конвейерной ленты, перемещающей включения в зону разгрузки. Устанавливаемая под магнитным валом пластина - делитель используется для отделения потока немагнитной составляющей материала от потока включений с малой магнитной восприимчивостью, изменяющих траекторию движения под воздействием мощного магнитного поля. На данном этапе происходит удаление 96-98\% металлических включений.

Далее материал проходит стадию дробления.
В патрубок загрузки роторной дробилки подается крошка размером 12х12 мм. Попав в зону дробления, материал измельчается до размеров от 0 до 6 мм.
После измельчения материал через патрубок выгрузки подается в устройство очистки крошки от текстильного корда.

Далее материал через циклон-сборник поступает на сетку вибросита, где происходит отделение крошки от текстильного корда. Текстильный корд удаляется с поверхности сетки. На данном этапе удаляется порядка 80\% текстильного корда. Отделенная крошка проваливается на лоток, где происходит ее транспортирование. Готовая крошка поступает на конвейер магнитный- 2, где происходит окончательное отделение крошки от металла.

Затем крошка подается на вибросито-2, где происходит окончательное отделение крошки от текстильного корда и осуществляется рассев крошки на фракции, которые самотеком поступают в бункера-накопители готового продукта или расфасовывается в мешки.
Текстильный корд отсасывается с вибросита и через газоходы попадает в пылевой циклон, где текстиль отделяется от воздуха, происходит его коагуляция и текстиль отсеивается в мягкие контейнеры типа "big-bag".




% TODO: 
\повопросу{Имеются ли в оборудовании, линии по переработке шин Ecogold 700, недостатки (дефекты), контрафактные детали? Если имеются, то какие и каковы причины  их возникновения? Являются ли обнаруженные дефекты оборудования существенными недостатками. неустранимыми недостатками и/или недостатками, которые не могут быть устранены без несоразмерных расходов или затрат времени, или выявляются неоднократно, либо проявляются вновь после устранения? Каковы причины возникновения в процессе эксплуатации открытого огня, случаев возгорания и задымления и есть ли риск повторного задымления и возгорания?}


......................................

В нормах Гражданского кодекса содержится определение контрафактных материальных носителей и контрафактных товаров, а также перечисляются противоправные действия, которые связаны с такими объектами. В ч. 4 ст. 1252 ГК РФ указано: «В случае, когда изготовление, распространение или иное использование, а также импорт, перевозка или хранение материальных носителей, в которых выражены результат интеллектуальной деятельности или средство индивидуализации, приводят к нарушению исключительного права на такой результат или на такое средство, такие материальные носители считаются контрафактными…». Уточняющие положения в области контрафактных товаров содержатся в положениях ч.1 ст.1515 и ч.4 ст. 1519 ГК РФ. К контрафактным следует относить товары, на которых любым образом незаконно используется зарегистрированный товарный знак.

Положения указанных норм позволяют сделать вывод о том, что производство и другое использование контрафактных товаров связано с противоправным использованием интеллектуальной собственности или средствами индивидуализации, при этом такие объекты должны быть выражены в материальном виде. То есть, например, раскрытие секрета производства контрафактом не будет, так же, как и продукция, подделанная без использования чужой интеллектуальной собственности не будет контрафактной.






% TODO: 
\повопросу{Являются ли обнаруженные дефекты оборудования производственными и/или эксплуатационными?}

...........................................

% TODO: 
\повопросу{В случае наличия производственных дефектов оборудования определить стоимость ремонтных работ, произведенных ООО «ИКАРА», для их устранения, а также стоимость фактического износа с учетом определенных затрат? }


..................................

% TODO: 
\повопросу{Меняет ли техническая документация в новой редакции условия эксплуатации и производительность для покупателя? Если да, то в каких разделах?}


В настоящее время, корректировка технической документации регулируется государственными стандартами  \cite{gost:260368} и \cite{gost:25032013}.  
Под корректировкой технической документации понимают процедуру разработки и внесения любых изменений в утвержденную техническую документацию.
Обстоятельства, при которых производят данный процесс

Корректировка технической документации необходима:

При разработке нового продукта (изделия, системы) – по итогам производимых испытаний: предварительных, приемочных, квалификационных.
В период выпуска продукции – при изменении маркировки, условного обозначения, а также при несоответствии новым требованиям стандарта в результате введения государством новых регулирующих документов.
На всех стадиях жизненного цикла продукции – при выявлении недоработок, приводящих к невозможности безопасной (для человека или окружающей среды) эксплуатации изделия.
В период производства продукта – при его совершенствовании.
При техническом перевооружении производства: замене технологической линии или ее отдельных узлов.
При монтаже систем, если какие-то агрегаты нужно заменить на аналогичные (указанные в документации устройства сняты с производства, срок действия сертификатов истек).

Выполнение корректировки технических документов

Изменение технической документации производится на основании извещения об изменении.

Корректировка документа, если она вызывает изменения в других документах, должна сопровождаться одновременной корректировкой всей взаимосвязанной документации.

Изменения могут вноситься в документы, представленные как в электронном виде, так и на бумажном носителе.

Внесение изменений в техническую документацию осуществляется рукописным, машинописным либо автоматизированным методом.

Допускается производить процедуру с помощью нескольких приемов:

Изготовлением новой версии документа, содержащей изменения. Применяется, как правило, для документов в электронном виде.
Заменой, добавлением, исключением отдельных листов документа.
Зачеркиванием, закрашиванием белым цветом, подчисткой, смывкой исключаемых данных и введением новой информации.

При данном процессе регистрируют все изменения, вносимые в документ. Сотрудники нашей компании аккуратно и грамотно выполнят корректировку документов. Осуществляем внесение изменений в техническую, в том числе конструкторскую документацию.

............

Методика проверки линии содержит информацию о производительности линии:

\enquote{Производительность линии по исходному сырью до 700 кг/час, по конечному продукту до 500 кг/час. При переработке покрышек с текстильным кордом необходимо снизить производительность линии до 490-560 кг/ч (по исходному сырью), в зависимости от наличия текстиля.}



.....................................
Станок для резки шин в документации имеет различающиеся названия.

На маркировке: \enquote{Станок для резки шин Г-650}
В методике проверки:  \enquote{Гильотина 700}
................................

% TODO: 
\повопросу{Были ли внесены изменения в механическую и электронную начинку шкафа, влияющие на технические характеристики?}


.................................

% TODO: 
\повопросу{Какие части/детали были изменены истцом в оборудовании, повлиявшие на технические характеристики оборудования?}




.......................

% TODO: 
\повопросу{Производилась ли своевременное и надлежащее техническое обслуживание оборудования?}


...........................

% TODO: 
\повопросу{Пригодно ли к использованию в климатических условиях (место нахождения 	Краснодарский край) оборудование - линия по переработке Ecogold 700 с заявленной производительностью в договоре купли-продажи № 67551 от 26 марта 2019 года? }


..................................











\section{ВЫВОДЫ}


\begin{enumerate}
	
\item \enquote{Какие изменения были внесены покупателем в оборудование - линия по переработке Ecogold 700 после его ввода в эксплуатацию; повлияли ли произведенные покупателем изменения оборудование - линия по переработке Ecogold 700 на его работоспособность и иные характеристики? Была ли необходимость, рациональность в «несении изменений в Оборудование со стороны покупателя?}

\item \enquote{Соответствует ли действительности показания/значения/режимы шкафа? Являются ли зафиксированные шкафом внештатные ситуации, подтверждением ненадлежащей эксплуатации оборудования? Вносились ли изменения в настройки шкафа либо в его конструкцию? }

\item \enquote{Есть ли износ (фактический износ) оборудование - линия по переработке Ecogold 700? Какова стоимость данного износа оборудование - линия по переработке Ecogold 700? }

\item \enquote{Соответствуют ли технические характеристики и производительность оборудования - линии по переработке Ecogold 700 - договору купли-продажи № 67551 от 26 марта 2019 года, а также технической и эксплуатационной документации, переданной покупателю в момент поставки оборудования? Какова реальная производительность оборудования в КГ/Ч? }

\item \enquote{Имеются ли в оборудовании, линии по переработке шин Ecogold 700, недостатки (дефекты), контрафактные детали? Если имеются, то какие и каковы причины  их возникновения? Являются ли обнаруженные дефекты оборудования существенными недостатками. неустранимыми недостатками и/или недостатками, которые не могут быть устранены без несоразмерных расходов или затрат времени, или выявляются неоднократно, либо проявляются вновь после устранения? Каковы причины возникновения в процессе эксплуатации открытого огня, случаев возгорания и задымления и есть ли риск повторного задымления и возгорания? }

\item \enquote{Являются ли обнаруженные дефекты оборудования производственными и/или эсплуатационными? }

\item \enquote{В случае наличия производственных дефектов оборудования определить стоимость ремонтных работ, произведенных ООО «ИКАРА», для их устранения, а также стоимость фактического износа с учетом определенных затрат? }

\item \enquote{Меняет ли техническая документация в новой редакции условия эксплуатации и производительность для покупателя? Если да, то в каких разделах? }

\item \enquote{Были ли внесены изменения в механическую и электронную начинку шкафа, влияющие на технические характеристики? }

\item \enquote{Какие части/детали были изменены истцом в оборудовании, повлияющие на технические характеристики оборудования? }

\item \enquote{Производилась ли своевременное и надлежащее техническое обслуживание оборудования?}

\item \enquote{Пригодно ли к использованию в климатических условиях (место нахождения 	Краснодарский край) оборудование - линия по переработке Ecogold 700 с заявленной производительностью в договоре купли-продажи № 67551 от 26 		марта 2019 года? }
\end{enumerate}


\vspace{10mm}
\noindent{Эксперт}  \hfill    \rule{45mm}{0.1 mm}     {Фефелов С.Л.}\\

\vspace{1mm}

\noindent{Эксперт}  \hfill    \rule{45mm}{0.1 mm}   {Мраморнов А.В.}\\
\vspace{7mm}
\relax



\noindent Приложения к заключению:\\

\noindent \textit{\small 
	%	Приложение № \Rownum. Расшифровка модельных опций ТС \тс \\
	Приложение № \Rownum. Фототаблица повреждений ТС \тс\\
	Приложение № \Rownum. Правоустанавливающие документы\\}

\pagebreak
%
