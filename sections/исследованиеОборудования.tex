\subsection{Описание объекта исследования}

\paragraph{Область применения}

\paragraph{Технические данные}

\paragraph{Гарантийные обязательства}

\paragraph{История ремонта и сервисного обслуживания}

\subsection{Данные осмотра оборудования}

\фотомасштаб{example-image}{}{170mm}

\фотомасштаб{example-image}{}{170mm}

\subsection{Данные осмотра оборудования № 1}


\paragraph{При  осмотре установлено:}

Один

\дварядом{example-image}{}{example-image}{}


\subsubsection{Данные осмотра оборудования № 2}

..................



\paragraph{При  осмотре установлено:}

........................

\дварядом{example-image}{}{example-image}{}




\subsection{Анализ результатов исследования}



%\фотомасштаб{example-image}{}{145mm}



% TODO: 
\повопросу{1.~Определить причины обрывов/разрывов тел жестких каркасно-сборных конструкций, а также деформации жестких каркасно-сборных конструкций на глубокорыхлителях КАМА ТГР 55.7-300, КАМА ТГР 55.9-400?}

Тест Тест
.................................

% TODO: 
\повопросу{2.~Какие конструктивные, эксплуатационные или иные факторы способствовали поломке? }




.......................

% TODO: 
\повопросу{3.~Имелись ли в глубокорыхлителях КАМА ТГР 55.7-300, КАМА ТГР 55.9-400 на момент поломок неисправности, способствовавшие или явившиеся причинами указанных повреждений? }


...........................

% TODO: 
\повопросу{4.~При положительном ответе на вопрос № 3 указать степень очевидности таких неисправностей (дефектов) и возможности их выявления при текущем или сервисном осмотре глубокорыхлителей специалистами соответственно эксплуатанта или сервисного центра. }


..................................


% TODO: 
\повопросу{5.~При выявлении дефектов  определить являются ли выявленные дефекты недостатки устранимыми без несоразмерных расходов и затрат, выявляются неоднократно, появляются вновь после устранения, препятствуют ли они нормальной работе глубокорыхлителей ТГР 55.7-300, КАМА ТГР 55.9-400? }


..................................


% TODO: 
\повопросу{6.~При выявлении дефектов, образовавшихся до поломки, определить характер дефекта заводской, производственный или эксплуатационный (приобретенный)? }


..................................

% TODO: 
\повопросу{7.~При выявлении дефектов в глубокорыхлителях КАМА ТГР 55.7-300, КАМА ТГР 55. 9-400  установить, имелась ли причинно-следственная связь между дефектом и причиной поломки (повреждения) глубокорыхлителей? }


..................................

% TODO: 
\повопросу{8. Соответствуют ли повреждения глубокорыхлителей с фото и видео приложенной флеш-карты повреждениям глубокорыхлителей КАМА ТГР 55.7-300. КАМА ТГР 55.9-400 согласно материалам дела № А32-31734/2020? }


................

% TODO: 
\повопросу{9. Если да, то имеются ли на глубокорыхлителях с фото и видео приложенной флеш-карты конструктивные изменения глубокорыхлителей КАМА ТГР 55.7-300. КАМА ТГР 55.9-400 в виде замены или укрепления срывных болтов? }


..................................


% TODO: 
\повопросу{10. Если да, то	могли ли конструктивные изменения в виде замены или укрепления срывных болтов повлиять на образование деформаций жестких каркасно-сборных конструкций глубокорыхлителей КАМА ТГР 55.7-300, КАМА ТГР 55.9-400? }


..................................


\section{ВЫВОДЫ}


\begin{enumerate}
	
\item  Определить причины обрывов/разрывов тел жестких каркасно-сборных конструкций, а также деформации жестких каркасно-сборных конструкций на глубокорыхлителях КАМА ТГР 55.7-300, КАМА ТГР 55.9-400?
\item 	Какие конструктивные, эксплуатационные или иные факторы способствовали поломке?
\item 	Имелись ли в глубокорыхлителях КАМА ТГР 55.7-300, КАМА ТГР 55.9-400 на момент поломок неисправности, способствовавшие или явившиеся причинами указанных повреждений?
\item 	При положительном ответе на вопрос № 3 указать степень очевидности таких неисправностей (дефектов) и возможности их выявления при текущем или сервисном осмотре глубокорыхлителей специалистами соответственно эксплуатанта или сервисного центра.
\item 	При выявлении дефектов  определить являются ли выявленные дефекты недостатки устранимыми без несоразмерных расходов и затрат, выявляются неоднократно, появляются вновь после устранения, препятствуют ли они нормальной работе глубокорыхлителей ТГР 55.7-300, КАМА ТГР 55.9-400?
\item 	При выявлении дефектов, образовавшихся до поломки, определить характер дефекта заводской, производственный или эксплуатационный (приобретенный)?
\item 	При выявлении дефектов в глубокорыхлителях КАМА ТГР 55.7-300, КАМА ТГР 55. 9-400  установить, имелась ли причинно-следственная связь между дефектом и причиной поломки (повреждения) глубокорыхлителей?
\item 	Соответствуют ли повреждения глубокорыхлителей с фото и видео приложенной флеш-карты повреждениям глубокорыхлителей КАМА ТГР 55.7-300. КАМА ТГР 55.9-400 согласно материалам дела № А32-31734/2020?
\item 	Если да, то имеются ли на глубокорыхлителях с фото и видео приложенной флеш-карты конструктивные изменения глубокорыхлителей КАМА ТГР 55.7-300. КАМА ТГР 55.9-400 в виде замены или укрепления срывных болтов?
\item 	Если да, то	могли ли конструктивные изменения в виде замены или укрепления срывных болтов повлиять на образование деформаций жестких каркасно-сборных конструкций глубокорыхлителей КАМА ТГР 55.7-300, КАМА ТГР 55.9-400?
\end{enumerate}


\vspace{10mm}

\noindent{Эксперт}  \hfill    \rule{45mm}{0.1 mm}   {Мраморнов А.В.}\\
\vspace{7mm}
\relax



\noindent Приложения к заключению:\\

\noindent \textit{\small 
	%	Приложение № \Rownum. Расшифровка модельных опций ТС \тс \\
	Приложение № \Rownum. Фототаблица повреждений ТС \тс\\
	Приложение № \Rownum. Правоустанавливающие документы\\}

\pagebreak
%
