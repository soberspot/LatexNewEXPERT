
%%%%%%%%%%%%%%%%%%%%%%%%%%%%%%%%%%%%%%%%%%%%%%%%%%%%%%%%%%%%%%%%%%%%%%%%%%%%%%%%%
%\subsection{Технические средства}  %% Список не удалять!!!
%\begin{itemize}
%%
%%%
%%%\item   Диагностический сканер SDconnect   с программным обеспечением Xentry Diagnostics v19.11.3.1
%%
%\item   Линейка масштабная магнитная с цветографической шкалой, 100мм
%%
%%%\item   Рулетка измерительная металлическая, 5м
%%%\item  Универсальный стенд для измерения углов установки колес Hunter Engineering %ProAlign с программным инструментом регулировки схождения колес без блокировки руля %автомобиля WinToe
%\item 	Цифровой фотоаппарат Canon 760D s/n 143032001327 с объективом Canon EF-S 18-135, тип используемой памяти: Transcend,  32Gb
%%
%%\item  Специализированное программное обеспечение для расчёта стоимости  восстановительного ремонта, содержащее нормативы трудоёмкости работ, регламентируемые изготовителями транспортного средства     AudaPadWeb, лицензионное соглашение № AS/\- APW-658  RU-P-409-
%\item  Специализированное программное обеспечение для расчёта стоимости  восстановительного ремонта, содержащее нормативы трудоёмкости работ, регламентируемые изготовителями транспортного средства  SilverDAT myClaim,
%лицензионный договор № 1422 от 05.02.2021 на право использования программы для ЭВМ от  DAT IP-Management und Vertriebs GmbH.
%
%%
%\item  Программа обработки фото-видео изображений ImageJ, разработчик  Wayne Rasband (wa-yne@codon.nih.gov),
%свободная лицензия GPL
%%
%\item  ПЭВМ под управлением операционной системы Windows 10 с установленным набором макрорасширений LaTeX системы компьютерной вёрстки TeX, cвободная лицензия LaTeX Project Public License (LPPL)
%%	
%\end{itemize}
%%%%%%%%%%%%%%%%%%%%%%%%%%%%%%%%%%%%%%%%%%%%%%%%%%%%%%%%%%%%%%%%%%%%%%%%%%%%%%%%%%%%%%%%%%%%%%%%%%%%%%

\subsection{Методы исследования}
\begin{itemize}
\item  Органолептический метод – исследование и оценка качества объектов с помощью органов чувств
\item 	Прямой измерительный метод – путем измерения размеров деталей специальными измерительными приборами
\item Расчётный метод (косвенный измерительный метод) – путём расчётов различных параметров на основе результатов измерений и других данных
\item Экспертный метод (метод экспертной оценки) — совокупности операций по выбору комплекса или единичных характеристик объекта, определению их действительных значений и оценкой экспертом соответствия их установленным требованиям и/или технической информации
%\item Метод натурной реконструкции??
\end{itemize}


%\subsection{Исходные данные}
%
%\begin{enumerate}
%	
%	\item Автомобиль \тс \, VIN \vin \, в повреждённом состоянии.
%	\item Цифровая копия видеозаписи \enquote{улица парковка 3\_23\_06\_2020 06.43.00.mp4}, формата  MPEG-4,  размером 33.4 MiB, длительностью 1 min 59 s, 12.275 FPS.
%	\item Светокопия постановления № 18810223177772659936 от 23.06.2020г. по делу об административном правонарушении, 2 лист.
%	\item Светокопия  решения к делу № 12-541/2020 УИД 23RS0041-01-2020-011330-91, 4 листа.
%%	
%%	
%\end{enumerate}

%%           
%\subsection{Обстоятельства дела}
%%
%%\begin{itemize}
%	%
%\item 
...............................................................
	

	%
%\end{itemize} 
%
%
\section{Исследование}
%

\subsection{Исследование предоставленных на экспертизу документов}
%
Согласно предоставленных документов,  исследованию подлежит легковой автомобиль  марки, модели, года выпуска:
\begin{figure}[H]
	\centering
	\includegraphics[width=0.8\linewidth]{example-image}
	\caption{Регистрационные данные исследуемого автомобиля}
	%\label{vin}
\end{figure}


%
Предоставленные для производства исследования документы и материалы позволяют установить следующую историю ремонтов и сервисного обслуживания  транспортного средства VIN \vin, проведенных официальным дилером изготовителя: Таблица \ref{tab:hist}.
%%%%%%%%%%%%%%%%%%%%%%%%%%%%%%%%%%%%%
% История автомобиля
%%%%%%%%%%%%%%%%%%%%%%%%%%%%%%%%%%%%%
%{\small 
%	\begin{longtable}{|p{16mm}|p{12mm}|p{29mm}|G{50mm}|G{41mm}|}
%		\caption[]{\footnotesize {\textbf{История ремонта и сервисного обслуживания по дате и пробегу}}} \label{tab:hist}\\
%		\hline
%		%\rowcolor[HTML]{C0C0C0} 
%		% Заголовки столбцов
%		\textbf{Дата} &\textbf{Пробег, км} &\textbf{№\,Заказ-наряда, накладной}& \textbf{Вид работы}& \textbf{Примечание} \\ \hline \endhead % повторение заголовка 
%		% Строки
%%		22.22.2019 &33\,000  & № 480279303-1 от 03.09.2019& Панель задка  & Замена, окраска \\ \hline
%%		%\rowcolor[HTML]{EFEFEF} 
%%		\Rownum & &n & Боковина задняя левая   & Замена, окраска \\ \hline
%		
%		\ист{arg1}{arg2}{arg3}{arg4}{arg5}
%		\ист{arg1}{arg2}{arg3}{arg4}{arg5}
%		\ист{arg1}{arg2}{arg3}{arg4}{arg5}
%		
%		
%		%%% ..............& 
%		% Обнуляем счетчик строк для следующей таблицы
%\end{longtable}}
%\setcounter{rownum}{0} %сброс счетчика строк в таблице
{\footnotesize \
	\begin{longtable}[h]{m{3mm}|m{14mm}|m{13mm}|m{35mm}|m{55mm}|m{18mm}}
	\caption[]{\footnotesize {\textbf{История ремонта и сервисного обслуживания по дате и пробегу}}} \label{tab:hist} \\ \hline
		\textit{\textbf{n/n}} 
		&\textit{\textbf{Дата}} 
		&\textit{\textbf{Пробег, км}}
		&\textit{\textbf{Документ}} 
		&\textit{\textbf{Содержание}} 
		&\textit{\textbf{Примечание}}\\ \hline \endhead
		

	\Rownum &19.09.2018.& -& Результаты расшифровки VIN& Выпуск исследуемого автомобиля изготовителем&  \\
	\hline
	\Rownum &03.03.2019& -& Требование (претензия) Уткиной Т.А к ООО «Формула-АЦК2» и ООО «Элерон» от 24.04.2023 г.& Заключение договора купли продажи нового автомобиля №43005590 от 03.03.2019 г. между Уткиной Т.А. и ООО «Формула-АЦК2»&  \\
	\hline
	\Rownum &05.05.2019.& -& Требование (претензия) Уткиной Т.А к ООО «Формула-АЦК2» и ООО «Элерон» от 24.04.2023 г.& Передача автомобиля от ООО «Формула-АЦК2» к Уткиной Т.А согласно договору купли продажи нового автомобиля №43005590 от 03.03.2019 г. &  \\
	\hline
	\Rownum &18.04.2020&27 807& Заказ-наряд № 420195627-1& Причина обращения клиента: ТО-2.
	Выполненные работы: замена масла 
	Рекомендации СТОА: нет & Оплата клиентом (Уткиным С.Д.) \\
	\hline
	\Rownum &22.01.2021&42 334&Заказ-наряд № 420203761-1& Причина обращения: ТО-3.
	Выполненные работы: замена масла 
	Рекомендации СТОА: нет & Оплата клиентом (Уткиным С.Д.) \\
	\hline
	\Rownum & 22.01.2021&42 334& Сервисное мероприятие П1-20CI& Причина обращения: сервисное мероприятие по гарантии.
	Выполненные работы: сервисное мероприятие П1 20CI
	Рекомендации СТОА: нет & Оплата по гарантии \\
	\hline
	\Rownum &21.11.2021&54 845&Заказ-наряд № 420211773-1& Причина обращения от клиента: «утопление» автомобиля при преодолении брода, у а/м заглох двигатель и более не заводится, на щитке приборов горит индикация «Привод неисправность, обратитесь в сервис», а/м доставлен на СТОА на эвакуаторе.
	ВЫПОЛНЕНО:Сервис по замене масла. Промывка/очистка масляной системы.
	КОММЕНТАРИИ СТОА: В БУ ДВС ошибки: Стартер не проворачивается заклинен или неисправность в электрической цепи + расходомер воздуха 1 время от времени нет сигнала. Заменён воздушный фильтр ДВС (влажное состояние). Удалена влага из трубопровода наддувочного воздуха, из области цилиндров ДВС после демонтажа свечей накаливания. Заменили масло в ДВС включая промывку/очистку масляной системы (эмульсия)Требуется проверка в ходе дальнейшей эксплуатации автомобиля, возможны скрытые дефекты. На момент нахождения а/м в сервисе эмульсии в АКПП и главной передаче заднего моста не выявлено& Оплата клиентом (Уткиным С.Д.)
	Сведений о разборке и дефектовке ДВС нет  \\
	\hline
	\Rownum &24.07.2022& 61 885& Заказ-наряд № 420217389-1& Причина обращения: ТО-61000, мойка а/м, уборка в моторном отсеке, регулировка стеклоомывателей лобового стекла.
	Выполненные работы: замена масла, инспекционный сервис - на момент нахождения а/м в сервисе ошибок в БУ нет, ходовая часть а/м исправна
	Рекомендации СТОА: выполнить работы по промывке топливной системы, очистке дроссельной заслонки& Оплата клиентом (Уткиной Т.А.) \\
	\hline
%	\Rownum &25.09.2022&69 441&Заказ-наряд № 420218652-2& Причина обращения: куда-то уходит охлаждающая жидкость, трижды доливалась примерно по 1-му литру.
%	Выполненные работы: замена насоса ОЖ. Гарантия - подменный автомобиль на 2 дня.
%	Рекомендации СТОА: нет& Оплата по гарантии \\
%	\hline
	\Rownum &05.02.2023&75 759&Заказ-наряд № 420221038-1. ФОРМУЛА-АЦК2& Причина обращения: при движении, в передней части автомобиля раздался хлопок, после чего авто перестал ускоряться. а двигатель не развивает мощность, по дороге в СТОА ДЦ на комбинации приборов горела индикация сбоя в работе привода, после перезапуска ДВС индикация потухла, а сам ДВС работает нестабильно и из выхлопной системы валит дым.
	Выполненные работы: диагностика, замена турбокомпресора& Оплата ООО «Формула-АЦК2»
	Сведений о разборке и дефектовке ДВС нет
	\\
	\hline
	\Rownum & 17.03.2023&75 759&Заказ-наряд № 420221741-1. ФОРМУЛА-АЦК2& Причина обращения: ТО-5.
	Выполненные работы: замена масла 
	Рекомендации СТОА: нет & Оплата клиентом (Уткиным С.Д.) \\
	\hline
	\Rownum & 19.03.2023& 75 759& Акт выполненных работ от 19.03.2023 г. к заказ-наряду № 420221038-1 от 05.02.2023г& Причина обращения клиента: нет
	Выполненные работы: ведомый поиск неисправностей, замена турбокомпрессора (турбонагнетателя), очистка масляной магистрали& Сведений о разборке и дефектовке ДВС нет \\
	\hline
	\Rownum & 07.04.2023&77 202& Заказ-наряд № 420222190-1 & Причина обращения: после замены турбокомпрессора (турбонагнетателя) периодически, при нажатии на педаль газа, ускорение автомобиля происходит с провалом/задержкой, как будто данный турбонагнетатель не создает необходимое давление.
	При запуске и работе холодного двигателя на прогревочных оборотах, после длительной стоянки, в моторном отсеке, с правой стороны от двигателя, проявляется посторонний вой/гул/цокот, которого раньше не было до проведения ремонта (клиент полагает, что это турбонагнетатель), периодически, когда а/м останавливается на светофоре, обороты двигателя поднимаются до 1100 об/мин и не опускаются до оборотов холостого хода.
	Выполненные работы: ведомый поиск неисправностей/ведомые функции
	Рекомендации СТОА: нет& - \\
	\hline
	\Rownum & 15.04.2023& 77 202& Акт выполненных работ от 15.04.2023 г. к заказ-наряду № 420222190-1 от 07.04.2023г& Причина обращения клиента: после замены турбокомпрессора (турбонагнетателя) периодически, при нажатии на педаль газа, ускорение автомобиля происходит с провалом/задержкой, как будто данный турбонагнетатель не создает необходимое давление.
	При запуске и работе холодного двигателя на прогревочных оборотах, после длительной стоянки, в моторном отсеке, с правой стороны от двигателя, проявляется посторонний вой/гул/цокот, которого раньше не было до проведения ремонта (клиент полагает, что это турбонагнетатель), периодически, когда а/м останавливается на светофоре, обороты двигателя поднимаются до 1100 об/мин и не опускаются до оборотов холостого хода
	Выполненные работы: ведомый поиск неисправностей/ведомые функции. На момент нахождения а/м в сервисе ошибок БУ ДВС не зарегистрировано. Посторонних звуков из области моторного отсека при работе, в том числе прогреве ДВС, при ускорении а/м во время тестовых поездок не возникало. Отклонений в динамике а/м при проведении тестовых поездок нет, провалов/задержки при ускорении не возникало, давление наддува в норме. Фактическое значение оборотов ДВС соответствует заданным 
	Рекомендации СТОА: заменить ветровое стекло и щетки стеклоочистителя (изношены), выполнить работа по промывке топливной системы и очистке дроссельной заслонки& Сведений о разборке и дефектовке ДВС нет \\
	\hline
	
		
\end{longtable}}\setcounter{rownum}{0}

\pagebreak

\noindent Проведя анализ истории автомобиля  можно выделить следующие, вероятно  значимые для исследования, события и признаки:
\begin{enumerate}
\item  
22.01.2021 г. при пробеге 42334 км  при преодолении брода двигатель автомобиля заглох. Вода попала во впускной тракт. Выполнены ремонтные мероприятия: удалена влага из трубопровода наддувочного воздуха, из области цилиндров ДВС после демонтажа свечей накаливания, заменен воздушный фильтр, произведена промывка масляной системы с целью удаления эмульсии, произведена замена масла ДВС. Попадание воды в АКПП, наличие эмульсии в АКПП специалистами сервисного центра не выявлено.
\item 
25.09.2022 г. при пробеге 69 441 км гарантийный ремонт.  Произведена замена насоса охлаждающей жидкости.
\item 
05.02.2023 г. при пробеге 75 759 км внезапное разрушение турбокомпрессора. Произведен гарантийный ремонт: замена турбокомпрессора.
\item 
07.04.2023 г. при пробеге 77 202 обращение в сервисный центр с жалобой на задержки, провалы при нажатии на педаль акселератора, появление нефункциональных шумов в моторном отсеке с правой стороны при пуске холодного мотора после длительной стоянки, периодическое увеличение оборотов холостого хода до 1100 об/мин при остановке на светофоре.
\end{enumerate}
\vspace{5mm}
Из материалов, предоставленных владельцем автомобиля, обращают внимание следующие моменты:\\
а) фрагмент видеозаписи, на котором отчетливо видно, что двигатель дымит, Рис.\ref{dimit} :\\

\begin{figure}[H]
	\centering
	\includegraphics[width=0.9\linewidth]{example-image}
	\caption{Кадр из видеозаписи, предоставленной владельцем автомобиля}
	\label{dimit}
\end{figure}
%\pagebreak
б) фрагмент видеозаписи, где автомобиль удерживает увеличенные обороты  холостого хода, Рис. \ref{fig:1000}:

\begin{figure}[H]
	\centering
	\includegraphics[width=0.9\linewidth]{example-image}
	\caption{Обороты холостого хода 1000 об/мин, педаль акселератора свободна}
	\label{fig:1000}
\end{figure}

Видеозаписи, из которых  сделаны данные фрагменты, были предоставлены владельцем автомобиля в подтверждение претензий по замене турбокомпрессора.


\vspace{3mm}

%%%%%%%%%%%%%%%%%%%%%%%%%%%%%%%%%%%%%%
% 
\subsection{Исследование транспортного средства}

Согласно публично доступных  каталогов запасных частей, размещенных в сети Интернет,  автомобиль с VIN \vin \ имеет следующие идентификационные параметры:
\begin{figure}[H]
	\centering
	\includegraphics[width=0.65\linewidth]{example-image}
	%\caption{Информация расшифровки VIN \vin \ по данным кталога \url{https://partsouq.com/}}
	%\label{vin}
\end{figure}

%\begin{figure}[H]
%	\centering
%	\includegraphics[width=0.8\linewidth]{модельКлип1}
%	%\caption{Информация расшифровки VIN \vin \ по данным кталога \url{https://partsouq.com/}}
%	%\label{vin}
%\end{figure}
%\begin{figure}[H]
%	\centering
%	\includegraphics[width=0.8\linewidth]{модельКлип2}
%	\caption{Информация расшифровки VIN \vin \ по данным кталога \url{https://partsouq.com/}}
%	\label{vin}
%\end{figure}

\subsubsection{Осмотр транспортного средства}

Осмотр и диагностическое исследование  автомобиля \тс \, \грз\, VIN \vin \, производились по адресу: г. Краснодар, ул. Дзержинского-231/2, на территории СТОА ООО «Элерон» 10.05.2023 г. с 10 час. 45 мин. до 16 час 36 мин.   в присутствии представителей ООО «Элерон»: Смирнова Ефима Борисовича – руководителя отдела сервиса и запасных частей, Балюченко Сергея Николаевича – мастера цеха, Куприенко Андрея Юрьевича – автодиагноста, а также представителей собственника исследуемого автомобиля: Уткиной Марины Александровны и Дадаян Феликса Вадимовича в условиях естественного и искусственного освещения с использованием производственных мощностей СТОА ООО «Элерон».

Осмотр автомобиля производился  органолептическим методом. В процессе осмотра выполнялась фото и видео съемка объекта исследования цифровой фотокамерой.\\ 
Маркировочные обозначения, нанесенные на кузове ТС соответствуют записям  в свидетельстве о регистрации ТС  9906 № 057213. Пробег автомобиля, согласно показаний одометра, на момент осмотра составлял 78~874 км.   Кузов автомобиля окрашен двухслойной эмалью черного цвета. Колеса автомобиля 18-ти дюймовые с 5-ю двойными спицами, темно-серого, почти черного цвета,  выполнены из легкого сплава. Шины YOKOHAMA ice GUARD 245/40/R18. Все четыре шины и колеса одинаковых марок, моделей и размерности.

\дварядом{example-image}{Исследуемый автомобиль  \тс \, \vin}{example-image}{Исследуемый автомобиль  \тс \, \vin}
\дварядом{example-image}{VIN на правом переднем брызговике}{\дварядом{}{}{}{}}{VIN на передней стойке}

 Износ шин колес одной оси равномерный, степень износа шин по осям незначительно отличается.   Стекло ветровое имеет трещину от правого нижнего угла до левого верхнего угла. Деформации  наружных панелей кузова аварийного характера отсутствуют. Состояние салона автомобиля удовлетворительное, соответствует возрасту и пробегу.  Передние фары имеют характерные признаки демонтажа.  Левая фара изготовлена 21.03.2017, правая 04.05.2018, \рис{фаралевая}, \рис{фараправая}.
 
 \дварядом{example-image}{Маркировка на корпусе фары левой  \тс \, \vin}{example-image}{Маркировка на корпусе фары правой  \тс \, \vin}
 
 При визуальном осмотре узлов и агрегатов в правой части масляного поддона ДВС обнаружены следы подтекания  моторного масла.
   
 Уровень охлаждающей жидкости на холодном двигателе на отметке «MAX», на прогретом двигателе - выше отметки «MAX». 
 
 Тормозная жидкость  светлого цвета, уровень «MAX» на расширительном бачке.
 
 Моторного масло темного цвета, без видимых посторонних включений, уровень моторного масла в прогретом двигателе: по электронному щупу – «НОРМА»,   измеренный более точным механическим способом с использованием мерного щупа по методике V.A.G,  соответствует «19 л»,  выше максимума 18~л. \рис{уровеньмасла}, \рис{охлаждающая}. Состояние тормозных дисков и тормозных колодок удовлетворительное,  замена не требуется. Люфты шарнирных соединений узлов подвески отсутствуют. Люфт рулевой системы отсутствует,  люфт карданной передачи отсутствует. 
 
\дварядом{example-image}{Мерный масляный щуп  \тс \, \vin}{example-image}{Уровень охлаждающей жидкости  \тс \, \vin}
 
При осмотре наружных элементов кузова  автомобиля визуально были установлены множественные остаточные признаки ремонта кузовных панелей, а именно следы устранения включений в слой ЛКП, следы устранения наплывов. Левое заднее крыло имеет искривленную форму поверхности, отличную от заводской.  Проверка толщины лакокрасочного покрытия  толщиномером  показала следующие результаты:\\

- капот – сколы ЛКП в передней части, толщина ЛКП неравномерная от 77 до 120 мк;

- крыло переднее левое – следы ранее выполненного ремонта, толщина ЛКП не неравномерная от 272 до 514 мкм в верхней части детали;

- дверь задняя левая – следы ранее выполненного ремонта, толщина ЛКП не неравномерная от 210 до 730 мк, максимальная толщина ЛКП наблюдается в задней части двери под наружной ручкой;

- боковина задняя левая – следы ранее выполненного ремонта, толщина ЛКП  неравномерная от 800  и более 2 000 мк;

- крыло переднее правое – следы ранее выполненного ремонта, толщина ЛКП неравномерная от 200 до 300 мк;

- панель крыши – не имеет следов ранее выполненного ремонта, толщина ЛКП от 110 до 130 мк;

- крышка багажника – не имеет следов ранее выполненного ремонта, толщина ЛКП от 110 до 120 мк;

- боковина задняя правая – не имеет следов ранее выполненного ремонта, толщина ЛКП от 110 до 120 мк;

- дверь задняя правая – не имеет следов ранее выполненного ремонта, толщина ЛКП от 110 до 120 мк;

- дверь передняя правая – не имеет следов ранее выполненного ремонта, толщина ЛКП от 110 до 120 мк;

- дверь передняя левая – не имеет следов ранее выполненного ремонта, толщина ЛКП от 116 до 130 мк.\\

Так же в процессе осмотра выявлены незначительные повреждения эксплуатационного характера бампера переднего в виде царапин и задиров  ЛКП и пластика в нижней правой части детали и защиты моторного отсека передней правой в виде трещины.\\

Отдельное внимание специалисты обращают на изменение в конструкции выпускной системы автомобиля, заключающееся в замене  глушителя двумя выпускными трубами, \рис{выпусклевый}, \рис{выпускправый}.\\

\дварядом{example-image}{Левая выпускная труба глушителя  \тс \, \vin}{example-image}{Правая выпускная труба глушителя  \тс \, \vin}

   \begin{figure}[H]
   	\centering
   	\includegraphics[width=0.9\linewidth]{example-image}
    \label{constr}
   	\caption{Автомобиль вид снизу. Стрелки указывают на видимые изменения конструкции выпускной системы ДВС автомобиля \тс VIN \vin}
     \end{figure}
   
    По завершении наружного осмотра автомобиля проведена диагностика ТС VIN \vin\, программно-аппаратными средствами   официального дилера, Рис. \ref{пд}.
       \begin{figure}[H]
    	\centering
    	\includegraphics[width=0.9\linewidth]{example-image}
    	\caption{Потокол диагностики от 10.05.2023}
    	\label{пд}
    \end{figure}
    
      Согласно протоколу диагностики от 10.05.2023, тип, модельный год, модификация, буквенное обозначение двигателя, автоматически определенный VIN полностью соответствуют наружной маркировке исследуемого автомобиля.
   
   \begin{figure}[H]
   	\centering
   	\includegraphics[width=0.9\linewidth]{example-image}
   	\caption{Идентификация ТС согласно протокола диагностики}
   	\label{модельсканер}
   \end{figure}
   
   В результате выполнения диагностического ведомого поиска неисправностей  зафиксирована ошибка "B1291FB:  Панель управления мультимедийной системы Клавиша со стрелкой вправо", тип ошибки 2: спорадическая. Иные  возможные  ошибки DTS не зафиксированы. \\
   
\subsubsection{Исследование двигателя}
     
 Автомобиль \тс VIN \vin \, оснащен  дизельным двигателем DESA объемом 2.0 литра и мощностью 190 л.с. с турбонаддувом. Управление двигателя осуществляется блоком управления BOSH 04L907309L. 
  
 Двигатель  представляет собой 2,0 л  рядный четырёх цилиндровый двигатель с чугунным блоком цилиндров (БЦ) и алюминиевой головкой блока цилиндров (ГБЦ) с 4-я клапанами на цилиндр. Оба впускных и оба выпускных клапана каждого цилиндра расположены поперёк ГБЦ, один за другим по направлению воздушного  потока. Конфигурация впускных и выпускных каналов в ГБЦ такова, что каждый из распредвалов активирует как впускные, так и выпускные клапаны. Распредвалы установлены в корпусе распредвалов (сверху ГБЦ) и образуют с ним единый неразборный узел. Привод распредвалов газораспределительного механизма  осуществляется зубчатым ремнём. Для уменьшения вибрации у двигателя имеются два балансирных вала, привод которых осуществляется через зубчатую передачу от коленвала. Наддув двигателя осуществляется турбокомпрессором  с изменяемой геометрией, который конструктивно выполнен за одно целое с выпускным коллектором. Впрыском топлива управляет система Common Rail с электромагнитными форсунками. Привод навесных агрегатов генератора и компрессора кондиционера осуществляется поликлиновым ремнём от коленвала. Система охлаждения двигателя жидкостная,  разделена на три контура. Первый контур — это микроконтур с очень небольшим объемом, включающий ГБЦ, охладитель EGR, теплообменник и электрический насос.  Второй или главный контур включает ГБЦ, БЦ, масляный радиатор двигателя и трансмиссии, передний радиатор и переключаемый насос охлаждения. Третий или низкотемпературный контур состоит из интеркулера встроенного впускного коллектора, переднего радиатора и электрического насоса охлаждающей жидкости. Для обеспечения такой конструкции системы охлаждения двигателя, рубашка охлаждения в ГБЦ разделена на верхнюю и нижнюю части. Нижняя часть имеет больший объём, чтобы обеспечить лучший отвод тепла из зоны камер сгорания в ГБЦ. Обе части отделены друг от друга в литом теле ГБЦ. Масляный насос с регулируемой производительностью. Может работать в двух режимах: с высоким (4 бар) или с низким (2 бар) давлением масла. Переключение с одного режима на другой происходит при частоте вращения коленчатого вала двигателя 3 000 об/мин. Масляный и вакуумный насосы размещены в одном корпусе и образуют единый узел. Нижняя часть корпуса насоса крепится винтами к блоку цилиндров. Оба насоса используют один общий приводной вал, который приводится зубчатым ремнём от коленвала.
 
 Для улучшения экологичности в двигателе используются следующие технические решения:
 
 -- регулирование фаз газораспределения;
 
 -- система рециркуляции отработанных газов высокого давления;
 
 -- система рециркуляции отработанных газов низкого давления;
 
 -- топливная система с давлением впрыска до 2000 бар;
 
 -- управление сгоранием топлива в зависимости от давления в цилиндре;
 
 -- модуль нейтрализации ОГ с накопительным нейтрализатором NOx.\\
 
Технические характеристики двигателя представлены ниже в таблице:
 
 \begin{center}
 	\begin{tabulary}{\textwidth}{LCL}
 		\hline 
 		\textbf{Параметр}      &   & \textbf{Значение}\\
 		\hline    
 		Объем двигателя, куб.см     &   &    	1968\\
 		Максимальная мощность, л.с.   &   &    	190 \\
 	Максимальный крутящий момент, Н*м (кг*м) при об./мин. & & 400 (41) / 1750 \\
 		Используемое топливо & &    Дизельное топливо\\
 		Расход топлива, л/100 км   & &     4.9\\
 		Тип двигателя   & &    Рядный, 4-цилиндровый\\
 	Система питания  & &    Common Rail\\
 	Максимальная мощность, л.с. (кВт) при об./мин   & &    190 (140) / 3800\\
 		Нагнетатель   & &     Турбина\\
 		Выброс CO2, г/км   & &     118 - 128\\
 		Количество клапанов на цилиндр  & &    4\\
 				Система старт-стоп   & &     да\\
 				Блок цилиндров   &   &    	чугунный R4 \\
 						Головка блока   &   &    	алюминиевая 16v \\
 						Диаметр цилиндра   &   &    81 мм \\
 						Ход поршня  &   &    95.5 мм \\
 							Степень сжатия &   &    15.5\\
 								Ход поршня  &   &    95.5 мм \\
 								Примерный ресурс  &   &    300 000 км \\
% 		\caption{Таблица основных технических характеристик двигателя DESA}
% 		\label{disel}
 	\end{tabulary}  
 \end{center}
  
Двигатель исследуемого автомобиля оснащен турбокомпрессором, имеющим маркировку ККК (Kühnle, Kopp \& Kausch) торговой марки  концерна BorgWarner,  фирменными эмблемами VOLKSWAGEN и AUDI, серийный номер (SERIAL NO) PE50223575 00042, артикул детали (PART.NO)  04L253010.S, \рис{дв}, \рис{мк}.





\дварядом{дв}{Вид на моторный отсек. Крышка ДВС снята}{мк}{Маркировочная табличка турбокомпрессора. Снимок через диагностческое зеркало}

Свечи накаливания  повреждений не имеют. Их общее техническое состояние удовлетворительное, рабочее.  

Компрессометром V.A.G  1763  с адаптером V.A.G 1763/8 проведен замер компрессии в ДВС согласно методике изготовителя.  Результаты измерений представлены ниже в таблице \ref{tab:6}.
 
 
 \begin{longtable}{G{25mm}|M{44mm}|G{70mm}}
 	\caption[]{\footnotesize {Таблица компрессии}} 
 	\label{tab:6}\\ 
 	\hline 
 	\hline  \toprule 
 	\bf  {\footnotesize  № цилиндра}  &\bf {\small Давление, bar} & \bf {\small Показания прибора} \\   \hline\hline  \toprule \endhead 
 	%%%%___________________________________________________________________    
 	\пов{Наименование детали - описание повреждений }{example-image}

\end{longtable}\setcounter{rownum}{0}

 	
По данным V.A.G, для нового двигателя  DESA 2.0 л TDI 140 кВт  значение компрессии должно находиться в диапазоне 25,0-31,0 бар,  граница износа 19,0 бар, а максимальная разница компрессии по цилиндрам не более 5 бар.  Таким образом, компрессия в цилиндрах исследуемого ДВС находится в допустимом диапазоне и указывают на исправное состояние цилиндро-поршневой группы.\\

В рамках настоящего исследования произведен осмотр турбокомпрессора в сборе с выпускным коллектором, который, со слов представителя ООО «Элерон» Балюченко С.Н., ранее был установлен на двигателе исследуемого автомобиля. Турбокомпрессор имеет маркировку  концерна ККК (Kühnle, Kopp \& Kausch), с фирменными эмблемами VOLKSWAGEN и AUDI, артикул (PART.NO) № 04L 253 056L V330, серийный № (SERIAL NO) UH50307492 00384.  Турбокомпрессор имеет значительные механические повреждения деталей, характерные для повреждений, полученных во время работы ДВС:

-- ротор турбокомпрессора разрушен в районе установки крыльчатки компрессора, ротор имеет явный, значительный радиальный и осевой люфт, гайка, которая крепит на роторе крыльчатку компрессора разрушена и отсутствует на штатном месте;

-- крыльчатка компрессора имеет механические повреждения в виде вмятин и задиров и следами наслоения нагара чёрного цвета;

-- корпус компрессора имеет механические повреждения в виде вмятин и задиров;

-- крыльчатки турбины имеет механические повреждения в виде вмятин и задиров и следы наслоения нагара чёрного цвета, соосность её и корпуса турбины нарушена;

-- в выпускном коллекторе двигателя имеется наслоения нагара чёрного цвета.

\дварядом{example-image}{Ранее установленный турбокомпрессор}{example-image}{Маркировочная табличка}

\дварядом{example-image}{Повреждения турбины}{example-image}{Повреждения турбины}


Далее произведена диагностика запуска и работы двигателя исследуемого автомобиля при неподвижном состоянии  автомобиля. 

В ходе данной диагностики установлено:

-- двигатель запускается штатно;

-- двигатель работает устойчиво, равномерно. Нефункциональные шумы, плавание оборотов, повышенная вибрация отсутствуют, ;

-- обороты холостого хода устойчивые, без колебаний, держатся на отметке – 900 об/мин,;

-- при запуске и работе двигателя, в том числе, при увеличении оборотов двигателя до 2 700 об/мин, двигатель не дымит;

-- запах выхлопных газов резкий, отчетливый, не характерный для автомобилей с исправной системой очистки выхлопных газов;

-- внутренние поверхности выхлопных труб слева и справа равномерно покрыты тонким, черным слоем сажи.
 	         	

\subsubsection{Дорожные испытания}
 
 Для дальнейшей проверки автомобиля были проведены дорожные испытания. Перед началом движения  проведена проверка исправности световых приборов, возможности регулировки рулевого колеса, сиденья водителя и пассажира, свеклоподъемников боковых дверей, проверена работа акустической системы. В статическом состоянии неисправности двигателя и  автомобиля в целом не обнаружены. Во время дорожных испытаний  автомобилем управлял представитель владельца Дадаян~Ф.В. Движение автомобиля осуществлялось по дорогам общего пользования в непосредственной близости от  сервисного центра по замкнутому «кольцевому» маршруту длинной 9 км: от ул. Дзержинского-231/2 по ул. Дзержинского в сторону ул. Ближнего Западного обхода, по ул. Ближний Западный обход в сторону ул. Ростовское Шоссе, по ул. Ростовское Шоссе в сторону ул. 3-я Трудовая, разворот на ул. Ростовское Шоссе в районе дома 30/5 и далее обратно по этому же маршруту до ул. Дзержинского-231/2, \рис{shemam}.  Полотно дороги сухое, асфальтобетон, температура окружающего воздуха +10-12\град С. 
   
 Применялся  равномерный режим движения на постоянной скорости, переменный режим движения с чередованием ускорения и торможения, подъемом и спуском,  движение на различных режимах АКПП: использовались  режимы АКПП «Динамик» (одно полное кольцо длинной 9 км) и «Комфорт» (одно полное кольцо длинной 9 км).
 
 \begin{figure}[H]
 	\centering
 	\includegraphics[width=0.9\linewidth]{example-image}
 	\caption{Схема маршрута дорожных испытаний ТС \тс }
 	\label{shemam}
 \end{figure}
 В ходе дорожных испытаний специалистами  проводилась фото и видео съемка,  автомобиль был подключен к диагностической системе ODIS. В реальном времени выполнялось считывание и запись измеряемых величин:\\
 -- Положение педали акселератора\\
 -- Температура охлаждающей жидкости\\
 -- Абсолютное давление на впуске\\
 -- Число оборотов коленчатого вала\\
 -- Давление топлива\\
 -- Давление наддува\\
 -- Фактическое значение датчика давления топлива\\
 -- Давление наддува перед дроссельной заслонкой\\
 -- Пробег, км\\
  
 В результате дорожных испытаний установлено следующее: нефункциональные шумы, перебои в работе двигателя, подвески, коробки перемены передач и других системах автомобиля отсутствуют.   Торможение автомобиля штатное, многократное торможение происходит с одинаковым усилием, неисправности элементов тормозной системы отсутствуют. Работа АКПП штатная и соответствует работе исправного агрегата. В процессе движения водитель автомобиля несколько раз сообщал о задержке  по нажатию на педаль акселератора, что  не  позволяло получить моментальный отклик двигателя на нажатие на педаль акселератора, несколько раз сообщал о "провалах  мощности" при подъеме и при нескольких перестроениях. При этом, анализ диагностических данных, полученных при дорожных испытаниях, не выявил недостатков или ошибок в работе двигателя автомобиля и его систем.  \\
 

\subsection{Анализ результатов исследования}

   Зафиксированные на исследуемом автомобиле  множественные следы устранения аварийных повреждений есть следствие ранее выполняшвегося кузовного ремонта.\\  


\смарт{example-image}{ДТП 12.02.2020 }{example-image}{ДТП 28.03.2023}
\смарт{example-image}{ДТП 16.04.2021}{example-image}{Расшифровка цветовой маркировки}

\vspace{4mm}

Официальный сайт ГИБДД МВД России \url{https://гибдд.рф/check/auto\#} содержит информацию о том, что исследуемый автомобиль  участвовал в дорожно-транспортных происшествиях: ДТП 15.02.2020 г. в 08 ч. 30 мин. в г. Краснодаре (повреждены передняя правая и задняя левая части автомобиля), в ДТП 25.03.2021 г. в 12 ч. 45 мин. в г. Краснодаре (повреждены передняя левая и задняя левая части автомобиля), в ДТП 16.06.2022 г. в 14 ч. 15 мин. в г. Краснодаре (повреждена передняя часть автомобиля).\\
Принимая во внимание факт отсутствия в истории ремонта и сервисного обслуживания официального дилера сведений о проведенных кузовных работах в совокупности с фактическим состоянием на момент исследования наружных конструктивных элементов, специалисты приходят к выводу о том, что устранение повреждений, полученных в зафиксированных ГИБДД дорожно-транспортных происшествиях проводилось  без соблюдения технологии ремонта и окраски кузовных элементов  в  сервисных центрах, не имеющих  технической возможности для приведения поврежденных элементов в надлежащее техническое состояние, соответствующее марке и модели ТС.\\


По совокупности результатов наружного осмотра,  диагностического исследования автомобиля,  дорожных испытаний  недостатки в работе ДВС автомобиля \тс VIN \vin \, не выявлены. 

Имеет место замена турбокомпрессора с заводским артикулом  04L253056L на турбокомпрессор с артикулом 04L253010.S. 

По  официальному электронному каталогу запчастей для автомобилей Volkswagen Group ETKA  можно проследить следующую хронологическую последовательность установки на двигатели DESA  турбокомпрессоров:


История детали :


деталь артикул 04L253010S   отмена 01/08/2013, заменена деталью с артикулом  04L253056C

отмена 01/01/2015, замена на 04L253056E

отмена 01/06/2016, замена на 04L253056H   

отмена 01/08/2016, замена на 04L253056L   

отмена 01/07/2019, замена на 04L253124.

Обменные детали,   альтернативно, с 21/06/2018 артикул  04L253056LX 

отмена 01/06/2019  замена на  04L253056L. 

Какие-либо бюллетени, инструкции, технические ноты, запрещающие или не рекомендующие к применению деталь с артикулом 
04L253010S изготовителем автомобиля не заявлены. Таким образом,  деталь с артикулом 04L253010S  является допустимой заменой детали с артикулом 04L253056L.
 
 Замечания представителя собственника исследуемого автомобиля Дадаян Ф.В.  о том, что при движения автомобиля как в режиме «Динамик», так и при движении в режиме «Комфорт» имели место по одному «провалу» мощности двигателя,  не подтверждаются объективными  данными диагностического исследования.\\ 

 В конструкцию выпускной  системы двигателя внесены изменения. Штатная схема выпуска отработанных газов, приведенная на фрагменте, Рис. \ref{fig:gl}:
  \begin{figure}[H]
 	\centering
 	\includegraphics[width=0.83\linewidth]{example-image}
 	\caption{Выпускная система автомобиля VIN \vin \, по данным Volkswagen Group ETKA. 2- задний глушитель  5- средний глуштель первоначально составляет с деталью 2 единое целое}
 	\label{fig:gl}
 \end{figure}
 
Штатная выпускная система  полностью заменена трубной конструкцией, глушители удалены, \рис{г1}, \рис{г2}, \рис{г3}, \рис{г4}, \рис{лев}, \рис{прав}:\\
 
 \дварядом{example-image}{Глушитель, фрагмент 1}{гexample-image}{Глушитель, фрагмент 2} 
 \дварядом{example-image}{Глушитель, фрагмент 3}{example-image}{Глушитель, фрагмент 4}
 \дварядом{example-image}{ГВыпускная труба слева}{example-image}{Выпускная труа справа}

\vspace{5mm}

  Ниже для сравнения приведены снимки выпускной трубы  идентичного автомобиля AUDI А4 VIN WAUZZZF41KA001292,  изготовленного в один день с исследуемым автомобилем \тс \, VIN \vin: %Вид со стороны заднего бампера, Рис. \ref{fig:4}:
  
   \дварядом{example-image}{Выпускная труба глушителя слева. Заводское исполнение}{example-image}{Справа в заводском исполнении в бампере установлена заглушка}
  
  \begin{figure}[H]
 	\centering
 	\includegraphics[width=0.9\linewidth]{example-image}
 	\caption{Автомобиль AUDI А4 VIN WAUZZZF41KA001292, в комплектации, полностью идентичной исследуемому VIN \vin}
 	\label{fig:4}
 \end{figure}
 
\begin{figure}[H]
	\centering
	\includegraphics[width=0.9\linewidth]{example-image}
	\caption{Идентификационная табличка автомобиля  AUDI А4 VIN WAUZZZF41KA001292}
	\label{fig:4}
\end{figure}

\vspace{4mm}

Анализ результатов дорожных испытаний, в совокупности с результатами  инструментальной и компьютерной диагностики исследуемого автомобиля \тс \, VIN \vin\,, позволяют специалистам  сделать вывод об отсутствии на момент производства исследования неисправностей двигателя исследуемого автомобиля. В то же время, на автомобиле имеет место изменение конструкции выпускной системы ДВС. Как правило, такие работы проводятся с целью получения дополнительной прибавки мощности двигателя.  Негативными последствиями подобного рода изменений  являются увеличение громкости выхлопа и появление резонанса в узком диапазоне оборотов ДВС вследствие возникновения турбулентности потоков газов, как правило, для данного двигателя в диапазоне 3500 об/мин. Наличие неприятного запаха выхлопа может указывать на изменения в системе нейтрализации и рециркуляции  отработанных газов. В таком случае, корректная работа системы управления двигателем становиться возможной только после внесения изменений, корректировок в штатную систему управления двигателем. В силу того, что в настоящее время доступно большое разнообразие  решений, не позволяющих штатной диагностической системой определить наличие изменений, внесенных в программу управления ДВС, разрешение вопроса по внесению изменений в ЭБУ ДВС требует дополнительных исследований с применением специализированных инженерных программных  продуктов.  

%резонатор Гельмгольца
Таким образом, проведенное исследование позволяет сделать вывод о том, что на момент настоящего исследования двигатель автомобиля \тс \, VIN \vin\, находится в исправном техническом состоянии. При этом  на автомобиле \тс \, VIN \vin\  имеет место изменение конструкции системы выпуска отработавших газов, что может оказывать негативное воздействие на работу двигателя в целом.

%\повопросу{2. Если имеются, то какие именно и в чем они выражаются?}
%
%\begin{table}[h]
%	\caption{Таблица зарегистрированных ошибок.}
%	\label{table:ошибки}
%	\begin{tabular}{c|m{45mm}|m{35mm}|m{63mm}}\hline
%		\textbf{  n/n} & \textbf{Код ошибки} & \textbf{Повторяемость} & \textbf{Описание} \\
%		\hline 
%		\Rownum & 0017 & Спорадическая & 16777089 U112100 Шина данных, нет сообщения (00001000 пассивн./спорадич.) \\
%		\hline
%	\end{tabular}
%\end{table}\setcounter{rownum}{0}


\section{Выводы}

\begin{enumerate}
\item \textbf{ В результате проведенного исследования  недостатки в работе ДВС автомобиля Audi А 4, VIN WAUZZZF49KA000830, 2018 года выпуска, принадлежащего Уткиной Т.А., заявленные в претензии собственника автомобиля, не установлены.}

\item  \textbf{Имеет место изменение конструкции системы выпуска отработанных газов двигателя автомобиля,  что  может оказывать негативное влияние на эксплуатационные характеристики автомобиля.% Имеющееся изменение конструкции является обратимым путем установления  системы отработавших газов, предусмотренной изготовителем транспортного средства.
}

\item  \textbf{Внесение изменений в конструкцию системы выпуска отработанных газов не предусмотрено изготовителем транспортного средства.} 
%
%

%	\vspace{5mm}
	
\end{enumerate}
    
\vspace{6mm}

\noindent{Эксперт}  \hfill    \rule{45mm}{0.1 mm}     {Фефелов С.Л.}\\[5mm]

\noindent{ Эксперт}  \hfill    \rule{45mm}{0.1 mm}   {Мраморнов А.В.}\\
%\vspace{7mm}
\relax

\vspace{5mm}

\relax
\noindent Приложение к заключению:\\
\textit{
	%	Приложение № 1. Расшифровка модельных опций ТС \тс \\
%	Приложение № \Rownum. Акт осмотра ТС \тс\\
%	Приложение № \Rownum. Фототаблица повреждений ТС\\
%	Приложение № \Rownum. Калькуляция стоимости восстановительных расходов ТС \тс\\
	%	Приложение № \Rownum. Цифровые копии регистрационных документов ТС\\
	%	Приложение № \Rownum. Цифровая копия постановления по делу об административном правонарушении дорожно-транспортном происшествии\\
	Приложение № \Rownum. Правоустанавливающие документы}
%
%%\begin{table}[h]
%%	\caption{Таблица возможных ошибок}
%%	\label{table:ошибки}
%%{\small 	\begin{tabular}{m{20mm}|m{90mm}|m{43mm}}\hline
%%		\textbf{Код ошибки} & \textbf{Описание} & \textbf{Неисправнось} \\
%%		\hline
%%		P0300 &
%%		Несколько или 1 цилиндр&
%%		Перебои в зажигании\\ \hline
%%		P0301&
%%		Цилиндр 1&
%%		Перебои в зажигании\\ \hline
%%		P0312&
%%		Цилиндр 12&
%%		Перебои в зажигании\\ \hline
%%		P0313&
%%		Распознаны перебои в зажигании при низком уровне топлива& 
%%		 -  \\ \hline
%%		P0314&
%%		Отдельный цилиндр (цил. не определен)&
%%		Перебои в зажигании\\ \hline
%%		P0320&
%%		Зажигание/распределитель, входная цепь числа оборотов двигателя&
%%		Ошибочная функция\\ \hline
%%		P0321&
%%		Зажигание/распределитель, входная цепь числа оборотов двигателя&
%%		Проблема диапазона измерений или мощности\\ \hline
%%		P0322&
%%		Зажигание/распределитель, входная цепь числа оборотов двигателя&
%%		Отсутствие сигнала\\ \hline
%%		P0323&
%%		Зажигание/распределитель, входная цепь числа оборотов двигателя&
%%		Перебои\\ \hline
%%		P0324&
%%		Датчик детонационного сгорания, неисправность в системе управления& 
%%		  -  \\ \hline
%%		P0325&
%%		Датчик детонационного сгорания 1 (банк 1 или отдельный датчик)&
%%		Ошибочная функция\\ \hline
%%		P0326&
%%		Датчик детонационного сгорания 1 (банк 1 или отдельный датчик)&
%%		Проблема диапазона измерений или мощности\\ \hline
%%		P0327&
%%		Датчик детонационного сгорания 1 (банк 1 или отдельный датчик)&
%%		Низкий уровень\\ \hline
%%		P0328&
%%		Датчик детонационного сгорания 1 (банк 1 или отдельный датчик)&
%%		Высокий уровень\\ \hline
%%		P0329&
%%		Датчик детонационного сгорания 1 (банк 1 или отдельный датчик)&
%%		Перебои\\ \hline
%%		P0334&
%%		Датчик детонационного сгорания 2 (банк 2)&
%%		Перебои\\ \hline
%%		P0335&
%%		Датчик положения коленчатого вала, цепь A&
%%		Ошибочная функция\\ \hline
%%		P0336&
%%		Датчик положения коленчатого вала, цепь A&
%%		Проблема диапазона измерений или мощности\\ \hline
%%		P0337&
%%		Датчик положения коленчатого вала, цепь A&
%%		Низкий уровень\\ \hline
%%		P0338&
%%		Датчик положения коленчатого вала, цепь A&
%%		Высокий уровень\\ \hline
%%		P0339&
%%		Датчик положения коленчатого вала, цепь A&
%%		Перебои\\ \hline
%%		P0340&
%%		Датчик положения распределительного вала, цепь A (банк 1)&
%%		Ошибочная функция\\ \hline
%%		P0341&
%%		Датчик положения распределительного вала, цепь A (банк 1)&
%%		Проблема диапазона измерений или мощности\\ \hline
%%		P0342&
%%		Датчик положения распределительного вала, цепь A (банк 1)&
%%		Низкий уровень\\ \hline
%%		P0343&
%%		Датчик положения распределительного вала, цепь A (банк 1)&
%%		Высокий уровень\\ \hline
%%		P0344&
%%		Датчик положения распределительного вала, цепь A (банк 1)&
%%		Перебои\\ \hline
%%		P0349&
%%		Датчик положения распределительного вала, цепь A (банк 2)&
%%		Перебои\\ \hline
%%		P0350&
%%		Катушка зажигания, первичный/вторичный контур&
%%		Ошибочная функция\\ \hline
%%		P0351&
%%		Катушка зажигания A, первичный/вторичный контур&
%%		Ошибочная функция\\ \hline
%%		P0362&
%%		Катушка зажигания L, первичный/вторичный контур&
%%		Ошибочная функция\\ \hline
%%		P0365&
%%		Датчик положения распределительного вала, цепь B (банк 1)&
%%		Ошибочная функция\\ \hline
%%		P0369&
%%		Датчик положения распределительного вала, цепь B (банк 1)&
%%		Перебои\\ \hline
%%		P0370&
%%		Тактовый сигнал с высокой разрешающей способностью, цепь A&
%%		Ошибочная функция\\ \hline
%%		P0371&
%%		Тактовый сигнал с высокой разрешающей способностью, цепь A&
%%		Слишком много импульсов\\ \hline
%%		P0372&
%%		Тактовый сигнал с высокой разрешающей способностью, цепь A&
%%		Слишком мало импульсов\\ \hline
%%		P0373&
%%		Тактовый сигнал с высокой разрешающей способностью, цепь A&
%%		Непостоянные импульсы\\ \hline
%%		P0374&
%%		Тактовый сигнал с высокой разрешающей способностью, цепь A&
%%		Отсутствие импульсов\\ \hline
%%		P0379&
%%		Тактовый сигнал с высокой разрешающей способностью, цепь B&
%%		Отсутствие импульсов\\ \hline
%%		P0385&
%%		Датчик положения коленчатого вала, цепь B&
%%		Ошибочная функция\\ \hline
%%		P0394&
%%		Датчик положения распределительного вала, цепь B&
%%		Перебои\\ \hline
%%	\end{tabular}}
%%\end{table}\setcounter{rownum}{0}
%
%
%%\textit{\textbf{МОДУЛЬ МОЩНОСТИ}}\\
%%\textbf{Дополнительная платная функция для прошивки контроллеров управления двигателем с пересчетом контрольной суммы и RSA.}\\
%%
%%\textbf{Дополнительная функция EDC17  Чтение/запись флэш-памяти блоков управления двигателем в режиме boot Tricore TC17x6 and 17x7 с пересчетом контрольной суммы.
%%	Поддерживаемые контроллеры: CP04, CP14, CP20, CP24, C46, C54.  Поддерживаемые контроллеры: EDC16U1, EDC16U3x, EDC16CPx, Simos pCR2.1.  Flash DSG - DQ250 и DQ200.}\\
%
%
%

%\includepdf[pages=-]{doc_f.pdf}
%\includepdf[pages=-]{doc.pdf}


%\includepdf[pages=-]{myfile.pdf}
%\includepdf[pages=-]{calc.pdf}