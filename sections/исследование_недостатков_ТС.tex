
%
%%%%%%%%%%%%%%%%%%%%%%%%%%%%%%%%%%%%%%%%%%%%%%%%%%%%%%%%%%%%%%%%%%%%%%%%%%%%%%%%%
%\subsection{Технические средства}  %% Список не удалять!!!
%\begin{itemize}
%%
%%%
%%%\item   Диагностический сканер SDconnect   с программным обеспечением Xentry Diagnostics v19.11.3.1
%%
%\item   Линейка масштабная магнитная с цветографической шкалой, 100мм
%%
%%%\item   Рулетка измерительная металлическая, 5м
%%%\item  Универсальный стенд для измерения углов установки колес Hunter Engineering %ProAlign с программным инструментом регулировки схождения колес без блокировки руля %автомобиля WinToe
%\item 	Цифровой фотоаппарат Canon 760D s/n 143032001327 с объективом Canon EF-S 18-135, тип используемой памяти: Transcend,  32Gb
%%
%%\item  Специализированное программное обеспечение для расчёта стоимости  восстановительного ремонта, содержащее нормативы трудоёмкости работ, регламентируемые изготовителями транспортного средства     AudaPadWeb, лицензионное соглашение № AS/\- APW-658  RU-P-409-
%\item  Специализированное программное обеспечение для расчёта стоимости  восстановительного ремонта, содержащее нормативы трудоёмкости работ, регламентируемые изготовителями транспортного средства  SilverDAT myClaim,
%лицензионный договор № 1422 от 05.02.2021 на право использования программы для ЭВМ от  DAT IP-Management und Vertriebs GmbH.
%
%%
%\item  Программа обработки фото-видео изображений ImageJ, разработчик  Wayne Rasband (wa-yne@codon.nih.gov),
%свободная лицензия GPL
%%
%\item  ПЭВМ под управлением операционной системы Windows 10 с установленным набором макрорасширений LaTeX системы компьютерной вёрстки TeX, cвободная лицензия LaTeX Project Public License (LPPL)
%%	
%\end{itemize}
%%%%%%%%%%%%%%%%%%%%%%%%%%%%%%%%%%%%%%%%%%%%%%%%%%%%%%%%%%%%%%%%%%%%%%%%%%%%%%%%%%%%%%%%%%%%%%%%%%%%%%

\subsection{Методы исследования}
\begin{itemize}
\item  Органолептический метод – исследование и оценка качества объектов с помощью органов чувств
\item 	Прямой измерительный метод – путем измерения размеров деталей специальными измерительными приборами
\item Расчётный метод (косвенный измерительный метод) – путём расчётов различных параметров на основе результатов измерений и других данных
\item Экспертный метод (метод экспертной оценки) — совокупности операций по выбору комплекса или единичных характеристик объекта, определению их действительных значений и оценкой экспертом соответствия их установленным требованиям и/или технической информации
%\item Метод натурной реконструкции??
\end{itemize}


%\subsection{Исходные данные}
%
%\begin{enumerate}
%	
%	\item Автомобиль \тс \, VIN \vin \, в повреждённом состоянии.
%	\item Цифровая копия видеозаписи \enquote{улица парковка 3\_23\_06\_2020 06.43.00.mp4}, формата  MPEG-4,  размером 33.4 MiB, длительностью 1 min 59 s, 12.275 FPS.
%	\item Светокопия постановления № 18810223177772659936 от 23.06.2020г. по делу об административном правонарушении, 2 лист.
%	\item Светокопия  решения к делу № 12-541/2020 УИД 23RS0041-01-2020-011330-91, 4 листа.
%%	
%%	
%\end{enumerate}

%%           
%\subsection{Обстоятельства дела}
%%
%%\begin{itemize}
%	%
%\item 
...............................................................
	

	%
%\end{itemize} 
%
%
\section{Исследование}
%

\subsection{Исследование предоставленных на экспертизу документов}
%
Анализ представленных материалов позволяет установить сервисные и ремонтные работы, проведенные на объекте исследования.  Результаты исследования представлены ниже в таблице \ref{tab:hist}.

%%%%%%%%%%%%%%%%%%%%%%%%%%%%%%%%%%%%%
% История автомобиля
%%%%%%%%%%%%%%%%%%%%%%%%%%%%%%%%%%%%%
{\small 
	\begin{longtable}{|p{16mm}|p{12mm}|p{29mm}|G{50mm}|G{41mm}|}
		\caption[]{\footnotesize {\textbf{История ремонта и сервисного обслуживания по дате и пробегу}}} \label{tab:hist}\\
		\hline
		%\rowcolor[HTML]{C0C0C0} 
		% Заголовки столбцов
		\textbf{Дата} &\textbf{Пробег, км} &\textbf{№\,Заказ-наряда, накладной}& \textbf{Вид работы}& \textbf{Примечание} \\ \hline \endhead % повторение заголовка 
		% Строки
%		22.22.2019 &33\,000  & № 480279303-1 от 03.09.2019& Панель задка  & Замена, окраска \\ \hline
%		%\rowcolor[HTML]{EFEFEF} 
%		\Rownum & &n & Боковина задняя левая   & Замена, окраска \\ \hline
		
		\ист{arg1}{arg2}{arg3}{arg4}{arg5}
		\ист{arg1}{arg2}{arg3}{arg4}{arg5}
		\ист{arg1}{arg2}{arg3}{arg4}{arg5}
		
		
		%%% ..............& 
		% Обнуляем счетчик строк для следующей таблицы
\end{longtable}}
\setcounter{rownum}{0} %сброс счетчика строк в таблице

%%%%%%%%%%%%%%%%%%%%%%%%%%%%%%%%%%%%%%
% 
\subsection{Исследование транспортного средства}
%
\subsection{Описание объекта исследования}

\par Из предоставленных материалов   экспертом установлена следующая общая информация об автомобиле, имеющая значение для дачи заключения:
\parbox[]{10cm}{}
\begin{itemize}
	\item[ ] 
	\begin{description}
		\item[Марка, модель] \hfill \тс
%		\item[Обозначение модели] \hfill \model  
		\item[VIN] \hfill \vin
		\item[Год выпуска] \hfill \год
		\item[Шасси] \hfill отсутствует
		\item[Цвет ЛКП] \hfill \цвет
		\item[Пробег] \hfill  \пробег\, км, считан с одометра
		%	\item[Дата начала эксплуатации] \hfill \началоэкспл
		\item[Дата производства] \hfill \датаизготовления
		\item[Двигатель] \hfill \двигатель
		%	\item[Объем двигателя] \hfill 1328 $ \text{см}^3 $
		\item[Свидетельство о регистрации] \hfill \свид
		%	\item[ПТС] \hfill\птс
	\end{description}
\end{itemize}

\subparagraph*{} Идентификационный код автомобиля (VIN)  \vin \, содержит следующую информацию о транспортном средстве, имеющую значение для 	дачи заключения (Рис. \ref{fig:vin} ):\\[3mm]
%	
%	\noindent\parbox[]{10cm}{
	%		\begin{itemize}
		%			\item[ ] 
		%			\begin{description}
			%				\item[Дата изготовления] \hfill \датаизготовления
			%				\item[Расположенние руля] \hfill Left
			%				\item[Двигатель] \hfill \двигатель
			%			%	\item[Объем двигателя] \hfill 1328 $ \text{см}^3 $
			%				\item[КПП] \hfill МКПП
			%				\item[Тип кузова] \hfill  \типкузова
			%				\item[Количество дверей] \hfill 4 
			%				%	\item[VDS] --
			%					
			%			\end{description}
		%	\end{itemize}}\\
%	
%\vspace{3mm}


Описание модели:
\begin{figure}[H]
	\centering
	\includegraphics[width=\linewidth]{example-image}
	\caption[]{{\footnotesize Модель  автомобиля VIN \vin\, по данным\textit{ \url{https://www.lexus-tech.eu/LegalNotice.aspx\&vin=}\вин}} }
	\label{fig:vin}
\end{figure}

\vspace{3mm}
%
%\begin{figure}[H]
%    \centering
%    \includegraphics[width=0.65\linewidth]{example-image}
%    \caption[]{Диаграмма рыночной стоимость автомобиля, аналогичного исследуемому \тс\ %, \textsl{источник:} %\url{https://spec.drom.ru}, \url{https://automama.ru/ocenka-avto}
	%    }
%    \label{fig:рыночная}
%\end{figure}
%
%Рыночная стоимость  автомобиля, аналогичного исследуемому \тс, по данным специализированных открытых источников на момент повреждения  составляла 590 000 (Пятьсот девяносто тысяч) рублей.\\
%Источник:  \url{https://automama.ru/ocenka-avto}, \url{https://spec.drom.ru}


\subsubsection{Осмотр транспортного средства}

%   \osm\, экспертом-техником проведён осмотр повреждённого транспортного средства \tc, регистрационный знак \grz. Осмотр проводился в сухую, ясную погоду с 10-00  до 10-30 на открытой площадке   по адресу: \местоосмотра.  При осмотре присутствовали \присутствовали, владелец транспортного средства \тс.  Соответствие маркировочных обозначений на кузове представленного ТС записям в регистрационных документах ТС экспертом-техником установлено. Видимые изменения конструкции ТС отсутствуют. Представленный на исследование автомобиль \тс\, регистрационный знак \грз\, имеет кузов типа <<\типкузова». Кузов автомобиля окрашен двухслойной   %лессирующей (с металликовым эффектом) 
%   эмалью (краской)  \colr \, цвета.
Осмотр транспортного средсвта экспертом не производился ввиду того, что на момент настоящего исследования автомобиль \тс \, регистрационный знак \грз \, восстановлен.
22.09.2020 авомобиль \тс \, регистрационный знак \грз\, был осмотрен специалистом ООО \enquote{АТБ-Сателлит}. В присутствии представителя ООО \enquote{ОПТИМА} Вовковича Владимира Васильевича составлен Акт осмотра  транспортного средства № 1722125, л.д. 24 и заключение к акту осомтра  № 1722125, л.д. 25. С повреждениями атомобиля участники осмотра согласилсись.  В процессе осмотра производилось фотографирование повреждений автомобиля \тс\, фотокамерой HUAWEI PAR-LX1. Цифровые копии фотоснимков в количестве 78 файлов формата jpg  с сохраненными техническмими данными изображений (EXIF) предоставлены эксперту. Фотоснимки удовлетворительного качества, информативны и пригодны для производства исследования и дачи заключения.

%\subsubsection{Исследование обстоятельств дорожно-транспортного происшествия и установлении причин возникновения повреждений транспортного средства}
%          
%    21.02.2020, водитель автомобиля \tca\, регистрационный номер E682KO123 не убедился в безопасности маневра при перестроении направо и допустил столкновение с автомбилем \тс, двигающимся не меняя полосы движения в попутном направлении  по ул. Гоголя в сторону ул. Красной. По факту дорожно-транспортного происшествия участниками происшествия  составлено извещение о дорожно-транспотном происшествии. Фотоизображения места ДТП представлены ниже на рис. \ref{ris:images/дтп1} и рис. \ref{ris:images/дтп2}.
%    
%    \begin{figure}[!h]\centering
	%        \parbox[t]{0.49\textwidth}
	%        {\centering
		%            \includegraphics[width=.49\textwidth]{example-image}
		%            \caption{\footnotesize {Фото места ДТП. На переднем плане автомобиль второго участника ДТП}}
		%            \label{ris:images/дтп1}}
	%        \hfil \hfil%раздвигаем боксы по горизонтали 
	%        \parbox[t]{0.49\textwidth}
	%        {\centering
		%            \includegraphics[width=.49\textwidth]{example-image}
		%            \caption{\footnotesize {Фото места ДТП. Автомобиль \тс}}
		%            \label{ris:images/дтп2}}
	%    \end{figure}
%    
%    Причины возникновения технических повреждений и возможность их отнесения к
%рассматриваемому ДТП исследованы при осмотре ТС. Для определения причины возникновения повреждений, указанных в Акте осмотра ТС  №  \NomerDoc\, (Приложение № 1) экспертом-техником изучены документы, представленные Заказчиком. По предоставленным документам экспертом установлена причина ДТП, установлены обстоятельства ДТП, выявлены повреждения ТС и установлены причины их образования. Проведено исследование характера выявленных повреждений, сопоставление повреждений ТС потерпевшего с повреждениями ТС иных участников ДТП в соответствии со сведениями, зафиксированными в документах о ДТП.  Проведена проверка взаимосвязанности повреждений на ТС с заявленными обстоятельствами ДТП. 
%
%В результате проведённых исследований эксперт-техник приходит к заключению о соответствии механических повреждений, имеющихся на \тс\, регистрационный знак \грз\, на момент осмотра заявленным обстоятельствам. 


%\subsubsection{Исследование наличия, характера и объёма технических повреждений}
%
%  Наличие, характер и объем технических повреждений транспортного средства \tc\, регистрационный знак \grz, исследованы в присутствии заинтересованных лиц,  зафиксированы в акте осмотра № \NomerDoc\,  (Приложение, <<Акт осмотра>> ),  и фотоматериалах (Приложение, <<Фототаблица>>) по принадлежности. Планируемые (предполагаемые) ремонтные воздействия для восстановления повреждённого  транспортного средства назначены экспертом-техником с учётом особенностей конструкции и рекомендаций изготовителя  транспортного средства, укрупненных показателей трудозатрат по кузовному ремонту и устранению перекосов проёмов и кузова легковых автомобилей иностранных производителей, приложение 3 к приложению к Положению Банка России от 19 сентября 2014 года № 432-П и приведены ниже в таблице \ref{tab:5}.

Повреждения транспортного средства \тс \, регистрационный знак \грз\, определены экспертом по материалам гражданского дела \delonum\, и представлены ниже в таблице \ref{tab:6}

%\pagebreak
\begin{longtable}{G{3mm}|M{90mm}|G{60mm}}
	\caption[]{\footnotesize {Повреждения автомобиля, установленные при его осмотре}} 
	\label{tab:6}\\ 
	\hline 
	\hline  \toprule 
	\bf  {\footnotesize  n/n}  &\bf {\small Наименование  детали с описанием повреждения} & \bf {\small Изображение} \\   \hline\hline  \toprule \endhead 
	%%%%___________________________________________________________________    
	\пов{Наименование детали - описание повреждений }{example-image}

\end{longtable}\setcounter{rownum}{0}

	
	%\subsubsection{Определение стоимости восстановительных расходов}
	\subsubsection{Определение стоимости восстановительного ремонта}
	
	
	В соответствии с действующей Единой методикой размер расходов на восстановительный ремонт определяется исходя из стоимости ремонтных работ (работ по восстановлению, в том числе окраске, контролю, диагностике и регулировке, сопутствующих работ), стоимости используемых в процессе восстановления транспортного средства деталей (узлов, агрегатов) и материалов взамен повреждённых. Расчёт размера расходов (в рублях) на восстановительный ремонт производится по формуле: 
	
	\begin{equation}\label{eq:cr}
		C_{\text{вр}}  =\sum{C_{ip}}= \sum\left({T_{ij}}\cdot {C_{i\text{нч}}}\right) + \sum{C_{ip^{\text{\,\,\,руб}}}} , \,\,\,\text{где:} 
	\end{equation}
	%\vspace{2mm}
	\begin{itemize}
		\item[ ]$ C_{ip} $ -- стоимость работ i-го вида: $C_\text {зам} $, $ C_\text{восст} $, $ C_\text{рег} $, $C_\text{контр} $, $ C_\text{антикор} $, $ C_\text{зч} $, $ C_\text{ом} $,$ C_\text{соп} $, $ C_\text{вм} $, руб;
		\item[ ]$ T_{ij} $ -- трудоёмкость j-й операции(комплекса) по i-му виду работ, руб;
		\item[ ]$ C_{i\text{нч}} $ -- стоимость нормо-часа по i-му виду работ, руб;
		\item[ ]$ C_{ip^\text{\,\,руб}} $ -- стоимость работ $ C_{ip} $, принятая непосредственно в денежном выражении, руб.
	\end{itemize}
	
	\par При определении стоимости восстановительного ремонта АМТС с учётом износа под износом следует понимать количественную меру физического старения АМТС и его элементов, достигнутого в результате эксплуатации, т.е. эксплуатационный износ.
	%
	Расчёт износа производится в  соответствии с Положением Банка России от «19» сентября 2014 года № 432-П «О единой методике определения размера расходов на восстановительный ремонт в отношении повреждённого транспортного средства» [3].
	Износ комплектующих изделий (деталей, узлов, агрегатов) рассчитывается по следующей формуле:
	%
	%
	%
	\begin{equation}\label{eq:I}
		\text{И}_{\text{ки}} 
		= 100\cdot\left( 1-e^ {-\left( \Delta_{T} \cdot T_{\text{КИ}} + \Delta_{L} \cdot L_{\text{КИ}} \right)}\right), \,\,\,\,\text{где:}   
	\end{equation}
	%
	\begin{itemize}
		\item[ ]$ \text{И}_{\text{ки}} $ -- износ комплектующего изделия (детали, узла, агрегата) (процентов); 
		\item[ ]$ e $ -- основание натуральных логарифмов (e =  2,72);
		\item[ ]$ \Delta_{T}$ --  коэффициент, учитывающий влияние на износ комплектующего 	изделия (детали, узла, агрегата) его срока эксплуатации;
		\item[ ]$ T_{\text{КИ}} $ -- срок эксплуатации комплектующего изделия (детали, узла, агрегата), (лет);
		\item[ ]$ \Delta_{L} $ -- коэффициент, учитывающий влияние на износ комплектующего (детали, узла, агрегата) величины пробега транспортного средства с этим комплектующим изделием;
		\item[ ]$ L_{\text{КИ}} $ -- пробег транспортного средства на дату дорожно-транспортного происшествия (тысяч километров).  
	\end{itemize}
	\vspace{5mm}
	\par Значения коэффициентов $ \Delta_{T}$  и $ \Delta_{L} $  для различных категорий и марок транспортных средств приведены в Приложении 5. сп. лит~[3]. При этом, на комплектующие изделия (детали, узлы, агрегаты), которые находятся в заведомо худшем состоянии, чем общее состояние транспортного средства в целом, и его основные части, вследствие влияния факторов, не учтённых при расчёте износа (например, проведение ремонта с нарушением технологии, не устранение значительных повреждений лакокрасочного покрытия), может быть начислен дополнительный индивидуальный износ. 
	Износ шины транспортного средства рассчитывается по следующей формуле:
	\begin{equation}\label{eq:sh}
		\text{И}_{\text{ш}} = \frac{\text{Н}_{\text{н}}-\text{Н}_{\text{ф}}}{\text{Н}_{\text{н}}-\text{Н}_{\text{доп}}} \cdot{100}\%,  \,\,\,\,\text{где:} 
	\end{equation}
	%
	\begin{itemize}
		\item[ ] $ \text{И}_{\text{ш}} $ -- износ шины, \%;
		\item[ ] $ \text{Н}_{\text{н}} $ -- высота рисунка протектора новой шины, мм;
		\item[ ] $\text{Н}_{\text{ф}} $ -- фактическая высота рисунка протектора шины, мм;
		\item[ ] $ \text{Н}_{\text{доп}} $ --минимально допустимая высота рисунка протектора шины в соответствии с требованиями законодательства Российской Федерации, мм.
	\end{itemize}
	%
	\vspace{5mm}
	\relax
	%\renewcommand\baselinestretch{1}\small\normalsize
	%
	Износ шины дополнительно увеличивается для шин с возрастом от 3 до 5 лет - на 15 процентов, свыше 5 лет - на 25 процентов.
	
	
	\subsubsection{Данные для расчёта}
	
	\noindent Объект экспертизы:  транспортное средство \tc\,
	регистрационный знак \грз;\\ 
	VIN: \вин;\\
	Пробег:    \пробег\,, км;\\
	Год выпуска:     \год;\\ 
	Дата  производства:  \датаизготовления;\\
	Дата ДТП:  \датадтп;\\
	
	На момент ДТП ввтомобиль эксплуатаировался с 01.02.2018 по 09.04.2020: 
	\begin{itemize}
		\item[ ]$ T_{\text{КИ}} = 2.5 $ 
		\item[ ]$ L_{\text{КИ}} = 57.646$
		\item[ ]$ \Delta_{T} = 0.045 $ 
		\item[ ]$ \Delta_{L} = 0.0024 $
	\end{itemize}
	\begin{equation}\label{eq:I}
		\text{И}_{\text{ки}} 
		= 100\cdot\left( 1-e^ {-\left( \Delta_{T} \cdot T_{\text{КИ}} + \Delta_{L} \cdot L_{\text{КИ}} \right)}\right) = 100\cdot\left( 1-e^ {-\left( 0.045 \cdot 2.5 + 0.0024 \cdot 57.646 \right)}\right) =  23.92 \%  
	\end{equation}
	
	Перечень ремонтных воздействий, необходимых для устранения повреждений ТС \тс\, представлен ниже в таблице \ref{tab:61}:\\
	%Рыночная стоимость ТС: \tc\,
	%регистрационный знак \grz \, по данным открытых специализированных информационных источников составляет: $640 000$ (Шестьсот сорок тысяч) рублей.\\
	%
	%
	%    \begin{equation}\label{eq:I}
		%    \text{И}_{\text{ки}} 
		%    = 100\cdot\left( 1-e^ {-\left( \Delta_{T} \cdot T_{\text{КИ}} + \Delta_{L} \cdot L_{\text{КИ}} \right)}\right), \,\,\,\,\text{где:}   
		%    \end{equation}
	%   
	%  
	%\pagebreak
	%\subsubsection{Ремонтные воздействия, необходимые для устранения повреждений}
	
	\setcounter{rownum}{0}
	
	\begin{longtable}{G{3mm}|M{130mm}|G{5mm}|G{5mm}|G{5mm}}
		\caption[]{Ремонтные воздействия, необходимые для устранения повреждений ТС \тс}
		\label{tab:61}\\
		\hline  \hline   \toprule 
		\bf  {\footnotesize  n/n}  &\bf {\small Наименование  детали и описание повреждения} & \bf {\small E} & \bf {\small I} & \bf {\small L}\\\hline \hline \toprule  \endhead 
		
		
%%%%______________________________________%%%%%%%%%%%%
%%%%%%%%%   ОПИСАНИЕ ПОВРЕЖДЕНИЙ   
%\\ps{ деталь - повреждение }{E}{I}{L} 
\акт{Крыло переднее правое }{}{\7}{\7}
\акт{Бампер передний }{\7 }{ }{\7 }
\акт{Кронштейн бампера переднего правый }{\7 }{ }{ }
\акт{Диск колеса правого переднего семиспицевый }{\7 }{ }{ }
\акт{Фара правая }{\7 }{ }{ }
\акт{Облицовка крыла правого переднего }{\7 }{ }{ }
\акт{Спойлер бампера переднего правая часть }{\7 }{ }{ }
\акт{Пленка  защитная антигравийная бампера переднего и крыла правого }{\7 }{ }{ }


\end{longtable}\setcounter{rownum}{0} 
		
		\textbf{\textit{E - заменить деталь, I - ремонтировать, L - окрасить} }
		
		
		%
		\renewcommand\baselinestretch{1.2}\small\normalsize 
		%
		\subsubsection{ Расчёт}
		
		\indent Полный расчёт стоимости восстановительных расходов на ремонт ТС с учётом износа в соответствии с правилами обязательного страхования гражданской ответственности владельцев транспортных средств выполнен в  лицензированном для решения задач в рамках ОСАГО программном комплексе   SilverDAT myClaim и приведён в Калькуляции № \NomerDoc.
		Расчёт износа произведён программой  SilverDAT и представлен  в калькуляции расчёта затрат № \NomerDoc.
		\indent Результаты расчёта  стоимости восстановительных расходов ТС \тс\, \грз\, представлены ниже:\\
		
		
		\begin{figure}[H]
			\centering
			\includegraphics[width=0.95\linewidth]{example-image}
			%    		\caption{}
			%    		\label{fig:screenshot001}
		\end{figure}
		
		%
		\begin{figure}[H]
			\centering
			\includegraphics[width=0.95\linewidth]{example-image}
			%    		\caption{}
			%    		\label{fig:screenshot001}
		\end{figure}
		\begin{figure}[H]
			\centering
			\includegraphics[width=0.95\linewidth]{example-image}
			%    		\caption{}
			%    		\label{fig:screenshot002}
		\end{figure}
		\medskip
		\renewcommand\baselinestretch{1.2}\small\normalsize
		
		
		\subparagraph{}Стоимость одного нормо-часа работ определена в соответствии с пунктом 3.8.1 Единой методики [3] путём применения электронных баз данных стоимостной информации, составляющая на момент ДТП 1100 руб/час.
		Трудоёмкость работ по разборке/сборке/замене  соответствует трудоёмкостям работ, рекомендованным заводом изготовителем ТС. Трудоёмкости окрасочных работ приняты согласно рекомендаций Единой методики, п.3.7.1. в соответствии с технологией  AZT (\url{http://www.schwacke.ru/down/azt _reparaturlackierung_ru.pdf}). Расчёт размера расходов на материалы произведён  согласно пункту 3.7.2 Приложения к Единой методике [3]. Артикулы запасных частей определены с помощью программы Audatex и электронных  каталогов запасных частей \url{emex.ru}, \url{partsouq.com}.
		Стоимость запасных частей определена в соответствии с пунктом 3.6.3 Единой методики путём применения электронных баз данных стоимостной информации (по утверждённому справочнику: \url{http://prices.autoins.ru/priceAutoParts/repair_parts.html} ).
		
		Стоимость работ и материалов по нанесению качественной (3M VentureShield, SunTek, Hexis Bodifence, Stek, Spectroll ) защитной пленки на кузовные элементы атомобиля  \тс \, регистрационный знак \грз\,- передний бампер,  правое переднее крыло и правая фара определена на основании анализа рынка соответствующих услуг, оказываемых специализированными предприятиями г. Краснодара: \url{https://unicar23.ru/prajs-list/},
		\url{https://vinyl-jam.ru/}, \url{http://protuning-company.ru/avtovinil-krasnodar.html}, \url{https://miart-detailing.com/uslugi/oklejka-vinilom/}, \url{https://www.vinylmonster.ru/krasnodar}, \url{https://autostudio23.ru/zashchita-kuzova-avtomobilya/}
		и составляет 20 000 рублей.
		
		
		\subparagraph{}Таким образом,  наиболее вероятная стоимость ремонта транспортного средства \tc\, регистрационный знак \грз, получившего повреждения в результате дорожно-транспортного происшествия  \датадтп\, составляет $277 032$ (Двести семьдесят семь тысяч тридцать два) рубля,  размер затрат на восстановительный ремонт ТС с учётом износа составляет  $ 219 601 $ (Двести девятнадцать тысяч шестьсот один) рубль, или с учётом округления составляет $ 219 600 $ (Двести девятнадцать тысяч шестьсот) рублей.
		
		%\subsection{Расчет рыночной стоимости ТС}


\par \indent Рыночная стоимость транспортного средства - наиболее вероятная стоимость, по которой транспортное средство может быть отчуждено на открытом рынке в условиях конкуренции, когда стороны сделки действуют разумно, располагая всей необходимой информацией, а на величине стоимости сделки не отражаются какие-либо чрезвычайные обстоятельства [4]. Рыночная стоимость транспортного средства  рассчитывается  для условий конкретных товарных рынков транспортных средств, соответствующих месту государственной регистрации транспортного средства потерпевшего.
\par Определение рыночной стоимости ТС может производится сравнительным и затратным подходом, а именно:  сравнительный анализ продаж (анализ информации о первичном и вторичном рынке АМТС в Российской Федерации);  затратный (с учетом износа АМТС).
\par В оценочной практике наиболее распространенным и предпочтительным является метод прямого сравнения продаж, в соответствии с которым стоимость объекта оценки определяется как средняя цена отобранных аналогов с последующими параметрическими и износными корректировками.   В цены аналогов не вносятся корректировки, если аналоги являются идентичными и равновозрастными объектами по отношению к объекту оценки. 
\par Общий алгоритм метода прямого сравнения продаж:
\begin{list}{-}{}
	\item сбор необходимой информации об объектах;
	\item  производится выборка цен на объекты;
\item  проверка полученной выборки на однородность путем расчета коэффициента вариации;
\item  при значении коэффициента вариации, не превышающем 20\%, определяется средняя цена, которая принимается за стоимость объекта [9]
\item  при превышении коэффициента вариации значения  20\%  выборка исследуется на наличие выбросов с использованием метода Граббса [9,10].
\end{list}
Рыночная стоимость ТС отражает его комплектацию, комплектность, фактическое техническое состояние, срок эксплуатации, пробег, условия, в которых оно эксплуатировалось, коньюктуру первичного и вторичного рынка ТС в регионе. В общем случае расчет рыночной стоимости производится по формуле:
\begin{equation}\label{eq:aa}
C_{\text{КТС}} = C_{cp}\left(1 \pm  \left( \frac{\text{П}_{\text{П}}}{100}\right) \pm\left( \frac{\text{П}_{\text{Э}}}{100}\right) \right) + C_{\text{ДОП}}, \text{руб}
\end{equation}

\noindent где: $C_\text{ср}$ -- средняя цена ТС, \text{руб};\\
$ \text{П}_\text{П}$ -- процентный показатель корректировки средней цены ТС по пробегу, \%;\\
$ C_\text{ДОП}$ -- дополнительное увеличение (уменьшение) стоимости в зависимости от его комплектации, наличия повреждений и акта их устранения, обновления составных частей, руб.\\
Корректировка средней цены ТС, исходя из его комплектности, опций комплектации, обновления составных частей, повреждений и акта их устранения определяется по формуле:
\begin{equation}\label{eq:b}
  C_\text{ДОП}=C_1\pm C_2\pm\left( C_P+C_M+C_\text{ЗЧ}\cdot\left( 1-\text{И}\right) +C_\text{УТС}\right), \text{руб}  
\end{equation} 
где:\\
$C_1$ --увеличение средней цены транспортного средства вследствие замены (обновления его частей в процессе эксплуатации, руб;\\
$ C_2$ -- изменение средней цены транспортного средства в зависимости опций его комплектации, руб;\\
 $ C_P $ -- стоимость ремонтных работ, руб;\\
 $C_M$--  стоимость материалов, руб;\\
 $C_\text{ЗЧ}  $ -- стоимость запасных частей, руб;\\
 $ C_\text{УТС} $ -- величина УТС, руб;\\
 $ \text{И} $ -- величина износа, \%;\\
 
Приоритетным способом определения рыночной стоимость TC
является метод сравнительного подхода, основанный на объективных справочных данных о ценах на подержанные автомобили  в регионе, где проводится оценка.
   
При определении стоимости транспортного средства сравнительным подходом экспертом были использованы  Интернет-источники сведений об аналогах (таблица \ref{tab:5}), содержащие краткое описание основных характеристик и технического состояния. 
 
\begin{longtable}{|p{5mm}|p{85mm}|c|p{60mm}|l|}
	\caption[]{\footnotesize {Описание ТС, идентичных оцениваему}} \label{tab:5}\\ 
	\hline
	%\rowcolor[HTML]{C0C0C0} 
	\rowcolor[HTML]{EFEFEF}
\bf	\text{n/n} &\bf  Описание аналога & \bf URL-адрес преложения  \\ \hline \endhead
		\toprule \centering
\Rownum  &\includegraphics[width=0.99\linewidth]{A_1} &{\noindent \scriptsize\ \url {https://auto.ru/krasnodar/cars/vaz/2107/1996-year/all}} \\ \hline 	\centering
%	\rowcolor[HTML]{EFEFEF} 
\Rownum  &\includegraphics[width=0.99\linewidth]{A_2} & {\noindent \scriptsize\ \url {https://auto.ru/krasnodar/cars/vaz/2107/1996-year/all}} \\ \hline 	\centering
\Rownum  &\includegraphics[width=0.99\linewidth]{A_3} &{\noindent \scriptsize\ \url {https://auto.ru/krasnodar/cars/vaz/2107/1996-year/all}} \\ \hline 	\centering
\Rownum  &\includegraphics[width=0.99\linewidth]{A_4} &{\noindent \scriptsize\ \url {https://auto.ru/krasnodar/cars/vaz/2107/1996-year/all}} \\ \hline 	\centering
\Rownum &\includegraphics[width=0.99\linewidth]{A_5} &{\noindent \scriptsize\ \url {https://auto.ru/krasnodar/cars/vaz/2107/1996-year/all}} \\ \hline 	\centering
\Rownum  &\includegraphics[width=0.99\linewidth]{A_6} &{\noindent \scriptsize\ \url {https://auto.ru/krasnodar/cars/vaz/2107/1996-year/all}}\\ \hline 	\centering
\Rownum  &\includegraphics[width=0.99\linewidth]{A_7} &{\noindent \scriptsize\ \url {https://krasnodar.drom.ru/data/2107/35131847}} \\ \hline 	\centering
%\Rownum  &\includegraphics[width=0.99\linewidth]{A_8} &{\noindent \scriptsize\ \url {https://https://krasnodar.drom.ru/lada/2107/34429407}} \\ \hline 	\centering \textbf{}
\Rownum &\includegraphics[width=0.99\linewidth]{A_9} &{\noindent \scriptsize\ 
\url {https://krasnodar.drom.ru/lada/2107/33456928}} \\ \hline
					
\end{longtable}
%}

%	1. Определяется \overline{X} – среднее значение цены в %промежуточной выборке по формуле:        
 %
%где n – объем выборки (количество элементов в выборке);
%X_i – цена\  i– го объекта-аналога в промежуточной выборке.
%
%	2. Проверяется полученная выборка на однородность путем расчета %коэффициента вариации. Степень однородности выборки оценивается по %коэффициенту вариации (VB ), измеряющему  рассеивание данных %относительно среднего значения:
%V_B=\frac{\sigma_1}{\overline{X}}100%,  
%где \sigma_1- среднеквадратическое (или стандартное) отклонение, %определяемое по формуле:
%\sigma_1=\sqrt{\frac{\sum_{i=1}^{n}{{(X}_i-\overline{X})}^2}{n-1}};
%D= \frac{\sum_{i=1}^{n}{{(X}_i-\overline{X})}^2}{n-1}, где D -%дисперсия выборки.
	%Для малой выборки принято считать, что если VB <15%, то %однородность выборки высокая; 15% <VB<<25%, однородность выборки %средняя,  25% < VB< 33% однородность выборки низкая [9].
	%4. .При значении коэффициента вариации, не превышающем 20%, %определяется средняя цена, которая принимается за стоимость объекта.
	%5.  При превышении коэффициента вариации значения  20%  выборка %исследуется на наличие выбросов с использованием метода Граббса.
	%Окончательно, рассчитанная средняя цена предложения корректируется %с учетом торга в зависимости  от активности рынка – чем выше %спрос/предложение, тем ниже корректировка на торг. Для данного ТС %наиболее вероятное значение корректировки – 10.9% (данные %аналитического агентства «Автостат» %%http://www.autostat.ru/news/20083/)











%%%%%%%%%% среднеквадратичное отклонение и дисперсия выборки
%где $ \sigma_1 $- среднеквадратическое (или стандартное) отклонение, определяемое по формуле:
%\begin{equation}\label{ab}
%\sigma_1=\sqrt{\frac{\sum\limits_{i=1}^{n}{{(X}_i-\overline{X})}^2}{n-1}}
%\end{equation}
%
%\begin{equation}\label{ac}
%D= \frac{\sum\limits_{i=1}^{n}{{(X}_i-\overline{X})}^2}{n-1}
%\end{equation},
%
%
%где $ D $ -дисперсия выборки.
%Для малой Для малой выборки принято считать, что если VB <15\%, то однородность выборки высокая; 15\% <$  V_B  $<<25\%, однородность выборки средняя,  25\% < VB< 33\% однородность выборки низкая 
%
%4. .При значении коэффициента вариации, не превышающем 20\%, определяется средняя цена, которая принимается за стоимость объекта.
%5.  При превышении коэффициента вариации значения  20\%  выборка исследуется на наличие выбросов с использованием метода Граббса.
%6.	Окончательно, рассчитанная средняя цена предложения корректируется с учетом торга в зависимости  от активности рынка – чем выше спрос/предложение, тем ниже корректировка на торг. Для данного ТС наиболее вероятное значение корректировки – 10.9\% (данные аналитического агентства «Автостат» http://www.autostat.ru/news/20083/)
%
%\par Рыночная стоимость ТС  учитывает его фактическое техническое состояние, условия, в которых оно эксплуатировалось.
% Ремонт кузовных составляющих транспортного средства является фактором, влияющим на уменьшение средней цены ТС. В соответствии с Методикой [1], Приложение 3.3 <<Процентный показатель корректирования средней цены КТС в зависимости от условий эксплуатации>>, Таблица 1, для автомобилей со сроком эксплуатации до 7 лет, при восстановлении трех и больше кузовных составных частей уменьшение рыночной стоимости составляет 10 \%. 

%\par При определении стоимости транспортного средства сравнительным подходом экспертом использованы  Интернет-источники сведений об аналогах (Таблица \ref{tab:5}), содержащие краткое описание основных характеристик и технического состояния. Поскольку при определении рыночной стоимости  эксперту-технику данные о техническом состоянии и условиях эксплуатации не заданы и объективно не могут быть  получены при осмотре ТС на момент исследования, то для расчета принимается следующее:
%\begin{list}{-}{}
%	\item техническое состояние ТС соответствует сроку эксплуатации;
%	\item тс не эксплуатировалось в режиме такси или специальных условий эксплуатации;
%	\item фактический пробег ТС на момент повреждения составлял 45 513 км;
%	\item  29.06.2018 г. ТС \тс \, участвовал в ДТП, в результате которого  автомобиль получил механические повреждения задней правой двери, заднего правого порога, заднего правого колеса,  подушки SRS справа.
%\end{list}

%Из открытых банков данных полиции следует, что автомобиль с VIN: \,  \вин\,  как минимум дважды становился участником ДТП.
%Первый раз 29.06.2018  06:40, извещение о ДТП № 030046913, в котором автомобиль получил повреждения задней правой двери, заднего правого порога, заднего правого колеса, подушки SRS справа, Рис. \ref{ris:images/d1} и второй раз 22.05.2019 06:50, извещение о ДТП № 030034947, в котором автомобиль получил повреждения деталей передней левой и задней частей кузова, Рис. \ref{ris:images/d2}.
%%
%\vspace{\baselineskip}
%%
%\begin{figure}[H]\centering
%	\parbox[t]{0.49\textwidth}
%	{\centering
%		\includegraphics[width=.49\textwidth]{images/d1}
%		\caption{\footnotesize {Повреждения в ДТП 29.06.2018 }}
%		\label{ris:images/d1}}
%	\hfil \hfil%раздвигаем боксы по горизонтали 
%	\parbox[t]{0.49\textwidth}
%	{\centering
%		\includegraphics[width=.49\textwidth]{images/d2}
%		\caption{\footnotesize {Повреждения в ДТП 22.05.2019}}
%		\label{ris:images/d2}}
%\end{figure}
%%
%\vspace{\baselineskip}
%
%{\noindent  \footnotesize \tikz \fill [red] (1,0.5) rectangle (0.1,0.1); --{\footnotesize  Вмятины, вырывы, заломы, перекосы, разрывы и другие повреждения с изменением геометрии элементов (деталей) кузова и эксплуатационных характеристик ТС.}\\
%	\tikz \fill [yellow] (1,0.5) rectangle (0.1,0.1); --  {\footnotesize Повреждения колёс (шин), элементов ходовой части, стекол, фар, указателей поворота, стоп-сигналов и других стеклянных элементов (в т.ч. зеркал), а также царапины, сколы, потертости лакокрасочного покрытия или пластиковых конструктивных деталей и другие повреждения без изменения геометрии элементов (деталей) кузова и эксплуатационных характеристик ТС.}\\[1mm]
%	
%\renewcommand\baselinestretch{1.2}\small\normalsize

%\par Вследствие вышеизложенного коэффициент $ C_\text{ДОП}$ принимается равным -10 \%, коэффициенты $ \text{П}_{\text{Э}} $ и $ \text{П}_{\text{П}} $ принимаем равными нулю.

Средняя цена  $C_\text{ср}$  определяется на базе  средней рыночной цены продажи совокупности идентичных ТС на дату оценки: 
\begin{equation}\label{C}
C_\text{ср} =   \frac{ \sum\limits_{i=1}^n{C_i}}{n}
\end{equation}
 $C_\text{ср} =(\sum\limits_{i=1}^n{C_i})/n= (35000+35000+38000+40000+35000+35000+40000+33000)/8 =36 375 $, руб.\\
\noindent где: $ C_i $ - цена предложения к продаже i-го ТС, \\
\indent i - количество предложений, i=8.
\par Поскольку используемая выборка состоит из цен предложений к продаже, то для приведения к средней рыночной цене покупки применяется коэффициент торга $ K_T $, значение которого согласно данных аналитического агентства «Автостат» http://www.autostat.ru/news/20083/, %рис. \ref{fig:avtostat} 
\, составляет 5 \%. Соответственно, средняя цена предложений должна быть скорректирована в соответствии с нижеприведенной формулой \ref{Cp}:
\begin{equation}\label{Cp}
C_\text{ср} =  \left( \sum\limits_{i=1}^n{C_i}\right)  \cdot K_T , \,\, \text{руб}
\end{equation}
%%%%%%%%%%%%%%%%%%%%%  График АВТОСТАТ (если нужно)
%\begin{figure}
%	\centering
%	\includegraphics[width=0.7\linewidth]{images/avtostat}
%	\caption{График аналитического агентства <<Автостат>> процента уторговывания по маркам автомобилей}
%	\label{fig:avtostat}
%\end{figure}
Тогда,\,\, рыночная стоимость $ C $ автомобиля \тс \, с учетом торга и факторов, влияющих на рыночную стоимость составляет: 
\begin{equation}\label{cp}
C = C_\text{ср} \cdot  K_T 
%-  C_\text{ДОП} 
= 36375 \cdot(100-5)
%)\cdot 0.9 
=  34556 \, \, \text{руб},
\end{equation} 
\noindent что с учетом округления составляет 35000  рублей.

\par Таким образом, рыночная стоимость ТС \тс, \, с учетом всех имеющихся сведений о состоянии ТС на момент повреждения \датадтп\,  составляет 35000 (Тридцать пять тысяч) рублей.

%\paragraph*{} Федеральным законом от 25.04.2002 года № 40-ФЗ «Об обязательном страховании гражданской ответственности владельцев транспортных средств» в редакции на момент заключения договоров страхования и страхового случая (далее – Закон об ОСАГО) установлено:
%(пункт 18 статьи 12) что размер подлежащих возмещению страховщиком убытков при причинении вреда имуществу потерпевшего определяется: \par а) в случае полной гибели имущества потерпевшего - в размере действительной стоимости имущества на день наступления страхового случая за вычетом стоимости годных остатков. Под полной гибелью понимаются случаи, при которых ремонт поврежденного имущества невозможен либо стоимость ремонта поврежденного имущества равна стоимости имущества на дату наступления страхового случая или превышает указанную стоимость.\par Настоящим исследованием установлено, что стоимость восстановительного ремонта ТС \тс \, превышает рыночную стоимость ТС на момент повреждения. Таким образом, при определении стоимости восстановительных расходов необходимо произвести расчет стоимости годных остатков ТС.
%\par  
%Положением Банка России от 19 сентября 2014 г. N 432-П "О единой методике определения размера расходов на восстановительный ремонт в отношении поврежденного транспортного средства <<Глава 5. Порядок расчета стоимости годных остатков в случае полной гибели транспортного средства>> установлен алгоритм расчета годных остатков ТС.




	\subsection{Расчет рыночной стоимости ТС}


\par \indent Рыночная стоимость транспортного средства - наиболее вероятная стоимость, по которой транспортное средство может быть отчуждено на открытом рынке в условиях конкуренции, когда стороны сделки действуют разумно, располагая всей необходимой информацией, а на величине стоимости сделки не отражаются какие-либо чрезвычайные обстоятельства [4]. Рыночная стоимость транспортного средства  рассчитывается  для условий конкретных товарных рынков транспортных средств, соответствующих месту государственной регистрации транспортного средства потерпевшего.
\par Определение рыночной стоимости КТС, имеющего повреждения, которые предусматривают проведение восстановительного ремонта, затраты на его проведение рассчитывают на дату оценки [1, п.2.10].
\par Определение рыночной стоимости ТС может производится сравнительным и затратным подходом, а именно:  сравнительный анализ продаж (анализ информации о первичном и вторичном рынке АМТС в Российской Федерации);  затратный (с учетом износа АМТС).
%%%%%%%   ЗАТРАТНЫЙ ПОДХОД
\par Ввиду отсутствия на рынке аналогичных объектов сравнения для исследуемого автомобиля (новизна, начало продаж модели с 09.08.2019, эксклюзивность модели) определение рыночной стоимости исследуемого автомобиля производится затратным подходом.

\par Цена нового автомобиля  $  \text{Ц}_\text{нов} $ снижается после его продажи на величину $ K_\text{сниж} $, которая для автомобилей различных категорий может составлять до 30\%. По данным специализированных интернет-площадок, уменьшение стоимости автомобиля аналогичного сегмента рынка после продажи (утеря новизны) составляет 20\%. (\url{https://cenamashin.ru/cena/bmw/z4, automata.ru}).
%\noindent тогда
\begin{equation}\label{f:r}
\text{Ц}_\text{новприв} = \text{Ц}_\text{нов} \cdot (1-K_\text{сниж}/100)
\end{equation}

Цена нового АМТС   должна быть уменьшена с учетом эксплуатационного износа   по формуле:

\begin{equation}\label{f:n}
\text{Ц} = \text{Ц}_\text{новприв} \cdot (1-\text{И}_\text{э}/100)
\end{equation}

\noindent $ \text{Ц}_\text{новприв} = \text{Ц}_\text{нов}(1-K_\text{сниж}/100)= 5692000(1-20/100) = 4553600  $\\
\noindent $ \text{И}_\text{э} = 0,27\cdot3,697 = 1 \%  $\\
$ \text{Ц} = \text{Ц}_\text{новприв}(1-\text{И}_\text{э}/100) = 4553600(1-1/100) = 4508064 $ (Четыре миллиона пятьсот восемь тысяч шестьдесят четыре) рубля, что с учетом округления составляет 4500000 (Четыре миллиона пятьсот  тысяч) рублей.






%%%%%%%%%%%%%    СРАВНИТЕЛЬНЫЙ ПОДХОД
%\par В оценочной практике наиболее распространенным и предпочтительным является метод прямого сравнения продаж, в соответствии с которым стоимость объекта оценки определяется как средняя цена отобранных аналогов с последующими параметрическими и износными корректировками.   В цены аналогов не вносятся корректировки, если аналоги являются идентичными и равновозрастными объектами по отношению к объекту оценки. 
%\par Общий алгоритм метода прямого сравнения продаж:
%\begin{list}{-}{}
%	\item сбор необходимой информации об объектах;
%	\item  производится выборка цен на объекты;
%\item  проверка полученной выборки на однородность путем расчета коэффициента вариации;
%\item  при значении коэффициента вариации, не превышающем 20\%, определяется средняя цена, которая принимается за стоимость объекта [9]
%\item  при превышении коэффициента вариации значения  20\%  выборка исследуется на наличие выбросов с использованием метода Граббса [9,10].
%\end{list}
%Рыночная стоимость ТС отражает его комплектацию, комплектность, фактическое техническое состояние, срок эксплуатации, пробег, условия, в которых оно эксплуатировалось, коньюктуру первичного и вторичного рынка ТС в регионе. В общем случае расчет рыночной стоимости производится по формуле:
%\begin{equation}\label{eq:aa}
%C_{\text{КТС}} = C_{cp}\left(1 \pm  \left( \frac{\text{П}_{\text{П}}}{100}\right) \pm\left( \frac{\text{П}_{\text{Э}}}{100}\right) \right) + C_{\text{ДОП}}, \text{руб}
%\end{equation}
%
%\noindent где: $C_\text{ср}$ -- средняя цена ТС, \text{руб};\\
%$ \text{П}_\text{П}$ -- процентный показатель корректировки средней цены ТС по пробегу, \%;\\
%$ C_\text{ДОП}$ -- дополнительное увеличение (уменьшение) стоимости в зависимости от его комплектации, наличия повреждений и акта их устранения, обновления составных частей, руб.\\
%Корректировка средней цены ТС, исходя из его комплектности, опций комплектации, обновления составных частей, повреждений и акта их устранения определяется по формуле:
%\begin{equation}\label{eq:b}
%  C_\text{ДОП}=C_1\pm C_2\pm\left( C_P+C_M+C_\text{ЗЧ}\cdot\left( 1-\text{И}\right) +C_\text{УТС}\right), \text{руб}  
%\end{equation} 
%где:\\
%$C_1$ --увеличение средней цены транспортного средства вследствие замены (обновления его частей в процессе эксплуатации, руб;\\
%$ C_2$ -- изменение средней цены транспортного средства в зависимости опций его комплектации, руб;\\
% $ C_P $ -- стоимость ремонтных работ, руб;\\
% $C_M$--  стоимость материалов, руб;\\
% $C_\text{ЗЧ}  $ -- стоимость запасных частей, руб;\\
% $ C_\text{УТС} $ -- величина УТС, руб;\\
% $ \text{И} $ -- величина износа, \%;\\
% 
%Приоритетным способом определения рыночной стоимость TC
%является метод сравнительного подхода, основанный на объективных справочных данных о ценах на подержанные автомобили  в регионе, где проводится оценка.
%   
%При определении стоимости транспортного средства сравнительным подходом экспертом были использованы  Интернет-источники сведений об аналогах (таблица \ref{tab:5}), содержащие краткое описание основных характеристик и технического состояния. 
% 
%\begin{longtable}{|p{5mm}|p{85mm}|c|p{60mm}|l|}
%	\caption[]{\footnotesize {Описание ТС, идентичных оцениваему}} \label{tab:5}\\ 
%	\hline
%	%\rowcolor[HTML]{C0C0C0} 
%	\rowcolor[HTML]{EFEFEF}
%\bf	\text{n/n} &\bf  Описание аналога & \bf URL-адрес преложения  \\ \hline \endhead
%		\toprule \centering
%\Rownum  &\includegraphics[width=0.99\linewidth]{A_1} &{\noindent \scriptsize\ \url {https://auto.ru/krasnodar/cars/vaz/2107/1996-year/all}} \\ \hline 	\centering
%%	\rowcolor[HTML]{EFEFEF} 
%\Rownum  &\includegraphics[width=0.99\linewidth]{A_2} & {\noindent \scriptsize\ \url {https://auto.ru/krasnodar/cars/vaz/2107/1996-year/all}} \\ \hline 	\centering
%\Rownum  &\includegraphics[width=0.99\linewidth]{A_3} &{\noindent \scriptsize\ \url {https://auto.ru/krasnodar/cars/vaz/2107/1996-year/all}} \\ \hline 	\centering
%\Rownum  &\includegraphics[width=0.99\linewidth]{A_4} &{\noindent \scriptsize\ \url {https://auto.ru/krasnodar/cars/vaz/2107/1996-year/all}} \\ \hline 	\centering
%\Rownum &\includegraphics[width=0.99\linewidth]{A_5} &{\noindent \scriptsize\ \url {https://auto.ru/krasnodar/cars/vaz/2107/1996-year/all}} \\ \hline 	\centering
%\Rownum  &\includegraphics[width=0.99\linewidth]{A_6} &{\noindent \scriptsize\ \url {https://auto.ru/krasnodar/cars/vaz/2107/1996-year/all}}\\ \hline 	\centering
%\Rownum  &\includegraphics[width=0.99\linewidth]{A_7} &{\noindent \scriptsize\ \url {https://krasnodar.drom.ru/data/2107/35131847}} \\ \hline 	\centering
%%\Rownum  &\includegraphics[width=0.99\linewidth]{A_8} &{\noindent \scriptsize\ \url {https://https://krasnodar.drom.ru/lada/2107/34429407}} \\ \hline 	\centering \textbf{}
%\Rownum &\includegraphics[width=0.99\linewidth]{A_9} &{\noindent \scriptsize\ 
%\url {https://krasnodar.drom.ru/lada/2107/33456928}} \\ \hline
%					
%\end{longtable}
%%}
%
%%	1. Определяется \overline{X} – среднее значение цены в %промежуточной выборке по формуле:        
% %
%%где n – объем выборки (количество элементов в выборке);
%%X_i – цена\  i– го объекта-аналога в промежуточной выборке.
%%
%%	2. Проверяется полученная выборка на однородность путем расчета %коэффициента вариации. Степень однородности выборки оценивается по %коэффициенту вариации (VB ), измеряющему  рассеивание данных %относительно среднего значения:
%%V_B=\frac{\sigma_1}{\overline{X}}100%,  
%%где \sigma_1- среднеквадратическое (или стандартное) отклонение, %определяемое по формуле:
%%\sigma_1=\sqrt{\frac{\sum_{i=1}^{n}{{(X}_i-\overline{X})}^2}{n-1}};
%%D= \frac{\sum_{i=1}^{n}{{(X}_i-\overline{X})}^2}{n-1}, где D -%дисперсия выборки.
%	%Для малой выборки принято считать, что если VB <15%, то %однородность выборки высокая; 15% <VB<<25%, однородность выборки %средняя,  25% < VB< 33% однородность выборки низкая [9].
%	%4. .При значении коэффициента вариации, не превышающем 20%, %определяется средняя цена, которая принимается за стоимость объекта.
%	%5.  При превышении коэффициента вариации значения  20%  выборка %исследуется на наличие выбросов с использованием метода Граббса.
%	%Окончательно, рассчитанная средняя цена предложения корректируется %с учетом торга в зависимости  от активности рынка – чем выше %спрос/предложение, тем ниже корректировка на торг. Для данного ТС %наиболее вероятное значение корректировки – 10.9% (данные %аналитического агентства «Автостат» %%http://www.autostat.ru/news/20083/)
%
%
%
%
%
%
%
%
%
%
%
%%%%%%%%%%% среднеквадратичное отклонение и дисперсия выборки
%%где $ \sigma_1 $- среднеквадратическое (или стандартное) отклонение, определяемое по формуле:
%%\begin{equation}\label{ab}
%%\sigma_1=\sqrt{\frac{\sum\limits_{i=1}^{n}{{(X}_i-\overline{X})}^2}{n-1}}
%%\end{equation}
%%
%%\begin{equation}\label{ac}
%%D= \frac{\sum\limits_{i=1}^{n}{{(X}_i-\overline{X})}^2}{n-1}
%%\end{equation},
%%
%%
%%где $ D $ -дисперсия выборки.
%%Для малой Для малой выборки принято считать, что если VB <15\%, то однородность выборки высокая; 15\% <$  V_B  $<<25\%, однородность выборки средняя,  25\% < VB< 33\% однородность выборки низкая 
%%
%%4. .При значении коэффициента вариации, не превышающем 20\%, определяется средняя цена, которая принимается за стоимость объекта.
%%5.  При превышении коэффициента вариации значения  20\%  выборка исследуется на наличие выбросов с использованием метода Граббса.
%%6.	Окончательно, рассчитанная средняя цена предложения корректируется с учетом торга в зависимости  от активности рынка – чем выше спрос/предложение, тем ниже корректировка на торг. Для данного ТС наиболее вероятное значение корректировки – 10.9\% (данные аналитического агентства «Автостат» http://www.autostat.ru/news/20083/)
%%
%%\par Рыночная стоимость ТС  учитывает его фактическое техническое состояние, условия, в которых оно эксплуатировалось.
%% Ремонт кузовных составляющих транспортного средства является фактором, влияющим на уменьшение средней цены ТС. В соответствии с Методикой [1], Приложение 3.3 <<Процентный показатель корректирования средней цены КТС в зависимости от условий эксплуатации>>, Таблица 1, для автомобилей со сроком эксплуатации до 7 лет, при восстановлении трех и больше кузовных составных частей уменьшение рыночной стоимости составляет 10 \%. 
%
%%\par При определении стоимости транспортного средства сравнительным подходом экспертом использованы  Интернет-источники сведений об аналогах (Таблица \ref{tab:5}), содержащие краткое описание основных характеристик и технического состояния. Поскольку при определении рыночной стоимости  эксперту-технику данные о техническом состоянии и условиях эксплуатации не заданы и объективно не могут быть  получены при осмотре ТС на момент исследования, то для расчета принимается следующее:
%%\begin{list}{-}{}
%%	\item техническое состояние ТС соответствует сроку эксплуатации;
%%	\item тс не эксплуатировалось в режиме такси или специальных условий эксплуатации;
%%	\item фактический пробег ТС на момент повреждения составлял 45 513 км;
%%	\item  29.06.2018 г. ТС \тс \, участвовал в ДТП, в результате которого  автомобиль получил механические повреждения задней правой двери, заднего правого порога, заднего правого колеса,  подушки SRS справа.
%%\end{list}
%
%%Из открытых банков данных полиции следует, что автомобиль с VIN: \,  \вин\,  как минимум дважды становился участником ДТП.
%%Первый раз 29.06.2018  06:40, извещение о ДТП № 030046913, в котором автомобиль получил повреждения задней правой двери, заднего правого порога, заднего правого колеса, подушки SRS справа, Рис. \ref{ris:images/d1} и второй раз 22.05.2019 06:50, извещение о ДТП № 030034947, в котором автомобиль получил повреждения деталей передней левой и задней частей кузова, Рис. \ref{ris:images/d2}.
%%%
%%\vspace{\baselineskip}
%%%
%%\begin{figure}[H]\centering
%%	\parbox[t]{0.49\textwidth}
%%	{\centering
%%		\includegraphics[width=.49\textwidth]{images/d1}
%%		\caption{\footnotesize {Повреждения в ДТП 29.06.2018 }}
%%		\label{ris:images/d1}}
%%	\hfil \hfil%раздвигаем боксы по горизонтали 
%%	\parbox[t]{0.49\textwidth}
%%	{\centering
%%		\includegraphics[width=.49\textwidth]{images/d2}
%%		\caption{\footnotesize {Повреждения в ДТП 22.05.2019}}
%%		\label{ris:images/d2}}
%%\end{figure}
%%%
%%\vspace{\baselineskip}
%%
%%{\noindent  \footnotesize \tikz \fill [red] (1,0.5) rectangle (0.1,0.1); --{\footnotesize  Вмятины, вырывы, заломы, перекосы, разрывы и другие повреждения с изменением геометрии элементов (деталей) кузова и эксплуатационных характеристик ТС.}\\
%%	\tikz \fill [yellow] (1,0.5) rectangle (0.1,0.1); --  {\footnotesize Повреждения колёс (шин), элементов ходовой части, стекол, фар, указателей поворота, стоп-сигналов и других стеклянных элементов (в т.ч. зеркал), а также царапины, сколы, потертости лакокрасочного покрытия или пластиковых конструктивных деталей и другие повреждения без изменения геометрии элементов (деталей) кузова и эксплуатационных характеристик ТС.}\\[1mm]
%%	
%%\renewcommand\baselinestretch{1.2}\small\normalsize
%
%%\par Вследствие вышеизложенного коэффициент $ C_\text{ДОП}$ принимается равным -10 \%, коэффициенты $ \text{П}_{\text{Э}} $ и $ \text{П}_{\text{П}} $ принимаем равными нулю.
%
%Средняя цена  $C_\text{ср}$  определяется на базе  средней рыночной цены продажи совокупности идентичных ТС на дату оценки: 
%\begin{equation}\label{C}
%C_\text{ср} =   \frac{ \sum\limits_{i=1}^n{C_i}}{n}
%\end{equation}
% $C_\text{ср} =(\sum\limits_{i=1}^n{C_i})/n= (35000+35000+38000+40000+35000+35000+40000+33000)/8 =36 375 $, руб.\\
%\noindent где: $ C_i $ - цена предложения к продаже i-го ТС, \\
%\indent i - количество предложений, i=8.
%\par Поскольку используемая выборка состоит из цен предложений к продаже, то для приведения к средней рыночной цене покупки применяется коэффициент торга $ K_T $, значение которого согласно данных аналитического агентства «Автостат» http://www.autostat.ru/news/20083/, %рис. \ref{fig:avtostat} 
%\, составляет 5 \%. Соответственно, средняя цена предложений должна быть скорректирована в соответствии с нижеприведенной формулой \ref{Cp}:
%\begin{equation}\label{Cp}
%C_\text{ср} =  \left( \sum\limits_{i=1}^n{C_i}\right)  \cdot K_T , \,\, \text{руб}
%\end{equation}
%%%%%%%%%%%%%%%%%%%%%%  График АВТОСТАТ (если нужно)
%%\begin{figure}
%%	\centering
%%	\includegraphics[width=0.7\linewidth]{images/avtostat}
%%	\caption{График аналитического агентства <<Автостат>> процента уторговывания по маркам автомобилей}
%%	\label{fig:avtostat}
%%\end{figure}
%Тогда,\,\, рыночная стоимость $ C $ автомобиля \тс \, с учетом торга и факторов, влияющих на рыночную стоимость составляет: 
%\begin{equation}\label{cp}
%C = C_\text{ср} \cdot  K_T 
%%-  C_\text{ДОП} 
%= 36375 \cdot(100-5)
%%)\cdot 0.9 
%=  34556 \, \, \text{руб},
%\end{equation} 
%\noindent что с учетом округления составляет 35000  рублей.

\par Таким образом, рыночная стоимость ТС \тс, \, с учетом всех имеющихся сведений о состоянии ТС на момент повреждения \датадтп\,  составляет  4508000 (Четыре миллиона пятьсот восемь тысяч )  рублей.






		%\subsection{Расчёт утраты товарной стоимости ТС}

\par В целях обеспечения единства практики применения судами законодательства, регулирующего отношения в области обязательного страхования гражданской ответственности владельцев транспортных средств, Пленум Верховного Суда Российской Федерации, руководствуясь статьей 126 Конституции Российской Федерации, статьями 2 и 5 Федерального конституционного закона от 5 февраля 2014 года N 3-ФКЗ "О Верховном Суде Российской Федерации", постановляет дать следующие разъяснениz
Общие положения Постановление Пленума Верховного Суда Российской Федерации от 26 декабря 2017 г. N 58 г. Москва "О применении судами законодательства об обязательном страховании гражданской ответственности владельцев транспортных средств" 

п. 37. К реальному ущербу, возникшему в результате дорожно-транспортного происшествия, наряду со стоимостью ремонта и запасных частей относится также утрата товарной стоимости, которая представляет собой уменьшение стоимости транспортного средства, вызванное преждевременным ухудшением товарного (внешнего) вида транспортного средства и его эксплуатационных качеств в результате снижения прочности и долговечности отдельных деталей, узлов и агрегатов, соединений и защитных покрытий вследствие дорожно-транспортного происшествия и последующего ремонта.

 Утрата товарной стоимости подлежит возмещению и в случае, если страховое возмещение осуществляется в рамках договора обязательного страхования в форме организации и (или) оплаты восстановительного ремонта повреждённого транспортного средства на станции технического обслуживания, с которой у страховщика заключён договор о ремонте транспортного средства, в установленном законом пределе страховой суммы.

\par Расчёт утраты товарной стоимости в настоящем исследовании производится согласно \emph {Методическим рекомендациям по проведению судебных автотехнических экспертиз и исследований колёсных транспортных средств в целях определения размера ущерба, стоимости восстановительного ремонта и оценки}  [6].

\par Утрата товарной стоимости (УТС) обусловлена снижением товарной стоимости из-за ухудшения потребительских свойств вследствие наличия дефектов (повреждений), или следов их устранения либо наличия достоверной информации, что дефекты (повреждения) устранялись [6, п. 8].

	УТС может быть рассчитана для ТС, находящихся как в повреждённом, так и в отремонтированном состоянии (при возможности установить степень повреждения).

УТС может определяться при необходимости выполнения одного из нижеперечисленных видов ремонтных воздействий или если установлено их выполнение:

-	устранение перекоса кузова или рамы ТС;

-	замена несъемных элементов кузова ТС (полная или частичная); ремонт съёмных или несъемных элементов кузова (включая оперение) ТС (в том числе пластиковых капота, крыльев, дверей, крышки багажника);

-	полная или частичная окраска наружных (лицевых) поверхностей кузова (включая оперение) ТС, бамперов;

-	полная или частичная разборка салона ТС, вызывающая нарушение качества заводской сборки.

УТС не рассчитывается:

а)	если срок эксплуатации легковых автомобилей превышает 5 лет;

б)	если легковые автомобили эксплуатируются в интенсивном режиме, а срок эксплуатации превышает 2,5 года;


в)	в случае замены кузова до оцениваемых повреждений (за исключением кузова грузового КТС, установленного на раме за кабиной);

%)	если КТС ранее подвергалось восстановительному ремонту (в том числе окраске - полной, наружной, частичной; «пятном с переходом») или имело аварийные повреждения, кроме повреждений, указанных в [6, п. 8.4];

д)	если КТС имело коррозионные повреждения кузова или кабины на момент происшествия.



Нижеприведенные повреждения не требуют расчёта УТС вследствие исследуемого происшествия, а их наличие до исследуемого происшествия не обуславливает отказ от расчёта УТС при таких повреждениях:

а)	эксплуатационных повреждениях ЛКП в виде меления, трещин, а также повреждений, вызванных механическими воздействиями - незначительных по площади сколов, рисок, не нарушающих защитных функций ЛКП составных частей оперения;

б)	одиночного эксплуатационного повреждения оперения кузова (кабины) в виде простой деформации, не требующего окраски, площадью не более 0,25 дм2;

в)	повреждения, которые приводят к замене отдельных составных частей, которые не нуждаются в окрашивании и не ухудшают внешний вид КТС (стекло, фары, бампера неокрашиваемые, пневматические шины, колёсные диски, внешняя и внутренняя фурнитура и т. п.). Если, кроме указанных составных частей, повреждены составные части кузова, рамы, кабины или детали оперения - крылья съёмные, капот, двери, крышка багажника, - то расчёт величины УТС должен учитывать все повреждения составных частей в комплексе;

г)	в случае окраски молдингов, облицовок, накладок, ручек, корпусов зеркал и других мелких наружных элементов, колёсных дисков.

В случае исследуемого события для автомобиля \тс\, VIN \vin\,  условия,  при которых производится расчёт УТС выполняются.\\


\par Величина УТС зависит от вида, характера и объёма повреждений и ремонтных воздействий по их устранению.
\par Величина УТС ($ C_\text{YTC} $) определяется на дату оценки (исследования) по формуле: 

\begin{equation}\label{uts}
C_{YTC} = C_{TC} \cdot \dfrac{\sum\limits_{i=1}^n K_{YTCi}}{100\%},\hspace{5mm} \text{руб.},
\end{equation}

\noindent где:\\
\noindent $ C_{TC} $ -- стоимость ТС на дату оценки (исследования), руб;\\
$ K_{YTCi} $ -- коэффициент УТС по i-му элементу КТС, ремонтному воздействию, \%.
 
\par Рыночная стоимость транспортного средства ( $ C_{TC} $ ), согласно п.6.1. Единой методики [3], принимается равной средней стоимости аналога на указанную
дату по данным имеющихся инфор\-мационно-справочных материалов,
содержащих сведения о средней стоимости транспортного средства.

\par  При ремонте съёмной составной части сумма стоимости ремонта (включая стоимость разборки для ремонта и при необходимости снятия детали для ремонта) и величины УТС (без учёта УТС вследствие окраски) не должна превышать суммы стоимости этой составной части (с учётом коэффициента износа) и стоимости работ по ее замене.

\par   Значение коэффициента УТС $ K_{\text{утсокр}} $ при подетальной окраске наружных поверхностей кузова ТС рассчитывается с учётом количества окрашиваемых кузовных составных частей и бамперов по формуле:

\begin{equation}\label{f:yc}
K_{\text{утсокр}}=K_{\text{утсокр(1)}}+K_{\text{утсокр(N-1)}}\cdot(N-1), \hspace{5mm} \% 
\end{equation}
        
\noindent где:\\
\noindent $ \text{К}_{\text{утсокр(1)}} $ - коэффициент УТС по окраске первой кузовной составной части или бампера, \%;\\
$ \text{К}_{\text{утсокр(N-1)}} $ - коэффициент УТС по окраске второй и каждой следующей кузовной составной части или бампера, \%;\\
N - количество окрашиваемых составных частей, по которым рассчитывается УТС.\\
Значения коэффициентов УТС ($ K_{YTC} $) определены по результатам экспертной практики и приведены в приложении [6, Приложение 2.9].

\par Для исследуемого автомобиля \тс \, соответствующие ремонтным воздействиям  коэффициенты УТС приведены ниже в таблице:

\begin{table}[H]
		%\caption{}
	\begin{tabular}{|p{5mm}|p{80mm}|c|c|c|}
	\hline 
	\textbf{п/п} & \textbf{Наименование детали} &\textbf{ К-замена }& \textbf{К-ремонт }&\textbf{ К-окраска} \\ 
	\hline 
	1 & Дверь задняя левая & -- & -- & 0,5 \\ 
	\hline 
%	2 & Бампер задний & -- & -- & 0,35 \\ 
%	\hline 
	2 & Боковина левая & -- & 0,2 & 0,35 \\ 
	\hline 
		3 & Порог левой боковины  & -- & 0,2 & 0,35 \\ 
	\hline 

	
\end{tabular} 

\end{table}

\vspace{7mm}

$  \sum\limits_{i=1}^n K_{YTCi} = 0.5+0.2+0.35+0.2+0.35 = 1.6$\\
  
  
Рыночная стоимость транспортного средства \тс\, на момент повреждения по данным справочника \url { https://automama.ru/krasnodar/cars/ } составляет 660 000 (Шестьсот шестьдесят тысяч) рублей.
  
$   C_{TC} = C_{TC} \cdot \dfrac{\sum\limits_{i=1}^n K_{YTCi}}{100} = 660000 \cdot 1.6/100 = 10560 $%, или с учетом округления 372000 (Триста семьдесят две тысячи) рублей.\\
(Десять тысяч пятьсот шестьдесят) рублей.

\par Таким образом, величина УТС автомобиля \тс\, составляет (Десять тысяч пятьсот шестьдесят) рублей.
		%\subsection{Расчёт стоимости годных остатков}



\par \indent  Согласно пп. <<a>> п. 18 ст. 12 Федерального закона  N 40-ФЗ  <<Об обязательном страховании гражданской ответственности владельцев транспортных средств>>  в случаях, при которых ремонт повреждённого имущества невозможен либо стоимость ремонта равна стоимости имущества на дату наступления страхового случая или превышает указанную стоимость размер подлежащих возмещению страховщиком убытков при причинении вреда имуществу потерпевшего определяется в размере действительной стоимости имущества на день наступления страхового случая за вычетом стоимости годных остатков. 

В нашем случае,  стоимость ремонта ТС \тс\, регистрационный знак \грз\, превышает его рыночную стоимость. Следовательно, в порядке, предусмотренном  Главой 5 Положения Банка России от 19 сентября 2014 г. № 432-П <<О единой методике определения размера расходов на восстановительный ремонт в отношении повреждённого транспортного средства>>  производится расчёт стоимости годных остатков. 

\par Под годными остатками автотранспортного средства понимаются работоспособные, имеющие остаточную стоимость детали (агрегаты, узлы) повреждённого автотранспортного средства, как правило, годные к дальнейшей эксплуатации, которые можно демонтировать с повреждённого автотранспортного средства и реализовать. 
Годные остатки должны отвечать следующим условиям:

1) деталь (агрегат, узел) не должна иметь повреждений, нарушающих ее целостность и товарный вид, а агрегат (узел), кроме того, должен находиться в работоспособном состоянии;

2) деталь (агрегат, узел) не должна иметь изменений конструкции, формы, целостности и геометрии, не предусмотренных изготовителем автотранспортного средства (например, дополнительные отверстия и вырезы для крепления несерийного оборудования);

3) деталь не должна иметь следов предыдущих ремонтных воздействий (следов правки, рихтовки, следов шпатлевки, следов частичного ремонта и т.д.).



 Стоимость годных остатков с учётом затрат на их демонтаж, дефектовку, хранение и продажу определяется по формуле:
 \begin{equation}\label{go}
C_{\text{ГО}}= C_{\text{Р}} \cdot K_{\text{В}}\cdot K_{\text{З}}\cdot K_{\text{ОП}} \cdot  \sum\limits_{i-1}^{n}\frac{C_i}{100}, \, \, \text{руб} 
\end{equation}
\noindent где: \,$ C_{\text{Р}} $ -- стоимость ТС в неповрежденном виде на момент происшествия;\\
$ K_{\text{З}} $-- коэффициент, учитывающий затраты на дефектовку, разборку, хранение, продажу;\\
$ K_{\text{В}} $ -- коэффициент, учитывающий срок эксплуатации АМТС на момент повреждения и спрос на его неповреждённые детали;\\
$ K_{\text{ОП}} $ -- коэффициент, учитывающий объем (степень) механических повреждений автомобиля;\\
$ C_i $ процентное соотношение (вес) стоимости неповреждённых элементов к стоимости автомобиля;\\
$ n  $- количество неповреждённых элементов (агрегатов, узлов).\\

Расчёт процентного соотношения (веса) стоимости неповреждённых элементов к стоимости ТС   \,\,
     % \begin{equation}\label{bb}
   $  \left( \sum\limits_{i-1}^{n}\frac{C_i}{100} \right)  $  
%   \end{equation}  
включает только установленные неповреждённые детали, узлы и агрегаты. Компоненты ТС, имеющие повреждения  вероятностного характера, и требующие диагностических работ для установления годности в расчёте не учитываются. 
 
  \begin{longtable}{|p{9cm}|p{4cm}|p{2cm}|}
 	\caption[]{\footnotesize {Таблица расчёта $ C_i $ }}
 	 \label{tab:7}\\
 	 \hline
 	 		Наименование агрегата, узла, детали & \%-ное соотношение (вес)  & Годные, \% \\
 	 		\hline \endhead
 		Кузовные детали, экстерьер, интерьер, в т.ч.: & 50 (45 \textless{}1\textgreater{}) & 0 \\
 		Передняя часть: & 14 &  \\
 		Капот & 1.9 & 1,9 \\
 		Крыло переднее (за 1 шт.) & 0.8 & 0,8 \\
 		Бампер передний (в сборе с усилителем, накладками и молдингами, спойлером) & 1.9 & 1,9 \\
 		Решетка (облицовка) радиатора & 0.8 & 0,8 \\
 		Лонжерон передний (за 1 шт.) & 0.8 & 0,8 \\
 		Брызговик крыла (за 1 шт.) & 1.4 & 1,4 \\
 		Стекло ветрового окна & 1.7 & 1,7 \\
 		Рамка радиатора & 1.4 & 1,4 \\
 		Щиток передка & 0.3 & 0,3 \\
 		Задняя часть: & 12 (14 \textless{}1\textgreater{}) & 0 \\
 		Бампер задний & 1.6 & 0 \\
 		Крыло заднее (боковина \textless{}1\textgreater{}) в сборе с арками (за 1 шт.) & 2.1 (3.1 \textless{}1\textgreater{}) & 0 \\
 		Стекло окна задка & 1.9 & 0 \\
 		Панель задка & 0.8 & 0 \\
 		Пол багажника & 0.8 & 0 \\
 		Облицовки багажника & 1.1 & 0 \\
 		Крышка багажника (дверь задка) & 1.6 & 0 \\
 		Средняя часть: & 24 (17 \textless{}1\textgreater{}) & 0 \\
 		Передняя стойка боковины (за 1 шт.) & 1.4 & 2,8 \\
 		Средняя стойка боковины с порогом и частью пола (за 1 шт.) & 1.4 (0 \textless{}1\textgreater{}) & 2,8 \\
 		Облицовки стоек боковины, порогов, уплотнители, центральная консоль, противосолнечные козырьки, плафоны освещения, коврики пола, зеркало заднего вида & 2.5 (2.1 \textless{}1\textgreater{}) & 2,5 \\
 		Двери в сборе с арматурой (за 1 шт.), & 1.9 & 5,7 \\
 		в т.ч. арматура дверей (за 1    дверной комплект) & 0.5 & 0 \\
 		Сиденья (все) & 1.1 & 1,1 \\
 		Панель крыши в сб. с обивкой, поперечинами и верх. частями стоек, & 3.5 & 2,7 \\
 		в т.ч. обивка панели крыши & 0.8 & 0 \\
 		Панель приборов в сборе с щитком приборов, решётками, вещевым ящиком, карманами и т.д. & 2.5 & 2,5 \\
 		Ремень безопасности передний (за 1 шт.) & 0.3 & 0,6 \\
 		Подушка безопасности пассажирская & 0.6 & 0,6 \\
 		Двигатель, навесное, охлаждение, впускная и выпускная система & 11 (13 \textless{}2\textgreater{}) & 13 \\
 		Двигатель в сборе без навесного оборудования & 4.9 & 0 \\
 		в т.ч. клапанная крышка & 0.5 & 0 \\
 		в т.ч. масляный поддон & 0.5 & 0 \\
 		в т.ч. блок цилиндров & 2.2 & 0 \\
 		Дроссельный узел в сборе с заслонкой, клапаном и датчиком & 1.4 & 0 \\
 		Генератор & 0.8 & 0 \\
 		Коллектор впускной & 0.5 & 0 \\
 		Коллектор выпускной & 0.5 & 0 \\
 		Радиатор охлаждения в сборе с кожухами, вентилятором & 0.8 & 0 \\
 		Стартер & 0.5 & 0 \\
 		Короб воздушного фильтра с патрубками & 0.5 & 0 \\
 		Выпускной тракт в сборе & 0.8 & 0 \\
 		Турбокомпрессор (турбонагнетатель) & 1.4 \textless{}2\textgreater{} & 0 \\
 		Интеркулер & 0.6 \textless{}2\textgreater{} & 0 \\
 		Топливная система & 2.5 & 2,5 \\
 		Бак топливный & 0.7 & 0 \\
 		Система подачи топлива & 1.8 & 0 \\
 		Трансмиссия & 4.5 & 4,5 \\
 		Усреднённый показатель с учётом всех возможных вариантов трансмиссии & 4.5 & 0 \\
 		Подвеска & 10 & 4 \\
 		Подвеска передняя в сборе с поперечиной & 5.5 (4.5 \textless{}4\textgreater{}) & 0 \\
 		Подвеска задняя в сборе с поперечиной & 4.5 (5.5 \textless{}4\textgreater{}) & 4,5 \\
 		Подвеска в сборе для полноприводных АМТС & 10 (5 \textless{}4\textgreater + 5 \textless{}4\textgreater{}) & 0 \\
 		Рулевое управление & 3 & 3 \\
 		Рулевая колонка в сборе с валом & 0.5 & 0 \\
 		Насос ГУР & 0.8 & 0 \\
 		Рулевой механизм & 1.2 & 0 \\
 		Рулевое колесо в сборе с подушкой безопасности & 0.5 & 0 \\
 		в т.ч.: подушка безопасности  водительская & 0.3 & 0 \\
 		Тормозная система & 3.5 & 3,5 \\
 		Главный тормозной цилиндр & 0.5 & 0 \\
 		Тормозной механизм колеса (за каждый колёсный узел) & 0.5 & 0 \\
 		Ручной (ножной) тормоз & 0.3 & 0 \\
 		Блок управления АБС & 0.7 & 0 \\
 		Электрооборудование & 12.5 & 0 \\
 		Провода свечные с катушками (комплект) & 0.5 & 0,5 \\
 		Монтажный блок & 0.5 & 0,5 \\
 		Блок управления двигателем & 1 & 1 \\
 		Фонари задние (за 1 шт.) & 0.5 & 1 \\
 		Зеркала заднего вида боковые (за 1 шт.) & 0.8 & 1,6 \\
 		Блок отопителя салона в сборе (корпус, двигатель, радиаторы) & 2.1 & 2,1 \\
 		Насос кондиционера & 0.5 & 0,5 \\
 		Конденсатор в сборе с осушителем, кожухом, вентилятором, трубками & 0.6 & 0,6 \\
 		Фары (за 1 шт.) & 1.1 & 1,1 \\
 		Жгут проводов ДВС & 0.9 & 0,9 \\
 		Жгут проводов панели приборов & 0.8 & 0,8 \\
 		Остальные жгуты проводов (все) & 0.3 & 0,3 \\
 		Фара противотуманная (за 1 шт.) & 0.8 & 1,6 \\ 
 		Прочее & 3/8 \textless{}1\textgreater{}/1 \textless{}2\textgreater /6 \textless{}3\textgreater{} & 4 \\
 		\hline
 		\textbf{ИТОГО,} \%: &  & \textbf{83,8}  \\
 		\hline	
 	\end{longtable}
 
\noindent  \begin{table}[H]
  	 \label{tab:KB}
	\caption{\footnotesize {Значения коэффициента Кв, учитывающего срок эксплуатации ТС}}
		 \begin{tabular}{|p{47mm} |p{53mm}| p{50mm}|}
	\hline
 		Срок эксплуатации автомобиля, лет & Значение Кв легковых автомобилей, малотоннажных грузовых на базе легковых и мототехники & Значение Кв грузовых автомобилей \\ \hline
 		0 - 5 (включительно)   &0.80                                                                                    & 0.80                             \\ \hline
 		6 - 10 (включительно)             & 0.65                                                                                    & 0.60                             \\ \hline
 		11 - 15 (включительно)            & 0.55                                                                                    & 0.50                             \\ \hline
 		16 - 20 (включительно)            & 0.40                                                                                    & 0.35                             \\ \hline
 		Более 20 лет                      & 0.35                                                                                    & 0.30                            \\ \hline
 	\end{tabular}
\end{table}


\noindent \begin{table}[H]
	\label{tab:KO}
	\caption{\footnotesize {Значение коэффициента $ K_{\text{оп}} $ , учитывающего объем (степень) механических повреждений автомобиля}} 
	\centering
\begin{tabular}{|p{47mm}| p{53mm}| p{50mm}|}
	\hline
	Объем механических повреждений & Соотношение стоимости неповреждённых элементов к стоимости автомобиля, Ci, \% & Значение коэффициента, учитывающего объем механических повреждений \\ \hline
	Незначительный     & 80 -    & 0.9 -  \\
	& 60 - 80      & 0.8 - 0.9      \\
	Средний    & 40 -     & 0.7 -    \\
	& 20 - 40     & 0.6 - 0.7         \\
	Значительный                   & 0 - 20                                                                        & 0.5 - 0.6                                                          \\ \hline
\end{tabular}
\end{table}

 \par
 
% \begin{equation}\label{k}
% C_{\text{ГО}}= C_{\text{Р}} \cdot K_{\text{В}}\cdot K_{\text{З}}\cdot K_{\text{ОП}} \cdot  \sum\limits_{i-1}^{n}\frac{C_i}{100} 
% \end{equation}
 
 Для исследуемого транспортного средства применимы следующие значения коэффициентов:\par
 $ K_{\text{В}} =0.8  $; 
 $ K_{\text{З}} =0.7 $; 
 $ K_{\text{ОП}} =0.8 $; $ \sum\limits_{i-1}^{n}\frac{C_i}{100} = 0.838 $ \par тогда:
 \par
$  C_{\text{ГО}} =  C_{\text{ГО}}= C_{\text{Р}} \cdot K_{\text{В}}\cdot K_{\text{З}}\cdot K_{\text{ОП}} \cdot  \sum\limits_{i-1}^{n}\frac{C_i}{100} =980000*0.8*0.7*0.8*0.838 = 367915 $ руб., или с учётом округления 370 000 (Триста семьдесят тысяч) рублей.
\par Таким образом, стоимость годных остатков ТС \тс \, \, составляет 370 000 (Триста семьдесят тысяч) рублей.



%
\paragraph{Анализ результатов}


\section{Выводы}

\begin{enumerate}
	\item \textbf{"Вывод по первому вопросу}\\[3mm]
	\item \textbf{"Вывод по второму вопросу}\\[3mm]
	\item \textbf{"Вывод по третьему вопросу}\\[3mm]
	\item \textbf{"Вывод  итд .......}\\[3mm]
	
	\vspace{5mm}
	
\end{enumerate}
    
\vspace{10mm}
%\noindent{Эксперт}  \hfill    \rule{45mm}{0.1 mm}     {Фефелов С.Л.}\\
%\vspace{10mm}
\noindent{Эксперт}  \hfill    \rule{45mm}{0.1 mm}   {Мраморнов А.В.}\\
\vspace{7mm}
\relax

\vspace{15mm}

\relax
\noindent Приложение к заключению:\\
\textit{
	%	Приложение № 1. Расшифровка модельных опций ТС \тс \\
	Приложение № \Rownum. Акт осмотра ТС \тс\\
	Приложение № \Rownum. Фототаблица повреждений ТС\\
	Приложение № \Rownum. Калькуляция стоимости восстановительных расходов ТС \тс\\
	%	Приложение № \Rownum. Цифровые копии регистрационных документов ТС\\
	%	Приложение № \Rownum. Цифровая копия постановления по делу об административном правонарушении дорожно-транспортном происшествии\\
	Приложение № \Rownum. Правоустанавливающие документы эксперта-техника\\
}

%\includepdf[pages=-]{myfile.pdf}
%\includepdf[pages=-]{calc.pdf}

%\includepdf[pages=-]{myfile.pdf}
%\includepdf[pages=-]{calc.pdf}