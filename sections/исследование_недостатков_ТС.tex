\subsection{  }
\par
%\subsection*{Исследование транспортного средства}

Согласно публично доступных  каталогов запасных частей, размещенных в сети Интернет,  автомобиль с VIN \vin \ имеет следующие идентификационные параметры:
\begin{figure}[H]
	\centering
	\includegraphics[width=0.65\linewidth]{example-image}
	%\caption{Информация расшифровки VIN \vin \ по данным кталога \url{https://partsouq.com/}}
	%\label{vin}
\end{figure}


Реестр колесных транспортных средств, размещенный на официальном сайте Росстандарта, \url{https://easy.gost.ru/opendata/7706406291-recallcampaigns/} не содержит сведения об отзывных компаниях исследуемого автомобиля INFINITI QX80, VIN JN1JANZ62U0100644



Представитель изготовителя	Общество с ограниченной ответственностью "Ниссан Мэнуфэкчуринг Рус". 194362, г. Санкт-Петербург, поселок Парголово, Комендантский проспект, д. 140, Российская Федерация,. ОГРН: 5067847096609. Тел. +7 (812) 303-63-00. Факс +7 (812) 303-63-01. 

Сборочный завод	Nissan Shatai Kyushu Co., Ltd., юридический и фактический адрес: 1-3, Shinhama-cho, Kanda-machi, Miyako-gun, Fukuoka 800-0321, Япония




\subsubsection{Осмотр транспортного средства}

Осмотр и диагностическое исследование  автомобиля \тс \, \грз\, VIN \vin \, производились по адресу: ..........

Осмотр автомобиля производился  органолептическим методом. В процессе осмотра выполнялась фото и видео съемка объекта исследования цифровой фотокамерой.\\ 
Маркировочные обозначения, нанесенные на кузове ТС соответствуют записям  в свидетельстве о регистрации ТС  .............

\дварядом{example-image}{Исследуемый автомобиль  \тс \, \vin}{example-image}{Исследуемый автомобиль  \тс \, \vin}
\дварядом{example-image}{VIN на правом переднем брызговике}{example-image}{VIN на передней стойке}



\дварядом{example-image}{  \тс \, \vin}{example-image}{  \тс \, \vin}



\дварядом{example-image}{  \тс \, \vin}{example-image}{  \тс \, \vin}

При осмотре наружных элементов кузова  автомобиля визуально были установлены .........
........


По завершении наружного осмотра автомобиля проведена диагностика ТС VIN \vin\, программно-аппаратными средствами   официального дилера, Рис. \ref{пд}.
\begin{figure}[H]
	\centering
	\includegraphics[width=0.9\linewidth]{example-image}
	\caption{}
	\label{пд}
\end{figure}

Согласно протоколу диагностики от ...  

\begin{figure}[H]
	\centering
	\includegraphics[width=0.9\linewidth]{example-image}
	\caption{Идентификация ТС согласно протокола диагностики}
	\label{модельсканер}
\end{figure}

В результате выполнения диагностического ведомого поиска неисправностей  ............



\дварядом{example-image}{}{example-image}{}

\дварядом{example-image}{}{example-image}{}


Далее произведена диагностика запуска и работы двигателя исследуемого автомобиля при неподвижном состоянии  автомобиля. 

В ходе данной диагностики установлено:

\subsection{По поставленным вопросам }
\par

\paragraph{{По первому вопросу}}
\par
%\textbf{1. \textsl{Какими элементами пассивной безопасности должен быть оснащён автомобиль INFINITI QX80 г.р.з. О376ХР123, VIN JN1JANZ62U0100644 согласно спецификации производителя, присутствует ли на данном автомобиле система пассивной безопасности?}}
\par
Приказом Федерального агентства по техническому регулированию и метрологии от 19 июля 2018 г. № 420-ст с 1 сентября 2018 года на территории Российской Федерации отменены национальные стандарты РФ ГОСТ Р группы 41, утратившие свою актуальность в связи с прямым применением с 28 декабря 2000 года Правил ООН, принятых в соответствии с международным Женевским Соглашением 1958 года.
\par
Технический регламент ТР ТС 018/2011 «О безопасности колесных транспортных средств» устанавливает требования к активной, пассивной, экологической безопасности и защите от несанкционированного доступа. Обязателен для всех автомобилей, ввозимых или производимых в России и странах ЕАЭС.
\par
Согласно Технического регламента таможенного союза ТР ТС 018/2011 «О безопасности колесных транспортных средств», \cite[п. 6]{0182011:tr},  «безопасность транспортного средства - состояние, характеризуемое совокупностью параметров конструкции и технического состояния транспортного средства, обеспечивающих недопустимость или минимизацию риска причинения вреда жизни или здоровью граждан, имуществу физических и юридических лиц, государственному или муниципальному имуществу, окружающей среде».  

Система пассивной безопасности автомобиля — это комплекс элементов и технологий, предназначенных для минимизации травм и защиты жизни водителя, пассажиров, а иногда и пешеходов в случае аварии. В отличие от активной безопасности (которая предотвращает ДТП, например, ABS или ESP), пассивная безопасность активируется во время или сразу после столкновения. 


Требования к пассивной безопасности: защита  при ДТП.

%Подушки безопасности (водительские, передние, боковые, шторки).
%
%Ремни безопасности с преднатяжителями и ограничителями нагрузки.
%
%Конструкция кузова с зонами программируемой  деформации.
%
%Активные подголовники для защиты от травм шеи.
%
%Краш-тесты:
%
%Автомобили должны проходить испытания по методикам, приближенным к Euro NCAP (например, фронтальный удар на 64 км/ч, боковой удар).

К элементам пассивной безопасности, в общем случае, относятся:\\
\textsl{{\textbf{1. Ремни безопасности}}}

 Основным удерживающим  защитным устройством, предназначенным для удержания водителя и пассажиров в креслах при резком торможении, столкновении или перевороте автомобиля являются трехточечные ремни безопасноти.
 
Функция: Удерживают пассажиров на местах, предотвращая удары о элементы салона или вылет из автомобиля.\\
Дополнения:\\
Преднатяжители: Мгновенно натягивают ремень при ударе, устраняя провисание.\\
Ограничители нагрузки: Ослабляют натяжение при критическом давлении, чтобы не повредить грудную клетку.\\
\textsl{{\textbf{2. Подушки безопасности (Airbags)}}}\\
Дополнительным средством защиты являются подушки безопасности, которые дополняют ремни безопасности, создавая мягкий барьер и уменьшая риск травм головы и верхней части тела.  

Подушки безопасности дополняют ремни безопасности, увеличивая общую безопасность, но не могут полностью заменить их. Поэтому рекомендуется всегда использовать ремни безопасности, а подушки безопасности рассматривать как дополнительный уровень защиты.

Типы:\\
Фронтальные (для водителя и переднего пассажира).\\
Боковые (защита таза и грудной клетки).\\
Шторки безопасности (предотвращают травмы головы при боковом ударе).\\
Коленные (защищают ноги водителя).\\
Центральная (между передними сиденьями, снижает травмы при боковых столкновениях).\\
Как работают: Срабатывают за 20–30 миллисекунд после удара, смягчая контакт с твердыми поверхностями.\\
\textsl{{\textbf{3. Конструкция кузова}}}\\
Зоны програмируемой деформации: специальные участки кузова, которые поглощают энергию удара, деформируясь в запрограммированных направлениях. Это снижает силу удара, передающуюся на пассажиров\\
Жесткий каркас салона: Изготавливается из высокопрочной стали, чтобы сохранить «жилое пространство» для пассажиров даже при сильном ударе.\\
Усиленные стойки и двери: Защищают от проникновения посторонних объектов в салон.\\
\textsl{{\textbf{4. Активные подголовники}}}\\
При ударе сзади активные подголовники автоматически перемещаются вперед и вверх, чтобы минимизировать риск травм шеи (так называемого "хлыстового удара").\\
\textsl{{\textbf{5. Стекла  триплекс}}}\\
Особенность: При разрушении рассыпаются на мелкие неострые осколки, уменьшая риск порезов.\\
\textsl{{\textbf{6. Системы защиты пешеходов}}}\\
Приподнимающийся капот: Создает буферную зону между двигателем и капотом при наезде на пешехода.\\
Мягкие бампера: Поглощают удар, снижая травмы ног.\\
\textsl{{\textbf{7. Детские удерживающие системы}}}\\
Автокресла: Фиксируют ребенка в безопасном положении.\\
Крепление ISOFIX: Жестко пристегивает кресло к кузову, исключая смещение.\\
\textsl{{\textbf{8. Система экстренного вызова (например, ГЛОНАС)}}}\\
Функция: Автоматически отправляет координаты ДТП и сигнал SOS в службы спасения через GPS/GSM.\\
\textsl{{\textbf{9. Защита топливной системы}}}\\
Автоматическое отключение топливного насоса при ударе, чтобы предотвратить утечку и возгорание.\\
\textsl{{\textbf{10.Система защиты при опрокидывании
Шторки безопасности и усиленная конструкция крыши помогают защитить пассажиров при опрокидывании автомобиля.}}}\\
\textsl{{\textbf{11. Травмобезопасные элементы салона}}}\\
Скрытые крепления руля и педалей: Уменьшают риск проколов и переломов.\\
Мягкие панели на торпедо и дверях: Снижают травмы при контакте.\\


Техническая документация автомобиля  INFINITI QX80 г.р.з. О376ХР123, VIN JN1JANZ62U0100644 указывает на оснащение автомобиля большим количеством элементов пассивной безопасности. Так, Одобрение типа транмортного средства № TC RU E-CH.MT02.00303, выданное органом по сертификации	“САТР-ФОНД”  в разделе "Оборудование транспортного средства " содержт запись, прямо указывающую на оснащение автомобилей INFINITI QX 80 с VIN JN1JANZ62U??????? элементами пассивной безопасности (выделены курсивом): 
 "омыватель фар, магнитола, проигрыватель компакт-дисков, отопитель, бортовой компьюте,\textsl{ подушки безопасности}, подогрев сидений первого и второго ряда, кондиционер, электронная система контроля устойчивости, система мониторинга давления воздуха в шинах, по заказу: навигационная система, DVD проигрыватель, люк в крыше, датчики парковки, \textit{устройство вызова экстренных оперативных служб}". 
 
 «Руководство по эксплуатации автомобиля INFINITI QX80» в разделе 1 «Безопасность» содержит перечисление  элементов пассивной безопасности, которыми должен быть оснащён исследуемый автомобиль: \textsl{сиденья водителя и пассажиров; подголовники сидений водителя и пассажиров; ремни безопасности водителя и пассажиров; преднатяжители ремней безопасности водителя и переднего пассажира; детские удерживающие системы; фронтальные подушки безопасности водителя и переднего пассажира; боковые подушки безопасности водителя и переднего пассажира; головные подушки безопасности (шторки) левая и правая; система «ЭРА-ГЛОНАС».} 
 
 % TODO: \usepackage{graphicx} required
 \begin{figure}[H]
 	\centering
 	\includegraphics[width=0.8\linewidth]{"../images/foto/расположение подушек"}
 	\caption{Система SRS исследуемого автомобиля (см. РЭ INFINITI QX80 на русском языке с.1-38, дополни-тельные материалы)}
 	\label{SRS}
 \end{figure}
 
Ведущий онлайн-каталог электронных запчастей (EPC) Microcat EPC (Nissan Microcat EPC Online Parts Catalog, \url{https://www.infomedia.com.au/parts/electronic-parts-catalogue/}), интегрирированный с системами управления дилерской деятельностью (DMS) и использующися автопроизводителями  Nissan INFINITI для предоставления информации о подлинных запчастях и аксессуарах для их автомобилей по всему миру, содержит  полный перечень оригинальных запчастей для  исследуемого автомобиля INFINITI QX80, VIN JN1JANZ62U0100644, включая все элементы пассивной безопасности, указанные изготовителем данного автомобиля.

Фронтальный краш-тест, проведенный  NHTSA присвоила QX80 рейтинг в три звезды из пяти в категории фронтального столкновения, что указывает на умеренный уровень защиты от травм 4. Боковой краш-тест: В категории бокового удара автомобиль получил пять звезд, что свидетельствует о высоком уровне безопасности в этом типе столкновений 5. Тест на опрокидывание: в этой категории QX80 также получил три звезды, что указывает на вероятность опрокидывания в 23.50\%, \url{https://www.motorbiscuit.com/the-infiniti-qx80-fares-poorly-in-this-crucial-safety-area/}.

Все вышеперечисленное указывает на то, что согласно спецификации производителя автомобиль INFINITI QX80 2019 годв выпуска должен быть оснащен различными элементами пассивной безопасности.

Натурное исследовании автомобиля INFINITI QX80 г.р.з. О376ХР123, VIN JN1JANZ62U0100644 проведенное  с применением  диагностического оборудования  Consalt III и Launch 431  показало, что система SRS Airbag  
 (раздел «Self-Diagnostic Results» (Результаты самодиагностики) для проверки ошибок (DTC — Diagnostic Trouble Codes)) ошибок не имеет (код NO DTC).   В разделе «Data Monitor» (Режим данных) была проверена активность датчиков. Сотояние датчиков удара (Impact Sensors),  подушек безопасности (Airbag Status) и натяжителей ремней безопасности  отображаются как «Normal»,  что одноверенно указываетна на наличие этих компонентов в автомобиле и на их исправное состояние. При демонтаже подушки безопасности водителя подтверждена конструкция двухступенчатой подушки безопасности. По данным изготовителя, двухступенчатая система позволяет подушке безопасности адаптироваться к различным условиям аварии, обеспечивая оптимальный уровень защиты в зависимости от силы удара, веса пассажира и использования ремня безопасности. В зависимости от серьезности аварии подушка может срабатывать в два этапа.  Первый этап: при незначительном столкновении подушка срабатывает частично, чтобы смягчить удар и минимизировать травмы.  Второй этап: в случае более серьезного столкновения подушка полностью надувается, обеспечивая максимальную защиту водителя. 
 
 По совокупности результатов проведенного исследования, экспертами установлено, что исследуемый автомобиль   INFINITI QX80 2019 года, согласно спецификации производителя,   оснащен следующим комплексом систем пассивной безопасности, направленных на защиту водителя и пассажиров в случае аварии:
 

 
\textbf{ \textbf{1. Подушки безопасности (Airbags)}}

Автомобиль оснащен усовершенствованной двухступенчатой  системой  фронтальных подушек для водителя и переднего пассажира подушек  (AABS) с датчиками ремней безопасности и классификации пассажиров.

 
 Боковые подушки безопасности (передние), встроенные в спинки сидений.
 
 Шторки безопасности (защита головы), охватывающие все три ряда сидений.
 
 Коленная подушка безопасности для водителя (защита ног).
 
 
Процесс развертывания начинается через 0,01 секунды после обнаружения аварии, подушка полностью надувается через 0,04 секунды и начинает сдуваться через 0,1 секунды, как указано в руководствах по безопасности NISSAN и INFINITI, \url{https://www.nhtsa.gov/vehicle-safety/air-bags}.
 
 
\textbf{\textbf{ 2. Ремни безопасности}\\}
 Трехточечные инерционные ремни для всех пассажиров.
 
 Пиропатронные натяжители ремней (Seatbelt Pretensioners) и электромеханическими преднатяжителями ремней безопасности для водителя и переднего пассажира:
 
 Автоматически натягивают ремни при аварии.
 
 Ограничители нагрузки ремней (Load Limiters):
 
 Снижают давление ремня на грудную клетку при резком рывке.
 
\textbf{\textbf{ 3. Конструкция кузова}\\}
 Усиленный каркас кузова (High-Strength Steel):
 
 Зоны программируемой деформации для поглощения энергии удара.
 
 Защита при боковом ударе:
 
 Усиленные стойки и пороги.
 
 Система защиты педалей (при фронтальном ударе педали смещаются, снижая риск травм ног).
 
\textbf{\textbf{ 4. Защита детей}\\}
 Крепления ISOFIX для детских кресел (второй ряд сидений).
 
 Замки ремней безопасности с защитой от детей.
 
\textbf{ \textbf{5. Дополнительные системы}\\}
 Активные подголовники (Active Head Restraints):
 
 Снижают риск травм шеи при ударе сзади.
 
 Система фиксации сидений (при аварии предотвращает смещение кресел).
 
 Защита при опрокидывании (усиленная крыша и каркас).
 
\textbf{\textbf{ 6. Особенности для российского рынка}:\\}
 адаптация к холодному климату (надежность работы датчиков и систем).
 
 
  \par
 В Infiniti QX80 2019 года, как и в подавлябщем большинстве совпеменных автомобилей  ремни безопасности являются  элементом системы безопасности,  обеспечивающем основную защиту и эффективны в различных типах аварий. Подушки безопасности дополняют ремни, увеличивая общую безопасность, но не могут полностью заменить их. Поэтому рекомендуется всегда использовать ремни безопасности, а подушки безопасности рассматривать как дополнительный уровень защиты.
 
 \textbf{{Согласно данных изготовителя автомобиля, 
 		эффективность подушек зависит от правильного использования ремней безопасности и положения сидений. Например, водитель должен находиться на расстоянии не менее 25 см от руля}}.\\
 
 \vspace{3mm}
 
\textbf{ {Таким образом, на  данном автомобиле присутствует система пассивной безопасности, соответствующая спецификации производителя автомобиля}}.\\






 


%
%
%
%\par
%Система пассивной безопасности Infiniti QX80 2019 года это система, направленая на защиту водителя и пассажиров в случае аварии.   В систему входят:
%
%
%
%{ Ремни безопасности:}
%
%\begin{itemize}
%	\item Ремни с преднатяжителями: автоматически натягиваются при резком торможении или ударе.
%	\item Ограничители нагрузки: регулируют усилие натяжения ремня, снижая риск травм грудной клетки.
%	\item Ремни для всех пассажиров, включая третий ряд.
%\end{itemize}
%
%
%{ Конструкция кузова:}
%
%\begin{itemize}
%	\item Зоны программируемой деформации: поглощают энергию удара, минимизируя деформацию салона.
%	\item Усиленный каркас салона (из высокопрочной стали) для защиты от сдавливания.
%\end{itemize}
%
%{ Активные подголовники (для передних сидений):}
%
%\begin{itemize}
%	\item Система защиты от хлыстовых травм (Whiplash Protection): подголовники автоматически смещаются вперед при ударе сзади, снижая риск травм шеи.
%\end{itemize}
%
%{ Система фиксации детских кресел:}
%
%\begin{itemize}
%	\item Крепления LATCH (Lower Anchors and Tethers for Children) для безопасной установки детских автокресел.
%\end{itemize}
%
%{ Защита при опрокидывании:}
%
%\begin{itemize}
%	\item Усиленные стойки крыши и система стабилизации кузова.
%	\item Шторки безопасности активируются при риске переворота.
%\end{itemize}
%
%{ Дополнительные элементы:}
%
%\begin{itemize}
%	\item Аварийный размыкатель аккумулятора: отключает питание при серьезном ДТП, снижая риск возгорания.
%	\item Структура педального узла: предотвращает смещение педалей в салон при фронтальном ударе.
%\end{itemize}
%\par
%{Infiniti QX80 2019 соответствует современным требованиям безопасности, включая:}
%
%\begin{itemize}
%	\item Сертификацию NHTSA (Национальное управление безопасностью движения на трассах США).
%	\item Технологии, одобренные IIHS (Страховой институт дорожной безопасности).
%\end{itemize}
%
%\par
%В Infiniti QX80 2019 года подушки безопасности являются ключевым элементом пассивной безопасности. 
%
%{Типы и расположение подушек безопасности}
%
%\begin{itemize}
%	\item В автомобиле установлено 8 подушек безопасности:
%	\item 2 фронтальные подушки (водитель и передний пассажир).
%	\item 2 боковые подушки (в передних сиденьях, защита грудной клетки).
%	\item 4 шторки безопасности (передние и задние боковые окна, защита головы для всех трех рядов).
%\end{itemize}
%
%{Особенности фронтальных подушек}
%
%\begin{itemize}
%	\item Двухступенчатое срабатывание:
% Система определяет силу столкновения и регулирует скорость надувания подушек:
%	\item При легком ударе подушки надуваются медленнее, чтобы снизить риск травм.
% При сильном ударе — мгновенно.
%\item Оптимизированная форма:
%Подушки водителя и пассажира имеют разную конструкцию, учитывая расстояние до руля и панели приборов.
%\end{itemize}
%
%{Боковые подушки (передние сиденья)}
%
%\begin{itemize}
%	\item Интегрированы в боковины сидений, а не в дверь, что улучшает защиту при боковом ударе.
%	\item Защита грудной клетки и таза:
% Снижают риск травм от контакта с дверью или предметами вне автомобиля.
%\end{itemize}
%
%{Шторки безопасности}
%
%Расположение: Верхние части боковых стоек (A, B, C, D-стоек).
%
%\begin{itemize}
%	\item Защита при перевороте и боковых ударах:
% Шторки остаются надутыми несколько секунд, чтобы предотвратить множественные удары головы.
%\item Покрытие всех трех рядов:
% Даже пассажиры третьего ряда защищены от контакта с окнами или стойками.
%\end{itemize}
%
%{Система управления подушками (ACM — Airbag Control Module)}
%
%\begin{itemize}
%	\item Датчики удара:
% Расположены в передней части, дверях и центральной стойке. Определяют направление, силу и тип удара (фронтальный, боковой, переворот).
%	\item Алгоритмы адаптации:
%Например, при столкновении на низкой скорости подушки могут не сработать, если система считает, что ремней безопасности достаточно.
%\end{itemize}
%
%{Взаимодействие с другими системами}
%
%\begin{itemize}
%	\item Ремни безопасности с преднатяжителями:
% Подушки срабатывают синхронно с натяжением ремней, фиксируя тело в оптимальном положении.
%	\item Отключение передней пассажирской подушки:
% Если на сиденье установлено детское кресло (об этом сигнализирует датчик веса), подушка автоматически деактивируется.
%\end{itemize}
%
%{Технологии материалов}
%
%\begin{itemize}
%	\item Вентиляционные отверстия в подушках:
% Позволяют контролировать скорость сдувания, смягчая удар.
%\item Термостойкая ткань:
% Снижает риск ожогов при контакте с горячим газом (используемым для надува).
%\end{itemize}
%
%{ Краш-тесты и эффективность}
%
%\begin{itemize}
%	\item Оценка NHTSA:
% Infiniti QX80 2019 получил 4 из 5 звезд за защиту при фронтальном и боковом ударах.
%	\item Защита головы и шеи:
% Шторки и активные подголовники снижают риск травм шеи (хлыстовых) на 30–40\%.
%\end{itemize}
%
%{Обслуживание и предупреждения}
%
%\begin{itemize}
%	\item Индикатор на приборной панели:
%Если горит лампа «Airbag», это указывает на неисправность системы (требуется диагностика).
%	\item Замена после срабатывания:
%Подушки и датчики должны быть заменены у официального дилера.
%	\item Ремонт только у профессионалов:
%Cамостоятельное вмешательство может привести к несанкционированному срабатыванию.
%\end{itemize}
%
%\textbf{\textit{Важно!
%Эффективность подушек зависит от правильного использования ремней безопасности и положения сидений. Например, водитель должен находиться на расстоянии не менее 25 см от руля}}.
%
%
%
  %уд

\paragraph{{По второму вопросу}}
\par
%\textbf{\textsl{2.	Находилась ли система пассивной безопасности в автомобиле INFINITI QX80 г.р.з. О376ХР123, VIN JN1JANZ62U0100644 в исправном состоянии на момент ДТП 26.05.2021 г. и находится ли она в исправном состоянии в настоящий момент?}}
\par

Согласно ГОСТ Р 27.102-2021. Надежность в технике. Надежность объекта. Термины и определения» исправное состояние (исправность) это такое состояние объекта, в котором все параметры объекта соответствуют всем требованиям, установленным в документации на этот объект. Сллтветственно, неисправное состояние (неисправность): состояние объекта, в котором хотя бы один параметр объекта не соответствует хотя бы одному из требований, установленных в документации на этот объект,  \cite[п. 12, п.13]{271022021:gost}. 

 Технический регламент таможенного союза ТР ТС 018/2011 «О безопасности колесных транспортных средств», Приложение № 8 «Требование к транспортным средствам, находящимся в эксплуатации» содержит перечень требований к элементам пассивной безопасности. Методы проверки элементов пассивной безопасности определены ГОСТ 33997-2016. «Колёсные транспортные средства. Требования к безопасности в эксплуатации и методы проверки» \cite{33997:gost}. 


Неисправность элементов пассивной безопасности  может быть вызвана повреждением:\\
-	элементов пассивной безопасности, входящих и не входящих в систему SRS;\\
-	электрической цепи, предназначенной для управления элементами пассивной безопасности, входящих в систему SRS.\\

В ходе проведения настоящей экспертизы для определения наличия (отсутствия) неисправностей элементов пассивной безопасности экспертами  были выполнены:\\
- исследование элементов пассивной безопансоти органолептическими методами;\\
- тест системы SRS методом самодиагностики автомобиля;\\
- компьютерная диагностика электронных систем автомобиля, включая тестирование системы SRS.\\

Органолептическими методами было определено, что на момент производства экспертизы:\\

На всех сиденьях: на ленте ремней отсутствуют потертости, разрывы, трещины; катушки  плавно  втягивают ремни; при резком рывке ремни  блокируются; язычки в замок вставляются с слышимым четким щелчком; при нажатии на кнопки разблокировки язычки  легко извлекаются;
замки и ремни надежно зафиксированы на кузове/сиденьях, без люфтов; коррозия, деформация в зоне установки ремней отсутствует.

Сиденья передние и задние - смонтированы штатно,  чехлы на всех сиденьях отсутствуют,  обивка сидений без следов потертостей, швы обивки без разрывов и повреждений. Каркас сидений без деформаций, крепление сидений к полу без люфта, болты сидений визуально затянуты. Подголовники сидений находятся в исходном положении, регулировка работает плавно, люфт подголовников отсутствует. Зона установки сидений без видимых деформаций и иных повреждений.


Визуальным осмотром подушек безопасности установлено, что 
целостность модулей подушек безопасности (передние, боковые, шторки, коленные) не нарушена,  передние подушки (в руле и панели приборов) наличие трещин, деформаций или следов срабатывания не имеют. 
Боковые подушки (в спинках передних сидений) и шторки (под потолочной обивкой) скрыты обшивкой,  целостность обшивки не нарушена. Повреждения коленных подушек (под панелью в области ног водителя и переднего пассажира) отсутствуют.   Облицовка салона без видимых признаков демонтажа и повреждений. Монтажное положение облицовочных элементов салона соответствует заводской установке. 

Подушка безопасности водителя скрыта под панелью приборов. Облицовка панели приборов на стороне водителя не повреждена, видимые признаки демонтажа панели приборов отсутствуют.

Подголовники передних сидений без видимых механических повреждений, ощутима податливость под лёгким давлением на подголовники, что указывает на  подвижность подголовников и отсутствие повреждений механизма выдвижения.


Визуальным осмотром кузова автомобиля на предмет деформации зон поглощения энергии установлено, что каркас салона автомобиля без признаков нарушения герметрии, пространственный карас рамы автомобиля нарушений плоскотности не имеет. Присутствуют  повреждения переднего бампера, усилителя и ударопоглотителя переднего бамепра, капота, левого перенего брызговика,  верхнего усилителя  переднего левого брызговика, верхнего усилителя арки колеса передней правой, панели рамки радиатора,  левого переднего краш-бокса.    

Датчик удара  фронтальных подушек безопасности на момент осмотра демонтирован с посадочного места,  датчик находится в электрической цепи автомобиля и штатным соединением  подключен к блоку подушек безопасности,  повреждения  жгута проводов датчика удара, разъемных соединений визуально отсутствуют.


Автомобиль INFINITI Qx80 оснащен развитой системой самодиагностики электронных компонентов, в том числе системой самодиагностической проверки системы подушек безопасности. Согласно «Руководству по эксплуатации автомобиля INFINITI QX80», стр. 5-12, при включении зажигания автомобиль запускает тест самодиагностики системы SRS. Индикатор SRS на приборной панели  работает согласно описанию руководства по эксплуатации автомобиля. При включении зажигания индикатор  загорается примерно на 7 секунд и гаснет; постоянное свечение или мигание или отсутствие свечения при вкючении зажигания, указывающее на неисправность системы отсутствует.

Согласно   теста самодиагностики,  система SRS исправна.

Далее, с использованием диагностического оборудования  СТОА «Финик-Авто»   сканеров CONSULT III PLUS V.80.21 и LAUNCH X431 была проведена компьютерная диагностика всех электронных систем автомобиля.

В результате диагностики электронных систем автомобиля установлено, что коды ошибок, указывающие на неисправности датчиков системы SRS, обрывы электрических цепей подушек безопасности, преднатяжителей ремней безопасности отсутствуют.


В заключении была произведена проверка функциональности компонентов системы подушек безопасности.

Фронтальный датчик подушек безопасности был отключен путем разъедитения электрического разъема и с помощью диагностического сканера произведена диагностика системы SRS. Коды неисправностей, соответствующие повреждению датчика зафиксированы.  Так же было произведено отключение модуля подушки безопасности водителя путем разъединения разъемов  модуля подушки безопасности и произведена диагностика системы SRS. По окончанию теста  зафиксированы коды неисправностей, соответствующие неисправностям фронтальной подушки безопасности водителя.  Следовательно, электрические и электронные компоненты системы подушек безопасности функционируют исправно.


\begin{itemize}
	\item B1031 – Цепь индикатора отключения подушки безопасности пассажира
	\item B1032 – Цепь индикатора включения подушки безопасности пассажира
	\item B1033 – Высокое сопротивление модуля подушки безопасности пассажира
	\item B1034 – Низкое сопротивление модуля подушки безопасности пассажира
	\item B1049 — высокое сопротивление модуля подушки безопасности водителя
	\item B1050 – Низкое сопротивление модуля подушки безопасности водителя
	\item B1053 — Разрыв цепи модуля подушки безопасности водителя
	\item B1054 — Замыкание на массу модуля подушки безопасности водителя
	\item B1055 – Модуль подушки безопасности водителя замыкается на батарею
	\item B1057 — Разрыв цепи модуля подушки безопасности пассажира
	\item B1058 – Модуль подушки безопасности пассажира замыкается на массу
	\item B1059 – Модуль подушки безопасности пассажира замыкается на батарею
\end{itemize}

Имитация условий срабатывания путем симуляции ускорения/замедления с помощью диагностического сканера   и проверка соответствующей реакции системы SRS  (например, фиксации преднатяжителей или индикатора) в настоящем исследовании не проводилась, так как экспертам было отказано судом в разрушающих методах исследования, а имитация срабатывания подушек безопасности на автомобиле после ДТП не может являеться 100\% безопасной.


Таким образом, в результате проведенного исследования экспертами установлено, что на момент ДТП 26.05.2021 г. система пассивной безопасности находилась в исправном техническом состоянии.  На момент производства экспертизы электронные и электрические элементы системы пассивной безопасности находятся в исправном техническом состоянии, элементы пассивной безопасности кузова автомобиля в фрональной левой части автомобиля в составе: бампера переднего, ударопоглотителя бампера переднего, усилителя бампера переднего, левого краш-бокса, левого переднего брызговика, капота деформированы, следовательно элементы пассивной безопасности кузова транспортного средства на момент настоящего исследования находятся в неисправном техническом состоянии. 

  % уд.

\paragraph{{По третьему вопросу}}
\par
\textsl{\textbf{3. 	Должны ли были сработать в автомобиле INFINITI QX80 г.р.з. О376ХР123, VIN JN1JANZ62U0100644 элементы пассивной безопасности в условиях ДТП 26.05.2021 г. и были ли достигнуты условия для их срабатывания?}}

Для открытия передних подушек безопасностей, автомобиль должен совершить жёсткое, блокирующее контактирование (с другим автомобилем или препятствием), при этом после столкновения он не должен двигаться вперёд по ходу своего движения (в данной случае пороговое значение замедления не будет достигнуто). Передние подушки безопасности практически в 100\% случаях должны раскрыться, если автомобиль после столкновения «отскочил» назад (на 180 градусов от направления своего движения) и у него значительно деформировались передние лонжероны (или рама).

При рассмотрении вопросов № 1 и № 2, было определено, что в результате ДТП 26.05.2021 г. такие элементы пассивной безопасности исследуемого автомобиля, не входящие в систему SRS, как   бампер передний, капот, усилитель и ударопоглотитель переднего бампера, левый передний краш-бокс усилителя бампера переднего и другие конструктивные элементы  передней зоны программируемой деформации кузова автомобиля были смяты по  заданным участкам контролируемого сминания  при фронтальном столкновении, поглотив часть кинетической энергии удара, тем самым  снизив силу воздействия на водителя.   Так как  при ДТП водитель не получил серьезных травм и увечий, то есть все основания полагать, что  элементы пассивной безопасности автомобиля сработали надлежащим образом. 

 
В то же время, в данном ДТП подушки безопасности, натяжители ремней безопасности так же входящие в систему SRS не сработали.


На основании изложенного эксперты приходят к обобщению, что в контексте поставленного вопроса целью  исследования  является  определение причины несрабатывания подушек безопасности в автомобиле Infiniti QX80 2019 года при столкновении с бескапотным грузовиком КамАЗ-5490-S5.


Подушки безопасности являются дополнительной защитой и разработаны для оптимальной работы в сочетании с ремнями безопасности. Как было установлено выше  по совокупности признаков,  в момент ДТП водитель автмобиля INFINITI QX80 не был пристегнут ремнями безопасности.

 В автомобилях Infiniti QX80 подушки безопасности, \cite{патент:US8801033B2} срабатывают при определенных условиях, которые зависят от типа датчиков, уровня удара, направления столкновения и ряда других факторов. Прямой зависимости между пристегнутыми ремнями безопасности и срабатыванием подушек безопасности нет, однако алгоритм активации подушек безопасности учитывает, были или небыли  прстегнуты водитель и пассжир ремнями безопасности.
 
  Основные условия срабатывания подушек безопасности включают:

1. \textbf{Сила удара}: Подушки безопасности срабатывают при ударах определенной силы, когда ускорение или замедление автомобиля превышает заданный порог. Системы определяют, достаточно ли серьезен удар для активации подушек.  Фронтальные подушки безопасности, как правило, предназначены для срабатывания при «умеренных или серьезных» фронтальных или почти фронтальных столкновениях, которые определяются как столкновения, эквивалентные удару о твердый неподвижный барьер на скорости от 13 км/ч до 22 км/ч или выше, что примерно  эквивалентно удару о припаркованный автомобиль аналогичного размера на скорости от 25,5 км/ч до 45 км/ч, \url{https://www.nhtsa.gov/vehicle-safety/air-bags}.

2. \textbf{Направление удара}:
- Фронтальные подушки безопасности срабатывают при фронтальных столкновениях средней и высокой тяжести. Обычно они активируются при столкновениях спереди под углом от 10 до 30 градусов, если сила удара соответствует пороговому значению.

- Боковые подушки безопасности активируются при боковых ударах и защищают грудную клетку и голову пассажира.

- Шторки безопасности срабатывают при ударах сбоку и помогают защитить голову пассажиров от боковых стекол и внешних объектов.

3. \textbf{Скорость движения}: Подушки могут не сработать на низких скоростях, так как система оценивает, достаточно ли опасна ситуация для водителя и пассажиров.

4.\textbf{ Расположение пассажиров и ремни безопасности}: Система безопасности отслеживает, пристегнуты ли ремни безопасности, а также вес и положение пассажиров. Например, если на переднем пассажирском сидении находится ребенок (это определяется датчиками веса), фронтальная подушка может быть отключена для безопасности ребенка. 

5. \textbf{Особые ситуации}: Подушки не сработают при перевороте автомобиля, если нет сильного фронтального или бокового удара. Также они не сработают при заднем столкновении, так как подушки предназначены для защиты от фронтальных и боковых ударов.


Административный материал, составленный сотрудниками ГАИ по факту  ДТП позволяет определить только направление удара, является мало информативным и не позволяет  определить физические характеристики  столкновения автомобилей (силы удара, скорости, замедления, энергии и т.д). В таком случае,  разрешение вопроса исследования  возможно на основании анализа  деформации транспортных средств, полученных в ДТП. При этом результаты исследования будут иметь вероятностный характер.  


Для расчета энергии удара по деформации автомобиля применим   принципы механики и физики. В частности, энергия удара (кинетическая энергия) преобразуется в работу по деформации, когда автомобиль сталкивается с препятствием или другим автомобилем.

%Основные шаги для расчета:
%
%1. \textbf{Определение кинетической энергии до столкновения}:
Кинетическая энергия \( E_k \) автомобиля до столкновения рассчитывается по формуле:
\[
E_k = \frac{1}{2} m v^2
\]
где:

\begin{itemize}
	\item \( m \) — масса автомобиля (в килограммах),
	\item \( v \) — скорость автомобиля до столкновения (в метрах в секунду).
\end{itemize}


 Из закона сохранения энергии сдедует,что кинетическая энергия автомобиля до столкновения равна энергии деформации:
{\large 	\[
	\frac{1}{2} m v^2 = E_d.
	\]}

Отсюда скорость \( v \):
{\large 	\[
	v = \sqrt{\frac{2 E_d}{m}}.
	\]}
где:

${E_d}$ - энергия, затраченная на деформацию

Для реального ДТП, зная    массы автомобилей, их взаимное положение в момент столкновения и их
остаточные скорости после удара, технически возможно установить скорости    этих автомобилей в момент удара исходя из затрат энергии на деформацию их    конструкций.  

Общепризнанной методикой, позволяющей точно определить ключевые параметры столкновения автомобилей, включая скорость автомобилей, энергию столкновения и механику удара является методика \textbf{CRASH3},  
основанная на анализе лабораторных испытаний автомобилей на удар.

	Методика {CRASH3} (Calspan Reconstruction of Accident Speeds on the Highway, версия 3), \cite{crash3} — это инструмент для реконструкции дорожно-транспортных происшествий, разработанный с целью определения скоростей транспортных средств в момент столкновения, параметров столкновения, перемещения после удара и оценки энергии деформации.  Методика  используется экспертами для получения количественных выводов о механике аварий на основе данных о повреждениях и условиях происшествия.  В 2004 году в п.5.2.3 методических рекомендаций по проведению независимой технической экспертизы транспортного средства при ОСАГО № 001МР/СЭ (утв. ГНИИ АТ Минтранса РФ, РФЦСЭ МЮ РФ, ЭКЦ МВД РФ, НПСО «Объединение транспортных экспертов») для  определения причин повреждений объектов экспертизы рекомендовалось использовать справочники с  результатами краш-тестов конкретных марок (моделей, модификаций) транспортных  средств, и, в частности результаты краш-тестов NHTSA, имеющиеся в свободном доступе в виде отчетов и в цифровом виде, вместе с программами для обработки данных.  В Российской Федерации проведение краш-тестов осуществляется в соответствии с ГОСТ Р 41.94-99 (Правила
	ЕЭК ООН № 94) "Единообразные предписания, касающиеся официального утверждения транспортных средств в отношении защиты водителя и пассажиров в случае лобового столкновения" и ГОСТ Р 41.95-99 (Правила ЕЭК ООН № 95) "Единообразные предписания, касающиеся официального утверждения транспортных средств в отношении защиты водителя и пассажиров в случае бокового столкновения". 
	
	
	
%	Используются методика CRASH3 для расчёта энергии удара, закон сохранения импульса, данные диагностики через CONSULT-III, визуальный анализ повреждений по фотографиям, а также учёт увеличения массы обоих автомобилей. Анализируется применимость данных краш-теста Nissan Armada SV 4WD 2018 года и стандарта PART 563---Event Data Recorders.
	
   
Соглано методики CRASH3 для расчета величины затрат энергии на деформацию автомобиля при   фронтальном столкновении требуется знать три коэффициента жесткости его передней  части: \textbf{\textit{ A}},  \textbf{\textit{B}} и \textbf{\textit{G}}.

Коэффициент \textit{A} (Н/м)
 -- удельное сопротивление конструкции до начала пластической деформации.    
 
 Коэффициент \textit{B} (Н/м2) -- удельная сила, необходимая для развития пластической деформации.
 
  Коэффициент \textit{G} (производный) - параметр, зависящий от \textit{A} и \textit{B}.
  
  
  Расчет коэффициентов
	
	Коэффициент \textbf{A} для разных значений скорости нулевой деформации $b_0$:
	
	{\large   \[A = \frac{mb_0b_1}{3.6^2L} \]}
	
	Коэффициент  \textit{B}::
	
	
	{\large    \begin{equation}
			B = \frac{mb_1}{3.6^2L}
			\label{A}
		\end{equation}
	}
	
	
	
	
	Величина коэффициента жесткости G автомобиля  для
	разных значений скорости нулевой деформации $ b_0 $составляет
	
	
	{\large \[G = \frac{A^2}{2B} \]}
	
	Если деформация \textit{С} есть средняя деформация автомобиля:
	
	{\large \[ C = C_{i}+(C_{i+1}-C_{i})\frac{l}{L_{i}}, 0 \leq l \leq L_{i}, \eqno(11) \]}
	
	тогда затраты энергии на дефрмацию на \textit{i}-ом участке деформированной поверхности автомобиля составляют:
	
	{\large \[ E_{i} = \int_{0}^{L_{i}}\left( AC + \frac{BC^2}{2} + G\right) dl = L_{i}\left( \frac{A}{2}(C_{i}+C_{i+1})+\frac{B}{6}(C^{2}_{i}+C_{i}C_{i+1}+C^{2}_{i+1})+G\right), \eqno(12)  \]
	}
	
Суммарные затраты энергии определяются сложением $E_{i}$ для всех участков с учётом аддитивности энергии.

Необходимые для   расчета значения жесткости конструкций    автомобилей возможно установить на основании экспериментальных данных  краш-тестов. 

Публично доступные отчеты по краш-тестам автомобилей размещяются, например,   на сайтах NHTSA (Национального управления безопасности дорожного движения США).


% TODO: \usepackage{graphicx} required
\begin{figure}[H]
	\centering
	\includegraphics[width=0.9\linewidth]{../images/foto/rating1}
	\caption{{\small Рейтинг автомобиля Infiniti QX80 2019 года выпуска, официально опубликованный на сайте NHTSA, \url{https://www.nhtsa.gov/vehicle/2019/INFINITI/QX80/SUV/AWD}. Технический отчет v10562R001.pdf, 28601.392 kB (M20195203 2019 Nissan Armada SUV NCAP Final Report.pdf)}}
	\label{fig:rating1}
\end{figure}




%INFINITI QX80  крупный люксовый внедорожник,  в основном ориентированный на рынки США, Ближнего Востока и России. Однако официальных результатов краш-тестов от ведущих международных организаций, таких как Euro NCAP, IIHS (Insurance Institute for Highway Safety) или NHTSA (National Highway Traffic Safety Administration), для этой модели нет.
%
%Причины отсутствия официальных рейтингов:
%
%Рыночная специфика: QX80 не продается в Европе, поэтому Euro NCAP его не тестирует.
%
%Американский рынок: В США QX80 позиционируется как нишевая модель, и IIHS/NHTSA также не публиковали данных о его краш-тестах (на момент 2023 года).
%
%Российские тесты: В России краш-тесты проводятся редко и не имеют международного признания. Для QX80 таких данных нет.
%
%	
%	
%Публично доступные отчеты по краш-тестам автомобилей размещяются  на сайтах NHTSA (Национального управления безопасности дорожного движения США), \url{https://www.nhtsa.gov/},  IIHS (Insurance Institute for Highway Safety (Институт страхования и безопасности на дорогах). Американская некоммерческая организация, занимающаяся исследованиями в области безопасности дорожного движения), \url{https://www.iihs.org/ratings}, 
%Европейская программа оценки новых автомобилей (Euro NCAP): Оценила безопасность Infiniti QX80 для европейского рынка, \url{https://www.euroncap.com}, Официальный сайт Австралийской программы оценки новых автомобилей (ANCAP), \url{Официальный сайт Австралийской программы оценки новых автомобилей (ANCAP)}.
	
.

Для автомобиля Infiniti QX80 NHTSA при оценке безопасности автомобиля использует данные полноприводно версии Nissan Armada.  Nissan и Infiniti — дочерние компании одного концерна (Nissan Motor Corporation).     Infiniti QX80 и Nissan Armada построены на одной платформе (Nissan F-Alpha) и имеют: 

идентичную раму (лестничная конструкция);

идентичные кузова и силовые элементы;

одинаковые зоны деформации;

схожие системы пассивной безопасности (подушки, ремни, усиление стоек).


Различаются Infiniti QX80 и Nissan Armada  материалами отделки салона, дополнительными опциями, влюяющими на комфорт, внешним дизайном.

Краш-тесты, проводимые NHTSA для базовой полноприводной модели Armada  полностью применимы к люксовой версии INFINITI QX80.

	База 	данных 	краш-тестов 	транспортных средств (Vehicle Crash Test Database) NHTSA находится  по адресу в интернете: \url{ https://www.nhtsa.gov/research-data/research-testing-databases#/vehicle}. Эта база данных  содержит технические 	данные, полученные в ходе различных видов испытаний, в том числе 	по программе оценки новых автомобилей NCAP. Доступ к базе данных 	свободный, регистрация на сайте не требуется. 
%
%Кроме краш-тестов по программе NCAP, база данных содержит краш-тесты на боковые удары одного транспортного средства в другое.

 Также 	на сайте NHTSA имеется программное обеспечение  для инженерного анализа, использующее локальные файлы UDS для испытаний на удар. Эти параметры включают в себя графики сигналов 	лабораторных акселерометров, устанавливаемых на транспортных средствах 	и манекенах водителя и пассажиров, и графики сигналов тензодатчиков в ячейках жестких неподвижных барьеров. %Загрузку файлов указанных 	сигналов на локальный компьютер можно произвести в ряде форматов 	NHTSA. 	%Краш-тесты можно выбрать из базы данных следующим образом: 	• посмотреть последние 10 краш-тестов (Browse the latest tests);
%	• выбрать краш-тесты по параметрам (Query by test parameters). Выбор 	включает в себя интервал номеров краш-тестов (Test No) или его ссылочный 	номер (Test Reference No), тип теста (Test Type) из выпадающего списка,
%	название контракта или исследования (Contract or Study Title), лабораторию- 	исполнителя краш-теста из выпадающего списка (Test Performer), интервал
%	угла удара (Impact Angle) в транспортное средство в градусах, конфигурацию 	краш-теста (Test Configuration) из выпадающего списка, интервал величины 	смещения (Offset Distance) в мм, интервал скорости удара (Closing Speed)
%	в км/ч, 	порядок 	сортировки 	(Order 	By)  найденных
%	краш-тестов 	из выпадающего 	списка 	(по умолчанию — 	по номеру) 	по возрастанию 	(Ascending) или убыванию (Descending), число записей на странице (Records 	Per Page) в списке найденных краш-тестов. 
%	

%База данных краш-тестов NHTSA может быть экспортирована целиком любым пользователем. Для помощи пользователям доступны описания форматов экспортируемых файлов. Файлы сжаты zip-архиватором, поля данных разделены символом «|». 	Для извлечения файлов и их импорта в другие программы потребуется 	вспомогательное 	программное
%обеспечение. Такие программы, как, например, Microsoft Access и Excel, поддерживают механизм импорта файлов ASCII и позволяют пользователю выбирать разделитель полей
%данных. Для установки программного обеспечения NHTSA
%со страницы  в интернете \url{https://www.nhtsa.gov/databases-and-software/signal-analysis-software-windows} надо скачать пакет установщика VSR\_RDDBSOFTWARE 	и запустить его на своем компьютере. В результате будут установлены 	15 программ, среди которых 
Для настоящего исследования требуется применить две програмны:

· LoadCellAnalysis — анализ сигналов в ячейках барьера,

· SignalBrowser — вычисления и анализ сигналов в краш-тесте.




	


%	\input{сравнение nissan infiniti}







\textbf{Особенности частичного перекрытия (40\%) :}\\
При частичном ударе нагрузка распределяется неравномерно.
Основная деформация будет сосредоточена в зоне контакта.
Энергопоглощающие элементы могут работать менее эффективно.	


Кабина КАМАЗ-5490 S5 спроектирована на базе кабины Mercedes-Benz Actros третьего поколения. Габариты кабины КАМАЗ-5490 S5 практически не отличаются от прототипа Mercedes-Benz Actros. Основные изменения касаются внешнего дизайна и адаптации под российские условия, но они минимально влияют на общие размеры. Каркас кабины КАМАЗ-5490 S5 сохраняет базовую конструкцию Mercedes-Benz Actros, но имеет усиления, адаптацию материалов и измененные настройки для работы в российских условиях. Эти изменения минимально влияют на базовую геометрию, но делают конструкцию более прочной и ремонтопригодной. Номер лицензионного соглашения на использование конструкции кабины Mercedes-Benz Actros для КАМАЗ-5490 S5 не является публичной информацией.


Спроси

Объяснить








Результаты проверки истории автомобиля на предмет предыдущих аварий и замены подушек исключают события, связанные с активацией системы SRS с начала эксплуатации автомобиля.	
\textbf{\textsl{	Вывод по третьему вопросу}}
\textbf{Скорость удара составила примерно 45-55 км/ч , что находится на границе порога срабатывания подушек безопасности для данного типа столкновения. Невысвобождение подушек является результатом комбинации факторов: скорости удара, типа столкновения и состояния ремней безопасности.}
	




\subsection{На что ссылаться при исследовании}    	

Для анализа системы подушек безопасности Infiniti QX80 можно обратиться к нескольким источникам технической литературы, которые обеспечат детальное понимание устройства, принципов работы и условий срабатывания системы подушек безопасности. Вот основные типы и конкретные примеры литературы, на которые можно сослаться:

1. **Руководство по ремонту и обслуживанию (Service Manual)**:
- **Infiniti QX80 Factory Service Manual** – это официальное руководство от производителя, в котором подробно описаны устройства системы безопасности, схема расположения датчиков, условия срабатывания и процедуры диагностики. Обычно в таких руководствах есть отдельный раздел для SRS (Supplemental Restraint System), включающий информацию о подушках безопасности и других компонентах.
- Эти руководства можно найти на официальном сайте Infiniti для дилеров и авторизованных сервисов или приобрести у авторизованных поставщиков.

2. **Руководство пользователя (Owner's Manual)**:
- **Infiniti QX80 Owner’s Manual** – руководство пользователя, в котором содержится базовая информация о работе системы подушек безопасности, условиях их срабатывания, рекомендациях по безопасности и мерах предосторожности. Хотя оно не такое подробное, как сервисное руководство, оно дает полезное представление о системе безопасности и правилах эксплуатации.

3. **Техническая документация по стандартам и требованиям безопасности**:
- **SAE (Society of Automotive Engineers) Standards** – стандарты SAE, такие как SAE J211 и J2570, которые охватывают системы пассивной безопасности, включая подушки безопасности. Эти стандарты описывают условия испытаний, параметры датчиков и требования к характеристикам срабатывания.
- **FMVSS (Federal Motor Vehicle Safety Standards)**, в частности, **FMVSS 208** – стандарт Министерства транспорта США, который регулирует требования к системам безопасности автомобилей, включая подушки безопасности, с подробным описанием условий их срабатывания и тестирования.

4. **Научные публикации и статьи**:
- Публикации по автомобилестроению, посвященные активной и пассивной безопасности, например, в журналах **SAE International Journal of Passenger Cars – Mechanical Systems** или **International Journal of Automotive Technology**. Эти статьи могут содержать исследовательские данные и эксперименты, касающиеся подушек безопасности, включая расчеты порогов замедления и анализа столкновений.

5. **Диагностическое оборудование и руководства по диагностике**:
- **Руководства по использованию диагностических сканеров для Infiniti** – для QX80 рекомендуется использовать дилерское оборудование, такое как **Consult III Plus**, которое предоставляет доступ к данным и кодам неисправностей, связанным с системой SRS. В руководствах к сканерам есть информация по тестированию и устранению неисправностей системы подушек безопасности.

Эти источники помогут составить полное представление о работе и диагностике системы подушек безопасности в Infiniti QX80.


\subsection{Конкретные данные срабатывания подушек безопасности}

Конкретные исследовательские данные, включая расчеты порогов замедления и условия анализа столкновений, касающиеся подушек безопасности для Infiniti QX80, обычно относятся к конфиденциальной информации производителя и не публикуются в открытом доступе. Однако можно воспользоваться аналогичными исследованиями, которые касаются систем безопасности автомобилей премиум-класса (включая большие внедорожники), а также данных по тестам, которые проводят независимые исследовательские организации. Вот несколько общих типов исследований и источников, которые могут предоставить полезную информацию по этому вопросу:

 1. **Исследования и публикации по стандартизации порогов срабатывания подушек безопасности**

- **SAE International и FMVSS**: Исследования, выполненные в рамках **SAE International** и **FMVSS 208**, касаются методик тестирования фронтальных и боковых подушек безопасности. В частности, **FMVSS 208** описывает требования к порогам замедления (обычно от 2 до 3 g для фронтальных подушек), которые определяются для срабатывания при фронтальных ударах и боковых столкновениях. Эти данные помогают понять, как производители, такие как Infiniti, рассчитывают и регулируют пороги замедления.
- **SAE J211 и J2570** – стандарты, которые определяют методы сбора данных в автомобильных столкновениях, а также методики установки и калибровки датчиков замедления, что необходимо для понимания порогов срабатывания.

 2. **Результаты краш-тестов и анализ данных IIHS и NHTSA**

- **IIHS (Insurance Institute for Highway Safety)** и **NHTSA (National Highway Traffic Safety Administration)** регулярно проводят независимые краш-тесты, которые предоставляют данные о системах безопасности, включая поведение подушек безопасности при разных уровнях замедления. Хотя результаты тестов не всегда раскрывают точные пороги замедления, можно получить представление о характеристиках безопасности Infiniti QX80, в том числе о срабатывании подушек безопасности при различных сценариях.
- Например, NHTSA проводит фронтальные и боковые испытания на скорости 35 миль в час (около 56 км/ч) с регистрацией данных о замедлении и воздействии на манекены, что помогает понять, при каких условиях система срабатывает.

 3. **Научные исследования и публикации**

- В научных журналах, таких как **SAE International Journal of Passenger Cars – Mechanical Systems** и **International Journal of Crashworthiness**, публикуются исследования, рассматривающие пороги срабатывания подушек безопасности, модели столкновений и данные по замедлениям для различных типов автомобилей, включая крупные внедорожники. Хотя конкретно Infiniti QX80 может не упоминаться, общие выводы и методологии применимы к анализу его системы.
- Например, исследования могут показывать, что для крупногабаритных автомобилей, подобных QX80, порог срабатывания подушек в случае лобового удара находится в диапазоне от **2,0 до 3,5 g** в зависимости от угла удара и условий замедления. Для боковых подушек этот порог может быть ниже, чтобы защитить от боковых ударов быстрее и при меньших замедлениях.

 4. **Данные и рекомендации производителей диагностического оборудования**

- Производители диагностического оборудования, такое как **Consult III Plus** от Nissan/Infiniti, используют документацию, в которой описаны предустановленные параметры и коды ошибок системы SRS (подушки безопасности), включая пороги срабатывания. С помощью такого оборудования можно провести диагностические тесты и симуляции, которые подтверждают или уточняют пороги срабатывания.

 5. **Учебные материалы и исследования по автомобильной безопасности**

- Учебные материалы и пособия по автомобильной безопасности, такие как **Vehicle Crash Mechanics** и **Automotive Accident Reconstruction**, также рассматривают пороги срабатывания и параметры замедления, применимые к большим внедорожникам. В этих изданиях часто приводятся модели расчета критических замедлений и инерционных порогов, которые помогают понять, как рассчитываются данные параметры для конкретных моделей.

Для Infiniti QX80 полезные параметры и конкретные значения можно получить, используя сочетание результатов краш-тестов, диагностики и методик, общих для современных систем безопасности в премиум-сегменте.


\vspace{10mm}

Фронтальные подушки безопасности обычно рассчитаны на срабатывание при "умеренных и сильных" фронтальных или околофронтальных столкновениях, которые определяются как столкновения, эквивалентные удару о твердый неподвижный барьер на скорости от 12 до 22 км/час или выше. (Это равносильно удару о припаркованный автомобиль аналогичного размера на скорости от 25 до 45 км/час или выше).






\subsection{Попытка получить пороги срабатывания из Consalt}


Доступ к точным значениям порогов срабатывания подушек безопасности для Infiniti QX80 в диагностическом оборудовании **Consult III Plus** ограничен и обычно является конфиденциальной информацией производителя. Эти значения в большинстве случаев не отображаются напрямую в программе и доступны только через внутренние коды неисправностей и данные, которые зашифрованы для защиты от несанкционированного вмешательства в систему безопасности.

В Consult III Plus можно получить доступ к следующим типам данных, которые косвенно дают понимание работы подушек безопасности:

1. **Коды неисправностей системы SRS (Supplemental Restraint System)**:
- Сканер Consult III Plus считывает коды ошибок и может указывать на конкретные сбои в датчиках замедления или других элементах системы, которые потенциально влияют на пороги срабатывания.

2. **Проверка и калибровка датчиков замедления**:
- Программа позволяет проверять работу и калибровать датчики ускорения (акселерометры) и другие сенсоры. Хотя точные пороги в g-силах не показываются, можно увидеть диагностические данные, которые показывают текущие замедления и ускорения в реальном времени.

3. **Тестовые параметры на манекенах (Simulation Testing)**:
- В некоторых случаях Consult III Plus может проводить тестовые симуляции, где параметры включения подушек безопасности проверяются при различных моделируемых ускорениях и замедлениях, но без точных значений порогов.

4. **Активные функции диагностики (Active Test Functions)**:
- Consult III Plus поддерживает активные тесты, такие как имитация столкновения или проверка системы SRS на срабатывание, но это не предоставляет численные значения порогов в открытом виде.

Пороговые значения для систем безопасности таких автомобилей, как Infiniti QX80, остаются скрытыми для большинства сервисных центров и доступны только инженерам производителя, так как это связано с безопасностью и защитой от вмешательства.




\textbf{Примечания:}

Современные системы SRS (Supplemental Restraint System) также используют комбинацию данных от датчиков ускорения, угла удара и наличия пассажиров (вес, положение), чтобы точнее оценивать необходимость срабатывания подушек.
Infiniti QX80 оснащен передовыми системами, которые могут использовать адаптивные алгоритмы для контроля порога срабатывания подушек в зависимости от конкретных условий столкновения, таких как скорость и нагрузка на автомобиль.

Для точных данных можно обратиться к официальным сервисным центрам Infiniti или к специализированной технической документации, доступной авторизованным специалистам.


\subsection{Анализ результатов исследования}



Система управления фронтальными подушками безопасности автомобиля
включает в себя следующие компоненты:

 датчики (сенсоры) удара – электромеханические устройства, измеряющие
величину замедления автомобиля вдоль его продольной оси (в проекции на
его продольную ось), и передающие данные модулю управления SRS в виде
электрического сигнала,

 модуль управления SRS – чип (или автономный бортовой компьютер),
принимающий и обрабатывающий сигналы от датчиков удара. В модуле
управления SRS имеется программа (алгоритм), согласно которому
указанный модуль определяет, наступило ли событие удара или другой
физический процесс, который превышает или соответствует порогу
обнаружения. Если порог обнаружения превышен, модуль управления SRS
дает команду на раскрытие подушек безопасности,

 подушки безопасности – устройство, состоящее из надувных мешков и
химических реагентов, наполняющих газом мешки при получении сигнала
от модуля управления SRS.




  
   
   Точный порог замедления для срабатывания подушек безопасности в Infiniti QX80, как и в других автомобилях, обычно не раскрывается производителем, так как это является конфиденциальной информацией. Однако, в автомобилях обычно используются значения замедления в диапазоне **от 2 до 3 g** (g – это ускорение свободного падения, примерно 9.8 м/с²), в зависимости от типа столкновения и угла удара.
   
   Примерные значения могут быть такими:
   
   - **Фронтальные подушки безопасности** срабатывают при фронтальных столкновениях с замедлением примерно 2-3 g, если удар соответствует заданным условиям (угол удара и место столкновения).
   - **Боковые подушки безопасности и шторки** могут иметь порог срабатывания ниже, так как боковые удары обычно требуют более быстрого реагирования системы.
   
   Эти параметры зависят также от наличия дополнительных факторов, таких как скорость автомобиля, вес пассажиров и данные датчиков положения. Системы управления подушками безопасности анализируют показания с датчиков несколько раз в миллисекунду, чтобы принять решение о срабатывании подушек максимально эффективно и безопасно.
   
   
   Да, на Infiniti QX80 подушки безопасности могут сработать, даже если ремни безопасности не пристегнуты. Однако следует учитывать несколько важных моментов:
   
   1. **Эффективность защиты**: Хотя подушки безопасности предназначены для защиты водителя и пассажиров, они работают наиболее эффективно в сочетании с ремнями безопасности. Ремни помогают удерживать человека в оптимальном положении, чтобы подушка могла защитить голову и грудную клетку. При непристегнутом ремне безопасность снижается, и подушка может нанести серьезные травмы, так как тело будет двигаться хаотичнее.
   
   2. **Система отслеживания ремней безопасности**: Многие современные автомобили, включая Infiniti QX80, используют системы, отслеживающие, пристегнуты ли ремни. Эти системы обычно напоминают водителю и пассажирам о необходимости пристегнуться, но на решение о срабатывании подушек это напрямую не влияет.
   
   3. **Фронтальные подушки**: Они сработают при сильном фронтальном ударе, даже если ремень не пристегнут. Однако важно помнить, что при таком сценарии травматизм может быть выше, поскольку тело не фиксируется ремнем, и подушка безопасности не сможет оптимально смягчить удар.
   
   4. **Боковые подушки и шторки безопасности**: Эти подушки также активируются при боковых ударах независимо от того, пристегнуты ремни или нет, поскольку они защищают от боковых воздействий и выброса из автомобиля через окно.
   
   Таким образом, подушки безопасности рассчитаны на срабатывание независимо от ремней, но максимальная безопасность достигается только при их использовании вместе.
   
   
   \subsection{title}
   
   
   % TODO: \usepackage{graphicx} required
   \begin{figure}[!]
   	\centering
   	\includegraphics[width=0.9\linewidth]{screenshot001}
   	\caption{Расстояние HW = 0.63 м}
   	\label{fig:screenshot001}
   \end{figure}
   
   
   
Лобовое стекло автомобиля Infiniti QX80 изготовлено из многослойного ламинированного стекла, состоящего из двух слоев стекла и промежуточного слоя полимера. Такое стекло разработано для повышения безопасности, чтобы при ударе оно не разлеталось на острые осколки, а трескалось, сохраняя целостность.

Согласно исследованиям, линейная скорость удара головой составляет около 80\% от начальной скорости в момент контакта.

Это означает, что при столкновении на скорости 30 км/ч голова пассажира может удариться о лобовое стекло со скоростью около 24 км/ч.

Однако, чтобы разбить лобовое стекло ударом головы, необходима значительная сила. Тесты показывают, что при столкновении на скорости 60 км/ч манекен, имитирующий голову массой 10 кг, может повредить стекло.

Таким образом, удар головой на скорости около 50-60 км/ч может привести к разрушению лобового стекла.

Важно отметить, что использование ремней безопасности и подушек безопасности значительно снижает риск таких травм и повреждений стекла.
   
Источник:   \url{https://dits-servis.by/polezno\-znat/testirovanie-stekla\-na\-prochnost\-tekh/}
   
   
   
   \subsection{title}
   
   
   % TODO: \usepackage{graphicx} required
   \begin{figure}[!]
   	\centering
   	\includegraphics[width=0.7\linewidth]{screenshot003}
   	\caption{Соотношение погрешностей в методе CRASH3}
   	\label{fig:screenshot003}
   \end{figure}
   
   


\блеклый{
Вернемся к краш-тесту №6221 автомобиля Toyota Yaris, упомянутому
выше. В таблице 15 на странице 25 отчета о краш-тесте автомобиля Toyota
Yaris находим, что ширина деформированной зоны составляла L=1.164 м.
Деформированная зона была разделена на 5 участков равной длины,
или L1=L2=L3=L4=L5=1.164/5=0.233 м,
на границах
которых
деформация
в шести измеренных точках составила С1=0.431 м, С2=0.491 м, С3=0.517 м,
С4=0.507 м, С5=0.497 м, С6=0.421 м.
Теперь, зная коэффициенты жесткости передней части автомобиля
Toyota Yaris и его фактическую деформацию в результате удара в жесткий
недеформируемый барьер, решим обратную задачу — используя значения
массы автомобиля Toyota Yaris m=1245 кг и значения коэффициентов
жесткости из нижней строки таблицы на рис. 4.5, вычислим затраты энергии
на деформацию автомобиля Toyota Yaris в краш-тесте. Ниже использована
формула (4.12) в виде суммирования по пяти участкам передней части этого
автомобиля.
}


{\large \[E=\sum_{i=1}^{i}L_{i}\left( \frac{A}{2}(C_{i}+C_{i+1})+\frac{B}{6}(C^{2}_{i}+C_{i}C_{i+1}+C^{2}_{i+1})+G\right), \eqno(13)\]}


тогда скорость автомобиля в момент удара в жесткое неподвижное препятствие в краш-тесте была

{\large \[v=3.6\sqrt\frac{2E}{m}\]}



\subsubsection{установление скоростей автомобилей в момент
	столкновения по их деформациям}
	

	
\блеклый{Из произведенного NHTSA краш-теста № 5061 на фронтальный удар 	аналогичного автомобиля Dodge в жесткий неподвижный барьер на скорости
	56.4 км/ч по формулам (4.5) — (4.7) можно вычислить коэффициенты
	жесткости передней части этого автомобиля. Они составляют A=131318.8 н/м,
	B=916062.7 н/м2, G=9412.4 н.}

Используя выражение (4.12), найдем затраты энергии на деформацию
передней части автомобиля в результате удара в инсценированном ДТП. Для этого просуммируем затраты на деформацию на указанных выше пяти участках его передней части


Энергетически эквивалентная скорость автомобиля, или скорость,
с которой этот автомобиль должен удариться своей передней частью
в жесткий неподвижный барьер, чтобы получить эквивалентные по затратам энергии деформации, с учетом его массы в инсценированном ДТП  составила

{\large 
\[ EES_{D}=3.6\sqrt\frac{2E_{D}}{m_{D}}=\]
}

\input{crash3}

%\input{скорость по деформации 1}


%\input{скорость по деформации 2}

%\input{только рассчеты}
%
%\input{только рассчеты c дополнениями}


%\input{ниссан или инфинити}

\input{обновленное}

\section{Выводы}
	На основании проведенного исследования, с учетом примененных методов и полученных результатов, эксперт пришел к следующим выводам:
\begin{enumerate}
\item \textbf{ }
\item  \textbf{ }
\item  \textbf{ }
\item \textbf{ }
\item  \textbf{ }
\item  \textbf{ }  
\end{enumerate}
    
\vspace{6mm}

\noindent{Эксперт}  \hfill    \rule{45mm}{0.1 mm}     {Фефелов С.Л.}\\[5mm]

\noindent{ Эксперт}  \hfill    \rule{45mm}{0.1 mm}   {Мраморнов А.В.}\\
