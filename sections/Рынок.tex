\subsection{Расчет рыночной стоимости ТС}


\subsubsection{Рыночная стоимость по данным открытых источников}
Рыночная стоимость  транспортного средства, аналогичного исследуемому \тс, по данным специализированных открытых источников на момент повреждения  составляла 
\рынок \,  (\!\числопрописью{\рынок}\!) рублей.\\
Источники:  \url{https://auto.ru/krasnodar/cars/mitsubishi/galant/2002-year/used/}, \url{https://spec.drom.ru}, \url{https://cenamashin.ru/}


% TODO: \usepackage{graphicx} required
\begin{figure}[H]
	\centering
	\includegraphics[width=0.8\linewidth]{../images/foto/рынок}
	\caption{График динамики рыночной стоимости аналогов по данным сайта https://cenamashin.ru/}
	\label{fig:111}
\end{figure}


\subsubsection{Расчет рыночной стоимости ТС}


\par \indent Рыночная стоимость транспортного средства - наиболее вероятная стоимость, по которой транспортное средство может быть отчуждено на открытом рынке в условиях конкуренции, когда стороны сделки действуют разумно, располагая всей необходимой информацией, а на величине стоимости сделки не отражаются какие-либо чрезвычайные обстоятельства [4]. Рыночная стоимость транспортного средства  рассчитывается  для условий конкретных товарных рынков транспортных средств, соответствующих месту государственной регистрации транспортного средства потерпевшего.
\par Определение рыночной стоимости ТС может производится сравнительным и затратным подходом, а именно:  сравнительный анализ продаж (анализ информации о первичном и вторичном рынке АМТС в Российской Федерации);  затратный (с учетом износа АМТС).
\par В оценочной практике наиболее распространенным и предпочтительным является метод прямого сравнения продаж, в соответствии с которым стоимость объекта оценки определяется как средняя цена отобранных аналогов с последующими параметрическими и износными корректировками.   В цены аналогов не вносятся корректировки, если аналоги являются идентичными и равновозрастными объектами по отношению к объекту оценки. 
%\par Общий алгоритм метода прямого сравнения продаж:
%\begin{list}{-}{}
%	\item сбор необходимой информации об объектах;
%	\item  производится выборка цен на объекты;
%\item  проверка полученной выборки на однородность путем расчета коэффициента вариации;
%\item  при значении коэффициента вариации, не превышающем 20\%, определяется средняя цена, которая принимается за стоимость объекта [9]
%\item  при превышении коэффициента вариации значения  20\%  выборка исследуется на наличие выбросов с использованием метода Граббса [9,10].
%\end{list}

% 
%Приоритетным способом определения рыночной стоимость TC
%является метод сравнительного подхода, основанный на объективных справочных данных о ценах на подержанные автомобили  в регионе, где проводится оценка.
   
При определении стоимости транспортного средства сравнительным подходом экспертом-техником были использованы  Интернет-источники сведений о наиболее близких  аналогах исследуемому ТС (таблица \ref{tab:5}). 
 
\begin{longtable}{|p{5mm}|p{85mm}|M{63mm}|}
	\caption[]{\footnotesize {Описание ТС, идентичных оцениваему}} \label{tab:5}\\ 
	\hline
\bf	\text{n/n} &\bf  Описание аналога & \bf URL-адрес преложения  \\ \hline \endhead
		\toprule \centering

\аналог{example-image}{https://auto.drom.ru/}
\аналог{example-image}{https://auto.drom.ru/}
\аналог{example-image}{https://auto.drom.ru/}
\аналог{example-image}{https://auto.drom.ru/}
\аналог{example-image}{https://auto.drom.ru/}



					

%}

%	1. Определяется \overline{X} – среднее значение цены в %промежуточной выборке по формуле:        
 %
%где n – объем выборки (количество элементов в выборке);
%X_i – цена\  i– го объекта-аналога в промежуточной выборке.
%
%	2. Проверяется полученная выборка на однородность путем расчета %коэффициента вариации. Степень однородности выборки оценивается по %коэффициенту вариации (VB ), измеряющему  рассеивание данных %относительно среднего значения:
%V_B=\frac{\sigma_1}{\overline{X}}100%,  
%где \sigma_1- среднеквадратическое (или стандартное) отклонение, %определяемое по формуле:
%\sigma_1=\sqrt{\frac{\sum_{i=1}^{n}{{(X}_i-\overline{X})}^2}{n-1}};
%D= \frac{\sum_{i=1}^{n}{{(X}_i-\overline{X})}^2}{n-1}, где D -%дисперсия выборки.
	%Для малой выборки принято считать, что если VB <15%, то %однородность выборки высокая; 15% <VB<<25%, однородность выборки %средняя,  25% < VB< 33% однородность выборки низкая [9].
	%4. .При значении коэффициента вариации, не превышающем 20%, %определяется средняя цена, которая принимается за стоимость объекта.
	%5.  При превышении коэффициента вариации значения  20%  выборка %исследуется на наличие выбросов с использованием метода Граббса.
	%Окончательно, рассчитанная средняя цена предложения корректируется %с учетом торга в зависимости  от активности рынка – чем выше %спрос/предложение, тем ниже корректировка на торг. Для данного ТС %наиболее вероятное значение корректировки – 10.9% (данные %аналитического агентства «Автостат» %%http://www.autostat.ru/news/20083/)











%%%%%%%%%% среднеквадратичное отклонение и дисперсия выборки
%где $ \sigma_1 $- среднеквадратическое (или стандартное) отклонение, определяемое по формуле:
%\begin{equation}\label{ab}
%\sigma_1=\sqrt{\frac{\sum\limits_{i=1}^{n}{{(X}_i-\overline{X})}^2}{n-1}}
%\end{equation}
%
%\begin{equation}\label{ac}
%D= \frac{\sum\limits_{i=1}^{n}{{(X}_i-\overline{X})}^2}{n-1}
%\end{equation},
%
%
%где $ D $ -дисперсия выборки.
%Для малой Для малой выборки принято считать, что если VB <15\%, то однородность выборки высокая; 15\% <$  V_B  $<<25\%, однородность выборки средняя,  25\% < VB< 33\% однородность выборки низкая 
%
%4. .При значении коэффициента вариации, не превышающем 20\%, определяется средняя цена, которая принимается за стоимость объекта.
%5.  При превышении коэффициента вариации значения  20\%  выборка исследуется на наличие выбросов с использованием метода Граббса.
%6.	Окончательно, рассчитанная средняя цена предложения корректируется с учетом торга в зависимости  от активности рынка – чем выше спрос/предложение, тем ниже корректировка на торг. Для данного ТС наиболее вероятное значение корректировки – 10.9\% (данные аналитического агентства «Автостат» http://www.autostat.ru/news/20083/)
%
%\par Рыночная стоимость ТС  учитывает его фактическое техническое состояние, условия, в которых оно эксплуатировалось.
% Ремонт кузовных составляющих транспортного средства является фактором, влияющим на уменьшение средней цены ТС. В соответствии с Методикой [1], Приложение 3.3 <<Процентный показатель корректирования средней цены КТС в зависимости от условий эксплуатации>>, Таблица 1, для автомобилей со сроком эксплуатации до 7 лет, при восстановлении трех и больше кузовных составных частей уменьшение рыночной стоимости составляет 10 \%. 

%\par При определении стоимости транспортного средства сравнительным подходом экспертом использованы  Интернет-источники сведений об аналогах (Таблица \ref{tab:5}), содержащие краткое описание основных характеристик и технического состояния. Поскольку при определении рыночной стоимости  эксперту-технику данные о техническом состоянии и условиях эксплуатации не заданы и объективно не могут быть  получены при осмотре ТС на момент исследования, то для расчета принимается следующее:
%\begin{list}{-}{}
%	\item техническое состояние ТС соответствует сроку эксплуатации;
%	\item тс не эксплуатировалось в режиме такси или специальных условий эксплуатации;
%	\item фактический пробег ТС на момент повреждения составлял 45 513 км;
%	\item  29.06.2018 г. ТС \тс \, участвовал в ДТП, в результате которого  автомобиль получил механические повреждения задней правой двери, заднего правого порога, заднего правого колеса,  подушки SRS справа.
%\end{list}

%Из открытых банков данных полиции следует, что автомобиль с VIN: \,  \вин\,  как минимум дважды становился участником ДТП.
%Первый раз 29.06.2018  06:40, извещение о ДТП № 030046913, в котором автомобиль получил повреждения задней правой двери, заднего правого порога, заднего правого колеса, подушки SRS справа, Рис. \ref{ris:images/d1} и второй раз 22.05.2019 06:50, извещение о ДТП № 030034947, в котором автомобиль получил повреждения деталей передней левой и задней частей кузова, Рис. \ref{ris:images/d2}.
%%
%\vspace{\baselineskip}
%%
%\begin{figure}[H]\centering
%	\parbox[t]{0.49\textwidth}
%	{\centering
%		\includegraphics[width=.49\textwidth]{images/d1}
%		\caption{\footnotesize {Повреждения в ДТП 29.06.2018 }}
%		\label{ris:images/d1}}
%	\hfil \hfil%раздвигаем боксы по горизонтали 
%	\parbox[t]{0.49\textwidth}
%	{\centering
%		\includegraphics[width=.49\textwidth]{images/d2}
%		\caption{\footnotesize {Повреждения в ДТП 22.05.2019}}
%		\label{ris:images/d2}}
%\end{figure}
%%
%\vspace{\baselineskip}
%
%{\noindent  \footnotesize \tikz \fill [red] (1,0.5) rectangle (0.1,0.1); --{\footnotesize  Вмятины, вырывы, заломы, перекосы, разрывы и другие повреждения с изменением геометрии элементов (деталей) кузова и эксплуатационных характеристик ТС.}\\
%	\tikz \fill [yellow] (1,0.5) rectangle (0.1,0.1); --  {\footnotesize Повреждения колёс (шин), элементов ходовой части, стекол, фар, указателей поворота, стоп-сигналов и других стеклянных элементов (в т.ч. зеркал), а также царапины, сколы, потертости лакокрасочного покрытия или пластиковых конструктивных деталей и другие повреждения без изменения геометрии элементов (деталей) кузова и эксплуатационных характеристик ТС.}\\[1mm]
%	
%\renewcommand\baselinestretch{1.2}\small\normalsize

%\par Вследствие вышеизложенного коэффициент $ C_\text{ДОП}$ принимается равным -10 \%, коэффициенты $ \text{П}_{\text{Э}} $ и $ \text{П}_{\text{П}} $ принимаем равными нулю.

Средняя цена  $C_\text{ср}$  определяется на базе  средней рыночной цены продажи совокупности идентичных ТС на дату оценки: 
\begin{equation}\label{C}
C_\text{ср} =   \frac{ \sum\limits_{i=1}^n{C_i}}{n}
\end{equation}
 $C_\text{ср} =(\sum\limits_{i=1}^n{C_i})/n= (\analoga+\analogb+\analogc+\analogd+\analoge)/\числоаналогов =\resulta $, руб.\\
\noindent где: $ C_i $ - цена предложения к продаже i-го ТС, \\
\indent i - количество предложений, i=\числоаналогов.

Таким образом,\,\, рыночная стоимость $ C $ автомобиля \тс \, составляет \resulta, что  с учетом округления составляет  \res \ рублей.

