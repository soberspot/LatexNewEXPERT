
%
%%%%%%%%%%%%%%%%%%%%%%%%%%%%%%%%%%%%%%%%%%%%%%%%%%%%%%%%%%%%%%%%%%%%%%%%%%%%%%%%%
%\subsection{Технические средства}  %% Список не удалять!!!
%\begin{itemize}
%%
%%%
%%%\item   Диагностический сканер SDconnect   с программным обеспечением Xentry Diagnostics v19.11.3.1
%%
%\item   Линейка масштабная магнитная с цветографической шкалой, 100мм
%%
%%%\item   Рулетка измерительная металлическая, 5м
%%%\item  Универсальный стенд для измерения углов установки колес Hunter Engineering %ProAlign с программным инструментом регулировки схождения колес без блокировки руля %автомобиля WinToe
%\item 	Цифровой фотоаппарат Canon 760D s/n 143032001327 с объективом Canon EF-S 18-135, тип используемой памяти: Transcend,  32Gb
%%
%%\item  Специализированное программное обеспечение для расчёта стоимости  восстановительного ремонта, содержащее нормативы трудоёмкости работ, регламентируемые изготовителями транспортного средства     AudaPadWeb, лицензионное соглашение № AS/\- APW-658  RU-P-409-
%\item  Специализированное программное обеспечение для расчёта стоимости  восстановительного ремонта, содержащее нормативы трудоёмкости работ, регламентируемые изготовителями транспортного средства  SilverDAT myClaim,
%лицензионный договор № 1422 от 05.02.2021 на право использования программы для ЭВМ от  DAT IP-Management und Vertriebs GmbH.
%
%%
%\item  Программа обработки фото-видео изображений ImageJ, разработчик  Wayne Rasband (wa-yne@codon.nih.gov),
%свободная лицензия GPL
%%
%\item  ПЭВМ под управлением операционной системы Windows 10 с установленным набором макрорасширений LaTeX системы компьютерной вёрстки TeX, cвободная лицензия LaTeX Project Public License (LPPL)
%%	
%\end{itemize}
%%%%%%%%%%%%%%%%%%%%%%%%%%%%%%%%%%%%%%%%%%%%%%%%%%%%%%%%%%%%%%%%%%%%%%%%%%%%%%%%%%%%%%%%%%%%%%%%%%%%%%

\subsection{Методы исследования}
\begin{itemize}
\item  Органолептический метод – исследование и оценка качества объектов с помощью органов чувств
\item 	Прямой измерительный метод – путем измерения размеров деталей специальными измерительными приборами
\item Расчётный метод (косвенный измерительный метод) – путём расчётов различных параметров на основе результатов измерений и других данных
\item Экспертный метод (метод экспертной оценки) — совокупности операций по выбору комплекса или единичных характеристик объекта, определению их действительных значений и оценкой экспертом соответствия их установленным требованиям и/или технической информации
%\item Метод натурной реконструкции??
\end{itemize}


%\subsection{Исходные данные}
%
%\begin{enumerate}
%	
%	\item Автомобиль \тс \, VIN \vin \, в повреждённом состоянии.
%	\item Цифровая копия видеозаписи \enquote{улица парковка 3\_23\_06\_2020 06.43.00.mp4}, формата  MPEG-4,  размером 33.4 MiB, длительностью 1 min 59 s, 12.275 FPS.
%	\item Светокопия постановления № 18810223177772659936 от 23.06.2020г. по делу об административном правонарушении, 2 лист.
%	\item Светокопия  решения к делу № 12-541/2020 УИД 23RS0041-01-2020-011330-91, 4 листа.
%%	
%%	
%\end{enumerate}

%%           
%\subsection{Обстоятельства дела}
%%
%%\begin{itemize}
%	%
%\item 
	Согласно постановлению \постановление \, по делу об административном правонарушении, \датадтп \,  в 12 часов 04 минуты \второйводитель,  управляя транспортным средством \тса \, двигалась по ул. Калинина со стороны ул. Передерия в сторону ул. Труда, и на пересечении ул. Калинина - ул. Герцена при развороте по зеленому сигналу светофора не уступил дорогу и создал помеху автомобилю \тс \, регистрационный знак \грз \, под управлением водителя Воткович В.В., двигающемуся во встречном направлении. В результате чего автомобиль \тса \, изменил траекторию движения, сместился вправо и столкнулся с автомобилем \tcb, двигающимся попутно справа. В результате столкновения на автомобиле \тс \, повреждено \повреждения.
	На автомобиле \тса \, видимых повреждений нет, на автомобиле \тсб \, повреждены заднее левое крыло, задняя левая дверь с накладкой, передняя левая дверь в задней части с ручкой.
	
	Виновным в совершении ДТП признана  \второйводитель.
	
	Размер страхового возмещения  владельцу автомобиля \тс \, ООО \enquote{ОПТИМА} составил 69680.80 по первичному обращению в страховую компанию плюс 33 100 рублей в качестве доплаты по претензии.
	
	ООО \enquote{ОПТИМА} не согласилось с размером страховой выплаты, которая по ее мнению должна была составить 226 899.67 (199 699 рублей величина восстановительных расходов плюс 27 200 рублей размер утраты товарной стоимости (УТС)) рулей и обратилось в суд.
	
	

	%
%\end{itemize} 
%
%
\section{Исследование}
%

\subsection{Исследование предоставленных на экспертизу документов}
%

В соответствии с открытыми каталогами запасных частей  автомобиль с VIN \vin \ имеет следующие идентификационные характеристики:

\begin{figure}[H]
	\centering
	\includegraphics[width=0.7\linewidth]{example-image}
	\caption{Информация расшифровки VIN \vin \ по данным кталога \url{https://partsouq.com/}}
	\label{vin}
\end{figure}
%
17.07.2020 г. ООО «Элерон», г. Ростов-на-Дону (Продавец), ООО «Интерлизинг», г. Санкт Петербург (Покупатель) и ИП Мирзаева А.М., Краснодарский край, Белореченский район (Лизингополучатель) заключили договор купли-продажи № КП-23-2674/20 от 17.07.2020 г. Предметом данного договора стал новый автомобиль марка, модель, год выпуска, VIN которого соответствуют записям в свидетельстве о регистрации ТС 9927 № 411797. Цена договора 6 850 000 руб.

Согласно договору купли-продажи № КП-23-2674/20 от 17.07.2020 исследуемый автомобиль  соответствует  автомобилю со следующими характеристиками:\\

\begin{figure}[H]
	\centering
	\includegraphics[width=0.99\linewidth]{1}
	\caption{Характеристики автомобиля.  Приложение №1 к договору купли-продажи № КП-23-2674/20 от 17.07.2020 г.}
	\label{vin}
\end{figure}

Полный перечень комплектации ТС содержится в договоре купли-продажи № КП-23-2674/20 от 17.07.2020.\\
Согласно  записям  выписки из электронной сервисной книжки автомобиля AUDI Q8, VIN  WAUZZZF15LD024287 поставка автомобиля приобретателю произведена   28.07.2020 г. Согласно условиям договора, автомобиль находится на расширенной гарантии до 27.07.2024 г. или достижения  пробег 120 000 км, при условии, что в первые 24 месяца пробег автомобиля не превысит 120 000 км. Гарантия на лако-красочное покрытие автомобиля составляет 36 месяцев с момента передачи автомобиля покупателю без ограничения пробега.\\[3mm]

В результате исследования предоставленных материалов, анализа актов выполненных работ  специалистами установлена  история ремонтов и сервисного обслуживания  транспортного средства VIN \vin. Результаты исследования представлены ниже в таблице \ref{tab:hist}.
%%%%%%%%%%%%%%%%%%%%%%%%%%%%%%%%%%%%%
% История автомобиля
%%%%%%%%%%%%%%%%%%%%%%%%%%%%%%%%%%%%%
%{\small 
%	\begin{longtable}{|p{16mm}|p{12mm}|p{29mm}|G{50mm}|G{41mm}|}
%		\caption[]{\footnotesize {\textbf{История ремонта и сервисного обслуживания по дате и пробегу}}} \label{tab:hist}\\
%		\hline
%		%\rowcolor[HTML]{C0C0C0} 
%		% Заголовки столбцов
%		\textbf{Дата} &\textbf{Пробег, км} &\textbf{№\,Заказ-наряда, накладной}& \textbf{Вид работы}& \textbf{Примечание} \\ \hline \endhead % повторение заголовка 
%		% Строки
%%		22.22.2019 &33\,000  & № 480279303-1 от 03.09.2019& Панель задка  & Замена, окраска \\ \hline
%%		%\rowcolor[HTML]{EFEFEF} 
%%		\Rownum & &n & Боковина задняя левая   & Замена, окраска \\ \hline
%		
%		\ист{arg1}{arg2}{arg3}{arg4}{arg5}
%		\ист{arg1}{arg2}{arg3}{arg4}{arg5}
%		\ист{arg1}{arg2}{arg3}{arg4}{arg5}
%		
%		
%		%%% ..............& 
%		% Обнуляем счетчик строк для следующей таблицы
%\end{longtable}}
%\setcounter{rownum}{0} %сброс счетчика строк в таблице


{\footnotesize \
	\begin{longtable}[h]{m{5mm}|m{14mm}|m{13mm}|m{35mm}|m{50mm}|m{18mm}}
	\caption[]{\footnotesize {\textbf{История ремонта и сервисного обслуживания по дате и пробегу}}} \label{tab:hist} \\ \hline
		\textit{\textbf{n/n}} 
		&\textit{\textbf{Дата}} 
		&\textit{\textbf{Пробег, км}}
		&\textit{\textbf{Документ}} 
		&\textit{\textbf{Содержание}} 
		&\textit{\textbf{Примечание}}\\ \hline \endhead
		

		\Rownum &27.05.2020& -- & Информация VIN & Дата изготовления автомобиля &Данные сайта  partsouq.com\\
		\hline
		\Rownum &17.07.2020& -- & Договор купли продажи автомобиля №КП-23-2674/20 от 17.07.2020г..& Покупка нового  автомобиля ООО «Интерлизинг» для ИП Мирзаева А.М.& -- \\
		\hline
		
		\Rownum &27.07.2020& 0 & 420198409-1 & Предпродажная подготовка & "Элерон" \\
		\hline
			\Rownum &28.07.2020& 0 & Выполнение  условий договора & Поставка ТС приобретателю & "Элерон" \\
		\hline
		
		
			\Rownum &20.11.2020& 14 258 & 420201825-1 & Замена масла, сажевого фильтра & "Элерон" \\
		\hline
		
		
		\Rownum &13.06.2021& 19 379& Акт выполненных работ №430092116-/ 430092116-1 от 13.09.2021 г.  & Причина обращения со слов клиента: диагностика.
		Выполненные работы: ведомый поиск неисправностей; диагностика ходовой части; геометрия передней и задней частей автомобиля проверить.
		Рекомендации сервиса: выполнить проверку развал-схождения; выполнить чистку дроссельных заслонок; произвести антибактериальную обработку климата; заменить лобовое стекло; приклеить плёнку на правой передней арке колеса и решетке радиатора & ООО «Формула-АЦК2»\\
		\hline
		\Rownum 	&13.09.2021& 29 203& Акт выполненных работ №430096159-1 от 13.09.2021 г. & Причина обращения со слов клиента: ошибка по замку задней левой двери.
		Выполненные работы: ведомый поиск неисправностей; обновление ПО блока управления коробки передач.
		Рекомендации сервиса: при включении задней скорости периодически не работает ра заднего вида - в момент нахождения автомобиля в сервисе неисправность не проявляется; & ООО «Формула-АЦК2». Гарантийный ремонт\\
		\hline
		\Rownum 	&19.09.2021& 29 203& Акт выполненных работ №430096060-/ 430096060-1 от 19.09.2021 г.  & Причина обращения со слов клиента: ТО-2.
		Выполненные работы: замена масла моторного НХ8 Syn 5W-30, масляного фильтра 059198405, топливного фильтра 4М0127434Н, воздушного фильтра 4М0133843С, салонного фильтра 
		PF2357, тормозных колодок 4M0698151K, кабеля предупредительных сигналов 4M0615121AB.
		Рекомендации сервиса: выполнить проверку развал-схождения; выполнить чистку дроссельных заслонок; произвести антибактериальную обработку климата; заменить лобовое стекло; приклеить плёнку на правой передней арке колеса и решетке радиатора& ООО «Формула-АЦК2».  ТО-2 \\
		\hline
		\Rownum 	&04.10.2021& 31 040& Акт выполненных работ №430097287-/ 430097287-1 от 04.10.2021 г.  & Причина обращения со слов клиента: замена щёток стеклоочистителей.
		Выполненные работы: замена щёток стеклоочистителя 4М8955425Е и 4М899800; проверка наружных световых приборов, проверка звукового сигнала; проверка уровня тормозной жидкости, регулировка форсунок омывателя ветрового стекла; проверка давления в шинах, диагностика ходовой части.
		Рекомендации сервиса: выполнить проверку развал-схождения; выполнить чистку дроссельных заслонок; произвести антибактериальную обработку климата; заменить лобовое стекло; приклеить плёнку на правой передней арке колеса и решетке радиатора& ООО «Формула-АЦК2»\\
		\hline
		\Rownum 	&28.05.2022 & 42 616 & Акт выполненных работ №430105673-1 от 28.05.2022 г.  & Причина обращения со слов клиента: высветилась неисправность тормозной системы, пропали тормоза. Автомобиль на эвакуатора.
		Выполненные работы: ведомый поиск неисправностей; замена прокладки 059129717S, уплотнительного кольца 059129796J, вакуумной трубки 059131057Q, кронштейна 059131338M, вакуумного шланга 059131375АP, вакуумной трубки 059131377АD, прокладки 059131599R, вакуумной трубки 059131995E, прокладки 059145774A, электромагнитного клапана 059906283C, пневматического клапана 059906627P, шланга ОЖ с быстрораз 4M0122449AC, воздушного фильтра 4M0133837S, гидроблока ABS 4M6614517BBBEF; переоборудование впуска вакуумного насоса на впуск со шлангом для чистого воздуха
		Рекомендации сервиса: нет &ООО «Формула-АЦК2». Гарантийный ремонт\\	\hline
		\Rownum &28.05.2022& 42 616 & Акт выполненных работ № 430107020-/430107020-1 от 28.05.2022 г. & Причина обращения со слов клиента: ТО, переклеить плёнку переднего правого стекла, заменить масло и фильтр АКП (масло и фильтр свои оригинал).
		Выполненные работы: замена масла 
		LGENSPECVN5W3O и фильтра АКПП оЕ671 3, замена салонного фильтра PF2357, замена плёнки AIR 90CL HPR (30,5*1,525) 46,5м2 Llumar/ с логотипом на лайнере, проверка наружных световых приборов, проверка звукового сигнала; проверка уровня масла в двигателе,  проверка уровня тормозной жидкости, регулировка форсунок омывателя ветрового стекла; проверка давления в шинах, диагностика ходовой части
		Рекомендации сервиса: нет&  ООО «Формула-АЦК2».   ТО \\
		\hline
		\Rownum 	&29.08.2022& 53 677& Акт выполненных работ № 430107437-1 от 29.08.2022 г. & Причина обращения со слов клиента: -
		Выполненные работы: ведомый поиск неисправностей, замена датчика числе оборотов ABS 01500000
		Рекомендации сервиса: нет& ООО «Формула-АЦК2». Гарантийный ремонт \\
		\hline
		\Rownum 	&29.08.2022& 53 677& Акт выполненных работ №430110328-1 от 29.08.2022 г. & Причина обращения со слов клиента: -
		Выполненные работы: ведомый поиск неисправностей, замена датчика перепада давления 26751939.
		Рекомендации сервиса: нет& ООО «Формула-АЦК2». Гарантийный ремонт \\
		\hline
		\Rownum 	& 30.09.2022 & 53 677& Акт выполненных работ №430111231-1 от 30.09.2022 г. & Причина обращения со слов клиента: -
		Выполненные работы: замена замков двери 58171950 и 58171950.
		Рекомендации сервиса: нет& ООО «Формула-АЦК2». Гарантийный ремонт  \\
		\hline
		
		
\end{longtable}}\setcounter{rownum}{0}

\noindent Анализ истории по характерным событиям и признакам позволяет выделить следующие значимые составляющие:
\begin{enumerate}
\item  
На момент настоящего исследования автомобиль находится на гарантии.
\item  
Сведения о нарушении графика ТО отсутствуют.
\item  
За весь период эксплуатации на автомобиле были выявлены неисправности системы замков  задних дверей  58171950 и 58171950, датчика перепада давления 26751939, датчика ABS переднего левого колеса, трубки с обратным клапаном 059131057Q,  гидроблока ABS 4M6614517BBBEF.
\item 
Причинами  всех неисправностей были признаны соответствующие производственные дефекты.  
\item 
Согласно предоставленных документов, все выявленные неисправности устранены по гарантии. 
\end{enumerate}


%%%%%%%%%%%%%%%%%%%%%%%%%%%%%%%%%%%%%%
% 
\subsection{Исследование транспортного средства}

Осмотр и диагностическое исследование  автомобиля AUDI Q8 г.р.з. А179ОК193, VIN WAUZZZF15LD024287  производились  по адресу: г. Краснодар, ул. Аэропортовская-4/а, на территории СТОА ООО «Формула-АЦК2» 03.04.2023 г. с 09 час. 15 мин. до 13 час 25 мин.  в условиях естественного и искусственного освещения с использованием производственных мощностей СТОА ООО «Формула-АЦК2» в присутствии инженера по гарантии СТОА ООО «Формула-АЦК2» Колесникова Александра Витальевича.\\
 05.04.2023 с 12:15 до 12:45 на территории  ООО «Формула-АЦК2» в присутствии Колесникова А.В. произведено дополнительное  диагностическое исследование ТС.

\subsubsection{Осмотр транспортного средства}

\дварядом{example-image}{Исследуемый автомобиль  \тс \, \vin}{1_2}{Исследуемый автомобиль  \тс \, \vin}
\дварядом{вин}{VIN под ветровым стеклом}{вин3}{VIN на передней стойке}

Осмотр автомобиля производился  органолептическим методом. В процессе осмотра выполнялась фото и видео съемка объекта исследования цифровой фотокамерой.\\ 
Маркировочные обозначения, нанесенные на кузове ТС соответствуют записям регистрационных документов. Кузов автомобиля окрашен двухслойной эмалью черного цвета, боковые элементы, включая молдинги,  капот имеют дополнительное защитное покрытие прозрачной виниловой пленкой. Автомобиль оснащен легкосплавными пятнадцатиспицевыми колесами черного цвета Audi Sport и шинами  Continental SportConntact 6 размерностью  285х40 R22 Y XL. Тягово-сцепное устройство на автомобиле отсутствует. Внешне автомобиль соответствует товарным образцам автомобилей Audi Q8 2020 модельного года.    Показание одометра (пробег)  53 689 км.  Бортовое время и дата ТС  не соответствует текущему. Настройки даты и времени ТС в начале осмотра: 16:34  19.11.2022. \\
Боковые двери, дверь задка, капот, крышка лучка заливной горловины топливного бака автомобиля  не опечатаны. Автомобиль комплектный. Уровни технических жидкостей находятся в рекомендуемых диапазонах.  Тормозная жидкость прозрачная, светлая, без видимых загрязнений. Охлаждающая жидкость розового цвета, светлая, прозрачная, без видимых загрязнений. Моторное масло темного цвета, по масломерному щупу уровень несколько ниже среднего значения. Каплепадение \cite{техрегтамсоюз:тр}, следы  подтекания рабочих жидкостей  или запотевания отсутствуют. Внешнее состояние деталей, узлов и агрегатов подкапотного пространства соответствует исправному, наружные поверхности покрыты  слоем пыли. Загрязнение воздушного фильтра незначительное, замена воздушного фильтра не требуется. \\
  Автомобиль   имеет  наружные повреждения  элементов, по внешним признакам соответствующие эксплуатационным повреждениям (см. Приложение: фототаблица):\\
 - капот – имеет незначительную вмятину площадью 1/4 дм2  с повреждением лакокрасочного покрытия (ЛКП) на передней кромке. \\
 - бампер передний имеет следы остаточной деформации  в нижней передней левой части, выраженные в   растрескивании на пощади 1 дм2 слоя ЛКП под защитной пленкой; в правой нижней части  утрачено ЛКП на площади 1/4 дм2. \\
 - решетка левая бампера переднего  не имеет  штатной фиксации  в проеме бампера, вероятно  повреждение креплений решетки;\\
  - решетка правая бампера переднего  имеет глубокие  задиры наружной поверхности  на площади 0,2 дм2. \\
 - решетка радиатора незначительно деформирована с повреждением ЛКП в месте, смежном с местом повреждения капота;\\
 - крыло переднее левое имеет плавную несложную деформацию в виде вмятины  под надписью «S Line» на площади поверхности 1,3 дм2;\\ 
 - облицовка арки боковины задней правой имеет повреждение защитной пленки.\\
В процессе осмотра салона автомобиля повреждения облицовки дверей, панели крыши, боковых стоек, панели приборов не выявлено. Облицовка передних и задних сидений кожаная, повреждения облицовки сидений отсутствует. При осмотре коврика салона, справа и слева на полу задних пассажиров имеются загрязненные участки веществом темного цвета. \\ Облицовка рулевого колеса без повреждений. Механизмы электрической регулировки передних сидений и рулевого колеса исправны.\\
На включение зажигания автомобиль реагирует штатно. Органы управления исправны, неисправности мультимедийной системы отсутствуют.\\ Проверка сопряжения автомобиля с мобильным телефоном не проводилась.\\
При  осмотре установленного на подъемнике автомобиля выявлены повреждения облицовки пола наружной правой в виде разрыва длиной 10 см в передней части;\\  - деформация теплозащитного экрана тоннеля пола левого;\\ - деформация  поперечины пола в виде вмятины диаметром 3см в правой передней части;\\
 - деформации облицовки  правой  топливного бака в виде разрыва материала детали передней части;\\
- нарушение целостности защитного кожуха компрессора пневмоподвески\\ - деформация ребра редуктора заднего моста. \\
Отдельно необходимо отметить наличие признаков остаточной деформации панели задка в задней нижней  левой части.\\
 Перечисленные повреждения, по совокупности морфологических  признаков, получены в процессе эксплуатации автомобиля и не имеют  причинной, следственной связи с неисправностями, указанными в предоставленных актах выполненных работ. \\
 Отдельное внимание при осмотре автомобиля было уделено практической проверке работоспособности стеклоподъемников и доводчиков дверных замков. На момент настоящего исследования механизмы открывания/закрывания всех дверей исправны и демонстрируют штатную работу.\\
   Таким образом, при проведении  осмотра автомобиля были установлены незначительные механические повреждения наружных элементов  транспортного средства, деформация нижнего  ребра охлаждения корпуса редуктора заднего моста, незначительная деформация, не изменяющая базовую геометрию детали, нижней поперечной балки,   загрязнение салонного коврика в районе заднего ряда пассажирских сидений. Все имеющиеся повреждения  получены в процессе эксплуатации транспортного средства.\\
   05.04.2023 в 08:30  проведена проверка  по открытой информационной базе ГИБДД об участии ТС с VIN  WAUZZZF15LD024287 в зарегистрированных дорожно-транспортных происшествиях, \url{https://xn--90adear.xn--p1ai/check/auto#WAUZZZF15LD024287}. Установлено, что данный автомобиль был участником одного ДТП. 08.03.2021 в 01:45   совершен наезд на препятствие. 
   \begin{figure}[H]
   	\centering
   	\includegraphics[width=0.7\linewidth]{дтп1}
%   	\label{vin}
   \end{figure}
\begin{figure}[H]
	\centering
	\includegraphics[width=0.7\linewidth]{дтп2}
%	\label{vin}
\end{figure}
\begin{figure}[H]
	\centering
	\includegraphics[width=0.7\linewidth]{дтп3}
	\caption{Данные сайта \url{https://гибдд.рф}}
	\label{гибдд}
\end{figure}  
   
 По завершении наружного осмотра автомобиля, с целью разрешения вопросов настоящего исследования, проведена диагностика ТС VIN WAUZZZF15LD024287.  Диагностирование автомобиля проводилось программно-аппаратными средствами   официального дилера.  Согласно протокола диагностики от 03.04.2023  12:22 марка, тип, модельный год, модификация, буквенное обозначение двигателя, автоматически определенный VIN полностью соответствуют наружной маркировке автомобиля.\\
 Первоначально было сделано два протокола диагностики.  03.04.2023  12:22 и  03.04.2023  12:25.  По причине недостаточного уровня  напряжения АКБ автомобиля и неустойчивой связи WI-FI протокол от 03.04.2023 12:22 к рассмотрению не принимается.  Для корректного прочтения  кодов DTC автомобиль был подключен к зарядному модулю АКБ, случайно возникшие коды DTC стерты.  \\
 Следующий протокол диагностики от 03.04.2023  12:25 коды ошибок не содержит. Система самодиагностики  указывает на исправное состояние ТС.\\
 % Для дальнейшей проверки автомобиля были проведены дорожные испытания автомобиля. Перед началом движения  проведена проверка исправности световых приборов, возможности регулировки рулевого колеса, сиденья водителя и пассажира, свеклоподъемники боковых дверей, работа акустической системы, работа двигателя в режиме холостого хода. В статическом состоянии неисправности автомобиля не обнаружены. Ввиду отсутствия сим-карты в мультимедийной системе, исправность функционала, зависящего от наличия сим-карты не проверялась.  Движение автомобиля осуществлялось по территории сервисного центра без выезда на дороги общего пользования. Полотно дороги сухое, асфальтобетон, температура окружающего воздуха +18\град С.  Применялся  равномерный режим движения на постоянной скорости, переменный режим движения с чередованием ускорения и торможения, проезд неровностей в виде стандартных "лежачих полицейских" попеременно левыми и правыми колесами. Нефункциональные шумы, перебои в работе двигателя, подвески, коробки перемены передач и других системах автомобиля отсутствуют.   Торможение автомобиля штатное, многократное торможение происходит с одинаковым усилием, неисправности элементов тормозной системы отсутствуют. Работа АКПП штатная и соответствует работе исправного агрегата.\\   
% По результатам  дорожных испытаний специалисты приходят к заключению, что исследуемый автомобиль по состоянию на 03.04.2023 года находится в исправном техническом состоянии.\\
% После дорожных испытаний 
Затем  электронные системы автомобиля диагностировались еще дважды: 03.04.2023  12:49 и  05.04.2023 в 12:26. 
Согласно протокола диагностики от 03.04.2023  12:49  автомобиль технически исправен. \\
Согласно протокола диагностики от  05.04.2023  12:26  автомобиль технически исправен.\\

Таким образом, из совокупности результатов наружного осмотра,
% , 
% дорожных испытаний и 
 диагностического исследования электронных систем автомобиля, следует, что транспортное средство \тс \, VIN \vin \, по состоянию на 05.04.2023 года находится в исправном техническом состоянии. Передний бампер, капот, крыло переднее левое, решетка радиатора, решетка левая и правая бампера переднего, арка боковины задней правой, салонный коврик в районе заднего ряда сидений, ребро охлаждения редуктора заднего моста, детали защиты днища автомобиля имеют несущественные повреждения эксплуатационного характера, не оказывающие влияние на общее техническое состояние автомобиля. \\
% 05.04.2023 в 12:26  исследуемый автомобиль  продиагностирован  повторно. Так же, как и 03.04.2023 диагностика подтвердила отсутствие неисправностей автомобиля \тс VIN \vin.
  

%\subsubsection{Исследование наличия, характера и объёма технических повреждений}
%
%  Наличие, характер и объем технических повреждений транспортного средства \tc\, регистрационный знак \grz, исследованы в присутствии заинтересованных лиц,  зафиксированы в акте осмотра № \NomerDoc\,  (Приложение, <<Акт осмотра>> ),  и фотоматериалах (Приложение, <<Фототаблица>>) по принадлежности. Планируемые (предполагаемые) ремонтные воздействия для восстановления повреждённого  транспортного средства назначены экспертом-техником с учётом особенностей конструкции и рекомендаций изготовителя  транспортного средства, укрупненных показателей трудозатрат по кузовному ремонту и устранению перекосов проёмов и кузова легковых автомобилей иностранных производителей, приложение 3 к приложению к Положению Банка России от 19 сентября 2014 года № 432-П и приведены ниже в таблице \ref{tab:5}.

%Повреждения транспортного средства \тс \, регистрационный знак \грз\, определены экспертом по материалам гражданского дела \delonum\, и представлены ниже в таблице \ref{tab:6}

%\pagebreak
\begin{longtable}{G{3mm}|M{90mm}|G{60mm}}
	\caption[]{\footnotesize {Эксплуатационные дефекты, установленные при  осмотре ТС}} 
	\label{tab:6}\\ 
	\hline 
	\hline  \toprule 
	\bf  {\footnotesize  n/n}  &\bf {\small Наименование  детали с описанием повреждения} & \bf {\small Изображение} \\   \hline\hline  \toprule \endhead 
	%%%%___________________________________________________________________    
	%\пов{Наименование детали- описание повреждения }{example-image}
\пов{Капот - сложная  деформация панели и каркаса  детали на площади более 80\% поверхности}{example-image}
%\пов{}{example-image}
%\пов{}{example-image}
%\пов{}{example-image}
%\пов{}{example-image}
%\пов{}{example-image}
%\пов{}{example-image}
%\пов{}{example-image}
%\пов{}{example-image}
%\пов{}{example-image}
\end{longtable}\setcounter{rownum}{0}
	
	%\subsubsection{Определение стоимости восстановительных расходов}

%		\subparagraph{} Протокол диагностики(длинный) 03.04.2023  12:49  
		
		%\input {sections/рынокОСАГО}
%	\input {sections/рынокНЕосаго}
		%\input {sections/утсОСАГО}
		%\input {sections/годныеОСАГО}

%
\pagebreak
\subsection{Анализ результатов исследования}

%\повопросу{1. Имеются ли в автомобиле \тс \, VIN \vin\, недостатки?}

Проведенное исследование позволяет утверждать, что на момент настоящего исследования  автомобиль \тс \, VIN \vin находится в исправном техническом состоянии. При этом  на автомобиле \тс \, VIN \vin\  выявлены незначительные механические повреждения наружных элементов кузова  транспортного средства, деформация нижнего  ребра охлаждения корпуса редуктора заднего моста, незначительная деформация не изменяющая базовую геометрию детали нижней поперечины пола,   загрязнение салонного коврика в районе заднего ряда пассажирских сидений, повреждение облицовок днища.  Все имеющиеся повреждения  получены в процессе эксплуатации транспортного средства, легко устранимы без существенных материальных и временных затрат  и не оказывают влияние на технические характеристики автомобиля.

%\повопросу{2. Если имеются, то какие именно и в чем они выражаются?}
%
%\begin{table}[h]
%	\caption{Таблица зарегистрированных ошибок.}
%	\label{table:ошибки}
%	\begin{tabular}{c|m{45mm}|m{35mm}|m{63mm}}\hline
%		\textbf{  n/n} & \textbf{Код ошибки} & \textbf{Повторяемость} & \textbf{Описание} \\
%		\hline 
%		\Rownum & 0017 & Спорадическая & 16777089 U112100 Шина данных, нет сообщения (00001000 пассивн./спорадич.) \\
%		\hline
%	\end{tabular}
%\end{table}\setcounter{rownum}{0}


\section{Выводы}

\begin{enumerate}
	\item \textbf{Автомобиль \тс \, VIN \vin находится в исправном техническом состоянии. При этом отдельные внешние элементы конструкции имеют незначительные повреждения, полученные в процессе эксплуатации транспортного средства,  не оказывающие влияние на его технические характеристики.}\\[3mm]
	\item \textbf{Автомобиль \тс \, VIN \vin \, на момент настоящего исследования имеет незначительные механические  повреждения  следующих деталей:\\
	капот – имеет незначительную вмятину площадью 1/4 дм2  с повреждением лакокрасочного покрытия (ЛКП) на передней кромке;\\
	бампер передний имеет следы остаточной деформации  в нижней передней левой части, выраженные в   растрескивании на пощади 1 дм2 слоя ЛКП под защитной пленкой; в правой нижней части  утрачено ЛКП на площади 1/4 дм2;\\ 
	решетка левая бампера переднего  не имеет  штатной фиксации  в проеме бампера, вероятно  повреждение креплений решетки;\\    
	решетка правая бампера переднего  имеет глубокие  задиры наружной поверхности  на площади 0,2 дм2;\\  
	решетка радиатора незначительно деформирована с повреждением ЛКП в месте смежном с местом повреждения капота;  
	крыло переднее левое имеет плавную несложную деформацию в виде вмятины  под надписью «S Line» на площади поверхности 1,3 дм2й;\\
	облицовка арки боковины задней правой имеет повреждение защитной пленки ;\\   
	коврик салона = справа и слева  на полу задних пассажиров имеются загрязненные участки веществом темного цвета; \\  
	повреждения облицовки пола наружной правой в виде разрыва длиной 10 см в передней части;\\  
	деформация теплозащитного экрана тоннеля пола левого;\\ 
	деформация  поперечины пола в виде вмятины диаметром 3см в правой передней части;\\  
	деформации облицовки  правой  топливного бака в виде разрыва материала детали передней части;\\   
	нарушение целостности защитного кожуха компрессора пневмоподвески;\\  
	деформация ребра охлаждения корпуса редуктора заднего моста;\\  
	деформации панели задка в задней нижней  левой части.   }\\[3mm]

	\vspace{5mm}
	
\end{enumerate}
    
\vspace{10mm}

\noindent{ Эксперт}  \hfill    \rule{45mm}{0.1 mm}   {Мраморнов А.В.}\\[15mm]

\noindent{Эксперт}  \hfill    \rule{45mm}{0.1 mm}     {Фефелов С.Л.}\\
\vspace{7mm}
\relax

\vspace{15mm}

\relax
\noindent Приложение к заключению:\\
\textit{
	%	Приложение № 1. Расшифровка модельных опций ТС \тс \\
%	Приложение № \Rownum. Акт осмотра ТС \тс\\
	Приложение № \Rownum. Фототаблица повреждений ТС\\
%	Приложение № \Rownum. Калькуляция стоимости восстановительных расходов ТС \тс\\
	%	Приложение № \Rownum. Цифровые копии регистрационных документов ТС\\
	%	Приложение № \Rownum. Цифровая копия постановления по делу об административном правонарушении дорожно-транспортном происшествии\\
	Приложение № \Rownum. Правоустанавливающие документы \\
}


%\includepdf[pages=-]{myfile.pdf}
%\includepdf[pages=-]{calc.pdf}

%\includepdf[pages=-]{myfile.pdf}
%\includepdf[pages=-]{calc.pdf}