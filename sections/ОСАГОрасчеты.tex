%%%
%%%
\def\итог{380950}
\def\итогизнос{215550}	

\par Из предоставленных материалов   экспертом-техником установлена следующая общая информация об автомобиле, имеющая значение для дачи заключения по поставленному вопросу:
\parbox[]{10cm}{}
\begin{itemize}
	\item[ ] 
	\begin{description}
		\item[Марка, модель] -- \тс
		%	\item[Заводское обозначение модели] --  
		\item[VIN] -- \vin
		\item[Шасси] -- \шасси
		\item[Год выпуска] -- \год
		\item[Цвет ЛКП] -- \цвет
		\item[Пробег] --  \пробег\, км, считан с одометра
		\item[Дата начала эксплуатации] -- \началоэкспл
		%	\item[Двигатель] -- \двигатель
		%	\item[Объем двигателя] -- 1328 $ \text{см}^3 $
		%	\item[Свидетельство о регистрации] -- \свид
		%	\item[ПТС] --\птс
	\end{description}
\end{itemize}
%
%\subparagraph*{} Идентификационный код автомобиля (VIN)  \vin \, содержит следующую информацию о транспортном средстве, имеющую значение для 	дачи заключения (Рис. \ref{fig:vin} ):\\[3mm]
%%	
%	\noindent\parbox[]{10cm}{
%		\begin{itemize}
%			\item[ ] 
%			\begin{description}
%				\item[Дата изготовления] \hfill \датаизготовления
%				\item[Расположенние руля] \hfill Left
%				\item[Двигатель] \hfill \двигатель
%			%	\item[Объем двигателя] \hfill 1328 $ \text{см}^3 $
%				\item[КПП] \hfill МКПП
%				\item[Тип кузова] \hfill  \типкузова
%				\item[Количество дверей] \hfill 5 
%				%	\item[VDS] --
%					
%			\end{description}
%	\end{itemize}}\\

%\vspace{3mm}

%

\vspace{3mm}

\subsubsection{Осмотр транспортного средства}


\subsubsection{Установлении причин возникновения повреждений транспортного средства}



\subsubsection{Исследование наличия, характера и объёма технических повреждений}


%\pagebreak

\begin{longtable}{G{3mm}|M{120mm}|G{30mm}}
	\caption[]{\footnotesize {Повреждения автомобиля, установленные при его осмотре}} 
	\label{tab:5}\\ 
	\hline 
	\hline  \toprule 
	\bf  {\footnotesize  n/n}  &\bf {\small Наименование  детали с описанием повреждения} & \bf {\small Изображение} \\   \hline\hline  \toprule \endhead 
	%%%%___________________________________________________________________    
	%\пов{Наименование детали- описание повреждения }{example-image}
\пов{Капот - сложная  деформация панели и каркаса  детали на площади более 80\% поверхности}{example-image}
%\пов{}{example-image}
%\пов{}{example-image}
%\пов{}{example-image}
%\пов{}{example-image}
%\пов{}{example-image}
%\пов{}{example-image}
%\пов{}{example-image}
%\пов{}{example-image}
%\пов{}{example-image}
\end{longtable}\setcounter{rownum}{0}
	
	
	
	\subsubsection{Определение стоимости восстановительных расходов}
	
	В соответствии с существующей экспертной методикой размер расходов на восстановительный ремонт определяется исходя из стоимости ремонтных работ (работ по восстановлению, в том числе окраске, контролю, диагностике и регулировке, сопутствующих работ), стоимости используемых в процессе восстановления транспортного средства деталей (узлов, агрегатов) и материалов взамен повреждённых. Расчёт размера расходов (в рублях) на восстановительный ремонт производится по формуле (\ref{eq:cr}): 
	
	\begin{equation}\label{eq:cr}
		C_{\text{вр}}  =\sum{C_{ip}}= \sum\left({T_{ij}}\cdot {C_{i\text{нч}}}\right) + \sum{C_{ip^{\text{\,\,\,руб}}}} , \,\,\,\text{где:} 
	\end{equation}
	%\vspace{2mm}
	\begin{itemize}
		\item[ ]$ C_{ip} $ -- стоимость работ i-го вида: $C_\text {зам} $, $ C_\text{восст} $, $ C_\text{рег} $, $C_\text{контр} $, $ C_\text{антикор} $, $ C_\text{зч} $, $ C_\text{ом} $,$ C_\text{соп} $, $ C_\text{вм} $, руб;
		\item[ ]$ T_{ij} $ -- трудоёмкость j-й операции(комплекса) по i-му виду работ, руб;
		\item[ ]$ C_{i\text{нч}} $ -- стоимость нормо-часа по i-му виду работ, руб;
		\item[ ]$ C_{ip^\text{\,\,руб}} $ -- стоимость работ $ C_{ip} $, принятая непосредственно в денежном выражении, руб.
	\end{itemize}
	
	\par При определении стоимости восстановительного ремонта АМТС с учётом износа под износом следует понимать количественную меру физического старения АМТС и его элементов, достигнутого в результате эксплуатации, т.е. эксплуатационный износ.
	%
	Расчёт износа производится в  соответствии с Положением Банка России от «19» сентября 2014 года № 432-П «О единой методике определения размера расходов на восстановительный ремонт в отношении повреждённого транспортного средства» [3].
	Износ комплектующих изделий (деталей, узлов, агрегатов) рассчитывается по следующей формуле (\ref{eq:I}):
	%
	%
	%
	\begin{equation}\label{eq:I}
		\text{И}_{\text{ки}} 
		= 100\cdot\left( 1-e^ {-\left( \Delta_{T} \cdot T_{\text{КИ}} + \Delta_{L} \cdot L_{\text{КИ}} \right)}\right), \,\,\,\,\text{где:}   
	\end{equation}
	%
	\begin{itemize}
		\item[ ]$ \text{И}_{\text{ки}} $ -- износ комплектующего изделия (детали, узла, агрегата) (процентов); 
		\item[ ]$ e $ -- основание натуральных логарифмов (e =  2,72);
		\item[ ]$ \Delta_{T}$ --  срок эксплуатации комплектующего изделия (детали, узла, агрегата) (лет);
		\item[ ]$ T_{\text{КИ}} $ -- стоимость работ $ C_{ip} $, принятая непосредственно в денежном выражении, руб;
		\item[ ]$ \Delta_{L} $ -- коэффициент, учитывающий влияние на износ комплектующего (детали, узла, агрегата) величины пробега транспортного средства с этим комплектующим изделием;
		\item[ ]$ L_{\text{КИ}} $ -- пробег транспортного средства на дату дорожно-транспортного происшествия (тысяч километров).  
	\end{itemize}
	\vspace{5mm}
	\par Значения коэффициентов $ \Delta_{T}$  и $ \Delta_{L} $  для различных категорий и марок транспортных средств приведены в п. 5. сп. лит~[3]. При этом, на комплектующие изделия (детали, узлы, агрегаты), которые находятся в заведомо худшем состоянии, чем общее состояние транспортного средства в целом, и его основные части, вследствие влияния факторов, не учтённых при расчёте износа (например, проведение ремонта с нарушением технологии, не устранение значительных повреждений лакокрасочного покрытия), может быть начислен дополнительный индивидуальный износ. 
	
	Износ шины транспортного средства рассчитывается по следующей формуле (\ref{eq:sh}):
	\begin{equation}\label{eq:sh}
		\text{И}_{\text{ш}} = \frac{\text{Н}_{\text{н}}-\text{Н}_{\text{ф}}}{\text{Н}_{\text{н}}-\text{Н}_{\text{доп}}} \cdot{100}\%,  \,\,\,\,\text{где:} 
	\end{equation}
	%
	\begin{itemize}
		\item[ ] $ \text{И}_{\text{ш}} $ -- износ шины, \%;
		\item[ ] $ \text{Н}_{\text{н}} $ -- высота рисунка протектора новой шины, мм;
		\item[ ] $\text{Н}_{\text{ф}} $ -- фактическая высота рисунка протектора шины, мм;
		\item[ ] $ \text{Н}_{\text{доп}} $ --минимально допустимая высота рисунка протектора шины в соответствии с требованиями законодательства Российской Федерации, мм.
	\end{itemize}
	%
	\vspace{5mm}
	\relax
	%\renewcommand\baselinestretch{1}\small\normalsize
	%
	Износ шины дополнительно увеличивается для шин с возрастом от 3 до 5 лет - на 15 процентов, свыше 5 лет - на 25 процентов.
	
	
	\subsubsection{Данные для расчёта}
	
	\noindent Объект экспертизы:  транспортное средство \tc\,
	регистрационный знак \грз;\\ 
	VIN: \вин;\\
	Пробег:    \пробег км (установлен по показаниям одометра);\\
	Год выпуска:     \год;\\ 
	Дата ввода в эксплуатацию:  \началоэкспл;\\
	Дата ДТП:  \датадтп;\\
	Перечень ремонтных воздействий представлен в таблице \ref{tab:осм}.
	
	\subsubsection{Ремонтные воздействия, необходимые для устранения повреждений}
	
	\setcounter{rownum}{0}
	
	\begin{longtable}{G{3mm}|M{130mm}|G{5mm}|G{5mm}|G{5mm}}
		\caption[]{Таблица ремонтных воздействий, необходимых для устранения повреждений ТС \тс, полученных в заявленных обстоятельствах}
		\label{tab:осм}\\
		\hline  \hline   \toprule 
		\bf  {\footnotesize  n/n}  &\bf {\small Наименование  детали и описание повреждения} & \bf {\small E} & \bf {\small I} & \bf {\small L}\\\hline \hline \toprule  \endhead 
		
	%	
%%%%______________________________________%%%%%%%%%%%%
%%%%%%%%%   ОПИСАНИЕ ПОВРЕЖДЕНИЙ   
%\\ps{ деталь - повреждение }{E}{I}{L} 
\акт{Крыло переднее правое }{}{\7}{\7}
\акт{Бампер передний }{\7 }{ }{\7 }
\акт{Кронштейн бампера переднего правый }{\7 }{ }{ }
\акт{Диск колеса правого переднего семиспицевый }{\7 }{ }{ }
\акт{Фара правая }{\7 }{ }{ }
\акт{Облицовка крыла правого переднего }{\7 }{ }{ }
\акт{Спойлер бампера переднего правая часть }{\7 }{ }{ }
\акт{Пленка  защитная антигравийная бампера переднего и крыла правого }{\7 }{ }{ }


\end{longtable}\setcounter{rownum}{0} 
		
		\textit{E - заменить деталь, I - ремонтировать, L - окрасить}
		%
		
		%\subsubsection{ Расчёт}
		
		%
		\renewcommand\baselinestretch{1.2}\small\normalsize
		
		\subsubsection{ Расчёт}
		
		\indent Полный расчёт стоимости восстановительных расходов на ремонт ТС с учётом износа в соответствии с правилами обязательного страхования гражданской ответственности владельцев транспортных средств выполнен в  лицензированном для решения задач в рамках ОСАГО программном комплексе   SilverDAT myClaim и приведён в Калькуляции № \NomerDoc.
		Расчёт износа произведён программой  SilverDAT myClaim и представлен  в калькуляции расчёта затрат № \NomerDoc.
		
		\indent Итоговые результаты расчёта  стоимости восстановительных расходов ТС \тс\, \грз\, представлены ниже:\\
		
		
		\begin{figure}[H]
			\centering
			\includegraphics[width=0.95\linewidth]{example-image}
			%    		\caption{}
			%    		\label{fig:screenshot001}
		\end{figure}
		
		%
		\begin{figure}[H]
			\centering
			\includegraphics[width=0.95\linewidth]{example-image}
			%    		\caption{}
			%    		\label{fig:screenshot001}
		\end{figure}
		\begin{figure}[H]
			\centering
			\includegraphics[width=0.95\linewidth]{example-image}
			%    		\caption{}
			%    		\label{fig:screenshot002}
		\end{figure}
		\medskip
		\renewcommand\baselinestretch{1.2}\small\normalsize
		
		
		%
		
		
		%
		%
		
		
		\subparagraph{}Стоимость одного нормо-часа работ определена в соответствии с пунктом 3.8.1 Единой методики [3] путём применения электронных баз данных стоимостной информации.
		Трудоёмкость работ по разборке/сборке/замене  соответствует трудоёмкостям работ, рекомендованным заводом изготовителем ТС. Трудоёмкости окрасочных работ приняты согласно рекомендаций Единой методики, п.3.7.1. в соответствии с технологией  AZT (\url{http://www.schwacke.ru/down/azt _reparaturlackierung_ru.pdf}). Расчёт размера расходов на материалы произведён  согласно пункту 3.7.2 Приложения к Единой методике [3]. Артикулы запасных частей определены с помощью программы SilverDAT и электронных  каталогов запасных частей \url{emex.ru}, \url{partsouq.com}.
		Стоимость запасных частей определена в соответствии с пунктом 3.6.3 Единой методики путём применения электронных баз данных стоимостной информации (по утверждённому справочнику: \url{http://prices.autoins.ru/priceAutoParts/repair_parts.html} ).
		
		\subparagraph{}Таким образом,  наиболее вероятная стоимость ремонта транспортного средства \tc\, регистрационный знак \грз, получившего повреждения в результате дорожно-транспортного происшествия  \датадтп\, составляет \итог \, (\числопрописью{\итог}) руб.,  размер затрат на восстановительный ремонт ТС с учётом износа составляет \итогизнос \,(\числопрописью{\итогизнос}) руб.
		
		% \input {sections/рынокОСАГО}
		%\input {sections/утсОСАГО}
		%\input {sections/годныеОСАГО}
		
		
%		4)  Стоимость восстановительного ремонта  транспортного средства \tc\, регистрационный знак \грз,\, \, получившего механические повреждения в результате дорожно-транспортного происшествия, имевшего место \датадтп\, составляет \итог\, (\числопрописью{\итог})  руб.\\[3mm]
%		
%		5) Размер затрат на проведение восстановительного ремонта с учётом износа (восстановительные расходы) транспортного средства \tc\, регистрационный знак \grz\, составляет  \итогизнос \, (\числопрописью{\итогизнос}) руб.\\[3mm]
		
		
		%    6) Стоимость годных остатков ТС \тс\, регистрационный знак \грз\, оставляет  $ 10\,560$ (Десять тысяч пятьсот шестьдесят) рублей.
		
		%    6) Величина утраты товарной стоимости транспортного средства \тс\,  регистрационный знак \грз\, составляет 