\setcounter{page}{1}
\clubpenalty=10000 
\widowpenalty=10000
%%%%%%%%%%%%%%%%%%%%%%%%%%%%%%%%%%%%%%%%
%      Шапка экспертной организации  
%%%%%%%%%%%%%%%%%%%%%%%%%%%%%%%%%%%%%%%%
%
%%%%%%%%%%%%%%%%%%%%%%%%%%%%%%%%%%%%%%%%%
%
%   Экспертная организация ООО Южнорегиональная экспертная группа
%
%%%%%%%%%%%%%%%%%%%%%%%%%%%%%%%%%%%%%%%%%
\noindent %\qrcode[height=21mm]{\NomerDoc от \окончено }  %%% Добавлен QR-Code
\begin{pspicture}(21mm,21mm)
\obeylines
\psbarcode{%
	%\NomerDoc от \окончено
	BEGIN:VCARD^^J
	VERSION:4.0^^J
	%N:Мраморнов; Александр; Вчеславович^^J
	FN:Александр Мраморнов^^J
%	ORG:IP Alexandr Mramornov^^J
	TITLE: эксперт
	ORG: ИП
	URL:http://www.yourexp.ru^^J
	EMAIL:4516611@gmail.com^^J
	TEL:+7-918-451-6611^^J
	ADR:г. Краснодар, с/т № 2 А/О «Югтекс», ул. Зеленая, 472^^J
	END:VCARD
}{width=1.0 height=1.0}{qrcode}%
\end{pspicture}
\begin{center}
	\normalsize\textbf{$\cdots$\\[-1.5mm] <<$\cdots$>> \\[-5mm]}
	%  
	\noindent\rule{\textwidth}{1pt}\\[-6mm]  % Горизонтальная линия
	% \line(1,0){460}% (1,0) -горизонтальная линия, и (0,1) - вертикальная 
\end{center}

\begin{center}
	\begin{footnotesize}\setstretch{0.3}
		%	\small\textbf\setlength   	%\raisebox{5mm}
		\vspace{-3.5mm}$\cdots$\\[0mm]
		Телефон: \quad $\cdots$, e-mail:\quad $\cdots$\\ [-2mm]{$\cdots$\quad$\cdots$}
	\end{footnotesize}	\\[10mm]
\end{center}


\begin{flushright}
	%Краснодар,
	$\cdots$, 2023    \\[8mm]
\end{flushright}
\begin{center}
	\LARGE\textbf{ ЗАКЛЮЧЕНИЕ ЭКСПЕРТА}
	\bigskip\\[0mm]
	%	{\normnumxtbf{\NomerDoc}}	}{den}
\end{center}
\par
\vspace{-3mm}\noindent по гражданскому делу \delonum \, по иску \isk \\[0mm]

%\raggedright 
%\def\hrf#1{\hbox to#1{\hrulefill}}
\noindent \textbf{№ $\cdots$}\hfill           \textbf{\окончено}\\%[2mm]
%Приостановлено\hfill      \datastop\\
%Возобновлено\hfill          \datarestart\\
%Окончено\hfill                \dataend\\%[4mm]

%\noindent\parbox[l][16mm]{16.5cm}
%{\def\hrf#1{\hbox to#1{\hrulefill}}
%	\noindent Начато\hfill            \datastart\\%[2mm]
%	%	Приостановлено\hfill      \datastop\\
%	%	Возобновлено\hfill          \datarestart\\
%	Окончено\hfill                \окончено\\%[4mm]
%}
%\relax

\begin{flushright}
	\noindent\parbox[l][10 mm]{5cm}
	{\def\hrf#1{\hbox to#1{\hrulefill}}
		\noindent Начато\hfill            \datastart\\%[2mm]
		%	Приостановлено\hfill      \datastop\\
		%	Возобновлено\hfill          \datarestart\\
		Окончено\hfill                \dataend
	}
\end{flushright}
\relax

\datastart г. ~в {\small $\cdots$} \,  при определении  \, \sud  \,  от \, \dataopr \, о назначении \opr \, по гражданскому делу \delonum \, поступили:

\begin{enumerate}\setlist{nolistsep}\item  Материалы гражданского дела \delonum \\[-2mm]
	%	\item  
\end{enumerate}Экспертиза произведена экспертом  $\cdots$  % Шапка организации ООО ЮРЕКСГРУП
%
%%   вопросы экспертизы
\subsection{Вопросы экспертизы}
%Заказчик поручает, а Исполнитель принимает на себя обязательство выполнить Заказчику  комплекс работ в виде автотехнических исследований автомобиля Mazda 6, VIN RUMGJ52\-6802007133 (дата начала гарантии 07.05.2018 г.), по следующим вопросам:
\begin{enumerate}
	\item  <<Связано ли повреждение панели рамки радиатора слева и брызговика с лонжероном переднего левого автомобиля ВАЗ 21099 с указанным ДТП?>>	
\end{enumerate}

\addcontentsline{toc}{section}{Использованные нормативы и источники информации}
%
%\left( \addcontentsline{toc}{section}{Использованные нормативы и источники информации}

\subsection{Использованные нормативы и источники информации}
%
\begin{enumerate}
\item 
Махнин\,Е.\,Л., Новоселецкий\, И.\,Н., Федотов\, С.\,В. \emph{Методические рекомендации по проведению судебных автотехнических экспертиз и исследований колёсных транспортных средств в целях определения размера ущерба, стоимости восстановительного ремонта и оценки} // -- М.: ФБУ РФЦСЭ при Минюсте России, 2018.-326 с.  ISBN 978-5-91133-185-6.
%
\item ТУ 017207-255-00232934-2014 \emph{Кузова автомобилей LADA. Технические требования при приёмке в ремонт, ремонте и выпуске из ремонта предприятиями дилерской сети ОАО "АВТОВАЗ"}//  ОАО НВП "ИТЦ АВТО", 2014
%
\item Смирнов  В.Л., Прохоров  Ю.С., Боюр В.С.  и др. \emph{Автомобили ВАЗ. Кузова. Технология ремонта, окраски и  антикоррозионной защиты. Часть II}// - Н.Новгород: АТИС, 2001.- 241с.
%
\item 
Савич Е.Л. \emph{Техническое  обслуживание  и  ремонт  легковых  автомобилей} : учеб. пособие / Е.Л. Савич, М.М. Болбас, В.К. Ярошевич ; под общ. ред. Е.Л. Савича. -Мн. : Вышэйшая школа,  2001. - 479 с. - ISBN985-06-0502-2.
%
\item 
Автомобили ВАЗ-2121, 21213, 21214, 2131 и их модификации: <<Трудоемкости работ (услуг) по техническому обслуживанию и ремонту>> /Куликов А.В., Христов П.Н., Климов В.Е.,  Боюр В.С., Рева В.В., Зимин В.А., Завьялова Н.Н., Хлыненкова Г.А. -- ИТЦТ "АвтоВАЗтехобслуживание", Тольяти -- 2005. 
%
\item
Автомобили LADA SAMARA и их модификации: <<Трудоемкости работ (услуг) по техническому обслуживанию и ремонту>> /Куликов А.В., Христов П.Н., Климов В.Е., Рева В.В., Боюр В.С., Васильев М.В., Фахрутдинов Р.В.,  Прудских Д.А., Гирко В.Б., Шмелева В.А., Зимин В.А. --  ОАО НВП "ИТЦ АВТО",  -- 2006. - 252 стр.
%
\item 
Автомобили LADA PRIORA. Трудоемкости работ (услуг) по техническому обслуживанию и ремонту /Куликов А.В., Христов П.Н., Климов В.Е., Рева В.В., Козлов П.Л., Боюр В.С., Прудских Д.А., Шмелева В.А., Зимин В.А. -- ООО "ИТЦТ АВОСФЕРА", Тольяти -- 2009. -- 344 с.
%
\item 
{Трудоемкости работ по техническому обслуживанию и ремонту автомобилей автомобилей Lada  Granta}/   \url{https://docplayer.ru/30250248-Trudoemkosti-rabot-po-teh\-nicheskomu-obsluzhivaniyu-i-remontu-avtomobiley-lada- granta.html}.
%
%
\item
{Специализированное программное обеспечение для расчёта стоимости  восстановительного ремонта, содержащее нормативы трудоёмкости работ, регламентируемые изготовителями транспортного средства}//   AudaPadWeb, лицензионное соглашение № AS/APW-658  RU-P-409-409435.
%
%
%
\item

{Специализированное программное обеспечение для расчёта стоимости  восстановительного ремонта, содержащее нормативы трудоёмкости работ, регламентируемые изготовителями транспортного средства ОАО «АвтоВАЗ», ЗАО «Джи-Эм-АвтоВАЗ», ОАО «СеАЗ» и ОАО «ЗМА»}//   Автосфера АС:Смета, v.3.9.11// ООО "ИТЦ «ИнтегроМаш», \url{https://autosmeta.pro}.
%
%
%
\item Информационный портал по техническому обслуживанию и ремонту автомобилей	 ВАЗ:\\ \url{www.autosphere.ru}.

%%
\end{enumerate}

%%%%%%%%%%%%%%%%%%%%%%%%%%%%%%%%%%%%%%%%%%%%%%%%%%%%%%%%%%%%%%%%%%%%%%%%%%%%%%%%%
\subsection{Технические средства}  %% Список не удалять!!!
\begin{itemize}
%
%%
%\item Диагностический сканер BOSH VCM II S/N 1324-88682639 c програмным обеспечением Mazda IDS - 115.02
%%\item   Диагностический сканер SDconnect   с программным обеспечением Xentry Diagnostics v19.11.3.1
%\item   Линейка масштабная магнитная с цветографической шкалой, 100мм
%%\item   Рулетка измерительная металлическая, 5м
%%\item  Универсальный стенд для измерения углов установки колес Hunter Engineering %ProAlign с программным инструментом регулировки схождения колес без блокировки руля %автомобиля WinToe
%\item 	Цифровой фотоаппарат Canon 760D s/n 143032001327 с объективом Canon EF-S 18-135, тип используемой памяти: Transcend,  32Gb
%\item  Специализированное программное обеспечение для расчёта стоимости  восстановительного ремонта, содержащее нормативы трудоёмкости работ, регламентируемые изготовителями транспортного средства     AudaPadWeb, лицензионное соглашение № AS/\- APW-658  RU-P-409-409435
\item Он-лайн программа моделирования кинематики подвески автомобиля // \url{http://www.vsusp.com/}
\item Он-лайн ресурс проверки метаданных цифровых изображений \url{http://exif.regex.info/exif.cgi}
\item  Программа обработки изображений с открытым исходным кодом ImageJ, разработанная для научных многомерных изображений.  Wayne Rasband (wa-yne@codon.nih.gov),
свободная лицензия GPL.
\item  ПЭВМ под управлением операционной системы Windows 10 с установленным набором макрорасширений LaTeX системы компьютерной вёрстки TeX, cвободная лицензия LaTeX Project Public License (LPPL). 
%	
\end{itemize}
%%%%%%%%%%%%%%%%%%%%%%%%%%%%%%%%%%%%%%%%%%%%%%%%%%%%%%%%%%%%%%%%%%%%%%%%%%%%%%%%%%%%%%%%%%%%%%%%%%%%%%
\subsection{Условные обозначения}

\begin{description}
%	 
%%\item[АВС] --антиблокировочная система
\item[АМТС] --автомототранспортное средство
%\item[ГРМ] -- газораспределительный механизм
\item[ДВС] --двигатель внутреннего сгорания
\item[ДТП] --дорожно--транспортное происшествие
\item[гос.\,рег.\,знак] --государственный регистрационный знак
\item[КТС] --колесное транспортное средство 
\item[ЛКП] --лакокрасочное покрытие
\item[л.д.] --лист дела
%%\item[Колесо турбины]  -- крыльчатка турбины
\item[ТС] --транспортное средство
%\item[ТK, ТКР] -- турбокомпрессор. Состоит из двух частей: турбины и компрессора, объединенных общим валом. Вал вращается в подшипниках, размещенных в центральном корпусе ТК
%\item[ЦПГ] -- цилиндро-поршневая группа
%\item[ЭБУ] --электронный блок управления
%%\item[FRAME] -- номер кузова транспортного средства, выпущенного для продажи на внутреннем рынке Японии и содержащий информацию производителя о транспортном средстве
%%\item[OBDII] -- On-board diagnostics. Протокол бортовой диагностики автомобиля
%%\item[SRS] -- Cистема пассивной защиты водителя и пассажиров
\item[VIN] --vehicle identification number, 17--значный идентификационный номер транспортного средства, соответствующий стандарту ISO 3779--2012.
%
\end{description}
%%%%%%%%%%%%%%%%%%%%%%%%%%%%%%%%%%%%%%%%%%%%%

\subsection{Методы исследования}

\begin{itemize}
\item  Органолептический метод – исследование и оценка качества объектов с помощью органов чувств
%\item 	Прямой измерительный метод – путем измерения размеров деталей специальными %измерительными приборами
\item Расчётный метод (косвенный измерительный метод) – путём расчётов различных параметров на основе результатов измерений и других данных
\item Экспертный метод (метод экспертной оценки) — совокупности операций по выбору комплекса или единичных характеристик объекта, определению их действительных значений и оценкой экспертом соответствия их установленным требованиям и/или технической информации
\item Графоаналитический метод  
\item Метод масштабного моделирования и проецирования
\end{itemize}
%%%%%%%%%%%%%%%%
%
\subsection{Термины и определения}
\begin{description}
	\item[Аварийные повреждения] --- повреждения, механизм образования которых определяется контактом с посторонними объектами, что привело к деформации или разрушению и к необходимости ремонта или замены составной части, или контактам с агрессивной средой, которая привела к необходимости ремонта (замены) составной части [1, часть II, п. 1.5].
	%	\item[Восстановительный ремонт]--- один из способов возмещения ущерба, состоящий в выполнении технологических операций ремонта КТС, действующий в сети торгово-сервисного обслуживания, созданной изготовителем этого КТС [1, часть II, п. 1.4].
%	\item[Годные остатки] --- работоспособные, имеющие остаточную стоимость детали (агрегаты, узлы) поврежденного автотранспортного средства, годные к дальнейшей эксплуатации, которые можно демонтировать с поврежденного автотранспортного средства и реализовать.
	\item[Дата исследования]--- дата, на которую проводятся расчеты и используются стоимостные данные КТС, запасных частей, материалов, нормо-часа ремонтных работ [1, часть II, п. 1.5].
	\item[Линия удара]--- линия, определяемая направлением вектора равнодействующего импульса сил, возникающих при контакте ТС при столкновении до прекращения взаимного внедрения деформирующихся при ударе частей. Положением линии удара на ТС определяются направление и величина момента импульса сил, возникающих при ударе, и, следовательно, направлением и интенсивность разворота ТС относительно центра масс после столкновения.  
	\item[Моделирование]--- исследование каких-либо явлений, процессов или систем объектов путем построения и изучения их моделей.
	\item[Морфологические признаки]--- признаки, отображающие внешнее и внутреннее строение объекта
	\item[Срок эксплуатации КТС]--- период времени от даты изготовления (даты выпуска) КТС, до даты оценки (исследования), определяемой условиями задачи исследования (независимо от даты его регистрации и начала использования по назначению (эксплуатации)).
\end{description}
%%%%%%%%%%%%%%%%%%%%%%%
\subsection{Исходные данные и объекты исследования}

19.07.2016 г.в 13 час.  42 мин. в г. Белореченске по ул. Красная, 66 водитель Терехов А.В., управляя автомобилем Renault, г/н Е431НЕ33 перед началом движения не предоставил преимущество и допустил столкновение с автомобилем \тс\, регистрационный знак \грз\, под управлением водителя Шамояна Р.О.  
\par Для производства исследования представлено:
\begin{enumerate}
\item материалы гражданского дела № \delonum, \, в том числе:
\begin{itemize}
	\item Копия акта осмотра № 6164 транспортного средства \тс\, составленного 29.07.2016 г.  специалистом  компании "РАНЭ" 
	\item Копия акта осмотра № 295 от 19 июля 2016 г., составленного индивидуальным предпринимателем Новиковым Олегом Николаевичем
	\item Копия свидетельства о регистрации ТС 23 38 № 793117
	\end{itemize}
\item Электронные копии цифровых фотоснимков поврежденного автомобиля \тс\, VIN \vin\, предоставленные в  количестве 16 файлов формата .jpg с сохраненными техническими данными EXIF, выполненные 19.07.2016 г. цифровым фотоаппаратом Сanon PowerShot SX150 IS 
\end{enumerate}

\subsection{Ранее по материалам дела выполнено}

\noindent Судебная автотехническая экспертиза, выполненная  экспертом Дереберя Н.В.\\
Повторная судебная автотехническая экспертиза, выполненная экспертом Алифиренко В.В.

\section{Исследование}
%
Установление связи между повреждениями автомобиля и заявленными событиями доро\-жно-транспортного происшествия есть вопрос  исследования   взаимосвязи между повреждением и механизмом, вызвавшим это повреждение. Разрешение поставленного вопроса предполагает исследование механизма происшествия с точки зрения анализа характера деформаций и причин образования повреждений компонентов транспортного средства.	В транспортной трасологии различают первичные следы -- следы, возникшие в процессе первичного, начального контакта транспортных средств между собой или транспортных средств с различными преградами, и вторичные следы-- следы, появившиеся в  процессе дальнейшего смещения и деформации вступивших в следовое взаимодействие объектов. Таким образом, если рассматривать объемный след в виде деформации, то различаются первичные деформации, характеризующиеся наличием признаков непосредственного контакта деталей и частей транспортных средств,  и вторичные деформации, характеризующиеся отсутствием признаков непосредственного контакта деталей и  частей транспортных средств. Вторичные деформации являются следствием первичных, контактных деформаций. Детали изменяют свою форму под воздействием сил и моментов, возникающих в случае контактных деформаций по законам механики и сопротивления материалов. Такие деформации могут располагаться на удалении от места непосредственного контакта воздействием силы удара, распространяющейся по деталям несущей конструкции кузова автомобиля.
 

 Из постановления об административном правонарушении и справки о дорожно-транс\-портном происшествии следует, что в результате столкновения левая часть автомобиля \тс\, взаимодействовала с передним бампером автомобиля 
 \tcb.
 
   В процессе следового взаимодействия контактирующие поверхности \тс \, в составе левого зеркала заднего вида, переднего левого крыла, передней левой двери деформирующим воздействием бампера переднего автомобиля \tcb\, получили повреждения в направлении спереди назад и слева направо.  Графо-аналитическим методом исследования фотоизображений  повреждений автомобиля \тс \, установлено, что повреждения левой стороны кузова автомобиля  содержат признаки динамического и статического деформирующих воздействий, в совокупности не противоречащие заявленному механизму дорожно-транспортного происшествия.
   
\begin{figure}[h!]\centering
	\parbox[t]{0.49\textwidth}
	{\centering
		\includegraphics[width=.49\textwidth]{example-image}
		\caption{\footnotesize {Поврежденный в исследуемом ДТП автомобиль \тс,\, вид спереди слева}}
		\label{ris:images/b3}}
	\hfil \hfil
	\parbox[t]{0.49\textwidth}
	{\centering
		\includegraphics[width=.49\textwidth]{example-image}
		\caption{\footnotesize {Автомобиль, аналогичный автомобилю второго участника ДТП \tcb}}
		\label{ris:images/b4}}
\end{figure}



\subsection{Исследование транспортного средства}
%
С момента повреждения автомобиля до момента настоящего исследования прошел значительный период времени (более трех лет). Исходя из имеющихся в распоряжении эксперта изображений автомобиля, на момент ДТП, \датадтп\, на кузове автомобиля, преимущественно в нижней части, присутствовали  очаги коррозионных повреждений. Коррозия - физико-химическое или химическое взаимодействие между металлом и средой,  самопроизвольный окислительно-восстановительный процесс разрушения металлов и сплавов вследствие взаимодействия с окружающей средой,   необратимо усиливающихся с течением времени.  Ретроспективный характер исследования объективно не позволяет произвести натурное исследование автомобиля в том состоянии, в котором  автомобиль находился сразу  после заявленного ДТП.  На основании изложенного, исследование автомобиля \тс\, VIN \вин\, производилось экспертом по предоставленным электронным копиям цифровых фотоснимков. Представленные фотоснимки удовлетворительного качества и содержат необходимую и достаточную информации для производства экспертизы.


\begin{figure}[!h]\centering
	\parbox[t]{0.49\textwidth}
	{\centering
		\includegraphics[width=.49\textwidth]{example-image}
		\caption{\footnotesize {Обзорный снимок поврежденной области исследуемого автомобиля }}
		\label{ris:images/tc2}}
	\hfil \hfil
	\parbox[t]{0.49\textwidth}
	{\centering
		\includegraphics[width=.49\textwidth]{example-image}
		\caption{\footnotesize {Масштабное изображение  поврежденной области кузова исследуемого автомобиля}}
		\label{ris:images/tc3}}
	
\end{figure}



\begin{figure}[!h]\centering
	\parbox[t]{0.49\textwidth}
	{\centering
		\includegraphics[width=.49\textwidth]{example-image}
		\caption{\footnotesize {Повреждение левой передней двери }}
		\label{ris:images/tc8}}
	\hfil \hfil
	\parbox[t]{0.49\textwidth}
	{\centering
		\includegraphics[width=.49\textwidth]{example-image}
		\caption{\footnotesize {Повреждение левого переднего крыла}}
		\label{ris:images/tc9}}
	
\end{figure}

Представленные  изображения первичных повреждений левой передней стороны автомобиля \тс\,,  получены в результате заявленного ДТП \датадтп.  Из материалов дела известно, что оспариваемыми  страховой компанией повреждениями являются повреждения    панели рамки рамки радиатора слева и  брызговика с лонжероном переднего левого, Рис. \ref{ris:images/tc5}, Рис. \ref{ris:images/tc4}, Рис. \ref{ris:images/tc6}, Рис. \ref{ris:images/tc7}:


  \begin{figure}[h]\centering
  	\parbox[t]{0.49\textwidth}
  	{\centering
  		\includegraphics[width=.49\textwidth]{example-image}
  		\caption{\footnotesize {Повреждение левой нижней части панели рамки радиатора  }}
  		\label{ris:images/tc5}}
  	\hfil \hfil
  	\parbox[t]{0.49\textwidth}
  	{\centering
  		\includegraphics[width=.49\textwidth]{example-image}
  		\caption{\footnotesize {Повреждение передней нижней части левого лонжерона}}
  		\label{ris:images/tc4}}
  \end{figure}
%%%%%%%%%%%%%%%%%%%%%%%%%%%%%%%%%%%%%%%%%%%  

\begin{figure}[h]
	\centering
	\includegraphics[width=0.98\linewidth]{example-image}
	\caption{{\footnotesize {Разрыв панели рамки радиатора внизу слева. Вид спереди}}}
	\label{ris:images/tc6}
\end{figure}
\pagebreak

\begin{figure}[!h]
	\centering
	\includegraphics[width=0.98\linewidth]{example-image}
	\caption{{\footnotesize {Разрыв панели рамки радиатора внизу слева. Вид снизу слева. "1" - }}}
	\label{ris:images/tc7}
\end{figure}

\begin{figure}[!h]
	\centering
	\includegraphics[width=0.98\linewidth]{example-image}
	\caption{{\footnotesize {Разрыв панели рамки радиатора внизу слева. Вид снизу слева}}}
	\label{ris:images/tc11}
\end{figure}

\begin{figure}[H]
	\centering
	\includegraphics[width=0.98\linewidth]{example-image}
	\caption{{\footnotesize {Разрыв панели рамки радиатора. Слева усталостные трещины, справа - деформация и разрыв}}}
	\label{ris:images/tc12}
\end{figure}


Анализ поверхностных  деформаций крыла переднего левого  автомобиля \тс\, показывает, что в процессе столкновения  автомобиль \тс\, находился в движении по направлению "вперед", так как  относительно продольной оси автомобиля \тс\,, вектор удара направлен слева направо и спереди назад, автомобиль Renault Master мог двигаться в направлении слева направо, пересекая траекторию движения \тс, Таким образом, согласно имеющимся материалам, механизм столкновения,    вероятно, соответствует перекрестному, косому, скользящему, эксцентричному, левому столкновению.  Ударное воздействие локализовалось в области левой передней стойки автомобиля, при этом крыло переднее левое деформировалось на глубину не менее 5см, передняя  кромка двери левой передней получила загиб на глубину не менее 3 см в направлении действия сил, верхняя петля и примыкающая к ней часть левой передней стойки так же получили деформацию.  

Из технических данных  цифровых изображений EXIF известно, что 
фотографирование поврежденного автомобиля производилось  2016:07:19 в 14:44:54,  согласно справки о дорожно-транспортном происшествии, ДТП произошло 2016:07:19 в 13:42 то есть через 1 час после заявленного страхового события. Актом осмотра № 295, л.д. 19, составленным ИП Новиковым О.Н. 19.07.2016г. зафиксировано, в том числе, повреждение панели рамки радиатора слева и брызговик с лонжероном  слева.

На фотоизображениях Рис. \ref{ris:images/tc11} и Рис. \ref{ris:images/tc12} эксперт обращает внимание на два различающихся внешними признаками  вида повреждений ( области "1" и "2"). Повреждения, отмеченные на Рис. \ref{ris:images/tc11} выделенной областью "1" и левый снимок на Рис. \ref{ris:images/tc12} имеют характерные для усталостного разрушения трещины металла, возникающие под воздействием знакопеременной нагрузки или периодической динамической нагрузки. Коррозия поверхности излома металла в нижней части балки однозначно указывают на повреждение, образовавшееся задолго до дорожно-транспортного происшествия \датадтп.  Повреждения, отмеченные областью "2", правый снимок на Рис.  \ref{ris:images/tc12} и на Рис.\ref{ris:images/tc6}  имеют характер пластической деформации с разрывом металла, при этом, поверхности металла на разрыве имеют чистый металлический цвет без загрязнений и следов коррозии. Совокупность морфологических признаков этой части повреждений нижней поперечины рамки радиатора позволяют эксперту полагать, что указанные повреждения могут являться вторичными деформациями вследствие ДТП от \датадтп.

 \begin{figure}[H]
 	\centering
 	\includegraphics[width=0.75\linewidth]{example-image}
 	\caption{{\footnotesize {Cхема деталей передней подвески ВАЗ 21099}}}
 	\label{ris:images/s1}
 \end{figure}

На Рис.\ref{ris:images/s1} приведена схема передней подвески автомобилей ВАЗ 21099, на которой стрелками показаны точки крепления подвески к кузову автомобиля.  Стрелка 6 указывает на область 1, где расположено крепление продольной растяжки к нижней поперечине рамки радиатора, стрелка 5 указывает на  область 2, место крепление поперечного рычага к основанию лонжерона и стрелка 4 указывает на точку 3 - место крепления стойки передней подвески к брызговику. Крепление рычага, растяжки и стойки выполнено через сайлентблоки. Практика ремонта автомобилей ВАЗ 2108-2109 показывает, что участок места крепления кронштейна рычага к панели рамки радиатора (на схеме отмечен "1"), в силу конструктивных особенностей, является  концентратором напряжений. 

 \begin{figure}[H]
	\centering
	\includegraphics[width=0.75\linewidth]{example-image}
	\caption{{\footnotesize {Направление   ударного воздействия  и реакции кузова ВАЗ 2109}}}
	\label{ris:images/s2}
\end{figure}

На рисунке \ref{ris:images/s2}  стрелка 1 показывает направление  вектора удара, стрелки 2 и 3 направление реакции кузова.

\pagebreak

\begin{multicols}{3}[\columnsep=1cm]
\noindent	\includegraphics[scale=0.5]{example-image}
	\columnbreak
	\includegraphics[scale=0.5]{example-image}
	\columnbreak
	\includegraphics[scale=0.5]{example-image}
\end{multicols}

\captionof{figure}{\footnotesize{Пример конечно-элементной модели реакции кузова автомобиля при боковом ударе (по материалам CompMechLab LLC, \url{http://fea.ru/compound/automotive}  ) }}

  \vspace{3mm}

На примере конечно-элементной модели автомобиля  наглядно видно, что при  ударном воздействии в боковую стойку в направлении слева направо левый передний лонжерон стремиться сместится в направлении, противоположном вектору удара. При этом, отклонение лонжерона от продольной оси может составлять нескольких сантиметров. Таким образом, именно в области левой нижней части панели рамки радиатора автомобиля \тс в месте сопряжения с передним левым лонжероном при боковом ударе создается дополнительная  концентрация напряжений.

 В случае исследуемого ДТП, имело место  сочетание ослабления места крепления кронштейна  рычага на панели рамки радиатора вследствие  усталости металла и образования дополнительной концентрации напряжений, возникшей в результате ударной нагрузки в область передней левой стойки автомобиля, что, в совокупности, привело к превышению предела прочности и  разрушению панели раки радиатора и передней нижней части левого лонжерона.  

\section{Вывод} 


\textbf{  <<Повреждение панели рамки радиатора слева и брызговика с лонжероном переднего левого автомобиля ВАЗ 21099 связано  с указанным ДТП.>>	}
  
  
  \vspace{20mm}
{Эксперт}\hfill           {Алифиренко В.В.}


