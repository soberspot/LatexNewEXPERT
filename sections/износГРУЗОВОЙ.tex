\subsection{Расчёт износа ТС \тс}

В общем случае, износ характеризует изменения определённого параметра (стоимости, эксплуатационных качеств и технических характеристик в виде посадок, зазоров, прочности, прозрачности и т.д.) или совокупности этих параметров относительно состояния, соответствующего новой составной части КТС в целом п. 7.4 [1]. Износ, используемый для определения стоимости восстановительного ремонта, характеризует изменения стоимости составной части в зависимости от ее оставшегося ресурса.

Подлежит обязательному учёту износ составных частей, нормативный ресурс которых меньше, чем ресурс КТС в целом, то есть тех составных частей, которые имеют постоянный нормальный износ и подлежат регулярной своевременной замене в соответствии с требованиями к эксплуатации транспортного средства.


%%%%%%%%%%%%%%%%%%%%%%%  Алгоритм расчета  %%%%%%%%%%%%%%%%%%%%%%%%%%%%%%

В соответствии п. 7.5 [1] коэффициент износа составных частей (И) КТС для автобусов и грузовых автомобилей при определении стоимости восстановительного ремонта рассчитывается по формуле:

\begin{equation}\label{eqsnos}
\text{И} =100\cdot(1-\exp^{-\omega}), \%  \,\,\,\, \text{где:}
\end{equation}

$ e $ - основание натурального логарифма, $ e \approx 2.72  $;

$ \omega $ - функция, зависящая от срока эксплуатации и пробега автобусов и грузовых автомобилей (Талица 4, приложение 2.4 к Методическим рекомендациям [1])


Для исследуемого автомобиля \тс\, регистрационный знак \грз \\ $\omega = 0.09\cdot\text{Д} + 0,002\cdot\text{П}, \,\,\, \text{где}  $ 

Д - срок эксплуатации, лет, Д = 4,7 года;

П - пробег, тыс. км., П = 101 тыс. км.

Тогда:

\begin{equation}\label{eqsnos1}
\text{И} =100\cdot(1-\exp^{-\omega}) = 100\cdot(1-\exp^{-(0.09\cdot\text{4,7} + 0,002\cdot\text{101})} = 47\%
\end{equation}

%
%
%\begin{itemize}
%	\item [] $ \text{И1} $ --усредненный показатель износа на 1000 км пробега, \%; 
%	\item [] $ \text{П} $ -- общий пробег (фактический или расчетный) за срок эксплуатации КТС, тыс.км;
%	\item [] $ \text{И2} $ -- усредненный показатель старения за 1 год эксплуатации, \%;
%	\item [] $ \text{Д} $ -- срок эксплуатации КТС (от даты изготовления КТС до момента, на который определяется износ), лет. 
%\end{itemize}

%\vspace{3mm}
%%%%%%%%%%%%%%%%%%%%%%%%  Нулевой и предельный износ %%%%%%%%%%%%%%%%%%%%%%%%%%%%%%


\par Согласно п. 7.8.\, Методики [1]  для случаев, не регулируемых законодательством об ОСАГО, для составных частей КТС значение износа принимается равным нулю, срок эксплуатации которых не превышает 5 лет.

В нашем случае, транспортное средство эксплуатировалось с 15.10.2014  по \датадтп, что составляет 4.7 года.

Автомобиль эксплуатировался в нормальном режиме , пробег на момент повреждения составлял меньше нормативного (нормативный пробег 29,1*4,7 = 136 тыс. км.);

Составные части остова ТС и оперения ранее не восстанавливали ремонтом;

Автомобиль, по определению, не эксплуатировался в режиме такси;

Автомобиль не эксплуатировался в регионе с тяжёлыми климатическими условиями.

На основании вышеизложенного, имеется достаточно оснований для того, чтобы  при определении размера ущерба согласно п. 7.8.\, Методики [1]  применить коэффициент износа (И) равным 0 (нулю).   
  
Утрата товарной стоимости для исследуемого автомобиля не рассчитывается (численное значение УТС принимается равным нулю), так как согласно п. 8.3  Методики [1]  для грузовых автомобилей возрастом старше 3 лет УТС не рассчитывается.
%
%Независимо от сферы правого регулирования, значение коэффициента износа принимается равным нулю для составных частей, непосредственно влияющих на безопасность движения. Номенклатура таких составных частей приведена в приложении 2.6. Методики [1].
%
%Независимо от сферы правого регулирования, значение износа принимается равным нулю для деталей из ремонтного комплекта, замена которых является частью технологического процесса обслуживания или ремонта (прокладки, фильтры, уплотнители и т.п.).  
 %,  предельное значение износа комплектующих транспортного средства  не должно превышать 80\% стоимости запасных частей. Для составных частей, имеющих срок эксплуатации более 12 лет, при отсутствии факторов снижения износа (проведенный капитальный ремонт, замена составных частей  и т.д) рекомендуемое значение износа составляет 80\%.



%\subsubsection{Данные для расчета}

%\noindent Объект исследования: транспортное средство \tc\,
%регистрационный знак: \грз\\ 
%Идентификационный номер VIN: \вин\\
%Пробег:    \пробег, км\\%(Установлен по показаниям одометра);\\
%Год выпуска:     \год\\ 
%Дата ввода в эксплуатацию:  \началоэкспл\\
%Дата повреждения:  \датадтп\\
%ПТС: \птс\\
%Рыночная стоимость ТС \tc\\ 
%регистрационный знак \grz составляет: $\cdot$ ($\cdot$) рублей;\\
%источник: \url{https://spec.drom.ru}
%\vspace{3mm}
%
%Из открытых банков данных полиции известно, что автомобиль с VIN  \вин\, как минимум дважды становился участником ДТП:\\
%15.06.2018  12:15, извещение о ДТП № 030043199\\  %, в котором автомобиль получил повреждения задней правой двери, заднего правого порога, заднего правого колеса, подушки SRS справа, Рис. \ref{ris:images/d1} 
%18.11.2019 12:30, извещение о ДТП № 790004991.\\%, в котором автомобиль получил повреждения деталей передней левой и задней частей кузова, Рис. \ref{ris:images/d2}.
%\noindent \textit{Источник:}  \url{https://xn--90adear.xn--p1ai/check/auto#\vin}
%\pagebreak


%\subsubsection{Расчет}

%Для исследуемого автомобиля \тс, параметры  расчета коэффициента износа приняты согласно справочным таблицам  ч. II, Приложение 2.4 [1], информация о пробеге предоставлена заказчиком, дата ввода в эксплуатацию принята согласно сведениям паспорта транспортного средства \птс:

%\begin{itemize}
%	\item [] $ \text{И1} = 0.3$ \,\% %--усредненный показатель износа на 1000 км пробега, \%; 
%	\item [] $ \text{П} = 142.1 $ \, тыс. км %-- общий пробег (фактический или расчетный) за срок эксплуатации КТС, тыс.км;
%	\item [] $ \text{И2} = 1.35 $ \, \% %-- усредненный показатель старения за 1 год эксплуатации, \%;
%	\item [] $ \text{Д} = 14 $ \, лет %-- срок эксплуатации КТС (от даты изготовления КТС до момента, на который определяется износ), лет. 
%\end{itemize}

%\begin{equation}\label{eqsnosr}
%\text{И} =\text{И1}\cdot\text{П}+\text{И2}\cdot \text{Д} = 0.3\cdot 142.1  + 1.35\cdot 14 = 61 \, \%
%\end{equation}
%
%\par Таким образом, по результатам  настоящего исследования износ транспортного средства принимается равным нулю по следующим основаниям: обстоятельства заявленного события не регулируются законодательством об ОСАГО,  срок эксплуатации КТС не превышает пяти лет, иные повреждения и следы ремонта отсутствуют, признаки интенсивной эксплуатации отсутствуют.

%Таким образом, согласно изложенному выше, величина износа транспортного средства \тс\, регистрационный знак \грз\, VIN \, \vin\, на момент дорожно-транспортного происшествия \датадтп\, составляет 0 \, \%. 

