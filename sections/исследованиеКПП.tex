
\clubpenalty=10000 
\widowpenalty=10000


\section{Исследование}
%
%\subsection{История ремонта и сервисного обслуживания}
%
%На основании предоставленных материалов составлена история ремонта и сервисного обслуживания транспортного средства \тс \, по датам и пробегу, Таблица \ref*{tab:hist}:
%
%{\small 
%\begin{longtable}{|p{16mm}|p{12mm}|p{29mm}|p{50mm}|p{35mm}|}
%\caption[]{\footnotesize {\textbf{История ремонта и сервисного обслуживания по дате и пробегу}}} \label{tab:hist}\\\hline\hline
%%%------------------------------------
%\toprule\textbf{Дата} &\textbf{Пробег, км} &\textbf{№\,Акта,Заказ-наряда, накладной}& \textbf{Вид работы}& \textbf{Примечание}\\\hline \toprule \endhead 
%%%-----------------------------------
%%%%%	% Строки
%	%
%%27.04.2018 & 5  & Заказ-наряд № 480261860-1 & Предпродажная подготовка  &  прим. \\ \hline
%%
%\hs{27.04.2018}{5000}{№ 7643}{расточка блока цилиндров}{просто так}
%\hs{27.04.2018}{5000}{№ 7643}{расточка блока цилиндров}{просто так}
%%\hs{arg1}{arg2}{arg3}{arg4}{arg5}
%%\hs{arg1}{arg2}{arg3}{arg4}{arg5}
%%\hs{arg1}{arg2}{arg3}{arg4}{arg5}
%%\hs{arg1}{arg2}{arg3}{arg4}{arg5}
%%\hs{arg1}{arg2}{arg3}{arg4}{arg5}
%%\hs{arg1}{arg2}{arg3}{arg4}{arg5}
%%\hs{arg1}{arg2}{arg3}{arg4}{arg5}
%%\hs{arg1}{arg2}{arg3}{arg4}{arg5}
%%\hs{arg1}{arg2}{arg3}{arg4}{arg5}
%%\hs{arg1}{arg2}{arg3}{arg4}{arg5}
%%
%%
%\end{longtable}}%\setcounter{rownum}{0} % Обнуляем счетчик строк для следующей таблицы
%

%
%\par 03.09.2019 автомобиль с посторонним стуком в ДВС на эвакуаторе доставлен  в сервисный центр ООО "Формула-МК" по адресу: г. Краснодар, ул. Аэропортовская, 4/1.  Первичная диагностика показала, что при увеличении оборотов до 2000 об/мин слышен стук в ДВС. При приеме ТС выявлено, что уровень масла ниже минимальной отметки, уровень охлаждающей жидкости на минимальном уровне, сигнализаторы или контрольные лампы на панели приборов не горят. Специалистами сервисного центра произведена замена масла, слито 3л масла, цвет масла темный, по субъективной оценке специалиста, выполнявшего замену масла, в слитом масле присутствовал запах бензина. Залито новое масло  до максимального уровня. После замены масла стук в ДВС не прошел. При считывании ошибок зафиксирована ошибка Р0524 (слишком низкое давление масла) на пробеге 32 674 км. Выполнена проверка согласно MESI по симптому № 21 <<Шум в двигателе>>. По итогам проверки, так как источник звука находится внутри ДВС, принято решение произвести частичную разборку для определения источника звука. Дополнительно выполнили проверку давления масла: нижний предел при 1500 об/мин - 2.4 бар; при 4500 об/мин - 4.4 бар. %\rem{ Какое давление масла должно быть по техдоку?} 
%Проверили компрессию для данного двигателя (степень сжатия 14) 1ц -6.5 кг/см2; 2ц-6.5 кг/см2; 3ц-6.5 кг/см2; 4ц-6.0 кг/см2. Выполнили снятие поддона ДВС и нижних головок шатуна. Вкладыш 4-го цилиндра имеет задиры, шатунная шейка коленвала 4го цилиндра имеет задиры, вкладыши 2 и 3 цилиндров имеют задиры. В маслозаборнике присутствуют металлические частицы.  На основании вышеизложенного, специалистами сервисного центра причиной возникновения неисправности названа эксплуатация автомобиля  с уровнем масла ниже рекомендованного заводом изготовителем.

\subsection{Исследование предоставленных на экспертизу документов}

%%%%%%%%%%%%%%%%%%%%%%%% ДЛЯ АВТОМОБИЛЯ
% \subparagraph*{}Из Электронной сервисной книжки  известна следующая информация об автомобиле, имеющая значение для дачи заключения:

Из регистрационных документов известна следующая информация об автомобиле, имеющая значение для дачи заключения:

\begin{description}
   \item[Марка, модель] --\тс
   \item[VIN] -- \vin
   \item[Год выпуска] --\год
   \item[Шасси] --отсутствует
   \item[Цвет ЛКП] --\цвет
   \item[Двигатель] --\двигатель
   \item[Тип КПП] --\кпп
   \item[Привод] --передний
   %\item[Трансмиссия] -- 
   \item[ПТС] --\птс
   \item[Свидетельство о регистрации] --\свид
   \item[Расположенние руля] --левое
\end{description}

\subparagraph*{} Идентификационный код автомобиля (VIN) \vin\, содержит следующую информацию о транспортном средстве, имеющую значение для 	дачи заключения:

\фото{example-image}{Расшифровка комплектации по VIN \vin}


\subsection{Исследование транспортного средства}

Исследование автомобиля \тс\, VIN  \vin\,  проводилось  специалистом \датаосмотра\, с использованием производственных мощностей автотехобслуживающего предприятия мультимарочного сервисного центра, расположенного по адресу: \местоосмотра\,  с 10-00 до 15-00 ,  в светлое время суток при естественном и искусственном освещении. При проведении осмотра присутствовали: \присутствовали.   Внешним осмотром установлено:\\
\begin{itemize}
\item 
Автомототранспортное средство \тс\, VIN  \vin\, соответствует товарным образцам автомобилей, собранным из сборочных комплектов  по технологии крупноузловой сборки южнокорейской компании  SsangYong  российской автомобилестроительной фирмой ООО «СОЛЛЕРС — Дальний Восток».  Имеет кузов типа «пятидверный универсал».  Кузов автомобиля окрашен рефлексной ("лессирующей", с "металлическим" эффектом) эмалью (краской) \colr\, цвета. Общий вид автотранспортного средства представлен на Рис. \ссылка{рис:спереди},  \ссылка{рис:видсзадисправа}. 
\item 
Маркировочные обозначения, нанесённые на кузове представленного ТС, цвет кузова, тип кузова, модель ТС, государственные регистрационные номера соответствуют записям  регистрационных документов ТС, Рис. \ссылка{рис:vin}, \ссылка{рис:vin2}, \ссылка{рис:vin4}, \ссылка{рис:св1}, \ссылка{рис:св2};
\item 
Внешние признаки внесения изменений в конструкцию ТС отсутствуют;
\item 
Тягово-сцепное устройство и признаки его установки на автомобиле отсутствуют;
%\item 
%Автомобиль предоставлен частично разобранным: ...
\item  
Показания одометра на момент осмотра составляют \пробег;
\item 
Автомобиль не имеет   повреждений аварийного характера;
\item 
Внешний вид ТС удовлетворительный, по совокупности характерных признаков, пробег автомобиля составляет не более 100 000 км;
\item 
Узлы и агрегаты, размещённые в моторном отсеке видимых повреждений, или признаков, указывающих на возможные повреждения, не имеют.
\end{itemize}

На момент обращения в сервисный центр  при начале движения автомобиль периодически трогается с незначительным рывком. Наиболее сильно рывки ощущаются при трогании на разогретом двигателе.Так же рывки (толчки) происходят в движении на прогретой до рабочей температуры АКПП. Наблюдаются толчки при переключении с 1-й на 2-ую передачу при резком нажатии педали акселератора, толчки при переключении на 3-ю передачу. Владелец автомобиля сообщает о периодическом "зависании" второй передачи. На панели приборов загорается индикатор ошибки, АКПП автомобиля \тс\,  переключается в аварийный режим. 
%Как правило, переключение в аварийный режим вызвано внутренними нарушениями  АКПП.


%
%Автомобиль предоставлен частично разобранным: демонтирован поддон двигателя, вкладыши коленчатого вала, шатунные катушки зажигания, свечи зажигания.  Отдельно представлена пластиковая емкость, содержащая 3 литра масла из двигателя ТС \тс.
%На момент осмотра на автомобиле имеются повреждения переднего бампера снизу слева в виде задиров, крыло заднее правое  имеет царапины ЛКП, бампер задний справа имеет царапины ЛКП, имеется повреждение лобового стекла. Давление в шинах колес передней и задней оси 2.3 бар, шины BRIDGESTONE TURA NZA 225/55R17 97V

Компьютерное диагностирование ЭБУ автомобиля показало неисправности трансмиссии.
С помощью диагностического оборудования получены  коды неисправностей (DTC) (Таблица \ref{table:ошибки}). 
\vspace{3mm}

\begin{table}[h]
    \caption{Таблица зарегистрированных ошибок.}
    \label{table:ошибки}
    \begin{tabular}{c|m{45mm}|m{35mm}|m{63mm}}\hline
      \textbf{  n/n} & \textbf{Код ошибки} & \textbf{Повторяемость} & \textbf{Описание} \\
        \hline 
        1 & P0700 & Фиксируется постоянно & TCU Signal Fault. Система управления трансмиссией (запрос MIL), подсистема электронный блок управления трансмиссией (TCU). Ошибка системы управления коробкой передач \\
        \hline
        2 & P1124 & Фиксируется спорадически & Accelerator Pedal Sensor MalfunctionStuck . Ошибка сенсора педали акселератора.\\ \hline
    \end{tabular}
\end{table}

Первоначально, с целью устранения неисправностей, специалистами сервисного центра  в АКПП автомобиля была произведена замена гидравлической жидкости. При замене использовалась гидравлическая жидкость Fuchs TITAN ATF 3292.    Слитая из АКПП гидравлическая жидкость густая,  тёмного цвета,  с включениями большого количества мелкодисперсных металлических частиц.  Вероятно, жидкость в АКПП не менялась весь период эксплуатации автомобиля (\пробег).  В результате замены жидкости %, после  адаптации АКПП, 
неисправности АКПП автомобиля сохранились, но стали менее выражены.  Ошибка (DTC) P0700, после cброса диагностическим сканером, в движении ТС вновь появляется стабильно. Ошибка P1124 - ошибка датчика педали акселератора, на исследуемом автомобиле имеет случайный характер, (алгоритмически формируется при  одновременном сигнале на ЭБУ от педалей газа и тормоза  при скорости более 25 км/ч), как правило,  связана с  работой реостата электронной педали акселератора. После сброса ошибка  не возникала.   Фактически, ошибка P0700 является информационной, указывает на имеющиеся неисправности в АКПП.    Неисправности АКПП, ощущаемые как толчки, рывки, вызваны гидроударом в связи с падением давления масла в коробке, что может быть обусловлено неисправностями как электрогидравлических компонентов, так и неисправностями механической части агрегата. В таком случае, установить причину неисправности КПП возможно только проведя исследование и диагностирование  компонентов АКПП, с необходимыми для этого  демонтажем и полной разборкой агрегата.

Исследуемый автомобиль оснащён автоматической коробкой перемены передач  первой модификации: 36100-34110 (Рис. \ссылка{рис:коробкамодель2}), агрегат изготавливался в период 15.10.2010-29.06.2011 и предназначался для установки на перенеприводную версию автомобиля \тс.

В процессе исследования произведена проверка электрических жгутов, разъёмов АКПП. Повреждения электрических разъёмов, жгута проводов АКПП отсутствуют.

Видимые возможные повреждения (механические повреждения, признаки воздействие воды, влаги)  электронного блока управления АКПП (TCU) отсутствуют.

Произведён демонтаж, разборка и поэлементный осмотр   компонентов АКПП.

Демонтирован гидроблок. Поверхности всех соленоидов электрорегуляторов VBS Norm-Low  и VBS Norm-High покрыты маслянистым осадком вещества тёмного цвета (Рис. \ссылка{рис:блоксоленоидов2}, \ссылка{рис:соленоид}); 

На корпусе оборотного датчика присутствуют  металлические  частицы (Рис. \ссылка{рис:Screenshot_1});

Дифференциал в удовлетворительном состоянии (Рис. \ссылка{рис:диференциал}), пригоден для повторного использования;

Роликовый подшипник дифференциала имеет признаки начального износа. Дорожки качения, тела качения и сепараторы повреждены малыми инородными абразивными частицами. Повреждения, препятствующие дальнейшей эксплуатации подшипника отсутствуют, (Рис. \ссылка{рис:подшипникрол3});

Обойма подшипника имеет признаки начального износа,  (Рис. \ссылка{рис:подшипникоб2});

Приводной вал в удовлетворительном состоянии (Рис. \ссылка{рис:вал}), пригоден для дальнейшего использования;

Роликовые конические подшипники приводного вала с признаками начального эксплуатационного износа, тела качения имеют незначительные царапины,  подшипники пригодны для дальнейшего использования, (Рис. \ссылка{рис:подшипникрол2});

Обойма подшипника с признаками незначительного эксплуатационного износа. Повреждения,препятствующие дальнейшей эксплуатации подшипника отсутствуют,   (Рис. \ссылка{рис:подшипникоб});

На магните-улавливателе  присутствует пастообразная масса частиц  темного цвета, обладающими магнитными свойствами (Рис. \ссылка{рис:магнит});

Масляный насос (Рис. \ссылка{рис:масляныйнасос}) со следами износа, внутренняя шестерня имеет  выработку рабочих поверхностей,  узел подлежит замене,  (Рис. \ссылка{рис:шестерня}, \ссылка{рис:шестерня2});

Фетровый фильтр тёмного, серо-жёлтого цвета, на материале фильтра  хорошо различимо присутствие большого количества мелкодисперсных частиц цвета металла  (Рис. \ссылка{рис:фильтр}), фильтр расходный компонент, повторное использование недопустимо;

Суппорт в удовлетворительном состоянии (Рис. \ссылка{рис:суппорт}), узел пригоден для дальнейшего использования;

Корпус тормозного барабана имеет поверхностные повреждения  тормозной лентой (Рис. \ссылка{рис:барабан}), деталь пригодна к повторному использованию после шлифовки корпуса;

Тормозная лента, средняя часть, темно-коричневого цвета с явными признаками перегрева, (Рис. \ссылка{рис:тормознаялента}, \ссылка{рис:тормознаялента2}), деталь подлежит замене; 

Шток сервопривода имеет характерную выработку, обрезиненная часть затвердевшая, эластичность низкая (Рис. \ссылка{рис:клапан2}, деталь подлежит замене);

Отверстие в корпусе АКПП под шток сервопривода имеет выработку (возможна реставрация корпуса втулкой или развёртка  отверстия при использовании  штока  ремонтного размера (Рис. \ссылка{рис:клапан3});

Пакеты фрикционов коричневого и темно-коричневого цвета с накладками из целлюлозы. Толщина всех пакетов фрикционов уменьшена на $ \approx $ 2-3 mm.  Фрикционные диски  с естественным эксплуатационным износом, поверхности стальных колец  третьей передачи   содержат характерные признаки, указывающие на проскальзывание, недостаточное сцепление фрикционов при передаче крутящего момента, дальнейшее использование пакетов фрикционов не допустимо. (Рис. \ссылка{рис:пакееще}, \ссылка{рис:пакееще2}, \ссылка{рис:пакееще3}, \ссылка{рис:пакет}, \ссылка{рис:пакет2});

Шестерни заднего планетарного механизма в удовлетворительном состоянии, узел пригоден к дальнейшей эксплуатации, (Рис. \ссылка{рис:планетарка}); 

Солнечная шестерня в удовлетворительном состоянии, пригодна для эксплуатации, (Рис. \ссылка{рис:валеще},  \ссылка{рис:вал2})

Посадочное место подшипника заднего планетарного механизма в корпусе АКПП имеет  выработку, диаметр увеличен на $ \approx $  2,5 мм (Рис. \ссылка{рис:подшипниквыроботка}, \ссылка{рис:подшипниквыроботка2}, для повторного использования необходимо восстановить посадочное место подшипника), в случае невозможности - заменить корпус АКПП;

Подшипник имеет признаки износа, к дальнейшей эксплуатации не пригоден;

Планетарный механизм АКПП в удовлетворительном состоянии, пригоден к дальнейшей эксплуатации;

Покрытия поршней утратили эластичность, повторное использование поршней не допустимо;


%Гидротрансформатор - 

 
\subsection{Анализ результатов исследования}

  В результате исследования деталей и узлов автоматической коробки перемены передач  автомобиля  \тс\, \вин\, установлено, что АКПП находится в неисправном состоянии вследствие физического износа компонентов агрегата, к эксплуатации непригодна. По совокупности установленных повреждений, причиной неисправного состояния АКПП является некачественное/несвоевременное обслуживание автомобиля.  Изготовителем рекомендована проверка состояния гидравлической жидкости через каждые 15 000 км или через 12 месяцев, замена гидравлической жидкости через каждые 60 000 км при эксплуатации в тяжелых условиях. На момент исследования пробег автомобиля составлял \пробег\, при этом уровень гидравлической жидкости соответствовал норме, жидкость имела тёмный, почти чёрный цвет, с включениями мелкодисперсных частиц вещества цвета металла. Физическое состояние гидравлической жидкости позволяет специалисту полагать, что в период эксплуатации автомобиля отсутствовал своевременный контроль состояния жидкости, а необходимая периодическая замена не производилась.
  В процессе эксплуатации гидравлическая жидкость окисляется, насыщается  продуктами износа, подвергается многократным циклам нагрева/охлаждения, что с течением времени приводит к утрате её физико-химических свойств, необходимых для нормальной эксплуатации агрегата.  
   Замена жидкости, выполненная специалистами СТОА, с целью устранения имеющихся неисправностей АКПП существенно не улучшила  работу агрегата, так как в результате  детального исследования компонентов АКПП установлено, что на момент замены гидравлической жидкости  в АКПП исследуемого автомобиля  уже присутствовали  механические повреждения части деталей, делающие невозможной нормальную эксплуатацию агрегата.
  
  С учётом вышеизложенного, исследованием установлено, что нормальная эксплуатация автомобиля \тс\, \вин\, возможна только после ремонта или замены АКПП.


\повопросу{1. Какие неисправности имеет коробка перемены передач \кпп\, транспортного средства \тс\, регистрационный знак \грз?}

На момент настоящего исследования  коробка перемены передач \кпп\, транспортного средства \тс\, регистрационный знак \грз\,
имеет повреждения  масляного насоса, тормозного барабана, тормозной ленты, подшипника и корпуса АКПП в виде износа посадочного места подшипника заднего планетарного механизма,  штока сервопривода и его посадочного отверстия в корпусе АКПП, уменьшение вследствие истирания толщин пакетов фрикционов, утрата эластичности поршней, имеющих эластомерные покрытия; загрязнены гидроблок, соленоиды электрорегуляторы VBS Norm-Low  и VBS Norm-Hig, масляный фильтр.

\повопросу{2. Какова причина их возникновения?}

Причиной возникновения неисправностей АКПП является  повреждения электрогидравлического блока и механических компонентов АКПП.

\повопросу{3. Является ли данная причина:
   производственной, т.е. недостатком сборки и/или материала;
      связанной с некачественным/несвоевременным обслуживанием автомобиля;
      связанной с неразрешенными/недопустимыми переделками агрегата и/или его систем;
      связанной с предыдущим ремонтом (если применимо);
      эксплуатационной, т.е. возникшей по причине неправильной/ненормальной эксплуатации;
       естественным износом в соответствии с пробегом автомобиля?>> 
}


В результате исследования не установлены какие-либо недостатки, отвечающие критериям "производственный недостаток" согласно ГОСТ 27.002-2015.

Изменения конструкции, признаки изменения конструкции исследуемого автомобиля отсутствуют.

Признаки, следы предыдущего ремонта АКПП отсутствуют.

Пробег автомобиля на момент настоящего исследования составляет \пробег. Общее состояние автомобиля соответствует состоянию автомобиля, с пробегом менее 100 000 км.

Состояние  деталей и узлов АКПП указывает на некачественное/несвоевременное техническое обслуживание агрегата, отсутствие контроля за состоянием гидравлической жидкости, что привело к выходу его из строя вследствие  преждевременного износа элементов конструкции АКПП.

 Имеющиеся повреждения, по совокупности морфологических признаков, имеют накопительный характер, образованы в процессе длительной эксплуатации АКПП при значительном  (несколько десятков тысяч км) перепробеге сервисного интервала замены гидравлической жидкости.
 
 

Таким образом, причина возникновения неисправностей АКПП автомобиля \тс\, VIN \тс\, имеет накопительный характер, обусловлена значительным (несколько десятков тысяч км) перепробегом сервисного интервала замены гидравлической жидкости в АКПП. Является эксплуатационной и возникла по причине неправильной/ненормальной эксплуатации ТС. 

%\повопросу{4. Какова стоимость их устранения?}


%
%\subparagraph{Первая возможная причина - }
%
%\subparagraph{Вторая возможная причина - }
%




\vspace{5mm}
\section{Выводы}

\begin{enumerate}
	\item \textbf{ На момент настоящего исследования  коробка перемены передач \кпп\, транспортного средства \тс\, регистрационный знак \грз\,
        имеет повреждения  масляного насоса, тормозного барабана, тормозной ленты, подшипника и посадочного места подшипника заднего планетарного механизма,  штока сервопривода и его посадочного отверстия в корпусе АКПП, уменьшение вследствие истирания толщин пакетов фрикционов, утрата эластичности поршней, имеющих эластомерные покрытия; загрязнены гидроблок, соленоиды электрорегуляторы VBS Norm-Low  и VBS Norm-Hig, масляный фильтр.  }
    \item \textbf{Причиной возникновения неисправностей является нарушение работы АКПП вследствие повреждения электрогидравлического блока и механических повреждений деталей АКПП.}
        \item \textbf{Причина возникновения неисправностей АКПП автомобиля \тс\, VIN \тс\,  имеет накопительный характер, обусловлена значительным (несколько десятков тысяч км) перепробегом сервисного интервала замены гидравлической жидкости в АКПП. Является  эксплуатационной и возникла по причине неправильной/ненормальной эксплуатации транспортного средства \тс VIN \vin.}
\end{enumerate}
\vspace{15mm}
\relax


%\vspace{20mm}

%\noindent {Специалист, инженер-механик}\hfill    {   Фефелов С. Л.}\\
{Специалист}\hfill           { Мраморнов А.В.}

\vspace{15mm}
%\noindet{\footnotesize  Страниц документа \pageref{LastPage} }
%\pagebreak
\begin{center}
 \textbf{{\Large    ФОТОТАБЛИЦА ПОВРЕЖДЕНИЙ}}
\end{center}

\фото{example-image}{Автомоиль \тс, вид спереди}
\фото{example-image}{Автомобиль \тс, вид сзади справа}
\дварядом{example-image}{Идентификационная табличка автомобиля \тс, установленная в подкапотном пространстве}{example-image}{Контрольная маркировочная табличка автомобиля \тс, установлена на левой средней стойке }


\фото{example-image}{Демонтированная и разобранная АКПП автомобиля \тс}
\фото{example-image}{Демонтированная и разобранная АКПП автомобиля \тс}

