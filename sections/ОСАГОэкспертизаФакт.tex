% !TeX spellcheck = <none>
\setcounter{page}{1}
\clubpenalty=100000  % Недопуск Висячей строки в начале страницы
\widowpenalty=100000 %Недопуск висячей строки в конце абзаца




\subsection{Описание объекта исследования}

	\par Из предоставленных материалов   экспертом установлена следующая общая информация об автомобиле, имеющая значение для дачи заключения:
 \parbox[]{10cm}{}
\begin{itemize}
	\item[ ] 
	\begin{description}
		\item[Марка, модель] \hfill \тс
    %    \item[Обозначение модели] \hfill \model  
		\item[VIN] \hfill \vin
		\item[Год выпуска] \hfill \год
		\item[Шасси] \hfill отсутствует
		\item[Цвет ЛКП] \hfill \цвет
		\item[Пробег] \hfill  \пробег\, км, считан с одометра
	%	\item[Дата начала эксплуатации] \hfill \началоэкспл
        \item[Дата производства] \hfill \датаизготовления
		\item[Двигатель] \hfill \двигатель
	%	\item[Объем двигателя] \hfill 1328 $ \text{см}^3 $
		\item[Свидетельство о регистрации] \hfill \свид
	%	\item[ПТС] \hfill\птс
	\end{description}
\end{itemize}

\subparagraph*{} Идентификационный код автомобиля (VIN)  \vin \, содержит следующую информацию о транспортном средстве, имеющую значение для 	дачи заключения (Рис. \ref{fig:vin} ):\\[3mm]
%	
%	\noindent\parbox[]{10cm}{
%		\begin{itemize}
%			\item[ ] 
%			\begin{description}
%				\item[Дата изготовления] \hfill \датаизготовления
%				\item[Расположенние руля] \hfill Left
%				\item[Двигатель] \hfill \двигатель
%			%	\item[Объем двигателя] \hfill 1328 $ \text{см}^3 $
%				\item[КПП] \hfill МКПП
%				\item[Тип кузова] \hfill  \типкузова
%				\item[Количество дверей] \hfill 4 
%				%	\item[VDS] --
%					
%			\end{description}
%	\end{itemize}}\\
%	
%\vspace{3mm}
	

Описание модели:
	\begin{figure}[H]
		\centering
		\includegraphics[width=\linewidth]{example-image}
		\caption[]{{\footnotesize Модель  автомобиля VIN \vin\, по данным\textit{ \url{https://www.lexus-tech.eu/LegalNotice.aspx\&vin=}\вин}} }
		\label{fig:vin}
	\end{figure}
	
\vspace{3mm}
%
%\begin{figure}[H]
%    \centering
%    \includegraphics[width=0.65\linewidth]{example-image}
%    \caption[]{Диаграмма рыночной стоимость автомобиля, аналогичного исследуемому \тс\ %, \textsl{источник:} %\url{https://spec.drom.ru}, \url{https://automama.ru/ocenka-avto}
%    }
%    \label{fig:рыночная}
%\end{figure}
%
%Рыночная стоимость  автомобиля, аналогичного исследуемому \тс, по данным специализированных открытых источников на момент повреждения  составляла 590 000 (Пятьсот девяносто тысяч) рублей.\\
%Источник:  \url{https://automama.ru/ocenka-avto}, \url{https://spec.drom.ru}


	\subsubsection{Осмотр транспортного средства}
	
%   \osm\, экспертом-техником проведён осмотр повреждённого транспортного средства \tc, регистрационный знак \grz. Осмотр проводился в сухую, ясную погоду с 10-00  до 10-30 на открытой площадке   по адресу: \местоосмотра.  При осмотре присутствовали \присутствовали, владелец транспортного средства \тс.  Соответствие маркировочных обозначений на кузове представленного ТС записям в регистрационных документах ТС экспертом-техником установлено. Видимые изменения конструкции ТС отсутствуют. Представленный на исследование автомобиль \тс\, регистрационный знак \грз\, имеет кузов типа <<\типкузова». Кузов автомобиля окрашен двухслойной   %лессирующей (с металликовым эффектом) 
%   эмалью (краской)  \colr \, цвета.
   Осмотр транспортного средсвта экспертом не производился ввиду того, что на момент настоящего исследования автомобиль \тс \, регистрационный знак \грз \, восстановлен.
   22.09.2020 авомобиль \тс \, регистрационный знак \грз\, был осмотрен специалистом ООО \enquote{АТБ-Сателлит}. В присутствии представителя ООО \enquote{ОПТИМА} Вовковича Владимира Васильевича составлен Акт осмотра  транспортного средства № 1722125, л.д. 24 и заключение к акту осомтра  № 1722125, л.д. 25. С повреждениями атомобиля участники осмотра согласилсись.  В процессе осмотра производилось фотографирование повреждений автомобиля \тс\, фотокамерой HUAWEI PAR-LX1. Цифровые копии фотоснимков в количестве 78 файлов формата jpg  с сохраненными техническмими данными изображений (EXIF) предоставлены эксперту. Фотоснимки удовлетворительного качества, информативны и пригодны для производства исследования и дачи заключения.
                       
%\subsubsection{Исследование обстоятельств дорожно-транспортного происшествия и установлении причин возникновения повреждений транспортного средства}
%          
%    21.02.2020, водитель автомобиля \tca\, регистрационный номер E682KO123 не убедился в безопасности маневра при перестроении направо и допустил столкновение с автомбилем \тс, двигающимся не меняя полосы движения в попутном направлении  по ул. Гоголя в сторону ул. Красной. По факту дорожно-транспортного происшествия участниками происшествия  составлено извещение о дорожно-транспотном происшествии. Фотоизображения места ДТП представлены ниже на рис. \ref{ris:images/дтп1} и рис. \ref{ris:images/дтп2}.
%    
%    \begin{figure}[!h]\centering
%        \parbox[t]{0.49\textwidth}
%        {\centering
%            \includegraphics[width=.49\textwidth]{example-image}
%            \caption{\footnotesize {Фото места ДТП. На переднем плане автомобиль второго участника ДТП}}
%            \label{ris:images/дтп1}}
%        \hfil \hfil%раздвигаем боксы по горизонтали 
%        \parbox[t]{0.49\textwidth}
%        {\centering
%            \includegraphics[width=.49\textwidth]{example-image}
%            \caption{\footnotesize {Фото места ДТП. Автомобиль \тс}}
%            \label{ris:images/дтп2}}
%    \end{figure}
%    
%    Причины возникновения технических повреждений и возможность их отнесения к
%рассматриваемому ДТП исследованы при осмотре ТС. Для определения причины возникновения повреждений, указанных в Акте осмотра ТС  №  \NomerDoc\, (Приложение № 1) экспертом-техником изучены документы, представленные Заказчиком. По предоставленным документам экспертом установлена причина ДТП, установлены обстоятельства ДТП, выявлены повреждения ТС и установлены причины их образования. Проведено исследование характера выявленных повреждений, сопоставление повреждений ТС потерпевшего с повреждениями ТС иных участников ДТП в соответствии со сведениями, зафиксированными в документах о ДТП.  Проведена проверка взаимосвязанности повреждений на ТС с заявленными обстоятельствами ДТП. 
%
%В результате проведённых исследований эксперт-техник приходит к заключению о соответствии механических повреждений, имеющихся на \тс\, регистрационный знак \грз\, на момент осмотра заявленным обстоятельствам. 


%\subsubsection{Исследование наличия, характера и объёма технических повреждений}
%
%  Наличие, характер и объем технических повреждений транспортного средства \tc\, регистрационный знак \grz, исследованы в присутствии заинтересованных лиц,  зафиксированы в акте осмотра № \NomerDoc\,  (Приложение, <<Акт осмотра>> ),  и фотоматериалах (Приложение, <<Фототаблица>>) по принадлежности. Планируемые (предполагаемые) ремонтные воздействия для восстановления повреждённого  транспортного средства назначены экспертом-техником с учётом особенностей конструкции и рекомендаций изготовителя  транспортного средства, укрупненных показателей трудозатрат по кузовному ремонту и устранению перекосов проёмов и кузова легковых автомобилей иностранных производителей, приложение 3 к приложению к Положению Банка России от 19 сентября 2014 года № 432-П и приведены ниже в таблице \ref{tab:5}.
 
  Повреждения транспорного средства \тс \, регистрационный знак \грз\, определены экспертом по материалам гражданского дела \delonum\, и представлены ниже в таблице \ref{tab:6}
 
  %\pagebreak
  \begin{longtable}{G{3mm}|M{90mm}|G{60mm}}
	\caption[]{\footnotesize {Повреждения автомобиля, установленные при его осмотре}} 
	\label{tab:6}\\ 
	\hline 
	\hline  \toprule 
	\bf  {\footnotesize  n/n}  &\bf {\small Наименование  детали с описанием повреждения} & \bf {\small Изображение} \\   \hline\hline  \toprule \endhead 
	%%%%___________________________________________________________________    
	%\пов{Наименование детали- описание повреждения }{example-image}
\пов{Капот - сложная  деформация панели и каркаса  детали на площади более 80\% поверхности}{example-image}
%\пов{}{example-image}
%\пов{}{example-image}
%\пов{}{example-image}
%\пов{}{example-image}
%\пов{}{example-image}
%\пов{}{example-image}
%\пов{}{example-image}
%\пов{}{example-image}
%\пов{}{example-image}
\end{longtable}\setcounter{rownum}{0}
  
%\subsubsection{Определение стоимости восстановительных расходов}
\subsubsection{Определение стоимости восстановительного ремонта}


 В соответствии с действующей Единой методикой размер расходов на восстановительный ремонт определяется исходя из стоимости ремонтных работ (работ по восстановлению, в том числе окраске, контролю, диагностике и регулировке, сопутствующих работ), стоимости используемых в процессе восстановления транспортного средства деталей (узлов, агрегатов) и материалов взамен повреждённых. Расчёт размера расходов (в рублях) на восстановительный ремонт производится по формуле: 
      
\begin{equation}\label{eq:cr}
C_{\text{вр}}  =\sum{C_{ip}}= \sum\left({T_{ij}}\cdot {C_{i\text{нч}}}\right) + \sum{C_{ip^{\text{\,\,\,руб}}}} , \,\,\,\text{где:} 
\end{equation}
%\vspace{2mm}
\begin{itemize}
	\item[ ]$ C_{ip} $ -- стоимость работ i-го вида: $C_\text {зам} $, $ C_\text{восст} $, $ C_\text{рег} $, $C_\text{контр} $, $ C_\text{антикор} $, $ C_\text{зч} $, $ C_\text{ом} $,$ C_\text{соп} $, $ C_\text{вм} $, руб;
	\item[ ]$ T_{ij} $ -- трудоёмкость j-й операции(комплекса) по i-му виду работ, руб;
	\item[ ]$ C_{i\text{нч}} $ -- стоимость нормо-часа по i-му виду работ, руб;
	\item[ ]$ C_{ip^\text{\,\,руб}} $ -- стоимость работ $ C_{ip} $, принятая непосредственно в денежном выражении, руб.
\end{itemize}

\par При определении стоимости восстановительного ремонта АМТС с учётом износа под износом следует понимать количественную меру физического старения АМТС и его элементов, достигнутого в результате эксплуатации, т.е. эксплуатационный износ.
%
Расчёт износа производится в  соответствии с Положением Банка России от «19» сентября 2014 года № 432-П «О единой методике определения размера расходов на восстановительный ремонт в отношении повреждённого транспортного средства» [3].
Износ комплектующих изделий (деталей, узлов, агрегатов) рассчитывается по следующей формуле:
%
%
%
\begin{equation}\label{eq:I}
\text{И}_{\text{ки}} 
= 100\cdot\left( 1-e^ {-\left( \Delta_{T} \cdot T_{\text{КИ}} + \Delta_{L} \cdot L_{\text{КИ}} \right)}\right), \,\,\,\,\text{где:}   
\end{equation}
%
\begin{itemize}
	\item[ ]$ \text{И}_{\text{ки}} $ -- износ комплектующего изделия (детали, узла, агрегата) (процентов); 
	\item[ ]$ e $ -- основание натуральных логарифмов (e =  2,72);
	\item[ ]$ \Delta_{T}$ --  коэффициент, учитывающий влияние на износ комплектующего 	изделия (детали, узла, агрегата) его срока эксплуатации;
	\item[ ]$ T_{\text{КИ}} $ -- срок эксплуатации комплектующего изделия (детали, узла, агрегата), (лет);
	\item[ ]$ \Delta_{L} $ -- коэффициент, учитывающий влияние на износ комплектующего (детали, узла, агрегата) величины пробега транспортного средства с этим комплектующим изделием;
	\item[ ]$ L_{\text{КИ}} $ -- пробег транспортного средства на дату дорожно-транспортного происшествия (тысяч километров).  
\end{itemize}
\vspace{5mm}
\par Значения коэффициентов $ \Delta_{T}$  и $ \Delta_{L} $  для различных категорий и марок транспортных средств приведены в Приложении 5. сп. лит~[3]. При этом, на комплектующие изделия (детали, узлы, агрегаты), которые находятся в заведомо худшем состоянии, чем общее состояние транспортного средства в целом, и его основные части, вследствие влияния факторов, не учтённых при расчёте износа (например, проведение ремонта с нарушением технологии, не устранение значительных повреждений лакокрасочного покрытия), может быть начислен дополнительный индивидуальный износ. 
Износ шины транспортного средства рассчитывается по следующей формуле:
\begin{equation}\label{eq:sh}
\text{И}_{\text{ш}} = \frac{\text{Н}_{\text{н}}-\text{Н}_{\text{ф}}}{\text{Н}_{\text{н}}-\text{Н}_{\text{доп}}} \cdot{100}\%,  \,\,\,\,\text{где:} 
\end{equation}
%
\begin{itemize}
	\item[ ] $ \text{И}_{\text{ш}} $ -- износ шины, \%;
	\item[ ] $ \text{Н}_{\text{н}} $ -- высота рисунка протектора новой шины, мм;
	\item[ ] $\text{Н}_{\text{ф}} $ -- фактическая высота рисунка протектора шины, мм;
	\item[ ] $ \text{Н}_{\text{доп}} $ --минимально допустимая высота рисунка протектора шины в соответствии с требованиями законодательства Российской Федерации, мм.
\end{itemize}
%
\vspace{5mm}
\relax
%\renewcommand\baselinestretch{1}\small\normalsize
%
Износ шины дополнительно увеличивается для шин с возрастом от 3 до 5 лет - на 15 процентов, свыше 5 лет - на 25 процентов.

                                                 
\subsubsection{Данные для расчёта}

\noindent Объект экспертизы:  транспортное средство \tc\,
регистрационный знак \грз;\\ 
VIN: \вин;\\
Пробег:    \пробег\,, км;\\
Год выпуска:     \год;\\ 
Дата  производства:  \датаизготовления;\\
Дата ДТП:  \датадтп;\\

На момент ДТП ввтомобиль эксплуатаировался с 01.02.2018 по 09.04.2020: 
\begin{itemize}
    \item[ ]$ T_{\text{КИ}} = 2.5 $ 
    \item[ ]$ L_{\text{КИ}} = 57.646$
    \item[ ]$ \Delta_{T} = 0.045 $ 
	\item[ ]$ \Delta_{L} = 0.0024 $
	\end{itemize}
\begin{equation}\label{eq:I}
	\text{И}_{\text{ки}} 
	= 100\cdot\left( 1-e^ {-\left( \Delta_{T} \cdot T_{\text{КИ}} + \Delta_{L} \cdot L_{\text{КИ}} \right)}\right) = 100\cdot\left( 1-e^ {-\left( 0.045 \cdot 2.5 + 0.0024 \cdot 57.646 \right)}\right) =  23.92 \%  
\end{equation}

Перечень ремонтных воздействий, необходимых для устранения повреждений ТС \тс\, представлен ниже в таблице \ref{tab:61}:\\
%Рыночная стоимость ТС: \tc\,
%регистрационный знак \grz \, по данным открытых специализированных информационных источников составляет: $640 000$ (Шестьсот сорок тысяч) рублей.\\
%
%
%    \begin{equation}\label{eq:I}
%    \text{И}_{\text{ки}} 
%    = 100\cdot\left( 1-e^ {-\left( \Delta_{T} \cdot T_{\text{КИ}} + \Delta_{L} \cdot L_{\text{КИ}} \right)}\right), \,\,\,\,\text{где:}   
%    \end{equation}
%   
%  
%\pagebreak
%\subsubsection{Ремонтные воздействия, необходимые для устранения повреждений}

\setcounter{rownum}{0}

\begin{longtable}{G{3mm}|M{130mm}|G{5mm}|G{5mm}|G{5mm}}
	\caption[]{Ремонтные воздействия, необходимые для устранения повреждений ТС \тс}
	\label{tab:61}\\
	\hline  \hline   \toprule 
	\bf  {\footnotesize  n/n}  &\bf {\small Наименование  детали и описание повреждения} & \bf {\small E} & \bf {\small I} & \bf {\small L}\\\hline \hline \toprule  \endhead 
	
	
%%%%______________________________________%%%%%%%%%%%%
%%%%%%%%%   ОПИСАНИЕ ПОВРЕЖДЕНИЙ   
%\\ps{ деталь - повреждение }{E}{I}{L} 
\акт{Крыло переднее правое }{}{\7}{\7}
\акт{Бампер передний }{\7 }{ }{\7 }
\акт{Кронштейн бампера переднего правый }{\7 }{ }{ }
\акт{Диск колеса правого переднего семиспицевый }{\7 }{ }{ }
\акт{Фара правая }{\7 }{ }{ }
\акт{Облицовка крыла правого переднего }{\7 }{ }{ }
\акт{Спойлер бампера переднего правая часть }{\7 }{ }{ }
\акт{Пленка  защитная антигравийная бампера переднего и крыла правого }{\7 }{ }{ }


\end{longtable}\setcounter{rownum}{0} 
	
	\textbf{\textit{E - заменить деталь, I - ремонтировать, L - окрасить} }
	
		
	%
	\renewcommand\baselinestretch{1.2}\small\normalsize 
%
\subsubsection{ Расчёт}
    
\indent Полный расчёт стоимости восстановительных расходов на ремонт ТС с учётом износа в соответствии с правилами обязательного страхования гражданской ответственности владельцев транспортных средств выполнен в  лицензированном для решения задач в рамках ОСАГО программном комплексе   SilverDAT myClaim и приведён в Калькуляции № \NomerDoc.
 Расчёт износа произведён программой  SilverDAT и представлен  в калькуляции расчёта затрат № \NomerDoc.
\indent Результаты расчёта  стоимости восстановительных расходов ТС \тс\, \грз\, представлены ниже:\\
  
  
\begin{figure}[H]
        	\centering
        	\includegraphics[width=0.95\linewidth]{example-image}
    %    		\caption{}
    %    		\label{fig:screenshot001}
        \end{figure}
  
    %
    \begin{figure}[H]
    	\centering
    	\includegraphics[width=0.95\linewidth]{example-image}
%    		\caption{}
%    		\label{fig:screenshot001}
    \end{figure}
    \begin{figure}[H]
    	\centering
    	\includegraphics[width=0.95\linewidth]{example-image}
%    		\caption{}
%    		\label{fig:screenshot002}
    \end{figure}
    \medskip
    \renewcommand\baselinestretch{1.2}\small\normalsize
    

\subparagraph{}Стоимость одного нормо-часа работ определена в соответствии с пунктом 3.8.1 Единой методики [3] путём применения электронных баз данных стоимостной информации, составляющая на момент ДТП 1100 руб/час.
Трудоёмкость работ по разборке/сборке/замене  соответствует трудоёмкостям работ, рекомендованным заводом изготовителем ТС. Трудоёмкости окрасочных работ приняты согласно рекомендаций Единой методики, п.3.7.1. в соответствии с технологией  AZT (\url{http://www.schwacke.ru/down/azt _reparaturlackierung_ru.pdf}). Расчёт размера расходов на материалы произведён  согласно пункту 3.7.2 Приложения к Единой методике [3]. Артикулы запасных частей определены с помощью программы Audatex и электронных  каталогов запасных частей \url{emex.ru}, \url{partsouq.com}.
Стоимость запасных частей определена в соответствии с пунктом 3.6.3 Единой методики путём применения электронных баз данных стоимостной информации (по утверждённому справочнику: \url{http://prices.autoins.ru/priceAutoParts/repair_parts.html} ).

Стоимость работ и материалов по нанесению качественной (3M VentureShield, SunTek, Hexis Bodifence, Stek, Spectroll ) защитной пленки на кузовные элементы атомобиля  \тс \, регистрационный знак \грз\,- передний бампер,  правое переднее крыло и правая фара определена на основании анализа рынка соответствующих услуг, оказываемых специализированными предприятиями г. Краснодара: \url{https://unicar23.ru/prajs-list/},
\url{https://vinyl-jam.ru/}, \url{http://protuning-company.ru/avtovinil-krasnodar.html}, \url{https://miart-detailing.com/uslugi/oklejka-vinilom/}, \url{https://www.vinylmonster.ru/krasnodar}, \url{https://autostudio23.ru/zashchita-kuzova-avtomobilya/}
 и составляет 20 000 рублей.

  
\subparagraph{}Таким образом,  наиболее вероятная стоимость ремонта транспортного средства \tc\, регистрационный знак \грз, получившего повреждения в результате дорожно-транспортного происшествия  \датадтп\, составляет $277 032$ (Двести семьдесят семь тысяч тридцать два) рубля,  размер затрат на восстановительный ремонт ТС с учётом износа составляет  $ 219 601 $ (Двести девятнадцать тысяч шестьсот один) рубль, или с учётом округления составляет $ 219 600 $ (Двести девятнадцать тысяч шестьсот) рублей.
      
% \input {sections/рынокОСАГО}
%\input {sections/утсОСАГО}
%\input {sections/годныеОСАГО}

\section{В ы в о д ы}

 Стоимость восстановительного ремонта, в рамках закона об ОСАГО,  автомобиля LEXUS RX300, государственный номер У755ТТ123, VIN:~JTJZAMCA302037447, 2018 года выпуска, цвет белый от полученных повреждений в результате ДТП, произошедшего 09.08.2020   составляет  $ 219 600 $ (Двести девятнадцать тысяч шестьсот) рублей.
    
\vspace{10mm}

 \noindent Эксперт  \hfill     \rule{4cm}{0.1 mm} \,\,\,  Мраморнов А.В.
 


\vspace{9mm}

\relax
\noindent Приложение к заключению:\\
\textit{
%	Приложение № 1. Расшифровка модельных опций ТС \тс \\
 %   Приложение № 1. Акт осмотра ТС \тс\\
 %   Приложение № 2. Фототаблица повреждений ТС\\
	Приложение № 1. Калькуляция стоимости восстановительных расходов ТС \тс\\
%	Приложение № 4. Цифровые копии регистрационных документов ТС\\
%	Приложение № 4. Цифровая копия постановления по делу об административном правонарушении дорожно-транспортном происшествии\\
	Приложение № 2. Правоустанавливающие документы\\
}

%\includepdf[pages=-]{myfile.pdf}
%\includepdf[pages=-]{calc.pdf}