\setcounter{page}{1}
\clubpenalty=100000  % Недопуск Висячей строки в начале страницы
\widowpenalty=100000 %Недопуск висячей строки в конце абзаца

%%%%%%%%%%%%%%%%%%%%%%%%%%%%%%%%%%%%%%%%%
%
%   Экспертная организация ООО Южнорегиональная экспертная группа
%
%%%%%%%%%%%%%%%%%%%%%%%%%%%%%%%%%%%%%%%%%
%\noindent\qrcode[height=21mm]{\NomerDoc от \dataend } 
\noindent %\qrcode[height=21mm]{\NomerDoc от \окончено }  %%% Добавлен QR-Code
\begin{pspicture}(21mm,21mm)
\obeylines
\psbarcode{%
	%\NomerDoc от \окончено
	BEGIN:VCARD^^J
	VERSION:4.0^^J
	%N:Мраморнов; Александр; Вчеславович^^J
	FN:Александр Мраморнов^^J
%	ORG:IP Alexandr Mramornov^^J
	TITLE: эксперт
	ORG: ИП
	URL:http://www.yourexp.ru^^J
	EMAIL:4516611@gmail.com^^J
	TEL:+7-918-451-6611^^J
	ADR:г. Краснодар, с/т № 2 А/О «Югтекс», ул. Зеленая, 472^^J
	END:VCARD
}{width=1.0 height=1.0}{qrcode}%
\end{pspicture}
 %%% Добавлен QR-Code
\vspace{-4mm}
\begin{center}
	\large\textbf{ИНДИВИДУАЛЬНЫЙ\quad ПРЕДПРИНИМАТЕЛЬ  \\[-1.5mm] МРАМОРНОВ  АЛЕКСАНДР ВЯЧЕСЛАВОВИЧ \\[-5.5mm]}
	%  
	\noindent\rule{\textwidth}{2pt}\\[-6mm]  % Горизонтальная линия
	% \line(1,0){460}% (1,0) -горизонтальная линия, и (0,1) - вертикальная 
\end{center}

\begin{center}
	\begin{footnotesize}\setstretch{0.3}
		%	\small\textbf\setlength   	%\raisebox{5mm}
		\vspace{-3.5mm}г. Краснодар, с/т № 2 А/О «Югтекс», ул. Зеленая, 472, 
		Телефон: 8-918-451-66-11, e-mail: 4516611@gmail.com\\ [-2mm]{ИНН\quad 231200665168\quad ОГРНИП \quad 310231220400043}
	\end{footnotesize}	\\[10mm]
\end{center}


\begin{flushright}
% 
	 \hfill	Краснодар, 2023    \\[8mm]
\end{flushright}
\begin{center}
	\LARGE\textbf{ЭКСПЕРТНОЕ ЗАКЛЮЧЕНИЕ}
	\bigskip\\[0mm]
	%	{\normnumxtbf{\NomerDoc}}	}{den}
\end{center}
\par
\vspace{-6mm}
%%%%%%%%%%%%%%%%%  ОСАГО
%\noindent независимой технической экспертизы по определению размера расходов на восстановительный ремонт транспортного средства   \тс  \\[2mm]
%%%%%%%%%%%%%%%%% НЕ ОСАГО
\noindent независимой технической экспертизы по определению размера ущерба, причинённого владельцу транспортного средства в результате дорожно-транспортного средства\\[2mm]
%%%%%%%%%%%%%%%%%%%%%%%%%
%\raggedright 
%\def\hrf#1{\hbox to#1{\hrulefill}}
\noindent \textbf{№ \NomerDoc}\hfill           \textbf{\dataend}\\%[2mm]
%Приостановлено\hfill      \datastop\\
%Возобновлено\hfill          \datarestart\\
%Окончено\hfill                \dataend\\%[4mm]

\noindent\parbox[l][16mm]{16.5cm}
{\def\hrf#1{\hbox to#1{\hrulefill}}
	\noindent Начато\hfill            \datastart\\%[2mm]
	%	Приостановлено\hfill      \datastop\\
	%	Возобновлено\hfill          \datarestart\\
	Окончено\hfill                \dataend
}
\relax

%%%%%%%%%%%% Если судебка
%
%\datastart г. ~в {\small ООО~ "ЮЖНО-РЕГИОНАЛЬНАЯ ЭКСПЕРТНАЯ ГРУППА"} \,  при определении  \, \sud  \,  от \, \dataopr \, о назначении \opr \, по гражданскому делу \delonum \, поступили:
%
%\begin{enumerate}\setlist{nolistsep}\item  Материалы гражданского дела \delonum \, в двух томах, том 1 на 276 листах, том 2  на 143 листах.\\[-2mm]
%	%	\item  
%\end{enumerate}
%
%%%%%%%%%%%%  Если независимая
\vspace{4mm}
Составлено на основании	договора № \NomerDoc\, от \dog. воздмездного оказания услуг по проведению независимой технической экспертизы (далее экспертиза)  транспортного средства и письменного заявления заказчика о проведении экспертизы. 

Заказчик  экспертизы:  \заказчик,  \адресзаказчика.

% Документ, удостоверяющий личность заказчика: водительское удостоверение    03\ 16\ 422344\ выдан 09.06.2011

%Транспортное средство виновника ДТП:  не предоставлялось.

\paragraph*{}
Экспертиза произведена  экспертом--техником
%{\small ООО "ЮЖНО-РЕГИОНАЛЬНАЯ ЭКСПЕРТНАЯ ГРУППА"}
\,  Мраморновым Александром Вячеславовичем, имеющим высшее  образование по специальности «техническая физика», диплом РВ №311964 от 28.02.1989, квалификация -- инженер-физик, специальное образование в области оценки: Диплом ПП-1 № 037211 Российской экономической академии им. Г.В. Плеханова, квалификация -- оценка и экспертиза объектов и прав собственности, специальное образование в области независимой технической экспертизы транспортных средств: Диплом ПП-I № 424167, квалификация: эксперт-техник (специализация 150210 специальности 190601.65 – Автомобили и автомобильное хозяйство), состоящий в Государственном реестре экспертов-техников (№ в реестре 256, https://data.gov.ru/opendata/7707211418-experts,  общий трудовой  стаж 30 лет, стаж  экспертной работы  12 лет. \par Заключение подготовлено по месту фактического расположения ИП по адресу: г. Краснодар, с/т № 2 А/О «Югтекс», ул. Зеленая, 472.
  % Шапка организации ИП если НЕТ ОСАГО

%%%%%%%%%%%%%%%%%%%%%%%%%%%%%%%%%%%%%%%%   вопросы экспертизы
%\subsection{Вопросы экспертизы}
%\subsection{Вопрос исследования}
\begin{enumerate}
	\item <<Какова стоимость восстановительного ремонта, в рамках закона об ОСАГО,  автомобиля LEXUS RX300, государственный номер У755ТТ123, VIN:~JTJZAMCA302037447, 2018 года выпуска, цвет белый, от полученных повреждений в результате ДТП, произошедшего 09.08.2020>>?
%\item  <<Установить наличие, характер и объем (степень) технических повреждений транспортного средства  \tc?>>
%\item  <<Установить причины возникновения технических повреждений транспортного средства \tc \,и возможность их отнесения к рассматриваемому дорожно-транспортному происшествию (далее ДТП)?>>
%\item <<Установить технологию, объем восстановительного  ремонта \!транспортного средства \tc?>>
%\item <<Установить размер затрат на восстановительный ремонт (с учётом износа) транспортного средства \tc?>>
%\item <<Определить размер ущерба, причиненного владельцу  транспортного средства \tc\,\грз\, \, в результате дорожно-транспортного происшествия, имевшего место \датадтп?>>
%\item <<Определить стоимость восстановительного ремонта  транспортного средства \tc\, регистрационный знак \грз,\, \, получившего механические повреждения в результате противоправных действий, имевших место \датадтп?>>
%\item <<Определить величину физического износа  транспортного средства \tc\,\грз\, \, получившего повреждения в результате дорожно-транспортного происшествия, имевшего место \датадтп?>>
%%
%	
%	\item
%	Какие неисправности имеет двигатель \двигатель\, самоходной машины  KOMATSU   АВТОПОГРУЗЧИК  FG15T-20   2007 года выпуска, заводской номер 661043?
%	
%	\item
%	Какова причина их возникновения?
%	
%	\item
%	Причина выхода из строя  двигателя \двигатель\,  имеет производственный или эксплуатационный характер?
%	
%	\item
%	Могли ли имеющиеся у  двигателя \двигатель\, неисправности возникнуть вследствие капитального ремонта?
%	\item  <<Связано ли повреждение панели рамки радиатора слева и брызговика с лонжероном переднего левого автомобиля ВАЗ 21099 с указанным ДТП?>>	
% \item  <<Что послужило причиной выхода двигателя автомобиля из строя?>>	   
%    \item  <<Является ли данная причина:
%\begin{itemize}
%        \item производственной, т.е. недостатком сборки и/или материала;
%        \item связанной с некачественным/несвоевременным обслуживанием автомобиля, включая ежедневный осмотр;
%        \item связанной с неразрешенными/недопустимыми переделками агрегата и/или его систем;
%        \item связанной с предыдущим ремонтом (если применимо);
%        \item эксплуатационной, т.е. возникшей по причине неправильной/ненормальной эксплуатации;
%        \item  естественным износом в соответствии с пробегом автомобиля?>> 
% 	\end{itemize}
%	
\end{enumerate}
%%%%%%%%%%%%%%%%%%
% ИТОГИ РСЧЕТА    
\def\итог{47426}
\def\итогизнос{47426}
\def\рынок{1000000}
\def\нормочас{930}

\subsection{Исходные данные и объекты исследования} 
Исходные  данные,  необходимые  для производства  исследования,  изложены   заказчиком при подаче заявления о проведения исследования   колесного  транспортного  средства (далее —  KTC):\\
- на момент повреждения \датадтп\, автомобиль \тс\,\грз\, находился в рабочем состоянии;\\
- автомобиль полностью комплектный;\\
- внешнее состояние автомобиля соответствует возрасту и пробегу.
   

	Для проведения исследования представлено:\\
\begin{enumerate}
\item Транспортное средство \тс\, \грз \, в поврежденном состоянии
\item Паспорт транспортного средства (ПТС) \птс
%\item Копия постановления \постановление об административном правонарушении дорожно-транспортном происшествии, имевшем место   \датадтп \, с участием  ТС \тс,\,\грз \, согласно которому  в результате дорожно-транспортного происшествия автомобиль \тс\, \грз\, получил механические повреждения. Повреждено:\, "\повреждения".
%\item Полис страхования  ОСАГО \polis.
	\end{enumerate}
%
%\subsection*{Обстоятельства происшествия}
%\datadtp г. автомобиль \тс \, г/н \грз\, под управлением водителя ****************************************
%%%%%%%%%%%%%%%%%%%%%%  
%
%\left( \addcontentsline{toc}{section}{Использованные нормативы и источники информации}

\subsection{Использованные нормативы и источники информации}
%
\begin{enumerate}
\item 
Махнин\,Е.\,Л., Новоселецкий\, И.\,Н., Федотов\, С.\,В. \emph{Методические рекомендации по проведению судебных автотехнических экспертиз и исследований колёсных транспортных средств в целях определения размера ущерба, стоимости восстановительного ремонта и оценки} // -- М.: ФБУ РФЦСЭ при Минюсте России, 2018.-326 с.  ISBN 978-5-91133-185-6.
%
\item ТУ 017207-255-00232934-2014 \emph{Кузова автомобилей LADA. Технические требования при приёмке в ремонт, ремонте и выпуске из ремонта предприятиями дилерской сети ОАО "АВТОВАЗ"}//  ОАО НВП "ИТЦ АВТО", 2014
%
\item Смирнов  В.Л., Прохоров  Ю.С., Боюр В.С.  и др. \emph{Автомобили ВАЗ. Кузова. Технология ремонта, окраски и  антикоррозионной защиты. Часть II}// - Н.Новгород: АТИС, 2001.- 241с.
%
\item 
Савич Е.Л. \emph{Техническое  обслуживание  и  ремонт  легковых  автомобилей} : учеб. пособие / Е.Л. Савич, М.М. Болбас, В.К. Ярошевич ; под общ. ред. Е.Л. Савича. -Мн. : Вышэйшая школа,  2001. - 479 с. - ISBN985-06-0502-2.
%
\item 
Автомобили ВАЗ-2121, 21213, 21214, 2131 и их модификации: <<Трудоемкости работ (услуг) по техническому обслуживанию и ремонту>> /Куликов А.В., Христов П.Н., Климов В.Е.,  Боюр В.С., Рева В.В., Зимин В.А., Завьялова Н.Н., Хлыненкова Г.А. -- ИТЦТ "АвтоВАЗтехобслуживание", Тольяти -- 2005. 
%
\item
Автомобили LADA SAMARA и их модификации: <<Трудоемкости работ (услуг) по техническому обслуживанию и ремонту>> /Куликов А.В., Христов П.Н., Климов В.Е., Рева В.В., Боюр В.С., Васильев М.В., Фахрутдинов Р.В.,  Прудских Д.А., Гирко В.Б., Шмелева В.А., Зимин В.А. --  ОАО НВП "ИТЦ АВТО",  -- 2006. - 252 стр.
%
\item 
Автомобили LADA PRIORA. Трудоемкости работ (услуг) по техническому обслуживанию и ремонту /Куликов А.В., Христов П.Н., Климов В.Е., Рева В.В., Козлов П.Л., Боюр В.С., Прудских Д.А., Шмелева В.А., Зимин В.А. -- ООО "ИТЦТ АВОСФЕРА", Тольяти -- 2009. -- 344 с.
%
\item 
{Трудоемкости работ по техническому обслуживанию и ремонту автомобилей автомобилей Lada  Granta}/   \url{https://docplayer.ru/30250248-Trudoemkosti-rabot-po-teh\-nicheskomu-obsluzhivaniyu-i-remontu-avtomobiley-lada- granta.html}.
%
%
\item
{Специализированное программное обеспечение для расчёта стоимости  восстановительного ремонта, содержащее нормативы трудоёмкости работ, регламентируемые изготовителями транспортного средства}//   AudaPadWeb, лицензионное соглашение № AS/APW-658  RU-P-409-409435.
%
%
%
\item

{Специализированное программное обеспечение для расчёта стоимости  восстановительного ремонта, содержащее нормативы трудоёмкости работ, регламентируемые изготовителями транспортного средства ОАО «АвтоВАЗ», ЗАО «Джи-Эм-АвтоВАЗ», ОАО «СеАЗ» и ОАО «ЗМА»}//   Автосфера АС:Смета, v.3.9.11// ООО "ИТЦ «ИнтегроМаш», \url{https://autosmeta.pro}.
%
%
%
\item Информационный портал по техническому обслуживанию и ремонту автомобилей	 ВАЗ:\\ \url{www.autosphere.ru}.

%%
\end{enumerate}
 %%%%   БИБЛИОГРАФИЯ
%
%%%%%%%%%%%%%%%%%%%%%%%%%%%%%%%%%%%%%%%%%%%%%%%%%%%%%%%%%%%%%%%%%%%%%%%%%%%%%%%%%
\subsection{Технические средства}  %% Список не удалять!!!
%
\begin{itemize}
%
%\item Диагностический сканер SDconnect   с программным обеспечением Xentry Diagnostics v19.11.3.1
\item  Линейка масштабная магнитная с цветографической шкалой, 100мм
\item  Рулетка измерительная металлическая, 5м.
%%\item Универсальный стенд для измерения углов установки колес Hunter Engineering %ProAlign с программным инструментом регулировки схождения колес без блокировки руля %автомобиля WinToe
\item Цифровой фотоаппарат Canon 760D s/n 143032001327 с объективом Canon EF-S 18-135, тип используемой памяти: Transcend,  32Gb.
%
%\item Специализированное программное обеспечение для расчёта стоимости  восстановительного ремонта, содержащее нормативы трудоёмкости работ, регламентируемые изготовителями транспортного средства     AudaPadWeb, лицензионное соглашение № AS/\- APW-658  RU-P-409-409435.

\item  Специализированное программное обеспечение для расчёта стоимости  восстановительного ремонта, содержащее нормативы трудоёмкости работ, регламентируемые изготовителями транспортного средства  SilverDAT myClaim, лицензионный договор № 1422 от 05.02.2022 на право использования программы для ЭВМ от  DAT IP-Management und Vertriebs GmbH.
%
\item  Программа обработки фото-видео изображений ImageJ, разработчик  Wayne Rasband (wa-yne@codon.nih.gov),
свободная лицензия GPL.
%
\item  ПЭВМ под управлением операционной системы Windows 10 с установленным набором макрорасширений LaTeX системы компьютерной вёрстки TeX, cвободная лицензия LaTeX Project Public License (LPPL). 
%	
	\end{itemize}
%%%%%%%%%%%%%%%%%%%%%%%%%%%%%%%%%%%%%%%%%%%%%%%%%%%%%%%%%%%%%%%%%%%%%%%%%%%%%%%%%%%%%%%%%%%%%%%
\subsection{Условные обозначения}
\begin{description}
%	 
%%\item[АВС] --Антиблокировочная система
\item[АМТС] --автомототранспортное средство
\item[ДВС] --двигатель внутреннего сгорания
\item[ДТП] --дорожно--транспортное происшествие
\item[гос.\,рег.\,знак] -государственный регистрационный знак
\item[КТС] --колесно транспортное средство 
\item[ЛКП] --лакокрасочное покрытие
%\item[л.д.] --Лист дела
%%\item[Колесо турбины]  -- крыльчатка турбины
\item[ТС] --транспортное средство
%\item[ТK, ТКР] -- Турбокомпрессор. Состоит из двух частей: турбины и компрессора, объединенных общим валом. Вал вращается в подшипниках, размещенных в центральном корпусе ТК
%\item[УУК] -- Углы установки колес
%\item[ЭБУ] --Электронный блок управления
%\item[DTC] --Diagnostic Trouble Codes, диагностические коды неисправностей
%\item[FRAME] --номер кузова транспортного средства, выпущенного для продажи на внутреннем рынке Японии и содержащий информацию производителя о транспортном средстве
%\item[OBDII] -- On-board diagnostics, протокол бортовой диагностики автомобиля
%\item[SRS] -- система пассивной защиты водителя и пассажиров
\item[VIN] --vehicle identification number, 17--значный идентификационный номер транспортного средства, соответствующий стандарту ISO 3779--2012.
%
\end{description}
%
\subsection{Термины и определения}
\begin{description}
	\item[Аварийные повреждения] --- повреждения, механизм образования которых определяется контактом с посторонними объектами, что привело к деформации или разрушению и к необходимости ремонта или замены составной части, или контактам с агрессивной средой, которая привела к необходимости ремонта (замены) составной части [1, часть II, п. 1.5].
	\item[Восстановительный ремонт]--- один из способов возмещения ущерба, состоящий в выполнении технологических операций ремонта КТС, действующий в сети торгово-сервисного обслуживания, созданной изготовителем этого КТС [1, часть II, п. 1.4].
%	\item[Годные остатки] --- работоспособные, имеющие остаточную стоимость детали (агрегаты, узлы) поврежденного автотранспортного средства, годные к дальнейшей эксплуатации, которые можно демонтировать с поврежденного автотранспортного средства и реализовать.
	\item[Дата исследования (оценки)]--- дата, на которую проводятся расчеты и используются стоимостные данные КТС, запасных частей, материалов, нормо-часа ремонтных работ [1, часть II, п. 1.5].
	\item[Срок эксплуатации КТС]--- период времени от даты изготовления (даты выпуска) КТС, до даты оценки (исследования), определяемой условиями задачи исследования (независимо от даты его регистрации и начала использования по назначению (эксплуатации)).
%	\item[Утрата товарной стоимости (УТС)]--- условное снижение рыночной стоимости КТС, восстановленного в соответствии с нормативными требованиями после повреждения, по сравнению с рыночной стоимостью подобного неповрежденного КТС. 
	\item[Характер повреждения]--- качественная характеристика повреждения, определяющая сущность и природу изменения состояния ТС (составной части)

\end{description}
\subsection{Методы исследования}
\begin{itemize}
\item  Органолептический метод – исследование и оценка качества объектов с помощью органов чувств.
\item 	Прямой измерительный метод – путём измерения  размеров  %деталей специальными %
мерительным инструментом %и специальными приборами
\item Расчётный метод (косвенный измерительный метод) – путём расчётов различных параметров на основе результатов измерений и других данных.
\item Экспертный метод (метод экспертной оценки) — совокупности операций по выбору комплекса или единичных характеристик объекта, определению их действительных значений и оценкой экспертом соответствия их установленным требованиям и/или технической информации.
%%	\item Метод натурной реконструкции
\end{itemize}
%
%
\subsection{Ограничения и пределы применения полученных результатов}

Следующие допущения и условия, ограничивающие пределы применения полученных результатов, являются неотъемлемой частью данного экспертного заключения.
\begin{itemize}
\item  {Результаты, полученные экспертом-техником, носят рекомендательный консультационный характер и не являются обязательными. Исполнитель высказывает своё субъективное суждение о наиболее вероятных будущих (абстрактных) расходах, их предполагаемом размере и дает заключение в пределах своей компетенции.}
\item { Под компетенцией эксперта-техника понимают его знания и опыт в области теории и методов экспертных исследований ТС, а также круг полномочий, представленных ему законом, и вопросов, которые он может решать на основе своих специальных познаний.
В компетенцию эксперта-техника входит исследование технического состояния поврежденного ТС в целях установления характера повреждений ТС, установления причины возникновения технических повреждений технологии, методов, стоимости его ремонта.}
\item  {Исполнитель в рамках своих обязательств по заключенному договору об экспертном обслуживании признает свою ответственность перед заказчиком и настоящим утверждает, что экспертное заключение выполнено профессионально, тщательно и с должной заботливостью и внимаем, %как это обычно принято для компетентного специалиста в области технической экспертизы ТС при ОСАГО,
а полученная величина восстановительных расходов, разумна и реальна.}
\item  {Исполнитель считает, что поскольку, по общему правилу, оценка доказательств является прерогативой и компетенцией органа дознания, следствия или суда, а в досудебном порядке - страховщика, постольку после проверки результатов экспертизы последним, их признания и принятия решения о выплате страхового возмещения этап возможного оспаривания достоверности исследований между заказчиком и исполнителем завершен.
Соответственно, обязанности Исполнителя по договору являются надлежаще исполненными в полном объеме и от исполнителя не требуется свидетельствовать по поводу произведённого исследования перед третьими лицами.}
\item  {Отдельные части настоящего экспертного исследования не могут трактоваться раздельно, а только в связи с полным текстом о проведенных расчетах.}
\item  {Исходные данные, использованные исполнителем при подготовке экспертного заключения, получены из надежных источников и считаются достоверными. Тем не менее, исполнитель не может гарантировать абсолютную точность, поэтому там, где это, возможно, делаются ссылки на источники информации.}
\item  {В процессе экспертного исследования специальная юридическая экспертиза документов, касающихся прав собственности на ТС, не проводилась.}
\item  {Суждения, содержащиеся в экспертном заключении, основываются на текущей ситуации на дату аварии и, в будущем, могут быть подвержены изменениям.
Исполнитель не принимает на себя никакой ответственности за изменение экономических, юридических и иных факторов, которые могут возникнуть после даты исследования и повлиять на результаты технической экспертизы.
\item При анализе скрытых повреждений экспертом-техником не принимается во внимание наличие или отсутствие записей о них в документах компетентных органов, в связи с отсутствием у сотрудников компетентных органов объективной возможности  идентификации таких повреждений на месте происшествия.
%Данное заключение составлено на основании Правил Независимой Технической Экспертизы и может применяться только при решении вопроса о выплате страхового возмещения по ОСАГО.
}\end{itemize}
%
%
\section{Исследование}
%
Настоящее исследование проводится на основании материалов, предоставленных Заказчиком, а также на основании данных, самостоятельно полученных экспертом-техником. Выводы, содержащиеся в настоящем Заключении, могут расцениваться как достоверные только в контексте того количества информации, на основании которого они были сделаны. При поступлении дополнительной или измененной информации данные выводы могут быть
скорректированы. 

\par  В качестве экспертной методики принимаются \emph{Методические рекомендации по проведению судебных автотехнических экспертиз и исследований колесных транспортных средств в целях определения размера ущерба, стоимости восстановительного ремонта и оценки} [1].


\par Размер ущерба (У) вследствие повреждения КТС принимается равным рыночной стоимости транспортного средства, если выполняется условие (\ref{11}):

\begin{equation}\label{11}
	C_\text{вр} + C_\text{утс} \geq  C_\text{ктс},    \text{где}
\end{equation}

$  C_\text{ктс} $ --  рыночная стоимость КТС, руб;

$ C_\text{вр}  $ -- стоимость восстановительного ремонта КТС, руб;

$ C_\text{утс} $ --  величина УТС, руб.

В этом случае, рыночная стоимость КТС определяется на заданную дату оценки, с учётом срока его эксплуатации технического состояния на момент происшествия.

В остальных случаях размер ущерба определяется по формуле (\ref{112}):

\begin{equation}\label{112}
	\text{У} = C_\text{вр} + C_\text{утс},    \text{руб.}
\end{equation}


 \par Из предоставленных материалов   экспертом-техником установлена следующая общая информация об автомобиле, имеющая значение для дачи заключения:\\
 \parbox[]{10cm}{}
	\begin{itemize}
		\item[ ] 
			\begin{description}
			\item[Марка, модель] -- \тс
			\item[VIN] -- \vin
			\vspace{6mm}
			%\textit{Источник: https://emex.ru/catalogs/original/?screen=units\&vin=/\вин}\\
			\begin{figure}[H]
				\centering
				\includegraphics[width=0.8\linewidth]{example-image}
				\caption{{\footnotesize {Исследуемое ТС \тс}}}
				\label{ris:images/1}
			\end{figure}
			
			\item[Год выпуска] -- \год
			\item[Шасси] -- Отсутствует
			\item[Цвет ЛКП] -- \цвет
			\item[Пробег] --  \пробег\,км%, считан с одометра
			\item[Двигатель] -- \двигатель
	    	\item[ПТС] --\птс	
		\end{description}
		\end{itemize}
%%%%%%%%   Информация, полученная при расшифровке ВИН
%	\subparagraph*{} Идентификационный код автомобиля (VIN)  \vin \, содержит следующую информацию о транспортном средстве, имеющую значение для 	дачи заключения:\\
%    Модель ТС: CHEVROLET LACETTI + NUBIRA + OPTRA (J200) [EUR]\\[3mm]
%    
%    
%\noindent\parbox[]{10cm}{
%\begin{itemize}
%	\item[ ] 
%	    \begin{description}
%%	\item{Модель} \hfill CHEVROLET J03 - LACETTI  (J200) [EUR]
%	\item[Дата изготовления] \hfill \началоэкспл
%	\item[Расположенние руля] \hfill Left
%	\item[Двигатель] \hfill \двигатель
%%	\item[Объем двигатель] \hfill 1998 $ \text{см}^3 $
%%	\item[КПП] \hfill АКПП
%	\item[Тип кузова] \hfill  Седан
%	\item[Количество дверей] \hfill 4 
%	%	\item[VDS] --
%    %	
%		\end{description}
%\end{itemize}}\\%[3mm]
%    

\vspace{3mm}

Модификация исследуемого автомобиля \тс \,   VIN  \vin \, 


\begin{figure}[H]
	\centering
	\includegraphics[width=0.99\linewidth]{example-image}
	\caption{{\footnotesize {Модификация ТС  VIN \vin  }}}
	\label{ris:images/модель}
\end{figure}

Рыночная стоимость аналогичного автомобиля составляет 1000000 (Один миллион) рублей, Рис. \ref{ris:images/цена}, 
\textit{ https://auto.ru/krasnodar/cars/skoda/superb/2012-year/used/}
\vspace{3mm}
%Пробег автомобиля  расчетный, согласно [1]  составляет 
%%%%%%%%%%%%%%%% Вставка изобржения общего вида авто
  \begin{figure}[H]
	\centering
	\includegraphics[width=0.6\linewidth]{example-image}
	\caption{{\footnotesize {Рыночная стоимость ТС \тс \,  в г. Краснодаре, \url{https://cenamashin.ru/cena/skoda/superb/2012}}}}
	\label{ris:images/цена}
\end{figure}




\relax
%
%\par За дату изготовления принимается 01.01.1996 г., при этом срок эксплуатации автомобиля \тс \, на момент исследования \датадтп, составляет 23,78 лет.  %%%%%% расчет в jupiter "разность дат"
%Исследуемый автомобиль не на гарантийном периоде эксплуатации и старше семи лет. 
%Пробег автомобиля \пробег, с учетом срока эксплуатации фактический среднегодовой пробег составляет \пробег/23,78 = 8495 км.
\subparagraph{Осмотр транспортного средства }


\noindent \begin{spacing}{1.2} { Осмотр транспортного средства \тс\, регистрационный знак  \грз\, проводился \osm\,с 11 час. 00 мин. до 11 час. 30 мин. в  ясную погоду  на открытой площадке по адресу: \местоосмотра. При осмотре присутствовал  владелец ТС \заказчик. %Виновник ДТП уведомлен надлежащим образом, на осмотр не явился.
Соответствие маркировочных обозначений на кузове представленного ТС записям в регистрационных документах ТС экс\-пертом-техником установлено. Видимые изменения конструкции ТС отсутствуют.  Представленный на исследование автомобиль \tc\, имеет кузов типа "\типкузова". Кузов автомобиля окрашен  
эмалью (краской) \colr цвета. Внешнее состояние автомобиля удовлетворительное, повреждения  кузовных элементов отсутствуют.  %Стекла боковин и двери задка  оклеены тонирующей плёнкой. % оснащен легкосплавными 15 дюймовыми пятиспицевыми колесами LS Wheels.  На наружных и внутренних поверхностях кузова автомобиля присутствуют признаки  эксплуатации, коррозионные повреждения отдельных кузовных элементов,  повреждения и следы ремонта.
\par Для определения причины возникновения повреждений, указанных в Акте осмотра ТС 
№ \NomerDoc \, (Приложение Акт осмотра) 
экспертом-техником изучены документы, представленные Заказчиком.  %и сведения о ДТП, с участием ТС \вин. % \, по данным открытых источников https://гибдд.рф. 
%По предоставленным документам экспертом-техником установлена причина ДТП, установлены обстоятельства ДТП, выявлены повреждения ТС и установлены причины их образования. 
Проведено исследование характера выявленных повреждений и сопоставление повреждений ТС  
%потерпевшего с повреждениями ТС иных участников ДТП в соответствии 
со сведениями, зафиксированными в представленных документах. %документах о ДТП. 
Проведена проверка взаимосвязанности повреждений на ТС с заявленными обстоятельствами, определён объем восстановительных работ.% (Приложение №1 Акт осмотра ТС). 
\par Наличие, характер и объем технических повреждений, а так же  планируемые (предполагаемые) ремонтные воздействия для восстановления поврежденного автомобиля  исследованы в присутствии заинтересованных лиц, зафиксированы в акте осмотра № \NomerDoc \,  от \osm  (Приложение №2),  и фотоматериалах по принадлежности (Приложение Фототаблица повреждений).  }
\end{spacing}
%
%%%%%%%%%%%%%%%%%% Если автомобиль новый, до пяти лет
%
%\par  настоящем исследовании износ транспортного средства принимается равным нулю по следующим основаниям: обстоятельства заявленного события не регулируются законодательством об ОСАГО,  срок эксплуатации КТС не превышает пяти лет, иные повреждения и следы ремонта отсутствуют, признаки интенсивной эксплуатации отсутствуют. 
%
%\relax
%\renewcommand\baselinestretch{0.86}\small\normalsize 
%\subsection{\underline{По  вопросу}\, \, \,	\textbf{\small{<<Установить наличие, характер и объем (степень) технических повреждений транспортного средства  \tc \,>>?}}}
%\renewcommand\baselinestretch{1.2}\small\normalsize
%%
%Первичное установление наличия и характера повреждений транспортного средства, в отношении которых определяются расходы на восстановительный ремонт, в соответствии с  Единой методикой определения размера расходов на восстановительный ремонт 
%в отношении поврежденного транспортного средства  
%утверждены Положением Банка России от «19» сентября 2014 года № 432-П (Единой методикой), должно производится во время осмотра транспортного средства и фиксироваться актом осмотра, в который  должны включаться сведения о повреждениях транспортного средства, с обязательной  характеристикой поврежденных элементов с указанием расположения, вида и объема повреждения.   
%
%В акте № 006/06/19  от \osm\, повреждений транспортного средства (ТС)  не содержатся установленные Единой методикой обязательные количественные показатели (характеристики) повреждений транспортного средства.  Отсутствие количественных характеристик повреждений в представленном акте осмотра предполагает его применение в настоящем исследовании исключительно в виде списка повреждений и перечня ремонтных воздействий, определенных экспертом-техником Яковлевым С.В. для восстановительного ремонта транспортного средства. %  не позволяет эксперту использовать указанный акт осмотра для объективного выбора необходимого и достаточного комплекса работ по восстановительному ремонту транспортного средства.
%
%На момент определения о проведении повторной судебной экспертизы, объект исследования неоднократно подвергался видоизменениям как при исследовании специалистом ООО "ЭКСПЕРТ", так и при производстве первичной судебной экспертизы экспертами ИП Куприянова Виктора Александровича. 
%
%\par
%Согласно данному акту осмотра № 1002  от \osm \, для устранения повреждений ТС необходимо было произвести следующие ремонтные работы:
%\vspace{\baselineskip}  % вставка пустой строки
% 
%\begin{longtable}{|p{1cm}|p{11cm}|p{3cm}|}
%\caption[]{\footnotesize {Ремонтные воздействия по акту осмотра № 006/06/19 от \, \osm \, ИП Яковлева С.В.}} \label{tab:4}\\ 
%	 \hline
%		\rowcolor[HTML]{C0C0C0} 
%	%	\multicolumn{1}{|c|}
%	%	{\cellcolor[HTML]{C0C0C0}N/N} & Наименование запчасти (материала) & Ремонтное воздействие 
%    %	
% \text{N/N} & Наименование запчасти (материала) & Ремонтное воздействие  \\ \hline \endhead
%		\Rownum  & Панель задка  & Замена, окраска \\ \hline
%		\rowcolor[HTML]{EFEFEF} 
%		\Rownum  & Боковина задняя левая   & Замена, окраска \\ \hline
%	    \Rownum  & Боковина задняя правая  & Замена, окраска  \\ \hline
%		
%\end{longtable}
%
%\setcounter{rownum}{0}
%%
%\par Таким образом, размер восстановительных расходов (затрат) \тс\,  может быть произведен в объеме сведений, содержащихся в представленной Таблице \ref{tab:4}.
%\vspace{\baselineskip}  % вставка пустой строки
%
%\renewcommand\baselinestretch{0.86}\small\normalsize 
%\subsection{\underline{По  вопросу}\, 	\textbf{\small{2. <<Установить причины возникновения технических повреждений транспортного средства \tc \, и возможность их отнесения к рассматриваемому дорожно-транспортному происшествию (ДТП)>>?}}}
%\renewcommand\baselinestretch{1.2}\small\normalsize
%
%
%Колесное транспортное средство сроком эксплуатации более 7 лет относится к категории транспортных средств с граничным сроком эксплуатации [1], для которой возможно применение ремонтных операций при условии экономической целесообразности и  технической возможности.
% 
%Из открытых банков данных полиции следует, что автомобиль с VIN  \вин\,  как минимум дважды становился участником ДТП.
%Первый раз 29.06.2018  06:40, извещение о ДТП № 030046913, в котором автомобиль получил повреждения задней правой двери, заднего правого порога, заднего правого колеса, подушки SRS справа, Рис. \ref{ris:images/d1} и второй раз 22.05.2019 06:50, извещение о ДТП № 030034947, в котором автомобиль получил повреждения деталей передней левой и задней частей кузова, Рис. \ref{ris:images/d2}.
%
%
%\begin{figure}[H]\centering
%	\parbox[t]{0.49\textwidth}
%	{\centering
%		\includegraphics[width=.49\textwidth]{images/d1}
%		\caption{\footnotesize {Повреждения в ДТП 29.06.2018 }}
%		\label{ris:images/d1}}
%	\hfil \hfil%раздвигаем боксы по горизонтали 
%	\parbox[t]{0.49\textwidth}
%	{\centering
%		\includegraphics[width=.49\textwidth]{images/d2}
%		\caption{\footnotesize {Повреждения в ДТП 22.05.2019}}
%		\label{ris:images/d2}}
%\end{figure}
%%
%{\noindent  \footnotesize \tikz \fill [red] (1,0.5) rectangle (0.1,0.1); --{\footnotesize  Вмятины, вырывы, заломы, перекосы, разрывы и другие повреждения с изменением геометрии элементов (деталей) кузова и эксплуатационных характеристик ТС.}\\
%\tikz \fill [yellow] (1,0.5) rectangle (0.1,0.1); --  {\footnotesize Повреждения колёс (шин), элементов ходовой части, стекол, фар, указателей поворота, стоп-сигналов и других стеклянных элементов (в т.ч. зеркал), а также царапины, сколы, потертости лакокрасочного покрытия или пластиковых конструктивных деталей и другие повреждения без изменения геометрии элементов (деталей) кузова и эксплуатационных характеристик ТС.}\\[1mm]
%%%%%%%%%%%%%%%%%%%%%%%%%%%%%%%%%%%%%%%%%%%%%%%%%%%%%%%%%%%%%%%%%%%%%%%%%%%}%%%%%%%%%%%%%%%%%%%%%%%%%%%%%%%%%%%%%%%%%%%%%%%%%%
%
%\renewcommand\baselinestretch{1.2}\small\normalsize
%\begin{spacing}{1.25}Таким образом, перечень повреждений, указанный в акте осмотра № 006/06/19 от \osm \, ИП Яковлева С.В. может соответствовать повреждениям автомобиля \тс\,, полученным в результате ДТП \датадтп. 
%\end{spacing}
%\renewcommand\baselinestretch{0.86}\small\normalsize 
%\subsection{\underline{По  вопросу}\, \, \,	\textbf{\small{3. <<Установить технологию, объем восстановительного  ремонта транспортного средства \tc \,>>?}}}
%%%%%%%%%%%%%%%%%%%%%%%%
%
%Для автомобилей, старше семи лет:
%\renewcommand\baselinestretch{1.2}\normalsize
%
%
%\renewcommand\baselinestretch{0.86}\small\normalsize 
%\subsection{\underline{По  вопросу}\, \, \,	\textbf{\small{<<Установить размер затрат на восстановительный ремонт (с учётом износа) транспортного средства \tc \,>>?}}}
%\renewcommand\baselinestretch{1.2}\small\normalsize
%




%



%%%%%%%%%%%%%%%%%%%%%%%%%%%%%%%%%%%%%%%%%%%%%%%%%%%%%%%%%%%%%%%%%%%%%%%%%%%%%%%%%%%%%%%%%%%%%%%%%%%%%%%%%%%%%%%%
%Стоимость восстановительных работ $ C_{\text{вр}} $ определяется на основании норм трудоёмкостей $ T_i $, \, предусмотренных заводом-изготовителем, и стоимостных параметров $ C_{i\text{нч}} $ (стоимости нормо-часа) работ по техническому обслуживанию и ремонту АМТС.  Расчет размера расходов (в рублях) на восстановительный ремонт производится по формуле:
%\begin{equation}\label{eq:cr}
%C_{\text{вр}}  =\sum{C_{ip}}= \sum\left({T_{ij}}\cdot {C_{i\text{нч}}}\right) + \sum{C_{ip^{\text{\,\,\,руб}}}} , \,\,\,\text{где:} 
%\end{equation}
%%\vspace{2mm}
%\begin{itemize}
%	\item[ ]$ C_{ip} $ -- стоимость работ i-го вида: $C_\text {зам} $, $ C_\text{восст} $, $ C_\text{рег} $, $C_\text{контр} $, $ C_\text{антикор} $, $ C_\text{зч} $, $ C_\text{ом} $,$ C_\text{соп} $, $ C_\text{вм} $, руб;
%	\item[ ]$ T_{ij} $ -- трудоёмкость j-й операции(комплекса) по i-му виду работ, руб;
%	\item[ ]$ C_{i\text{нч}} $ -- стоимость нормо-часа по i-му виду работ, руб;
%	\item[ ]$ C_{ip^\text{\,\,руб}} $ -- стоимость работ $ C_{ip} $, принятая непосредственно в денежном выражении, руб.
%\end{itemize}
%%
%\par При определении стоимости восстановительного ремонта АМТС с учётом износа под износом следует понимать количественную меру физического старения АМТС и его элементов, достигнутого в результате эксплуатации, т.е. эксплуатационный износ.
%%
%Расчёт износа производится в  соответствии с Положением Банка России от «19» сентября 2014 года № 432-П «О единой методике определения размера расходов на восстановительный ремонт в отношении повреждённого транспортного средства» [2].
%Износ комплектующих изделий (деталей, узлов, агрегатов) рассчитывается по следующей формуле:
%%
%\begin{equation}\label{eq:I}
%\text{И}_{\text{ки}} 
%= 100\cdot\left( 1-e^ {-\left( \Delta_{T} \cdot T_{\text{КИ}} + \Delta_{L} \cdot L_{\text{КИ}} \right)}\right), \,\,\,\,\text{где:}   
%\end{equation}
%%
%\begin{itemize}
%	\item[ ]$ \text{И}_{\text{ки}} $ -- износ комплектующего изделия (детали, узла, агрегата) (процентов); 
%	\item[ ]$ e $ -- основание натуральных логарифмов (e =  2,72);
%	\item[ ]$ \Delta_{T}$ --  срок эксплуатации комплектующего изделия (детали, узла, агрегата) (лет);
%	\item[ ]$ T_{\text{КИ}} $ -- стоимость работ $ C_{ip} $, принятая непосредственно в денежном выражении, руб
%	\item[ ]$ \Delta_{L} $ --коэффициент, учитывающий влияние на износ комплектующего (детали, узла, агрегата) величины пробега транспортного средства с этим комплектующим изделием;
%	\item[ ]$ L_{\text{КИ}} $ --пробег транспортного средства на дату дорожно-транспортного происшествия (тысяч километров).  
%		\end{itemize}
%\vspace{5mm}
%\par Значения коэффициентов $ \Delta_{T}$  и $ \Delta_{L} $  для различных категорий и марок транспортных средств приведены в п.5. сп. лит~[2]. При этом, на комплектующие изделия (детали, узлы, агрегаты), которые находятся в заведомо худшем состоянии, чем общее состояние транспортного средства в целом, и его основные части, вследствие влияния факторов, не учтённых при расчете износа (например, проведение ремонта с нарушением технологии, не устранение значительных повреждений лакокрасочного покрытия), может быть начислен дополнительный индивидуальный износ. 
%Износ шины транспортного средства рассчитывается по следующей формуле:
%\begin{equation}\label{eq:sh}
%\text{И}_{\text{ш}} = \frac{\text{Н}_{\text{н}}-\text{Н}_{\text{ф}}}{\text{Н}_{\text{н}}-\text{Н}_{\text{доп}}} \cdot{100}\%  \,\,\,\,\text{где:} 
%\end{equation}
%%
%\begin{itemize}
%	\item[ ] $ \text{И}_{\text{ш}} $ -- износ шины, \%;
%	\item[ ] $ \text{Н}_{\text{н}} $ -- высота рисунка протектора новой шины, мм;
%	\item[ ] $\text{Н}_{\text{ф}} $ -- фактическая высота рисунка протектора шины, мм;
%	\item[ ] $ \text{Н}_{\text{доп}} $ --минимально допустимая высота рисунка протектора шины в соответствии с требованиями законодательства Российской Федерации, мм.
%\end{itemize}
%%
%\renewcommand\baselinestretch{1}\small\normalsize
%Износ шины дополнительно увеличивается для шин с возрастом от 3 до 5 лет - на 15 процентов, свыше 5 лет - на 25 процентов. 
%%%%%%%%%%%%%%%%%%%%%%%  Нулевой и предельный износ %%%%%%%%%%%%%%%%%%%%%%%%%%%%%%
%\par $ ^*$Согласно п. 7.8.\, Методики [1]  для случаев, не регулируемых законодательством об ОСАГО, для составных частей КТС значение износа принимается равным нулю, срок эксплуатации которых не превышает 5 лет,  предельное значение износа комплектующих транспортного средства  не должно превышать 80\% стоимости запасных частей. Для составных частей, имеющих срок эксплуатации более 12 лет, при отсутствии факторов снижения износа (проведенный капитальный ремонт, замена составных частей  и т.д) рекомендуемое значение износа составляет 80\%.
%%%%%%%%%%%%%%%%%%%%%%%%%%%%%%%%%%%%%%%%%%%%%%%%%%%%%%%%%%%%%%%%%%%%%%%%%%%%%%%%%%%5



\indent Повреждения автомобиля \тс\, установленные по результатам осмотра, отражены в таблице \ref{tab:5}: 

\begin{longtable}{M{125mm}|M{30mm}}
\caption[]{\footnotesize {Повреждения автомобиля, установленные при его осмотре}} \label{tab:5}\\ \hline
\bf {\small Наименование  детали с описанием повреждения} & \bf {\small Изображение} \\ \hline \endhead
%
% {\small    } & \imt{fp1} \\ \hline  %фото повреждений
%
{\small Фонарь задний левый - разбит   } & \imt{example-image} \\ \hline  %фото повреждений
{\small Панель крышки багажника - несложная деформация в левой угловой части  с загибом металла и повреждением ЛКП,  \s{0.01} } & \imt{example-image} \\ \hline
{\small Крыло заднее левое  - повреждение ЛКП угловой смежной с крышкой багажника частью (скол ЛКП до гунта, \s{0.001})  } & \imt{example-image} \\ \hline
{\small Бампер задний - деформация левой части, повреждение ЛКП, \s{0.02}   } & \imt{example-image} \\ \hline
{\small Датчик парковки левый внешний - наслоение вещества синего цвета   } & \imt{example-image} \\ \hline

\end{longtable}

\par В результате исследования   экспертом-техником установлено, что для устранения повреждений \тс \, необходимо  выполнить следующие  работы:
%\begin{center}
	\begin{longtable}{M{65mm}M{85mm}}
\hline 
\textbf{Наименование детали}        & \textbf{Ремонтное воздействие}\\
\hline {\small Панель крышки багажника }      &   {\small  Ремонт, окраска}\\
{\small Фонарь задний левый  } &  {\small Заменить } \\
{\small Крыло заднее левое  } &  {\small Окраска  } \\
{\small Бампер задний  } &  {\small Ремонт, окраска  } \\
{\small Датчик парковки левый внешний  } &  {\small Проверка, окраска  } \\
\hline\\
	\end{longtable}  

%
\textbf{Произвести  необходимые для выполнения  ремонта разборочно-сборочные, подготовительные и вспомогательные работы в соответствии с требованиями завода–изгото\-ви\-теля транспортного средства.}\\
%%
%\end{center}
\renewcommand\baselinestretch{1.2}\small\normalsize


\par  
В соответствии с принятой экспертной методикой размер расходов на восстановительный ремонт определяется исходя из стоимости ремонтных работ (работ по восстановлению, в том числе окраске, контролю, диагностике и регулировке, сопутствующих работ), стоимости используемых в процессе восстановления транспортного средства деталей (узлов, агрегатов) и материалов взамен поврежденных.\\
%                                         
Стоимость восстановительного ремонта АМТС ( $ C_\text{вp} $) определяется по формуле:
%
\begin{equation}\label{eq:r}
	C_\text{вp} =C_p + C_\text{м}% + C_\text{зч}\cdot(1-\dfrac{\text{И}}{100})
\end{equation}
%
\noindent где:
%
\begin{itemize}
	%	
	\item[ ]$C_\text {р} $ --  стоимость ремонтных работ по восстановлению КТС, руб.;
	\item[ ]$ C_\text{м} $ --  стоимость необходимых ремонтных материалов, руб.;
	\item[ ]$ C_\text{зч} $ --  стоимость новых запасных частей, руб;
%	\item[ ] $ \text{И} $ -- коэффициент износа составной части, подлежащей замене, \%.
\end{itemize}
%
\vspace{5mm}
\renewcommand\baselinestretch{1.2}\small\normalsize

%Коэффициент износа составных частей (И) КТС (кроме автобусов и грузовых автомобилей) при определении стоимости восстановительного ремонта рассчитывается по формуле:
%
%\begin{equation}\label{eqsnos}
%	\text{И} =\text{И1}\cdot\text{П}+\text{И2}\cdot \text{Д},\, \%  \,\,\,\,,
%\end{equation}
%\noindent где:
%\begin{itemize}
%	\item [] $ \text{И1} $ --усреднённый показатель износа на 1000 км пробега, \%; 
%	\item [] $ \text{П} $ -- общий пробег (фактический или расчетный) за срок эксплуатации КТС, тыс.км;
%	\item [] $ \text{И2} $ -- усреднённый показатель старения за 1 год эксплуатации, \%;
%	\item [] $ \text{Д} $ -- срок эксплуатации КТС (от даты изготовления КТС до момента, на который определяется износ), лет. 
%\end{itemize}
%%
%Для исследуемого автомобиля \тс, параметры для расчета коэффициента износа приняты согласно справочным таблицам [1, ч. II, Приложение 2.4]:
%\begin{equation}\label{eqsnosr}
%	\text{И} =\text{И1}\cdot\text{П}+\text{И2}\cdot \text{Д} = 0.3\cdot 238  + 1,35\cdot 15 = 72^* \, \%
%\end{equation}

\par
Расчёт стоимости применённых запасных частей произведён на основании выборки цен запасных частей, по данным специализированных магазинов:

% TODO: \usepackage{graphicx} required
\begin{center}
	\includegraphics[width=0.98\linewidth]{example-image}
%	\caption{Дата обращения  20.04.2022}
\end{center}


 Расчёт стоимости восстановительного ремонта выполнен в программе \silver.\\

\indent Результаты расчёта представлены ниже:
\vspace{3mm}
%
%\begin{figure}[H]
%	\centering
%	\includegraphics[width=0.95\linewidth]{example-image}
%	%%	\caption{}
%	%%	\label{fig:screenshot001}
%\end{figure}
\begin{figure}[H]
	\centering
	\includegraphics[width=0.95\linewidth]{example-image}
%%	\caption{}
%%	\label{fig:screenshot001}
\end{figure}
%\begin{figure}[H]
%	\centering
%	\includegraphics[width=0.9\linewidth]{images/screenshot002}
%%%	\caption{}
%	\label{aud}
%\end{figure}
\medskip
\renewcommand\baselinestretch{1.2}\small\normalsize
%%%%%%%%%%%%%%%%  Не ОСАГО
Полный текст калькуляции представлен в Приложении  <<Калькуляция стоимости восстановительного ремонта>>.\\ 
%%%%%%%%%%%%%%%%
Стоимость коммерческого нормо-часа работ применена  с учётом условий регионального рынка услуг и сложившихся средних расценок по видам работ, типу ТС, а также по маркам и моделям ТС  и   составляет \нормочас \, р/ч для данного транспортного средства (\url{http://prices.autoins.ru/priceAutoParts/}). Трудоёмкость работ по разборке/сборке/замене  соответствует трудоёмкости работ, рекомендованной заводом изготовителем ТС [1, часть II, п. 7.32], а так же рекомендованные значения оценочной трудоёмкости ремонта кузовных составных частей [1, часть II, п. 7.33]. Расчёт стоимости ремонта, согласно положениям Методики [1] производится с учётом  применения оригинальных запасных частей. %, которые поставляются изготовителем КТС авторизованным ремонтным организациям. 
%Техническое состояние запасных частей учитывается коэффициентом износа, что в совокупности с установкой оригинальных запасных частей в максимальной степени отвечает понятию «восстановительный ремонт», то есть восстановления состояния КТС, при котором используются установленные изготовителем составные части, но с использованным частично ресурсом.  

Согласно положений экспертной методики [1],  для  автомобиля VIN  \vin \, величина $ C_\text{утс} $,  в силу ограничений по сроку эксплуатации, не рассчитывается и принимается равной нолю, следовательно:

\begin{equation}\label{112}
	\text{У} = C_\text{вр},    \text{руб.}
\end{equation}


%

\par
Таким образом, в результате проведённых расчётов (см. Приложение, <<Калькуляция стоимости восстановительного ремонта № \NomerDoc>>) определена стоимость восстановительного ремонта транспортного средства  \тс, составляющая  \итог\, (\числопрописью{\итог}) руб. %без учёта износа, стоимость восстановительных расходов, с учётом уменьшения стоимости запасных частей вследствие их износа,  составляет \итогизнос\, (\числопрописью{\итогизнос}) руб. рублей.
При соблюдении граничных условий согласно п.7.2. Медодики [1] размер ущерба вследствие повреждения КТС принимается равным стоимости восстановительного ремонта КТС.
%%%%%%%%%%%%%%% УТС, годные,  рыночная
%\nopagebreak
% \input{rinok}   %% Расчет рыночной стоимости ТС
% \paragraph*{Расчет стоимости годных остатков.}
 Стоимость годных остатков с учетом затрат на их демонтаж, дефектовку, хранение и продажу определяется по формуле:
 \begin{equation}\label{go}
C_{\text{ГО}}= C_{\text{Р}} \cdot K_{\text{В}}\cdot K_{\text{З}}\cdot K_{\text{ОП}} \cdot  \sum\limits_{i-1}^{n}\frac{C_i}{100}, \, \, \text{руб} 
\end{equation}
\noindent где: \,$ C_{\text{Р}} $ -- стоимость ТС в неповрежденном виде на момент происшествия;\\
$ K_{\text{З}} $-- коэффициент, учитывающий затраты на дефектовку, разборку, хранение, продажу;\\
$ K_{\text{В}} $ -- коэффициент, учитывающий срок эксплуатации АМТС на момент повреждения и спрос на его неповрежденные детали;\\
$ K_{\text{ОП}} $ -- коэффициент, учитывающий объем (степень) механических повреждений автомобиля;\\
$ C_i $ процентное соотношение (вес) стоимости неповрежденных элементов к стоимости автомобиля;\\
$ n  $- количество неповрежденных элементов (агрегатов, узлов).\\

Расчет процентного соотношения (веса) стоимости неповрежденных элементов к стоимости ТС   \,\,
     % \begin{equation}\label{bb}
   $  \left( \sum\limits_{i-1}^{n}\frac{C_i}{100} \right)  $  
%   \end{equation}  
включает только установленные неповрежденные детали, узлы и агрегаты. Компоненты ТС, имеющие повреждения  вероятностного характера, и требующие диагностических работ для установления годности в расчете не учитываются. 
 
  \begin{longtable}{|p{9cm}|p{4cm}|p{2cm}|}
 	\caption[]{\footnotesize {Таблица расчета $ C_i $ }}
 	 \label{tab:7}\\
 	 \hline
 	 		Наименование агрегата, узла, детали & \%-ное соотношение (вес)  & Годные, \% \\
 	 		\hline \endhead
 		Кузовные детали, экстерьер, интерьер, в т.ч.: & 50 (45 \textless{}1\textgreater{}) & 0 \\
 		Передняя часть: & 14 &  \\
 		Капот & 1.9 & 1,9 \\
 		Крыло переднее (за 1 шт.) & 0.8 & 0,8 \\
 		Бампер передний (в сборе с усилителем, накладками и молдингами, спойлером) & 1.9 & 1,9 \\
 		Решетка (облицовка) радиатора & 0.8 & 0,8 \\
 		Лонжерон передний (за 1 шт.) & 0.8 & 0,8 \\
 		Брызговик крыла (за 1 шт.) & 1.4 & 1,4 \\
 		Стекло ветрового окна & 1.7 & 1,7 \\
 		Рамка радиатора & 1.4 & 1,4 \\
 		Щиток передка & 0.3 & 0,3 \\
 		Задняя часть: & 12 (14 \textless{}1\textgreater{}) & 0 \\
 		Бампер задний & 1.6 & 0 \\
 		Крыло заднее (боковина \textless{}1\textgreater{}) в сборе с арками (за 1 шт.) & 2.1 (3.1 \textless{}1\textgreater{}) & 0 \\
 		Стекло окна задка & 1.9 & 0 \\
 		Панель задка & 0.8 & 0 \\
 		Пол багажника & 0.8 & 0 \\
 		Облицовки багажника & 1.1 & 0 \\
 		Крышка багажника (дверь задка) & 1.6 & 0 \\
 		Средняя часть: & 24 (17 \textless{}1\textgreater{}) & 0 \\
 		Передняя стойка боковины (за 1 шт.) & 1.4 & 2,8 \\
 		Средняя стойка боковины с порогом и частью пола (за 1 шт.) & 1.4 (0 \textless{}1\textgreater{}) & 2,8 \\
 		Облицовки стоек боковины, порогов, уплотнители, центральная консоль, противосолнечные козырьки, плафоны освещения, коврики пола, зеркало заднего вида & 2.5 (2.1 \textless{}1\textgreater{}) & 2,5 \\
 		Двери в сборе с арматурой (за 1 шт.), & 1.9 & 5,7 \\
 		в т.ч. арматура дверей (за 1    дверной комплект) & 0.5 & 0 \\
 		Сиденья (все) & 1.1 & 1,1 \\
 		Панель крыши в сб. с обивкой, поперечинами и верх. частями стоек, & 3.5 & 2,7 \\
 		в т.ч. обивка панели крыши & 0.8 & 0 \\
 		Панель приборов в сборе с щитком приборов, решетками, вещевым ящиком, карманами и т.д. & 2.5 & 2,5 \\
 		Ремень безопасности передний (за 1 шт.) & 0.3 & 0,6 \\
 		Подушка безопасности пассажирская & 0.6 & 0,6 \\
 		Двигатель, навесное, охлаждение, впускная и выпускная система & 11 (13 \textless{}2\textgreater{}) & 13 \\
 		Двигатель в сборе без навесного оборудования & 4.9 & 0 \\
 		в т.ч. клапанная крышка & 0.5 & 0 \\
 		в т.ч. масляный поддон & 0.5 & 0 \\
 		в т.ч. блок цилиндров & 2.2 & 0 \\
 		Дроссельный узел в сборе с заслонкой, клапаном и датчиком & 1.4 & 0 \\
 		Генератор & 0.8 & 0 \\
 		Коллектор впускной & 0.5 & 0 \\
 		Коллектор выпускной & 0.5 & 0 \\
 		Радиатор охлаждения в сборе с кожухами, вентилятором & 0.8 & 0 \\
 		Стартер & 0.5 & 0 \\
 		Короб воздушного фильтра с патрубками & 0.5 & 0 \\
 		Выпускной тракт в сборе & 0.8 & 0 \\
 		Турбокомпрессор (турбонагнетатель) & 1.4 \textless{}2\textgreater{} & 0 \\
 		Интеркулер & 0.6 \textless{}2\textgreater{} & 0 \\
 		Топливная система & 2.5 & 2,5 \\
 		Бак топливный & 0.7 & 0 \\
 		Система подачи топлива & 1.8 & 0 \\
 		Трансмиссия & 4.5 & 4,5 \\
 		Усредненный показатель с учетом всех возможных вариантов трансмиссии & 4.5 & 0 \\
 		Подвеска & 10 & 4 \\
 		Подвеска передняя в сборе с поперечиной & 5.5 (4.5 \textless{}4\textgreater{}) & 0 \\
 		Подвеска задняя в сборе с поперечиной & 4.5 (5.5 \textless{}4\textgreater{}) & 4,5 \\
 		Подвеска в сборе для полноприводных АМТС & 10 (5 \textless{}4\textgreater + 5 \textless{}4\textgreater{}) & 0 \\
 		Рулевое управление & 3 & 3 \\
 		Рулевая колонка в сборе с валом & 0.5 & 0 \\
 		Насос ГУР & 0.8 & 0 \\
 		Рулевой механизм & 1.2 & 0 \\
 		Рулевое колесо в сборе с подушкой безопасности & 0.5 & 0 \\
 		в т.ч.: подушка безопасности  водительская & 0.3 & 0 \\
 		Тормозная система & 3.5 & 3,5 \\
 		Главный тормозной цилиндр & 0.5 & 0 \\
 		Тормозной механизм колеса (за каждый колесный узел) & 0.5 & 0 \\
 		Ручной (ножной) тормоз & 0.3 & 0 \\
 		Блок управления АБС & 0.7 & 0 \\
 		Электрооборудование & 12.5 & 0 \\
 		Провода свечные с катушками (комплект) & 0.5 & 0,5 \\
 		Монтажный блок & 0.5 & 0,5 \\
 		Блок управления двигателем & 1 & 1 \\
 		Фонари задние (за 1 шт.) & 0.5 & 1 \\
 		Зеркала заднего вида боковые (за 1 шт.) & 0.8 & 1,6 \\
 		Блок отопителя салона в сборе (корпус, двигатель, радиаторы) & 2.1 & 2,1 \\
 		Насос кондиционера & 0.5 & 0,5 \\
 		Конденсатор в сборе с осушителем, кожухом, вентилятором, трубками & 0.6 & 0,6 \\
 		Фары (за 1 шт.) & 1.1 & 1,1 \\
 		Жгут проводов ДВС & 0.9 & 0,9 \\
 		Жгут проводов панели приборов & 0.8 & 0,8 \\
 		Остальные жгуты проводов (все) & 0.3 & 0,3 \\
 		Фара противотуманная (за 1 шт.) & 0.8 & 1,6 \\ 
 		Прочее & 3/8 \textless{}1\textgreater{}/1 \textless{}2\textgreater /6 \textless{}3\textgreater{} & 4 \\
 		\hline
 		\textbf{ИТОГО,} \%: &  & \textbf{83,8}  \\
 		\hline	
 	\end{longtable}
 
\noindent  \begin{table}[H]
  	 \label{tab:KB}
	\caption{\footnotesize {Значения коэффициента Кв, учитывающего срок эксплуатации ТС}}
		 \begin{tabular}{|p{47mm} |p{53mm}| p{50mm}|}
	\hline
 		Срок эксплуатации автомобиля, лет & Значение Кв легковых автомобилей, малотоннажных грузовых на базе легковых и мототехники & Значение Кв грузовых автомобилей \\ \hline
 		0 - 5 (включительно)              & 0.80                                                                                    & 0.80                             \\ \hline
 		6 - 10 (включительно)             & 0.65                                                                                    & 0.60                             \\ \hline
 		11 - 15 (включительно)            & 0.55                                                                                    & 0.50                             \\ \hline
 		16 - 20 (включительно)            & 0.40                                                                                    & 0.35                             \\ \hline
 		Более 20 лет                      & 0.35                                                                                    & 0.30                            \\ \hline
 	\end{tabular}
\end{table}


\noindent \begin{table}[H]
	\label{tab:KO}
	\caption{\footnotesize {Значение коэффициента $ K_{\text{оп}} $ , учитывающего объем (степень) механических повреждений автомобиля}} 
	\centering
\begin{tabular}{|p{47mm}| p{53mm}| p{50mm}|}
	\hline
	Объем механических повреждений & Соотношение стоимости неповрежденных элементов к стоимости автомобиля, Ci, \% & Значение коэффициента, учитывающего объем механических повреждений \\ \hline
	Незначительный                 & 80 - 100                                                                      & 0.9 - 1                                                            \\
	& 60 - 80                                                                       & 0.8 - 0.9                                                          \\
	Средний                        & 40 - 60                                                                       & 0.7 - 0.8                                                          \\
	& 20 - 40                                                                       & 0.6 - 0.7                                                          \\
	Значительный                   & 0 - 20                                                                        & 0.5 - 0.6                                                          \\ \hline
\end{tabular}
\end{table}

 \par
 
% \begin{equation}\label{k}
% C_{\text{ГО}}= C_{\text{Р}} \cdot K_{\text{В}}\cdot K_{\text{З}}\cdot K_{\text{ОП}} \cdot  \sum\limits_{i-1}^{n}\frac{C_i}{100} 
% \end{equation}
 
 Для исследуемого транспортного средства применимы следующие значения коэффициентов:\par
 $ K_{\text{В}} =0.8  $; 
 $ K_{\text{З}} =0.7 $; 
 $ K_{\text{ОП}} =0.8 $; $ \sum\limits_{i-1}^{n}\frac{C_i}{100} = 0.838 $ \par тогда:
 \par
$  C_{\text{ГО}} =  C_{\text{ГО}}= C_{\text{Р}} \cdot K_{\text{В}}\cdot K_{\text{З}}\cdot K_{\text{ОП}} \cdot  \sum\limits_{i-1}^{n}\frac{C_i}{100} =980000*0.8*0.7*0.8*0.838 = 367915 $ руб., или с учетом округления 370 000 (Триста семьдесят тысяч) рублей.
\par Таким образом, стоимость годных остатков ТС \тс \, \, составляет 370 000 (Триста семьдесят тысяч) рублей.

      %% Расчет стоимости годных остатков
%\subsection{Расчет стоимости годных остатков}

\par В случаях правовых отношений, регулируемых Гражданским кодексом РФ, расчет стоимости годных остатков и определение стоимости  реального ущерба с его уменьшением на стоимость годных остатков не предусмотрены [1, ч.II,п.9.6]. Вместе с тем, экспертная практика свидетельствует о возможности постановки перед экспертом задачи определения стоимости годных остатков вне  поля действия законодательства об ОСАГО.
\par Под годными остатками автотранспортного средства понимаются работоспособные, имеющие остаточную стоимость детали (агрегаты, узлы) поврежденного автотранспортного средства, как правило, годные к дальнейшей эксплуатации, которые можно демонтировать с поврежденного автотранспортного средства и реализовать. 
Годные остатки должны отвечать следующим условиям:

1) деталь (агрегат, узел) не должна иметь повреждений, нарушающих ее целостность и товарный вид, а агрегат (узел), кроме того, должен находиться в работоспособном состоянии;

2) деталь (агрегат, узел) не должна иметь изменений конструкции, формы, целостности и геометрии, не предусмотренных изготовителем автотранспортного средства (например, дополнительные отверстия и вырезы для крепления несерийного оборудования);

3) деталь не должна иметь следов предыдущих ремонтных воздействий (следов правки, рихтовки, следов шпатлевки, следов частичного ремонта и т.д.).

Под стоимостью годных остатков понимается наиболее вероятная стоимость, по которой они могут быть реализованы, учитывая затраты на их демонтаж, дефектовку, ремонт, хранение и продажу.
К годным остаткам не могут быть отнесены [1, ч.II,п.10.3] составные части:

- демонтаж которых требует работ, связанных с применением газосварочного и электродугового резания;

-имеющие изменения конструкции, формы, нарушения целостности, не предусмотренные изготовителем ТС;

- подвергшиеяся ранее ремонтным воздействиям (например, правке, рихтовке, шпатлеванию  и т.д.);

- влияющие на безопасность дорожного движения.Номенклатура таких составных частей приведена в приложении  2.6 методики [1];

- имеющие коррозионные повреждения;

-требующие ремонта.


Стоимость годных остатков автотранспортного средства может рассчитываться только при соблюдении следующего условия: 

- полная гибель автотранспортного средства в результате ДТП. Под полной гибелью понимается случай, когда стоимость восстановительного ремонта поврежденного ТС превышает его рыночную стоимость на момент повреждения, или проведение восстановительного ремонта технически невозможно.
 
Расчет стоимости годных остатков не производится в следующих случаях:

- когда автотранспортное средство не подлежит, с учетом технического состояния, разборке на запасные части;

- когда, в силу региональных особенностей вторичного  рынка запасных частей, годные остатки данного автотранспортного средства не пользуются спросом.

Учитывая срок эксплуатации ТС \тс \, (23 года), предельную величину физического износа (80\%), региональные особенности вторичного рынка запасных частей, общее техническое состояние и  степень повреждения исследуемого транспортного средства,  в данном случае,   под стоимостью годных остатков понимается стоимость металлической массы автомобиля (металлического лома).
 $  \text{Сго} = \text{Мтс*Сметаллолома}  $, (руб/кг).
 
Стоимость металлолома в городе Краснодаре по данным организаций, занимающихся  приемом лома цветных и черных металлов, составляет от 6,0 до 7,5 рублей за килограмм в зависимости от размеров и качественных характеристик сдаваемого лома (ниже сравнительная таблица цены лома категории 12А1, автомобильный и бытовой легковесный лом). 


\begin{table}[H]
	
\begin{tabular}{|p{8mm}|l|p{26mm}|l|p{32mm}|l|p{18mm}|l|p{10mm}|c|p{8mm}}
	\hline 
№п/п	& {\small Наименование  предприятия} &{\small Адресс } &{\small Телефон}  & 	{\small Стоимость 1 кг, прием лома (руб)} \\
	\hline 
{\small 1}	& {\small Булатов И.П.} &{\small г.Краснодар,  Восточно-Кругликовская, 38 } &	{\small +7 (918) 430-16-70}  & {\small 6,5} \\ 
	\hline 
{\small 2}	&{\small  ВТМ-ЮГПЛЮС} &	{\small г.Краснодар , ул.Текстильная, 3 } &{\small +7 (861) 227-57-07 }&{\small  7,3 }\\ 
	\hline 
{\small 3}	&{\small Евростандарт } &{\small г.Краснодар пос.Знаменский ул.Богатырская 17}  &{\small 89181535143}  & {\small 6,0} \\ 
	\hline 
{\small 4}	& {\small Метализам} & {\small г.Краснодар ул.Уралская, 141/1} &	 {\small 8(918) 467-11-68 } & {\small 	6,5} \\ 
	\hline 
{\small 5}	& {\small Промышленный ресурс} & {\small Краснодар, ул. Соколова, 54}	 &{\small (861)2701773 } & {\small 	7,2 }\\ 
	\hline
\end{tabular} 

\end{table}
		

Средняя цена приема  лома :	 		6.7 руб/кг.

%По информации, представленной организациями, занимающимися приемом и переработкой лома, данная цена не включает в себя стоимость разборки, прессовки и транспортные издержки. В дальнейших расчетах, расходы на проведение указанных выше мероприятий будут вычтены из среднерыночной стоимости приема «чистого» лома.  Поэтому в дальнейших расчетах стоимость  лома металла принята в размере 6,0 рублей за кг.
 
\noindent Если:
масса транспортного средства = 1049 руб. (по данным паспорта ТС);\\
\indent масса неметаллических изделий  ≈ 300 кг. \\
Тогда:  $ \text{Сго = (1049-300)Х6.7= 5 018} $ руб.

\par Таким образом, результаты проведенного  исследования позволяют сделать вывод о том, что стоимость годных остатков транспортного средства \тс\,, с учетом округления,   составляет 5000 (Пять тысяч) рублей.\\


   %% Расчет стоимости годных остатков
%\subsection{Расчет утраты товарной стоимости ТС}


\par Утрата товарной стоимости (УТС) обусловлена снижением товарной стоимости из-за ухудшения потребительских свойств вследствие наличия дефектов (повреждений), или следов их устранения либо наличия достоверной информации, что дефекты (повреждения) устранялись [1,п. 8].

	УТС может быть рассчитана для КТС, находящихся как в поврежденном, так и в отремонтированном состоянии (при возможности установить степень повреждения).

УТС может определяться при необходимости выполнения одного из нижеперечисленных видов ремонтных воздействий или если установлено их выполнение:

-	устранение перекоса кузова или рамы КТС;

-	замена несъемных элементов кузова КТС (полная или частичная); ремонт съемных или несъемных элементов кузова (включая оперение) КТС (в том числе пластиковых капота, крыльев, дверей, крышки багажника);

-	полная или частичная окраска наружных (лицевых) поверхностей кузова (включая оперение) КТС, бамперов;

-	полная или частичная разборка салона КТС, вызывающая нарушение качества заводской сборки.

УТС не рассчитывается:

а)	если срок эксплуатации легковых автомобилей превышает 5 лет;

б)	если легковые автомобили эксплуатируются в интенсивном режиме, а срок эксплуатации превышает 2,5 года;


в)	в случае замены кузова до оцениваемых повреждений (за исключением кузова грузового КТС, установленного на раме за кабиной);

г)	если КТС ранее подвергалось восстановительному ремонту (в том числе окраске - полной, наружной, частичной; «пятном с переходом») или имело аварийные повреждения, кроме повреждений, указанных в [1, п. 8.4];

д)	если КТС имело коррозионные повреждения кузова или кабины на момент происшествия.



Нижеприведенные повреждения не требуют расчета УТС вследствие исследуемого происшествия, а их наличие до исследуемого происшествия не обуславливает отказ от расчета УТС при таких повреждениях:

а)	эксплуатационных повреждениях ЛКП в виде меления, трещин, а также повреждений, вызванных механическими воздействиями - незначительных по площади сколов, рисок, не нарушающих защитных функций ЛКП составных частей оперения;

б)	одиночного эксплуатационного повреждения оперения кузова (кабины) в виде простой деформации, не требующего окраски, площадью не более 0,25 дм2;

в)	повреждения, которые приводят к замене отдельных составных частей, которые не нуждаются в окрашивании и не ухудшают внешний вид КТС (стекло, фары, бампера неокрашиваемые, пневматические шины, колесные диски, внешняя и внутренняя фурнитура и т. п.). Если, кроме указанных составных частей, повреждены составные части кузова, рамы, кабины или детали оперения - крылья съемные, капот, двери, крышка багажника, - то расчет величины УТС должен учитывать все повреждения составных частей в комплексе;

г)	в случае окраски молдингов, облицовок, накладок, ручек, корпусов зеркал и других мелких наружных элементов, колесных дисков.

В случае исследуемого события для автомобиля \тс\, VIN \vin\, все условия  при которых производится расчет УТС выполняются.\\


\par Величина УТС зависит от вида, характера и объема повреждений и ремонтных воздействий по их устранению.
\par Величина УТС ($ C_\text{YTC} $) определяется на дату оценки (исследования) по формуле: 

\begin{equation}\label{uts}
C_{YTC} = C_{KTC} \cdot \dfrac{\sum\limits_{i=1}^n K_{YTCi}}{100\%}, \text{руб.},
\end{equation}

\noindent где:\\
\noindent $ C_{KTC} $ -- стоимость КТС на дату оценки (исследования), руб;\\
$ K_{YTCi} $ -- коэффициент УТС по i-му элементу КТС, ремонтному воздействию, \%.
 


\par  При ремонте съемной составной части сумма стоимости ремонта (включая стоимость разборки для ремонта и при необходимости снятия детали для ремонта) и величины УТС (без учета УТС вследствие окраски) не должна превышать суммы стоимости этой составной части (с учетом коэффициента износа) и стоимости работ по ее замене.

\par   Значение коэффициента УТС $ K_{\text{утсокр}} $ при подетальной окраске наружных поверхностей кузова КТС рассчитывается с учетом количества окрашиваемых кузовных составных частей и бамперов по формуле:

\begin{equation}\label{f:yc}
K_{\text{утсокр}}=K_{\text{утсокр(1)}}+K_{\text{утсокр(N-1)}}\cdot(N-1), %/ 
\end{equation}
        
\noindent где:\\
\noindent $ \text{К}_{\text{утсокр(1)}} $ - коэффициент УТС по окраске первой кузовной составной части или бампера, \%;\\
$ \text{К}_{\text{утсокр(N-1)}} $ - коэффициент УТС по окраске второй и каждой следующей кузовной составной части или бампера, \%;\\
N - количество окрашиваемых составных частей, по которым рассчитывается УТС.\\
Значения коэффициентов УТС ($ K_{YTC} $) определены по результатам экспертой практики и приведены в приложении [1, Приложение 2.9].

\par Для исследуемого автомобиля \тс \, соответствующие ремонтным воздействиям  коэффициенты УТС приведены ниже в таблице:

\begin{table}[H]
		%\caption{}
	\begin{tabular}{|p{5mm}|p{80mm}|c|c|c|}
	\hline 
	\textbf{п/п} & \textbf{Наименование детали} &\textbf{ К-замена }& \textbf{К-ремонт }&\textbf{ К-окраска} \\ 
	\hline 
	1 & Наружная окраска кузова & -- & -- & 5 \\ 
	\hline 
	2 & Бампер задний & -- & -- & 0,35 \\ 
	\hline 
	3 & Дверь правая & -- & 0,2 & -- \\ 
	\hline 
	4 & Панель задка & 0,3 & -- & -- \\ 
	\hline 
	5 & Крыло заднее левое & 0,5 & -- & -- \\ 
	\hline 
	6 & Арка колеса наружная & 0,2 & -- & -- \\ 
	\hline 
	7 & Надставка левая задняя & 0,2 & -- & -- \\ 
	\hline 
	8 &  Устранить перекос проема левой двери & -- & 0,5 & -- \\ 
	\hline 
	8 & Нарушение целостности заводской сборки при полной разборке/сборке салона & -- & 1 & -- \\ 
	\hline 
	
\end{tabular} 

\end{table}

\vspace{7mm}

$  \sum\limits_{i=1}^n K_{YTCi} = 5+1+0.35+0.2+0.3+0.5+0.2+0.2+0.5 = 8.25 $\\
  
$   C_{KTC} = C_{KTC} \cdot \dfrac{\sum\limits_{i=1}^n K_{YTCi}}{100} = 4508000 \cdot 8.25/100 = 371910 $, или с учетом округления 372000 (Триста семьдесят две тысячи) рублей.\\

Таким образом, величина УТС автомобиля \тс\, составляет 372 000 (Триста семьдесят две тысячи) рублей.

%%%%%%%%%%%%%%%%%%%%%%%%%%%%%%%   Условие расчета ущерба
%\par Согласно п.7.2. Медодики [1] размер ущерба вследствие повреждения КТС принимается равным рыночной стоимости КТС, если соблюдается условие:
%\begin{equation}\label{2}
%C_{BP} + C_{YTC} \geqslant C_{KTC},
%\end{equation} 
%где: $C_{KTC} $ -- рыночная стоимость КТС, руб;\\
%\indent $C_{YTC} $ -- величина УТС, руб.\\
%В этом случае рыночная стоимость определяется на заданную дату оценки, с учетом срока его эксплуатации и технического состояния на момент происшествия. \\
%Согласно п. 9.6. Методики [1], в случаях правовых отношений, регулируемым Гражданским кодексом РФ, расчет стоимости годных остатков и определение стоимости реального ущерба с его уменьшением на стоимость годных остатков не предусмотрены.

%%%%%%%%%%%%%%%%%%%%%%%%%%%%%%%%

\section{Выводы}
%
%
%При условии достоверности предоставленных данных в части повреждений ТС \тс \, при заявленных обстоятельствах, 
%эксперт приходит к следующим выводам:\par
\begin{enumerate}
% \item  Наличие, характер и объем (степень) технических повреждений, причиненных ТС, определены при осмотре и зафиксированы в Акте осмотра \NomerDoc.
% \, и фототаблице повреждений, являющимися неотъемлемой частью настоящего экспертного заключения.
% \\[-2mm]
%\item  Направление, расположение и характер повреждений определены путем сопоставления полученных повреждений, изучения административных материалов по рассматриваемому событию, и  являются  следствиями рассматриваемого ДТП (события).\\[-2mm]
%\item  Технология и объем необходимых ремонтных воздействий зафиксированы в калькуляции \NomerDoc \, по определению стоимости восстановительного ремонта транспортного средства \тс. Расчетная стоимость восстановительного ремонта составляет 1\,085\,696 (Один миллион восемьдесят пять тысяч шестьсот девяносто шесть) рублей.
%\\[-2mm]
\item Размер ущерба, причинённого владельцу  транспортного средства \tc\,\- \грз\,  составляет \итог\, (\числопрописью{\итог}) руб.% без учёта износа,  с учётом уменьшения стоимости запасных частей вследствие их износа,  составляет 16 600  (Шестнадцать тысяч шестьсот) рублей.\\[-2mm]   
%\item Размер ущерба, причиненного владельцу  транспортного средства \tc \, в результате дорожно-транспортного происшествия, имевшего место \датадтп\,
%составляет 80\~000 (Восемьдесят тысяч) рублей.\\[-2mm]  
%\item Рыночная стоимость транспортного средства ТС \тс\, на момент повреждения составляет 35000 (Тридцать пять тысяч) рублей.\\[-2mm]
%\item Стоимость годных остатков ТС \тс \, \, составляет 5 000 (Пять тысяч) рублей.
\end{enumerate}
\vspace{8mm}
\noindent{Эксперт-техник}      \hfill                        {Мраморнов А.В.}
\vspace{6mm}
\relax

Приложения к заключению:\\
\noindent \textit{\small 
	%	Приложение № \Rownum. Расшифровка модельных опций ТС \тс \\
    Приложение № \Rownum. Калькуляция стоимости восстановительного ремонта ТС\\
	Приложение № \Rownum. Акт осмотра ТС \тс\\
	Приложение № \Rownum. Фототаблица повреждений ТС \тс\\
%	Приложение № \Rownum. Цифровые копии регистрационных документов ТС\\
%	Приложение № \Rownum. Цифровая копия постановления по делу об административном правонарушении дорожно-транспортном происшествии\\
	Приложение № \Rownum. Правоустанавливающие документы\\}


 \FPdiv\x{6}{3}
%\vspace{20mm}


%\noindent{Эксперт-техник}      \hfill                        {Мраморнов А.В.}

%\includepdf[pages=-]{foto.pdf}
%\includepdf[pages=-]{calc.pdf}
%\includepdf[pages=-]{myfile.pdf}