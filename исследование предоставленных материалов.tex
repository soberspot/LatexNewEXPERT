Согласно предоставленных документов,  исследованию подлежит легковой автомобиль  марки, модели, года выпуска:
\begin{figure}[H]
	\centering
	\includegraphics[width=0.8\linewidth]{example-image}
	\caption{Регистрационные данные исследуемого автомобиля}
	%\label{vin}
\end{figure}


%
Предоставленные для производства исследования документы и материалы позволяют установить следующую историю ремонтов и сервисного обслуживания  транспортного средства VIN \vin, проведенных официальным дилером изготовителя: Таблица \ref{tab:hist}.
%%%%%%%%%%%%%%%%%%%%%%%%%%%%%%%%%%%%%
% История автомобиля
%%%%%%%%%%%%%%%%%%%%%%%%%%%%%%%%%%%%%
%{\small 
	%	\begin{longtable}{|p{16mm}|p{12mm}|p{29mm}|G{50mm}|G{41mm}|}
		%		\caption[]{\footnotesize {\textbf{История ремонта и сервисного обслуживания по дате и пробегу}}} \label{tab:hist}\\
		%		\hline
		%		%\rowcolor[HTML]{C0C0C0} 
		%		% Заголовки столбцов
		%		\textbf{Дата} &\textbf{Пробег, км} &\textbf{№\,Заказ-наряда, накладной}& \textbf{Вид работы}& \textbf{Примечание} \\ \hline \endhead % повторение заголовка 
		%		% Строки
		%%		22.22.2019 &33\,000  & № 480279303-1 от 03.09.2019& Панель задка  & Замена, окраска \\ \hline
		%%		%\rowcolor[HTML]{EFEFEF} 
		%%		\Rownum & &n & Боковина задняя левая   & Замена, окраска \\ \hline
		%		
		%		\ист{arg1}{arg2}{arg3}{arg4}{arg5}
		%		\ист{arg1}{arg2}{arg3}{arg4}{arg5}
		%		\ист{arg1}{arg2}{arg3}{arg4}{arg5}
		%		
		%		
		%		%%% ..............& 
		%		% Обнуляем счетчик строк для следующей таблицы
		%\end{longtable}}
		%\setcounter{rownum}{0} %сброс счетчика строк в таблице
		{\footnotesize \
			\begin{longtable}[h]{m{3mm}|m{14mm}|m{13mm}|m{35mm}|m{55mm}|m{18mm}}
				\caption[]{\footnotesize {\textbf{История ремонта и сервисного обслуживания по дате и пробегу}}} \label{tab:hist} \\ \hline
				\textit{\textbf{n/n}} 
				&\textit{\textbf{Дата}} 
				&\textit{\textbf{Пробег, км}}
				&\textit{\textbf{Документ}} 
				&\textit{\textbf{Содержание}} 
				&\textit{\textbf{Примечание}}\\ \hline \endhead
				
				\Rownum & -- & -- & -- & -- \\
				\hline
				\Rownum & -- & -- & -- & -- \\
				\hline
				\Rownum & -- & -- & -- & -- \\
				\hline
				
				
		\end{longtable}}\setcounter{rownum}{0}
		
		\pagebreak
		
		\noindent Проведя анализ истории автомобиля  можно выделить следующие, вероятно  значимые для исследования, события и признаки:
		\begin{enumerate}
			\item  .............
			\item .................
			\item .............
		\end{enumerate}
		
		
		%%%%%%%%%%%%%%%%%%%%%%%%%%%%%%%%%%%%%%