%\begin{itemize}
%	%
%\item 
	Согласно постановлению \постановление \, по делу об административном правонарушении, \датадтп \,  в 12 часов 04 минуты \второйводитель,  управляя транспортным средством \тса \, двигалась по ул. Калинина со стороны ул. Передерия в сторону ул. Труда, и на пересечении ул. Калинина - ул. Герцена при развороте по зеленому сигналу светофора не уступил дорогу и создал помеху автомобилю \тс \, регистрационный знак \грз \, под управлением водителя Воткович В.В., двигающемуся во встречном направлении. В результате чего автомобиль \тса \, изменил траекторию движения, сместился вправо и столкнулся с автомобилем \tcb, двигающимся попутно справа. В результате столкновения на автомобиле \тс \, повреждено \повреждения.
	На автомобиле \тса \, видимых повреждений нет, на автомобиле \тсб \, повреждены заднее левое крыло, задняя левая дверь с накладкой, передняя левая дверь в задней части с ручкой.
	
	Виновным в совершении ДТП признана  \второйводитель.
	
	Размер страхового возмещения  владельцу автомобиля \тс \, ООО \enquote{ОПТИМА} составил 69680.80 по первичному обращению в страховую компанию плюс 33 100 рублей в качестве доплаты по претензии.
	
	ООО \enquote{ОПТИМА} не согласилось с размером страховой выплаты, которая по ее мнению должна была составить 226 899.67 (199 699 рублей величина восстановительных расходов плюс 27 200 рублей размер утраты товарной стоимости (УТС)) рулей и обратилось в суд.
	
	

	%
%\end{itemize}