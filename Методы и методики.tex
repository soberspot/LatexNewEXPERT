
Для выполнения поставленных задач были применены следующие методы:
\begin{itemize}
	\item  Органолептический метод – исследование и оценка качества объектов с помощью органов чувств
	\item 	Прямой измерительный метод – путем измерения размеров деталей специальными измерительными приборами
	\item Расчётный метод (косвенный измерительный метод) – путём расчётов различных параметров на основе результатов измерений и других данных
	\item Экспертный метод (метод экспертной оценки) — совокупности операций по выбору комплекса или единичных характеристик объекта, определению их действительных значений и оценкой экспертом соответствия их установленным требованиям и/или технической информации
	%\item Метод натурной реконструкции??
	\item [Метод 1, например, методика CRASH3 для расчёта энергии деформации],
	\item [Метод 2, например, закон сохранения импульса],
	\item [Метод 3, например, анализ данных краш-теста аналогичного автомобиля],
	\item [Метод 4, например, анализ стандартов, таких как PART 563---Event Data Recorders].
\end{itemize}

%	\subsection{Методы и методики, применённые при проведении экспертизы}
%Для выполнения поставленных задач были применены следующие методы:
%\begin{itemize}
%	\item [Метод 1, например, методика CRASH3 для расчёта энергии деформации],
%	\item [Метод 2, например, закон сохранения импульса],
%	\item [Метод 3, например, анализ данных краш-теста аналогичного автомобиля],
%	\item [Метод 4, например, анализ стандартов, таких как PART 563---Event Data Recorders].
%\end{itemize}
%
