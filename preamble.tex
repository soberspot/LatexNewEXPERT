% !TEX TS-program = xelatex
% !BIB TS-program = biber
%%%%%%%%%%%%%%%%%%%%%%%%%%%%
\documentclass[a4paper]{article}
\usepackage[utf8]{inputenc}  % Кодировка исходного текста
\usepackage{polyglossia}    %для русского языка
\usepackage[T2A]{fontenc}
\defaultfontfeatures{Ligatures={TeX},Renderer=Basic} 
\setmainfont[Ligatures={TeX, Historic}]{Times New Roman}
\usepackage[left=20mm, top=20mm, right=10mm, bottom=20mm,  footskip=10mm]{geometry} 
\usepackage[fontsize=12pt]{scrextend}
\usepackage{setspace}
\onehalfspacing % Полутерный интервал
%\singlespacing % Одиночный интервал
%\doublespacing % Двойной интервал
%\setstretch{1.25} % Произвольный интервал 
\usepackage{sectsty}
%\allsectionsfont{\centering} - Все разделы по середине 
\sectionfont{\centering\fontspec{Times New Roman}} 
\subsectionfont{\centering\fontspec{Times New Roman}} 
\subsubsectionfont{\raggedright\fontspec{Times New Roman}} 


\usepackage{style/stylefile}  % СТИЛЕВОЙ ФАЙЛ
\usepackage{duckuments}

%%%%%%%%%%%  Библиолграфия
.
\bibliographystyle{gost2008}

\addbibresource{C:/BibLatex/TechnicalExpertise.bib}

%\DeclareSourcemap{
%	\maps[datatype=bibtex]{
%		\map{
%			\step[fieldsource=langid, match=russian, final]
%			\step[fieldset=presort, fieldvalue={a}]
%		}
%		\map{
%			\step[fieldsource=langid, notmatch=russian, final]
%			\step[fieldset=presort, fieldvalue={z}]
%		}
%	}
%}



\usepackage[nottoc,notlot,notlof]{tocbibind} %добавляет в содержание список источников
\makeatletter % |список 
\bibliographystyle{ugost2008} % |литературы
\renewcommand{\@biblabel}[1]{#1.}% |с \makeatother % |точкой




%%%%%%%%%%% Поля документа
\usepackage{geometry}
\geometry{top=20mm}
\geometry{bottom=20mm}
\geometry{left=25mm}
\geometry{right=20mm}
\usepackage{setspace} % Интерлиньяж  
%\onehalfspacing % Интерлиньяж 1.5
% произвольный интервал
\setstretch{1.25}
%


