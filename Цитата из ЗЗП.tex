\textbf{{\Large Закон РФ от 07.02.1992 N 2300-1 (ред. от 11.06.2021) "О защите прав потребителей"}}

\textbf{Статья 19. Сроки предъявления потребителем требований в отношении недостатков товара}


1. Потребитель вправе предъявить предусмотренные статьей 18 настоящего Закона требования к продавцу (изготовителю, уполномоченной организации или уполномоченному индивидуальному предпринимателю, импортеру) в отношении недостатков товара, если они обнаружены в течение гарантийного срока или срока годности.



Существенным недостатком товара является неустранимый недостаток или недостаток, который не может быть устранен без несоразмерных расходов или затрат времени, или выявляется неоднократно, или проявляется вновь после его устранения, или другие подобные недостатки.


(в ред. Федерального закона от 21.12.2004 N 171-ФЗ)