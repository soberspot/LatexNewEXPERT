\begin{enumerate}
	
\item 	    9. ГОСТ 25346-2013 (ISO 286-1:2010) «Основные нормы взаимозаменяемости. Характеристики изделий геометрические. Система допусков на линейные размеры. Основные положения, допуски, отклонения и посадки»
	10. ГОСТ 2.601-95 «Эксплуатационные документы»
	11. ГОСТ 2789-73 « Шероховатость поверхности. Параметры и характеристики»
	12. ГОСТ 18322-2016 «Система технического обслуживания и ремонта техники. Термины и определения»
	13. ГОСТ 27.002­2015 «Надежность в технике. Термины и   определения». 
	14. ГОСТ 12.2.003-74 ССБТ. Оборудование производственное. Общие требования безопасности.
	  
	  В.И. Анурьев. Справочник конструктора машиностроителя. Т. 1-3,  М. Машиностроение, 2001, 920 с.
	  
	  Программа обработки графических изображений ImageJ 1.51d,  распространяется под свободной лицензией GNU LGPL License.
	
	\item   Федеральный закон «Об обязательном страховании гражданской ответственности владельцев транспортных средств» от 25.04.2002 г. № 40-ФЗ.
	\item  Положения Банка России от «19» сентября 2014 года № 431-П «О правилах обязательного страхования гражданской ответственности владельцев транспортных средств».
	\item  Положение Банка России от «19» сентября 2014 года № 432-П «О единой методике определения размера расходов на восстановительный ремонт в отношении повреждённого транспортного средства».
	\item  Положение ЦБ РФ № 433-П «О правилах проведения независимой технической экспертизы транспортного средства» от 19 сентября 2014 г.
	\item  Технический регламент Таможенного союза <<О безопасности колёсных транспортных средств>> (ТР ТС - 018 - 2011).
	%\item  Методические рекомендации по проведению судебных автотехнических экспертиз и исследований колёсных транспортных средств в целях определения размера ущерба, стоимости восстановительного ремонта и оценки / Е. Л. Махнин, И. Н. Новоселецкий, С. В. Федотов и [др.]. - М. : ФБУ РФЦСЭ при Минюсте  России, 2018. - 326 с.
	\item  Технологическое руководство «Приёмка, ремонт и выпуск из ремонта кузовов легковых автомобилей предприятиями автотехобслуживания» РД 37.009.024-92.
	\item  Предотвращение страхового мошенничества в автостраховании  (практическое  пособие)  М.  2005.
	%\item  Исследование транспортных средств в целях определения стоимости восстановительного ремонта и оценки: курс лекций / под общ. ред. д-ра юрид. наук, профессора С.А. Смирновой; Министерство юстиции Российской Федерации, Федеральное бюджетное учреждение Рос. Федер. центр судеб экспертизы. - М.: ФБУ РФЦСЭ при Минюсте России, 2017. - 286 с.
	\item  Методика окраски и расчёта стоимости лакокрасочных материалов для проведения окраски ТС – AZT. 
%	\item  Сервис по автоматической расшифровке VIN номеров – AudaVIN.
%	\item  Сервис РСА для проверки текущего договора ОСАГО,  http://86.62.95.12:8080/dkbm-web-1.0/bsostate.htmhttp://prices.autoins.ru/spares/.
	\item  Онлайн сервис РСА средней стоимости запасной части и нормочаса в экономическом районе,    http://prices.autoins.ru/priceAutoParts/.
	\item \url{https://www.lexus-tech.eu/} - Он-лайн ресурс по предоставлению технической информации по ремонту автомобилей TOYOTA/LEXUS, поставляемых на европейский рынок компанией Toyota Motor Europe.
	\item  	Материалы тематических веб-сайтов сети Интернет\\
	\url{https://partsouq.com}\\
	\url{https://emex.ru}
	\url{https://audatex.ru}
\end{enumerate}