%\left( \addcontentsline{toc}{section}{Использованные нормативы и источники информации}

%\subsection{Использованные нормативы и источники информации}
%
\begin{enumerate}
\item	
Положение Банка России от 19 сентября 2014 года № 432-П {О единой методике определения размера расходов на восстановительный ремонт в отношении повреждённого транспортного средства} // Вестник банка России, № 93 (1571). Нормативные акты и оперативная информация 	Центрального банка Российской Федерации. Москва, 2014
\item 
Махнин\,Е.\,Л., Новоселецкий\, И.\,Н., Федотов\, С.\,В. Методические рекомендации по проведению судебных автотехнических экспертиз и исследований колёсных транспортных средств в целях определения размера ущерба, стоимости восстановительного ремонта и оценки -- М.: ФБУ РФЦСЭ при Минюсте России, 2018.-326 с.  ISBN 978-5-91133-185-6.
%
%
\item  
Безопасность движения автомобильного транспорта. Анализ дорожных происшествий. Б.Е. Боровский – Лениздат, 1984.
\item 
Корухов\,Ю.\,Г., Замиховский\, М.\,И. Криминалистическая фотография и видеозапись для экспертов-автотехников. Практическое пособие М.: ИПК РФЦСЭ при МЮ РФ, 2006г.
\item 
Чава\,И.\,И. Судебная автотехническая экспертиза // Учебно-методическое пособие для  экспертов,    судей, следователей, дознавателей и адвокатов. НП «Судэкс», Москва, 2014.
\item
Краснопевцев\,Б.,В. Фотограмметрия // Учебное пособие. МИИГАиК, 2008.
\item 
Чалкина\,А.\,В.  Осмотр, фиксация и моделирование механизма образования внешних повреждений автомобилей с использованием их масштабных изображений / А.\,В. Чалкин, А.\,Л. Пушнов, В.\,В. Чубченко // Учебное пособие.  М.:ВНКЦ МВД СССР 1991г.
\item 
Расследование дорожно-транспортных происшествий. Селиванов И.А. – «Лига-Разум», М. 1998.
%\item 
% Основы судебно-экспертного исследования технического состояния транспортных средств. – КНИИСЭ, 1987.
\item  
Транспортно-трасологическая экспертиза по делам о дорожно-транспортных происшествиях (диагностическое исследование). Выпуск 1-2 – ВНИИСЭ, М. 1988;
\item 
 СВОД методических и нормативно-технических документов в области экспертного исследования обстоятельств дорожно-транспортного происшествия – ВНИИСЭ, М. 1993.
\item 
 Решение отдельных типовых задач судебной автотехнической экспертизы – ВНИИСЭ, М. 1988.
%\item 
% Методическое пособие для следователей и экспертов «Исследование механизма и условий взаимодействия в трасологии и судебной баллистике». Бергер В.Е., Грановский Г.Л., Прищепа В.М. – ВНИИСЭ, М. 1980.
%\item 
% Судебная дорожно-транспортная экспертиза. Суворов Ю.Б. – «Экзамен», М. 2003.
%
%
%\item ТУ 017207-255-00232934-2014 \emph{Кузова автомобилей LADA. Технические требования при приёмке в ремонт, ремонте и выпуске из ремонта предприятиями дилерской сети ОАО "АВТОВАЗ"}//  ОАО НВП "ИТЦ АВТО", 2014
%%
%\item Смирнов  В.Л., Прохоров  Ю.С., Боюр В.С.  и др. \emph{Автомобили ВАЗ. Кузова. Технология ремонта, окраски и  антикоррозионной защиты. Часть II}// - Н.Новгород: АТИС, 2001.- 241с.
%
\item Судебная автотехническая экспертиза. Институт повышения квалификации Российского Федерального Центра Судебной Экспертизы, М. 2007.
%
%\item 
%Савич Е.Л. \emph{Техническое  обслуживание  и  ремонт  легковых  автомобилей} : учеб. пособие / Е.Л. Савич, М.М. Болбас, В.К. Ярошевич ; под общ. ред. Е.Л. Савича. -Мн. : Вышэйшая школа,  2001. - 479 с. - ISBN985-06-0502-2.
%%
%\item 
%Автомобили ВАЗ-2121, 21213, 21214, 2131 и их модификации: <<Трудоемкости работ (услуг) по техническому обслуживанию и ремонту>> /Куликов А.В., Христов П.Н., Климов В.Е.,  Боюр В.С., Рева В.В., Зимин В.А., Завьялова Н.Н., Хлыненкова Г.А. -- ИТЦТ "АвтоВАЗтехобслуживание", Тольяти -- 2005. 
%%
%\item
%Автомобили LADA SAMARA и их модификации: <<Трудоемкости работ (услуг) по техническому обслуживанию и ремонту>> /Куликов А.В., Христов П.Н., Климов В.Е., Рева В.В., Боюр В.С., Васильев М.В., Фахрутдинов Р.В.,  Прудских Д.А., Гирко В.Б., Шмелева В.А., Зимин В.А. --  ОАО НВП "ИТЦ АВТО",  -- 2006. - 252 стр.
%%
%\item 
%Автомобили LADA PRIORA. Трудоемкости работ (услуг) по техническому обслуживанию и ремонту /Куликов А.В., Христов П.Н., Климов В.Е., Рева В.В., Козлов П.Л., Боюр В.С., Прудских Д.А., Шмелева В.А., Зимин В.А. -- ООО "ИТЦТ АВОСФЕРА", Тольяти -- 2009. -- 344 с.
%%
%\item 
%{Трудоемкости работ по техническому обслуживанию и ремонту автомобилей автомобилей Lada  Granta}/   \url{https://docplayer.ru/30250248-Trudoemkosti-rabot-po-teh\-nicheskomu-obsluzhivaniyu-i-remontu-avtomobiley-lada- granta.html}.
%%%
%%%
%\item
%{Специализированное программное обеспечение для расчёта стоимости  восстановительного ремонта, содержащее нормативы трудоёмкости работ, регламентируемые изготовителями транспортного средства}//   AudaPadWeb, лицензионное соглашение № AS/APW-658  RU-P-409-409435.
%%
%%
%%
%\item
%{Специализированное программное обеспечение для расчёта стоимости  восстановительного ремонта, содержащее нормативы трудоёмкости работ, регламентируемые изготовителями транспортного средства ОАО «АвтоВАЗ», ЗАО «Джи-Эм-АвтоВАЗ», ОАО «СеАЗ» и ОАО «ЗМА»}//   Автосфера АС:Смета, v.3.9.11// ООО "ИТЦ «ИнтегроМаш», \url{https://autosmeta.pro}.
%%
%
%%
%\item Информационный портал по техническому обслуживанию и ремонту автомобилей	 ВАЗ:\\ \url{www.autosphere.ru}.

%%
\end{enumerate}
