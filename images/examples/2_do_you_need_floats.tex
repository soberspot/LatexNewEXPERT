\documentclass[12pt,a4paper]{article}
% пример использования \captionof{}{}
% examples with \captionof{}{}

\usepackage[labelformat=simple]{subcaption}
\renewcommand\thesubfigure{(\alph{subfigure})} % I want (a), not just a

\usepackage[demo]{graphicx} % опция demo вставляет черные прамоугольники вместо картинок
                            % demo option prints black squares instead of pics

\usepackage{array} % для параметра m{} в таблицах
                   % provides m{} parameter for tabulars

\begin{document}

The code below inserts a picture and provides a caption.
{
 \centering
 \includegraphics{mypicture}
 \captionof{figure}{xxxx}\label{fig:mypic}
}

\vspace*{1cm}

Picture in a table:

\noindent
\begin{tabular}{m{0.5\linewidth}m{0.5\linewidth}}
     Here just some text text text text text text text 
     text text text text text text text text text
     text text text text text text text text text 
     text text text text text text text text text 
     text text text text text text text text text
     text text text text text text text text text 
     text text text text text text text text text   
    &
     \centering
     \includegraphics{roman.jpg}
     \captionof{figure}{Usage of captionof}\label{fig:intab}
\end{tabular}

\clearpage %%%%%%%%%%%%%%%%%%%%%%%%%%%%%%%%%%%%%%%%%%%

\begin{figure}\centering % В одном флоат...
  % сначала картинка...
  \includegraphics[width=7cm]{fig/roman_b}
  \captionof{figure}{Figure part of the float}\label{fig:fig}
\vspace*{3em}
  % потом таблица
  \begin{tabular}{c|cc}
    aaaaaaa & bbbbbbb & cccccccccc \\ \hline
    zzzzzzz & xxxxxxx & vvvvvvvvv  \\
    aaaaaaa & bbbbbbb & cccccccccc \\
    zzzzzzz & xxxxxxx & vvvvvvvvv  \\
    aaaaaaa & bbbbbbb & cccccccccc \\
    zzzzzzz & xxxxxxx & vvvvvvvvv  \\
    zzzzzzz & xxxxxxx & vvvvvvvvv 
  \end{tabular}
  \captionof{table}{Table part of the float}\label{fig:table}
\end{figure}

\clearpage %%%%%%%%%%%%%%%%%%%%%%%%%%%%%%%%%%%%%%%%%%%
\newsavebox{\leftpic}

\begin{figure}[t]\centering
% Левая картинка (а) помещена в бокс, чтобы измерить её высоту
\sbox{\leftpic}%
{% Левая картинка (a):
 \begin{subfigure}[b]{0.45\linewidth}\centering
   \includegraphics[height=7cm]{fig/roman_a1}
   \caption{Subfigure A}\label{fig:2a}
 \end{subfigure}%
}
 %------------------------
 % Вывысти картинку, сохраненную в боксе
 \usebox{\leftpic}
 \quad % немного пустого места между левой и правой картинками
 % Две правые картинки в минипейдж, 
 %   - высота которого равна высоте левой картинки: \ht\leftpic
 %   - материал будет растянут вертикально: [s] + \vfill 
 \begin{minipage}[b][\ht\leftpic][s]{0.45\linewidth}
  \begin{center}
   \includegraphics[height=2.5cm]{fig/roman_b}
   \captionof{subfigure}{Subfigure B}\label{fig:2b}
  \end{center}

  \vfill

  \begin{center}
   \includegraphics[height=2.5cm]{fig/roman_c}
   \captionof{subfigure}{Subfigure C}\label{fig:2c}
  \end{center}
 \end{minipage}
\caption{Fancy placement of subfigures}
\label{fig:2}
\end{figure}

\clearpage %%%%%%%%%%%%%%%%%%%%%%%%%%%%%%%%%%%%%%%%%%%

\begin{figure}[!t]\centering
\setcounter{subfigure}{0}
\addtocounter{figure}{1}
\def\hgt{15cm}
\def\wdt{5cm}
 \begin{minipage}[b][\hgt][s]{0.45\linewidth}
  \begin{center}
   \includegraphics[width=\wdt]{fig/roman_b}
   \captionof{subfigure}{Subfigure A}\label{fig:3a}
  \end{center}

  \vfill

  \begin{center}
   \includegraphics[width=\wdt]{fig/roman_c}
   \captionof{subfigure}{Subfigure B}\label{fig:3bb}
  \end{center}

   \vfill

  \begin{center}
   \includegraphics[width=\wdt]{fig/roman_aa}
   \captionof{subfigure}{Subfigure C}\label{fig:3c}
  \end{center}
 \end{minipage}
%
 \quad
%
 \begin{minipage}[b][\hgt][s]{0.45\linewidth}
  \begin{center}
   \includegraphics[angle=90,width=\wdt]{fig/roman_b}
   \captionof{subfigure}{Subfigure D}\label{fig:3d}
  \end{center}

  \vfill

  \begin{center}
   \includegraphics[width=7cm,height=5cm]{fig/roman_c}
   \captionof{subfigure}{Subfigure E}\label{fig:3e}
  \end{center}
 \end{minipage}
 \addtocounter{figure}{-1}
 \caption{Fancy placement of subfigures}\label{fig:3}
\end{figure}

Example of a reference: Figure \ref{fig:3} but also \ref{fig:3bb}.

\clearpage %%%%%%%%%%%%%%%%%%%%%%%%%%%%%%%%%%%%%%%%%%%%%%%%%%%%%

Figures \ref{fig:x1}-\ref{fig:x5} show the same trick as 
in Figure \ref{fig:3} but with figures instead of 
subfigures


\begin{figure}[!t]\centering
\def\hgt{15cm}
\def\wdt{5cm}
 \begin{minipage}[b][\hgt][s]{0.45\linewidth}
  \begin{center}
   \includegraphics[width=\wdt]{fig/roman_b}
   \captionof{figure}{caption x1}\label{fig:x1}
  \end{center}

  \vfill

  \begin{center}
   \includegraphics[width=\wdt]{fig/roman_c}
   \captionof{figure}{caption x2}\label{fig:x2}
  \end{center}

   \vfill

  \begin{center}
   \includegraphics[width=\wdt]{fig/roman_aa}
   \captionof{figure}{caption x3}\label{fig:x3}
  \end{center}
 \end{minipage}
%
 \quad
%
 \begin{minipage}[b][\hgt][s]{0.45\linewidth}
  \begin{center}
   \includegraphics[angle=90,width=\wdt]{fig/roman_b}
   \captionof{figure}{caption x4}\label{fig:x4}
  \end{center}

  \vfill

  \begin{center}
   \includegraphics[width=7cm,height=5cm]{fig/roman_c}
   \captionof{figure}{caption x5}\label{fig:x5}
  \end{center}
 \end{minipage}
%
\end{figure}



%%%%%%%%%%%%%%%%%%%%%%%%%%%%%%%%%%%%%%%%%%%%%%%%%%%%%%%%%%%%%%


\end{document}
