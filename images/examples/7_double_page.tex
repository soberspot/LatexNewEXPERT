\documentclass[12pt,a4paper,twoside]{book}
% пример использования Double Page Float и \ContinuedFloat
% examples with Double Page Float & \ContinuedFloat

\usepackage{dpfloat}

\usepackage{caption}
\usepackage[labelformat=simple]{subcaption}
\renewcommand\thesubfigure{(\alph{subfigure})} % I want (a), not just a

%---------------------------------------
\usepackage{lipsum} % вставляет куски текста / dummy text
\usepackage[demo]{graphicx} % опция demo вставляет черные прямоугольники вместо картинок
                            % demo option prints black squares instead of pics

% для краткости, новое окружение / just a shortcut
\newenvironment{mysubfig}
{\begin{subfigure}{0.5\linewidth}\centering}
{\end{subfigure}}


\begin{document}

See very long Figure \ref{fig:long}, which consists of two parts:
the first is on page \pageref{fig:A} and the other on page \pageref{fig:I}.
If printed in a book (e.g., PhD dissertation), it is better to put them 
on two consecutive pages: the first part on even (i.e. left) page
and the second on an odd (i.e. right) page.

\begin{figure}
 \begin{leftfullpage}
  \begin{mysubfig} % 1-й "подрисунок" / 1st subifure
     \includegraphics{qq}
     \caption{AAAAAA}\label{fig:A}
  \end{mysubfig} 
  \begin{mysubfig}   % 2-й "подрисунок" / 2nd subifure
     \includegraphics{qq}
     \caption{BBBBBB}\label{fig:B}
  \end{mysubfig} 
  \begin{mysubfig}   % 3-й "подрисунок" / 3rd subifure
     \includegraphics{qq}
     \caption{CCCCCC}\label{fig:C}
  \end{mysubfig} 
  \begin{mysubfig}   % 4-й "подрисунок" / 4th subifure
     \includegraphics{qq}
     \caption{DDDDDD}\label{fig:D}
  \end{mysubfig} 
  \begin{mysubfig}   % 5-й "подрисунок" / 5th subifure
     \includegraphics{qq}
     \caption{EEEEEE}\label{fig:E}
  \end{mysubfig} 
  \begin{mysubfig}   % 6-й "подрисунок" / 6th subifure
     \includegraphics{qq}
     \caption{FFFFFF}\label{fig:F}
  \end{mysubfig} 
  \begin{mysubfig}   % 7-й "подрисунок" / 7th subifure
     \includegraphics{qq}
     \caption{GGGGGG}\label{fig:G}
  \end{mysubfig} 
  \begin{mysubfig}   % 8-й "подрисунок" / 8th subifure
     \includegraphics{qq}
     \caption{HHHHHH}\label{fig:H}
  \end{mysubfig} 
  %---------------------
  \caption{LEFT PART of A very long figure with subfigures}\label{fig:long}
 \end{leftfullpage}
\end{figure}

% - - - -

\begin{figure} \ContinuedFloat % продолжение рисунка
 \begin{fullpage}
  \begin{mysubfig}   % 9-й "подрисунок" / 9th subifure
     \includegraphics{qq}
     \caption{IIIIII}\label{fig:I}
  \end{mysubfig} 
  \begin{mysubfig}   % 10-й "подрисунок" / 10th subifure
     \includegraphics{qq}
     \caption{JJJJJJ}\label{fig:J}
  \end{mysubfig} 
  \begin{mysubfig}   % 11-й "подрисунок" / 11th subifure
     \includegraphics{qq}
     \caption{KKKKKK}\label{fig:K}
  \end{mysubfig} 
  \begin{mysubfig}   % 12-й "подрисунок" / 12th subifure
     \includegraphics{qq}
     \caption{LLLLLL}\label{fig:L}
  \end{mysubfig} 
  \begin{mysubfig}   % 13-й "подрисунок" / 13th subifure
     \includegraphics{qq}
     \caption{MMMMMM}\label{fig:M}
  \end{mysubfig} 
  \begin{mysubfig}   % 14-й "подрисунок" / 14th subifure
     \includegraphics{qq}
     \caption{NNNNNNN}\label{fig:N}
  \end{mysubfig} 
   %----------------------
  \caption{RIGHT PART of A very long figure with subfigures (continued)} % \label здесь не нужна!!! / no need for \label here

 \end{fullpage}
\end{figure}

Just dummy text: \lipsum[1-2]


%%%%%%%%%%%%%%%%%%%%%%%%%%%%%%%%%%%%%%%%%%%%%%%%%%%%%%%%%%%%%%


\end{document}
