В исследовании применяются следующие термины и определения:
\begin{description}
	\item
	[Аварийные повреждения] -- повреждения, механизм образования которых определяется контактом с посторонними объектами, что привело к деформации или разрушению и к необходимости ремонта или замены составной части, или контактам с агрессивной средой, которая привела к необходимости ремонта (замены) составной части, \cite[часть II, п. 1.5]{minust:2018}.
	%	\item[Восстановительный ремонт]-- один из способов возмещения ущерба, состоящий в выполнении технологических операций ремонта КТС, действующий в сети торгово-сервисного обслуживания, созданной изготовителем этого КТС [1, часть II, п. 1.4].
	%	\item[Годные остатки] -- работоспособные, имеющие остаточную стоимость детали (агрегаты, узлы) поврежденного автотранспортного средства, годные к дальнейшей эксплуатации, которые можно демонтировать с поврежденного автотранспортного средства и реализовать.
	\item
	[Дата исследования]-- дата, на которую проводятся расчёты и используются стоимостные данные КТС, запасных частей, материалов, нормо-часа ремонтных работ;\autocite[часть II, п. 1.5]{dor:2016}.% [1, часть II, п. 1.5].
	%    \item [Декоративные свойства лакокрасочного покрытия] -- способность лакокрасочного покрытия придавать окрашенной
	%    поверхности заданный цвет и блеск
	\item
	[Дефект] -- это каждое отдельное несоответствие объекта требованиям, установленным документацией; \autocite{18322:gost}
	\item
	[Исправное состояние (исправность)] -- состояние объекта, при котором он соответствует всем требованиям, установленным в документации на него;
%	\item
%	[Конструктивный отказ] -- отказ, возникший по причине, связанной с несовершенством или нарушением установленных правил и (или) норм проектирования и конструирования;
	%\item
	%[Защитные свойства лакокрасочного покрытия] --
	%    Способность лакокрасочного покрытия предотвращать или замедлять
	%    коррозию металлических или разрушение неметаллических поверхностей в
	%    условиях агрессивного воздействия внешних факторов.
	%\item
	%[Лак] -- продукт, который после нанесения на поверхность образует твёрдую прозрачную 	плёнку, обладающую защитными, декоративными или специальными техническими свойствами.
	\item
	[Лакокрасочное покрытие (ЛКП)] -- сплошное покрытие, полученное в результате нанесения 	одного или нескольких слоёв лакокрасочного материала на окрашиваемую поверхность
	%\item
	%[Линия удара]-- линия, определяемая направлением вектора равнодействующего импульса сил, возникающих при контакте ТС при столкновении до прекращения взаимного внедрения деформирующихся при ударе частей. Положением линии удара на ТС определяются направление и величина момента импульса сил, возникающих при ударе, и, следовательно, направлением и интенсивность разворота ТС относительно центра масс после столкновения.  
%	\item
%	[Моделирование]-- исследование каких-либо явлений, процессов или систем объектов путем построения и изучения их моделей;
%	\item
%	[Малозначительный дефект] -- дефект, который существенно не влияет на использование продукции по назначению и ее долговечность;
%	\item
%	[Механизм отказа] -- процесс, который приводит к отказу
%	\item
%	[Морфологические признаки]-- признаки, отображающие внешнее и внутреннее строение объекта;
	%\item
	%[Недостаток лакокрасочного покрытия] -- отклонение лакокрасочного покрытия от 	требований нормативно-технической документации, образовавшееся в процессе нанесения и
	%    формирования лакокрасочного покрытия (производственный недостаток)
%	\item
%	[Неработоспособное состояние (неработоспособность)] -- состояние объекта, в котором он не способен выполнять хотя бы одну требуемую функцию по причинам, зависящим от него или из-за профилактического технического обслуживания;
%	\item
%	[Неисправное состояние (неисправность) ] -- это состояние объекта, при котором он не соответствует хотя бы одному из требований, установленных в документации на него;
	\item
	[Неустранимый дефект] -- дефект, устранение которого технически невозможно или экономически нецелесообразно;
	\item
	[Отказ]  -–  событие, заключающееся в нарушении работоспособного состояния объекта;
	
	\item[Пластичность] --  способность  материала
	приобретать  необратимые  изменения  формы  под действием нагрузки;
%	\item
%	[Производственный отказ] -- отказ, возникший по причине, связанной с несовершенством или нарушением установленного процесса изготовления или ремонта, выполняемого на ремонтном предприятии;
%	\item
%	Производственный (технологический) дефект] -- дефект, вызванный нарушением установленной технологии изготовления детали, узла, агрегата;
	\item
	[Работоспособное состояние] -- состояние объекта, в котором он способен выполнять требуемые функции;
	\item[Предел упругости ] -- свойство вещества, максимальная нагрузка, после снятия которой не возникает остаточных (пластических) деформаций;
	\item
	[ Прочность] --свойство материала сопротивляться разрушению под действием внешних сил;
	\item
	[Скрытый отказ] -- отказ, не обнаруживаемый визуально или штатными методами и сред-ствами контроля и диагностирования, но выявляемый при проведении технического обслуживания или специальными методами диагностирования;
	%    \item[Срок эксплуатации КТС]-- период времени от даты изготовления (даты выпуска) КТС, до даты оценки (исследования), определяемой условиями задачи исследования (независимо от даты его регистрации и начала использования по назначению (эксплуатации))
	\item
	[Упругость] --  способность  материалов  изменять форму  под  действием  нагрузки  и  возвращаться  в исходное состояние после снятия нагрузки
	\item
	[Устранимый дефект] -- дефект, устранение которого возможно путем технического;
	обслуживания или ремонта
	
	
	%    \item[Эмаль] -- жидкий или порошкообразный продукт, содержащий пигменты, который после
	%    нанесения на поверхность образует непрозрачную плёнку, обладающую защитными,
	%    декоративными или специальными техническими свойствами.
	\item
	[Эксплуатационный отказ] -- отказ, возникший по причине, связанной с нарушением уста-новленных правил и (или) условий эксплуатации;
%	\item
%	[Явный отказ] -- отказ, обнаруживаемый визуально или штатными методами и средствами контроля и диагностирования при подготовке объекта к применению или в процессе его применения;
	
	\item[Экспертное исследование] --- процесс исследования объектов, представленных на экспертизу с целью получения новых знаний об объекте исследования, характеризующийся объективностью, воспроизводимостью, доказательностью, точностью \cite[п.3.74]{58197:gost} 
	
%	
%	\item[Экспертиза качества автотранспортного средства] --- научно-техническая услуга, заключающаяся в проведении прикладного исследования с применением системы специальных, научных и технических, познаний в области конструирования, производства и эксплуатации объекта экспертизы, выполняемая экспертом, являющимся специалистом в данной области, с целью установления определённых параметров, определяющих качество, работоспособность, причины и время возникновения дефектов, повреждений и неисправностей, а также возможность их обнаружения, и представ-ления научно обоснованного письменного акта экспертного исследования об установленных фактах, отражающего порядок и результаты исследований ГОСТ Р58197-2018 п. 3.72
	
	
%	\item[Экспертный причинный анализ (для целей настоящего стандарта)] --- Исследование причинной связи между выявленными дефектами транспортного средства (его деталями, узлами, механизмами), конструктивными реше-ниями, нормами технологии изготовления (обслуживания и ремонта), условиями хранения; нарушениями условий и правил эксплуатации, установленных изготовителем ГОСТ Р58197-2018 п. 3.75
	
	\item[Эксплуатационный дефект] --- дефект, возникший в результате нарушения установленных правил и (или) условий эксплуатации объекта, установленных изготовителем. Примечание - При применении указанного термина следует указывать, какие правила и (или) условия эксплуатации, установленные изготовителем автомототранспортного средства в эксплуатационных документах, нарушил его владелец. \cite[п.76]{58197:gost} 
	
	\item[Эксплуатация] --- стадия жизненного цикла изделия (автомототранспортного средства), на которой реализуется, поддерживается и восстанавливается его качество. Примечание - Эксплуатация изделия включает в себя: использование по назначению, транспортирование, хранение, техническое обслуживание и ремонт [ГОСТ 25866]. Для целей исследования качества следует считать началом эксплуатации (началом полноценного функционирования) момент завершения процесса сборки объекта и его переход в сложнонапряженное состояние, вызывающее износ конструкции, \cite[п.3.77]{58197:gost} 
	
	\item[Серийная комплектация  АМТС (серийное оборудование)] --- оборудование, которое устанавливается заводом-изготовителем на всех АМТС данной модификации (серии) в обязательном порядке. 
	
	\item[Условия эксплуатации] --- совокупность внешних факторов, оказывающих влияние на расходование ресурса АМТС (износ АМТС). К ним относятся: режим движения и нагрузка на АМТС, дорожные и климатические условия, качество топлива, смазочных материалов, технического обслуживания и мастерства вождения. 
	
	\cite{vin:z}
	
	\cite{molchanov:dlb}
	
	%%%%%%%%%%%%%%%%%%%%%%%%%%%%%%%%%%
	%Дефект – неисправное состояние объекта, характеризуемое выходом его параметров за допустимые пределы, но не делающее его неработоспособным. Дефект может предшествовать отказу агрегата.
%	Исправное состояние – состояние объекта, при котором он соответствует всем требованиям нормативно-технической и (или) конструкторской (проектной) документации
%	Неисправность (неисправное состояние) – состояние объекта, при котором он не соответствует хотя бы одному из требований, указанных в нормативно-технической и/или конструкторской документации.
%	Конструктивный дефект – дефект, вызванный нарушением установленных норм проектирования (конструирования) агрегата.
%	Изнашивание – процесс изменения размеров детали в результате отделения с ее поверхности частиц материала.
%	Повреждение – событие, характеризуемое нарушением исправного состояния объекта при сохранении его работоспособности.
%	Производственный (технологический) дефект – дефект, вызванный нарушением установленной технологии изготовления объекта.
%	Устранимый дефект – дефект, устранение которого возможно путем технического обслуживания или ремонта объекта.
%	Неустранимый дефект – дефект, устранение которого технически невозможно или экономически нецелесообразно и требует замены объекта в сборе.
%	Отказ – выход из строя или поломка агрегата в процессе его работы из-за неправильной эксплуатации, плохого обслуживания (ремонта) и/или конструктивно-производственных дефектов. Выход из строя (поломка) характеризуется таким отклонением параметров объекта от нормативных, при которых он становится неработоспособен.
%	Перегрев – состояние агрегата, при котором температура деталей становится намного выше их нормальной рабочей температуры.
%	Работоспособное состояние – состояние объекта, при котором значения всех параметров, характеризующих способность выполнять заданные функции, соответствуют требованиям нормативно-технической (проектной) документации. 
%	Ремонт – комплекс операций по восстановлению исправности или работоспособности изделий и восстановлению ресурсов изделий или их составных частей.
%	Масляное голодание – режим работы трущихся деталей в условиях недостаточной подачи смазочного материала в зазор между ними, что приводит к соприкосновению деталей по микронеровностям, увеличению трения и температуры деталей, задирам и заклиниванию.
%	Капитальный ремонт (ПО ГОСТ 18322-2016. Система технического обслуживания и ремонта техники. Термины и определения) – Ремонт, выполняемый для восстановления исправности и полного или близкого к полному восстановлению ресурса изделия с заменой или восстановлением любых его частей, включая базовые.  
%	Примечание. Значение близкого к полному ресурсу устанавливается в нормативно-технической документации.
%	Предельное состояние – состояние объекта, при котором его дальнейшая эксплуатация недопустима или нецелесообразна, либо восстановление его работоспособного состояния недопустимо или нецелесообразно.
%	Текущий ремонт – ремонт, выполняемый для обеспечения или восстановления работоспособности изделия и состоящий в замене и (или) восстановление отдельных частей.
%	Техническое состояние – совокупность подверженных изменению в процессе производства или эксплуатации свойств объекта, характеризуемая в определенный момент времени признаками, установленными технической документацией на этот объект.  
%	Примечание. Видами технического состояния являются исправность, работоспособность, неисправность, неработоспособность и т.д.
%	Эксплуатационный отказ – отказ, возникший по причине, связанной с нарушением установленных правил и (или) условий эксплуатации.
%	Явный отказ – отказ, обнаруживаемый визуально или штатными методами и средствами
%	контроля и диагностирования при подготовке объекта к применению или в процессе
%	его применения.
%	
	
	
	
	
\end{description}