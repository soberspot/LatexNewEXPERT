%\subsection{Вопросы экспертизы}
%\subsection{Вопрос исследования}
\begin{enumerate}
	
	\item \enquote{Какие изменения были внесены покупателем в оборудование - линия по переработке Ecogold 700 после его ввода в эксплуатацию; повлияли ли произведённые покупателем изменения оборудование - линия по переработке Ecogold 700 на его работоспособность и иные характеристики? Была ли необходимость, рациональность во внесении изменений в Оборудование со стороны покупателя?}
	
	\item \enquote{Соответствует ли действительности показания/значения/режимы шкафа? Являются ли зафиксированные шкафом внештатные ситуации, подтверждением ненадлежащей эксплуатации оборудования? Вносились ли изменения в настройки шкафа либо в его конструкцию?}
	
	\item \enquote{Есть ли износ (фактический износ) оборудование - линия по переработке Ecogold 700? Какова стоимость данного износа оборудование - линия по переработке Ecogold 700?}
	
	\item \enquote{Соответствуют ли технические характеристики и производительность оборудования - линии по переработке Ecogold 700 - договору купли-продажи № 67551 от 26 марта 2019 года, а также технической и эксплуатационной документации, переданной покупателю в момент поставки оборудования? Какова реальная производительность оборудования в КГ/Ч?}
	
	\item \enquote{Имеются ли в оборудовании, линии по переработке шин Ecogold 700, недостатки (дефекты), контрафактные детали? Если имеются, то какие и каковы причины  их возникновения? Являются ли обнаруженные дефекты оборудования существенными недостатками. неустранимыми недостатками и/или недостатками, которые не могут быть устранены без несоразмерных расходов или затрат времени, или выявляются неоднократно, либо проявляются вновь после устранения? Каковы причины возникновения в процессе эксплуатации открытого огня, случаев возгорания и задымления и есть ли риск повторного задымления и возгорания?}
	
	\item \enquote{Являются ли обнаруженные дефекты оборудования производственными и/или эксплуатационными?}
	
	\item \enquote{В случае наличия производственных дефектов оборудования определить стоимость ремонтных работ, произведённых ООО «ИКАРА», для их устранения, а также стоимость фактического износа с учётом определённых затрат?}
	
	\item \enquote{Меняет ли техническая документация в новой редакции условия эксплуатации и производительность для покупателя? Если да, то в каких разделах?}
	
	\item \enquote{Были ли внесены изменения в механическую и электронную начинку шкафа, влияющие на технические характеристики?}
	
	\item \enquote{Какие части/детали были изменены истцом в оборудовании, повлияющие на технические характеристики оборудования?}
	
	\item \enquote{Производилась ли своевременное и надлежащее техническое обслуживание оборудования?}
	
	\item \enquote{Пригодно ли к использованию в климатических условиях (место нахождения 	Краснодарский край) оборудование - линия по переработке Ecogold 700 с заявленной производительностью в договоре купли-продажи № 67551 от 26 		марта 2019 года?}
	
%	\item <<Какова стоимость восстановительного ремонта, в рамках закона об ОСАГО,  автомобиля LEXUS RX300, государственный номер У755ТТ123, VIN:~JTJZAMCA302037447, 2018 года выпуска, цвет белый, от полученных повреждений в результате ДТП, произошедшего 09.08.2020>>?
%\item  <<Установить наличие, характер и объем (степень) технических повреждений транспортного средства  \tc?>>
%\item  <<Установить причины возникновения технических повреждений транспортного средства \tc \,и возможность их отнесения к рассматриваемому дорожно-транспортному происшествию (далее ДТП)?>>
%\item <<Установить технологию, объем восстановительного  ремонта \!транспортного средства \tc?>>
%\item <<Установить размер затрат на восстановительный ремонт (с учётом износа) транспортного средства \tc?>>
%\item <<Определить размер ущерба, причиненного владельцу  транспортного средства \tc\,\грз\, \, в результате дорожно-транспортного происшествия, имевшего место \датадтп?>>
%\item <<Определить стоимость восстановительного ремонта  транспортного средства \tc\, регистрационный знак \грз,\, \, получившего механические повреждения в результате противоправных действий, имевших место \датадтп?>>
%\item <<Определить величину физического износа  транспортного средства \tc\,\грз\, \, получившего повреждения в результате дорожно-транспортного происшествия, имевшего место \датадтп?>>
%%
%	
%	\item
%	Какие неисправности имеет двигатель \двигатель\, самоходной машины  KOMATSU   АВТОПОГРУЗЧИК  FG15T-20   2007 года выпуска, заводской номер 661043?
%	
%	\item
%	Какова причина их возникновения?
%	
%	\item
%	Причина выхода из строя  двигателя \двигатель\,  имеет производственный или эксплуатационный характер?
%	
%	\item
%	Могли ли имеющиеся у  двигателя \двигатель\, неисправности возникнуть вследствие капитального ремонта?
%	\item  <<Связано ли повреждение панели рамки радиатора слева и брызговика с лонжероном переднего левого автомобиля ВАЗ 21099 с указанным ДТП?>>	
% \item  <<Что послужило причиной выхода двигателя автомобиля из строя?>>	   
%    \item  <<Является ли данная причина:
%\begin{itemize}
%        \item производственной, т.е. недостатком сборки и/или материала;
%        \item связанной с некачественным/несвоевременным обслуживанием автомобиля, включая ежедневный осмотр;
%        \item связанной с неразрешенными/недопустимыми переделками агрегата и/или его систем;
%        \item связанной с предыдущим ремонтом (если применимо);
%        \item эксплуатационной, т.е. возникшей по причине неправильной/ненормальной эксплуатации;
%        \item  естественным износом в соответствии с пробегом автомобиля?>> 
%\item  Определить причины обрывов/разрывов тел жестких каркасно-сборных конструкций, а также деформации жестких каркасно-сборных конструкций на глубокорыхлителях КАМА ТГР 55.7-300, КАМА ТГР 55.9-400?
%\item 	Какие конструктивные, эксплуатационные или иные факторы способствовали поломке?
%\item 	Имелись ли в глубокорыхлителях КАМА ТГР 55.7-300, КАМА ТГР 55.9-400 на момент поломок неисправности, способствовавшие или явившиеся причинами указанных повреждений?
%\item 	При положительном ответе на вопрос № 3 указать степень очевидности таких неисправностей (дефектов) и возможности их выявления при текущем или сервисном осмотре глубокорыхлителей специалистами соответственно эксплуатанта или сервисного центра.
%\item 	При выявлении дефектов  определить являются ли выявленные дефекты недостатки устранимыми без несоразмерных расходов и затрат, выявляются неоднократно, появляются вновь после устранения, препятствуют ли они нормальной работе глубокорыхлителей ТГР 55.7-300, КАМА ТГР 55.9-400?
%\item 	При выявлении дефектов, образовавшихся до поломки, определить характер дефекта заводской, производственный или эксплуатационный (приобретенный)?
%\item 	При выявлении дефектов в глубокорыхлителях КАМА ТГР 55.7-300, КАМА ТГР 55. 9-400  установить, имелась ли причинно-следственная связь между дефектом и причиной поломки (повреждения) глубокорыхлителей?
%\item 	Соответствуют ли повреждения глубокорыхлителей с фото и видео приложенной флеш-карты повреждениям глубокорыхлителей КАМА ТГР 55.7-300, КАМА ТГР 55.9-400 согласно материалам дела № А32-31734/2020?
%\item 	Если да, то имеются ли на глубокорыхлителях с фото и видео приложенной флеш-карты конструктивные изменения глубокорыхлителей КАМА ТГР 55.7-300, КАМА ТГР 55.9-400 в виде замены или укрепления срывных болтов?
%\item 	Если да, то	могли ли конструктивные изменения в виде замены или укрепления срывных болтов повлиять на образование деформаций жестких каркасно-сборных конструкций глубокорыхлителей КАМА ТГР 55.7-300, КАМА ТГР 55.9-400?
% 	\end{itemize}
%	
\end{enumerate}