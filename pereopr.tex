% example-image

%\renewcommand{\chaptername}{Заключение эксперта}
\renewcommand{\refname}{\large{Справочные материалы и нормативные документы}}
\renewcommand{\bibname}{\footnotesize{Список использованных источников}}

\renewcommand{\epsilon}{\ensuremath{\varepsilon}}
\renewcommand{\phi}{\ensuremath{\varphi}}
\renewcommand{\kappa}{\ensuremath{\varkappa}}
\renewcommand{\le}{\ensuremath{\leqslant}}
\renewcommand{\leq}{\ensuremath{\leqslant}}
\renewcommand{\ge}{\ensuremath{\geqslant}}
\renewcommand{\geq}{\ensuremath{\geqslant}}
\renewcommand{\emptyset}{\varnothing}

%%%%%%%%%%%%%%%%%%%%%%%%%%%%%%%%%   ПРОИЗВОЛЬНЫЙ СЧЕТЧИК
\newcounter{@nnnn}  % задаём имя счёчика 
\setcounter{@nnnn}{0}  % устанавливаем его первое значение

\newcommand{\пп}{\stepcounter{@nnnn}%
	\arabic{@nnnn}}



\newcounter{@nnn}  % задаём имя счёчика 
\setcounter{@nnn}{0}  % устанавливаем его первое значение

\newcommand{\z}[2]{\par\addtocounter{@nnn}{1}  % формируем комманду 
	{\bf \arabic{@nnn}.   Работы по заказ-наряду  #1, произведеные на автомобиле #2:}}


%%%%%%%%%%%%%%%%%%%%%%%  Подсчет строк в таблице
%\newcounter{rownum}
\setcounter{rownum}{0}
\newcommand{\Rownum}{\stepcounter{rownum}%
\arabic{rownum}}

%\def\contentsname{Содержание}
%Аннотация  \abstractname
%Часть       \partname
%Глава        \chaptername
%Список литературы  \refname
%Рис.                \figurename
%Таблица           \tablename
%Литература       \bibname

%Предметный указатель  \indexname
%Приложение                \appendixname
%Содержание          \contentsname
%Список иллюстраций \listfigurename
%Список таблиц        \listtablename
%\addto\captionsrussian{\def\refname{Список используемой литературы}}

%%%%%%%%%%%%%%   Размещение изображений
%\textfloatsep — расстояние между флоатс (в верхней или нижней части страницы) и текстом (по умолчанию, около 20pt)
%\floatsep — вертикальное расстояние между двумя флоатс (около 12pt)
%\intextsep — расстояние между флоатс вставленным "прямо здесь" (параметр h) и текстом (около 12pt)
%\abovecaptionskip и \belowcaptionskip — расстояние над и под подписью к флоат
\setcounter{totalnumber}{10}
\setcounter{topnumber}{10}
\renewcommand{\topfraction}{1}
\renewcommand{\textfraction}{0}
%%%%%%  Больше плавающих объектов на страницу
 \setlength{\textfloatsep}{10pt plus 1.0pt minus 2.0pt}
 \setlength{\floatsep}{5pt plus 1.0pt minus 1.0pt}
 \setlength{\intextsep}{5pt plus 1.0pt minus 1.0pt}
 
  
 %%%%%%%%%%%%%%%%   Число прописью
 \newcommand{\числопрописью}[1]
 {\numnameru{#1}}
 %%%%%%%%%%%%%%%%%%%%%%%%%%%%%%%%%%%%%%%%%%%%%%%%%%%%%%%%%%%%%%%
%%%%%%%  Увеличить межсимвольный интервал
%%%%%%%%%
\newcommand{\растянуть}[2]
{
 \addfontfeature{LetterSpace=#2}
 {#1}
 \addfontfeature{LetterSpace=0.0}
}
%%%%%%%%%%%%%%%%%%%%%%%%%%%%%%%%%%%%%%%%%%%%%%%%%%%%%%%%%%%%%%%
%
%   Заметка на полях  (ремарка)
%
%%%%%%%%%%%%%%%%%%%%%%%%%%%%%%%%%%%%%%%%%%%%%%%%%%%%%%%%%%%%%%%%
\newcommand{\rem}[1]
{
\marginpar{\scriptsize\textcolor{red}{#1}}
}
\newcommand{\рем}[1]
{
	\marginpar{\scriptsize\textcolor{red}{#1}}
}

%%%%%%%%%%%%%%%%%%%%%%%%%%%%%%%%%%%%%%%%%%%%%%%%%%%% ПЕРЕОПРЕДЕЛЕНИЕ ФОРМАТИРОВАНИЯ ЯЧЕЕК ТАБЛИЦЫ%%%%%%%%%%
%
\newcolumntype{P}[1]{>{\centering\arraybackslash}p{#1}}   %  \centering   \raggedleft  \raggedright
\newcolumntype{M}[1]{>{\raggedright\arraybackslash}m{#1}} %
\newcolumntype{G}[1]{>{\centering\arraybackslash}m{#1}} %

%%%%%%%%%%%% ВСАВКА с масштабированием ИЗОБРАЖЕНИЯ 2х3  В ТАБЛИЦУ
\newcommand{\imt}[1]
{\includegraphics[width=62mm, height=42mm, keepaspectratio=false]{#1}}

%%%% Переопределение команды для
%  Её вызов — \imgh{45.25mm}{zb}{Пример}
%  Первый параметр — ширина
%  Второй параметр — название файла
%  Третий параметр — название подписи к изображению
\newcommand{\imgh}[3]
{
	\begin{figure}[hpt!]
		\center{\includegraphics[width=#1]{#2}}
		\caption{\small {#3}}
		\label{ris:#2}
	\end{figure}
}


\newcommand{\imgroot}[4]
{
	\begin{figure}[hpt!]
		\center{\includegraphics[angle=#4,width=#1]{#2}}
		\caption{\small {#3}}
		\label{ris:#2}
	\end{figure}
}

%%Собственный  простейший список без нумерации и с обычными межстрочными интервалами
\newenvironment{compactlist}{
    \begin{list}{{$\bullet$}}{
            \setlength\partopsep{0pt}
            \setlength\parskip{0pt}
            \setlength\parsep{0pt}
            \setlength\topsep{0pt}
            \setlength\itemsep{0pt}
            \setlength{\itemindent}{\leftmargin}
            \setlength{\leftmargin}{0pt}
        }
    }{
    \end{list}
}
%%%%%%%%%%%%%%%%%%%%%%%%%%%%%%%%%%%%
%%
%% ПЕРЕОПРЕДЕЛЕНИЕ ДЛЯ ЗАПИСИ СТРОКИ АКТА ОСМОТРА
%%

\newcommand{\акт}[4]{\Rownum  & {\small #1}& #2  & #3 & #4\\  \toprule}

%%%%%%%%%%%%%%%
%% Переопределение для ЗАКЛЮЧЕНИЯ. Таблица ввода повреждений  с фото

\newcommand{\пов}[2]{\Rownum  & {\small #1 }&  \imt{#2}\\ \hline \toprule}

%%%%%%%%%%%%%%%%
%%%  Переопределение длятаблицы ИСТОРИИ РЕМОНТА и сервисного обслуживания

\newcommand{\ист}[5]{{\footnotesize #1} & {\footnotesize #2} & {\footnotesize #3} & {\footnotesize #4}  & {\footnotesize #5} \\ \hline}

%%%%%%%%%%%%%%%%%%%%%%%%%%%%%%%%%%%
%%% ПЕРЕОПРЕДЕЛЕНИЕ ДЛЯ ТАБЛИЦЫ с Игдексом и Двумя Столбцами

\newcommand{\два}[2]{\small \Rownum  & {\small #1 }&  \small #2\\ \hline \toprule}




%%% ПЕРЕОПРЕДЕЛЕНИЕ ДЛЯ ТАБЛИЦЫ без индекса  и тремя Столбцами

\newcommand{\три}[3]{& \small #1 & \small #2 & \small #3\\ \hline \toprule}


%%% ПЕРЕОПРЕДЕЛЕНИЕ ДЛЯ ТАБЛИЦЫ с Индексом и Пятью Столбцами

\newcommand{\пять}[5]{\small \Rownum  & \small #1 &  \small #2&\small #3&\small #4&\small #5\\ \hline \toprule}


%%%%%%%%%%%%%%%%%%%%%%%%%%%%%%%%%%%
\newcommand{\dee}{
	% вертикальные промежутки:
	\topsep=0pt % вокруг списка
	\parsep=0pt % между абзацами
	\itemsep=0pt % между пунктами % горизонтальные промежутки: \itemindent=0pt % абзацный выступ
	\labelsep=1ex % расстояние до метки
	\leftmargin=\parindent % отступ слева
	\rightmargin=0pt} % отступ справа
%%

%%%%%%%%%%%% Нумерованный список
\newcommand{\be}{\begin{enumerate}}
\newcommand{\en}{\end{enumerate}}

%%%% Вставить цитату
\newcommand{\цитата}[1]
{
	\begin{quote}
		\textcolor{gray}{#1}
	\end{quote}
}

\newcommand{\блеклый}[1]
{\textcolor{gray}{#1}[0.7]}

\newcommand{\сноска}[1]{\footnote{#1}}

\newcommand{\икс}{$x$}
\newcommand{\игрек}{$y$}
\newcommand{\зет}{$z$}
\newcommand{\audaОСАГО}{Audatex AudaWeb, в модуле ОСАГО ПРО}
\newcommand{\auda}{Audatex AudaWeb}

%%%%%%%%%%%%%%%%%%%%%%%%%%% ЧЕК БОКСЫ
\newcommand{\cmark}{\ding{51}}%$\checkmark $
\newcommand{\xmark}{\ding{55}}%
\newcommand{\done}{{$\square$}{\hspace{-6.5pt}\cmark}}
\newcommand{\wontfix}{{$\square$}{\hspace{-6.5pt}\xmark}}
%%%%%%%%%%%%%%%%%%%%%%%%%%%%%%%%%%%%%%%%%%

\newcommand{\г}{$\checkmark $}
\newcommand{\7}{$\checkmark $}
\newcommand{\галка}{\ding{51}}
\newcommand{\х}{\ding{55}}
\newcommand{\градус}{\circ}
\newcommand{\чек}{$\square$}

\newcommand{\чекг}{\done}
\newcommand{\чекх}{\wontfix}


% Площадь пореждений, М2
\newcommand{\plp}[1]{$S_{\text{повреждений}} \approx#1\, m^2$}
\newcommand{\плп}[1]{$S_{\text{повреждений}} \approx#1\, m^2$}

% Подпись
\newcommand{\подпись}[2]{\noindent #1    \hfill     \rule{4cm}{0.1 mm} \,\,\,  #2\\}


\newcommand{\подписьспечатью}{\begin{figure}[H]
		\begin{subfigure}{0.6\textwidth}
			\includegraphics[width=48mm]{п2} 
		\end{subfigure}
		\begin{subfigure}{0.4\textwidth}
			\includegraphics[width=40mm]{п3}
		\end{subfigure}
	\end{figure}
	\vspace{-54mm}
	\noindent \подпись{Эксперт-техник}{Мраморнов А.В.}
	\relax
	\vspace{35mm}}

\newcommand{\podpis}[2]{
	\parbox[b]{4cm}{#1}
	\hspace{2.5cm}
	\tikz[baseline=2pt]{\draw(0,0) to node[below=-2pt]{\scriptsize подпись}(3.5cm,0);}
	\hspace{1.5cm}
	\tikz[baseline=2pt]{
		\def\familywidth{\textwidth-4cm-2.5cm-3.5cm-1.5cm-10pt}
		\draw(0,0) to node[below=-2pt]{\scriptsize инициалы и фамилия}(\familywidth,0);
		\node[anchor=west](f) at (5pt,7pt){#2};
	}
}

\newcommand{\угол}[1]{$ #1^\circ $}
\newcommand{\град}[1]{$ #1^\circ $}
%%%% Стиль для колонтитулов

\newcommand{\грз}{\grz}
\newcommand{\вин}{\vin}
\newcommand{\датадтп}{\datadtp}
\newcommand{\датадоговора}{\dog}
\newcommand{\начато}{\datastart}
\newcommand{\датаосмотра}{\osm}
\newcommand{\датазаключения}{\zkl}
\newcommand{\страховойполис}{\polis}
\newcommand{\протокол}{\pr}
\newcommand{\повреждения}{\pov}
\newcommand{\иск}{\isk}
\newcommand{\тс}{\tc}
\newcommand{\окончено}{\dataend}
\newcommand{\прибл}{$ \approx $}
\newcommand{\тса}{\tca}
\newcommand{\тсб}{\tcb}
\newcommand{\ссылка}{\ref}
\newcommand{\рис}[1]{Рис. \ref{рис:#1}}
%\newcommand{}{}

%%%%%%%%%%%%%%%% ПЕРЕОПРЕДЕЛЕНИЕ  "По вопросу"     \повопросу{вопрос}

\newcommand{\повопросу}[1]
{\noindent{\small \underline{\textbf{По  вопросу \,}  }{\textbf{#1}}}}
%%%%%%%%%%%%%%%%%%%%%%%%%%%%%%%%%%%%%%%%%%%%%



\newcommand{\фото}[2]
{
    \begin{figure}[H]
        \center{\includegraphics[width=0.85\textwidth]{#1}}
        \caption{\small {#2}}
        \label{рис:#1}
    \end{figure}
}



\newcommand{\фотоб}[2]
{
	\begin{figure}[H]
		\center{\includegraphics[width=0.99\textwidth]{#1}}
		\caption*{\small {#2}}
		\label{рис:#1}
	\end{figure}
}



\newcommand{\фот}[2]
{
	\begin{figure}[H]
		\center{\includegraphics[width=0.99\textwidth]{#1}}
		\caption{\small {#2}}
	%	\label{рис:#1}
	\end{figure}
}

%%%%%%%%%%%%%%% ДВА РИСУНКА РЯДОМ            \дварядом{файл1}{подпись1}{файл2}{подпись2}
\newcommand{\дварядом}[4]{\begin{figure}[H]\centering
        \parbox[t]{0.49\textwidth}
        {\centering
            \includegraphics[width=.49\textwidth,  height=.32\textwidth]{#1}
            \caption{\footnotesize {#2}}
            \label{#1}}
        \hfil \hfil
        \parbox[t]{0.49\textwidth}
        {\centering
            \includegraphics[width=.49\textwidth, height=.32\textwidth]{#3}
            \caption{\footnotesize {#4}}
            \label{#3}}

\end{figure}}

%%%%%%%%%%%%%%%%%%%%%%%%%%%%%%%%%%%
% Два рядом с одной общей подписью
%%%%%%%%%%%%%%%%%%%%%%%%%%%%%%%%%%%

\newcommand{\дварисунка}[5]{\begin{figure}[H]
	\begin{minipage}{0.49\textwidth}
		\includegraphics[width=\linewidth,  height=.64\linewidth]{#1}
		\subcaption{#2}
	\end{minipage}
	\hfill
	\begin{minipage}{0.49\textwidth}
		\includegraphics[width=\linewidth,  height=.64\linewidth]{#3}
		\subcaption{#4}
	\end{minipage}

	\caption{#5}
	\label{рис:#1}
\end{figure}}





%%%% СТС две стороны рядом
\newcommand{\стс}[4]{\begin{figure}[H]
    \centering
    \parbox[t]{0.49\textwidth}
    {\centering
        \includegraphics[width=.49\textwidth,  height=1.45\linewidth]{#1}
        \caption{\footnotesize {#2}}
        \label{рис:#1}}
    \hfil \hfil
    \parbox[t]{0.49\textwidth}
    {\centering
        \includegraphics[width=.49\textwidth,  height=1.45\linewidth]{#3}
        \caption{\footnotesize {#4}}
        \label{рис:#3}}

\end{figure}}


\newcommand{\смарт}[4]{\begin{figure}[H]
		\centering
		\parbox[t]{0.49\textwidth}
		{\centering
			\includegraphics[width=80mm,  height=140mm]{#1}
			\caption{\footnotesize {#2}}
			\label{рис:#1}}
		\hfil \hfil
		\parbox[t]{0.49\textwidth}
		{\centering
			\includegraphics[width=80mm,  height=140mm]{#3}
			\caption{\footnotesize {#4}}
			\label{рис:#3}}
\end{figure}}

%%%%% ФОТО РЯДОМ С ТЕКСТОМ
%
%\newcommand{\фотосправа}[2]{
%    \begin{SCfigure}
%        \centering {\footnotesize \caption{#2}
%            \includegraphics[width = 0.6 \textwidth]{foto/#1}
%            \label{рис:#1}
%    \end{SCfigure}}



%%%% Переопределение команды для
%  Её вызов — \фотомасштаб{45.25mm}{название файла}{подпись рисунка}

%  Первый параметр — название файла
%  Второй параметр — название подписи к изображению
%  Третий параметр — ширина
\newcommand{\фотомасштаб}[3]
{
    \begin{figure}[H]
        \center{\includegraphics[width=#3]{#1}}
        \caption{\small{#2}}
        \label{рис:#1}
    \end{figure}
}



%  Её вызов — \фотоповорот{45.25mm}{название файла}{подпись рисунка}{угол поворота}
%  Первый параметр — ширина
%  Второй параметр — название файла
%  Третий параметр — название подписи к изображению
\newcommand{\фотоповорот}[4]
{
    \begin{figure}[hpt!]
        \center{\includegraphics[angle=#4,width=#1]{#2}}
        \caption{\small {#3}}
        \label{рис:#2}
    \end{figure}
}

%%% ИЗМЕРИТЬ ШИРИНУ СТРАНИЦЫ
\newcommand{\ширина}{\the\textwidth\\
    \printinunitsof{mm}\prntlen{\textwidth}}
