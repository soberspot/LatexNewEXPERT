       
%-----------------------------------------------------   
% ИТОГИ РСЧЕТА    
\def\итог{380950}
\def\итогизнос{215550}
\def\рынок{100000}	       
\def\нормочас{950}
	       
	       

\subsection{Использованные нормативы и источники информации}

\begin{enumerate}
\item   Федеральный закон «Об обязательном страховании гражданской ответственности владельцев транспортных средств» от 25.04.2002 г. № 40-ФЗ.
\item  Положения Банка России от «19» сентября 2014 года № 431-П «О правилах обязательного страхования гражданской ответственности владельцев транспортных средств».
\item  Положение Банка России от «19» сентября 2014 года № 432-П «О единой методике определения размера расходов на восстановительный ремонт в отношении повреждённого транспортного средства».
\item  Положение ЦБ РФ № 433-П «О правилах проведения независимой технической экспертизы транспортного средства» от 19 сентября 2014 г.
\item  Технический регламент Таможенного союза <<О безопасности колёсных транспортных средств>> (ТР ТС - 018 - 2011).
%\item  Методические рекомендации по проведению судебных автотехнических экспертиз и исследований колёсных транспортных средств в целях определения размера ущерба, стоимости восстановительного ремонта и оценки / Е. Л. Махнин, И. Н. Новоселецкий, С. В. Федотов и [др.]. - М. : ФБУ РФЦСЭ при Минюсте  России, 2018. - 326 с.
\item  Технологическое руководство «Приёмка, ремонт и выпуск из ремонта кузовов легковых автомобилей предприятиями автотехобслуживания» РД 37.009.024-92.
\item  Предотвращение страхового мошенничества в автостраховании  (практическое  пособие)  М.  2005.
%\item  Исследование транспортных средств в целях определения стоимости восстановительного ремонта и оценки: курс лекций / под общ. ред. д-ра юрид. наук, профессора С.А. Смирновой; Министерство юстиции Российской Федерации, Федеральное бюджетное учреждение Рос. Федер. центр судеб экспертизы. - М.: ФБУ РФЦСЭ при Минюсте России, 2017. - 286 с.
\item  Методика окраски и расчёта стоимости лакокрасочных материалов для проведения окраски ТС – AZT. 
\item  Сервис по автоматической расшифровке VIN номеров – AudaVIN.
\item  Сервис РСА для проверки текущего договора ОСАГО,  http://86.62.95.12:8080/dkbm-web-1.0/bsostate.htmhttp://prices.autoins.ru/spares/.
\item  Онлайн сервис РСА средней стоимости запасной части и нормочаса в экономическом районе,    http://prices.autoins.ru/priceAutoParts/.
\item  	Материалы тематических веб-сайтов сети Интернет\\
\url{https://partsouq.com}\\
\url{https://emex.ru}
\url{https://audatex.ru}
\end{enumerate}


\subsection{Технические средства}

\begin{itemize}
\item  Рулетка измерительная металлическая, 0-5000мм, «HORTZ» №451, отклонение от действительной длины ± 1,20мм;
\item  Линейка измерительная металлическая, ГОСТ 427-75, заводской номер 51118, 0-500мм, цена деления 1мм, пг ± 0,15мм;
\item Телескопическая измерительная линейка OMAS TM-2;
\item  Линейка масштабная магнитная с цветографической шкалой, 100 мм;
\item  Цифровой фотоаппарат  Canon 760D s/n 143032001327  с объективом Canon EF-S 18-135;
%\item Специализированное программное обеспечение для расчёта стоимости  восстановительного ремонта, содержащее нормативы трудоёмкости работ, регламентируемые изготовителями транспортного средства     AudaPadWeb, лицензионное соглашение № AS/\- APW-658  RU-P-409-409435.
\item  Специализированное программное обеспечение для расчёта стоимости  восстановительного ремонта, содержащее нормативы трудоёмкости работ, регламентируемые изготовителями транспортного средства  SilverDAT myClaim,
лицензионный договор № 1422 от 05.02.2021 на право использования программы для ЭВМ от  DAT IP-Management und Vertriebs GmbH.
%\item  MotorData — интерактивная справочно-информационная система по диагностике и ремонту автомобилей. http://motordata.ru/ru
\item  Программа обработки фото-видео изображений ImageJ, разработчик  Wayne Rasband (wayne@codon.nih.gov) , is at the Research Services Branch, National Institute of Mental Health, Bethesda, Maryland, USA. Свободная лицензия GPL;
\item  ПЭВМ под управлением операционной системы Windows 10 с установленным набором офисных программ LebreOffice, лицензия: Mozilla Public License версия 2.0, \url{http://mozilla.org/MPL/2.0/.}
\end{itemize}
\subsection{Ограничения и пределы применения полученных результатов}

Следующие допущения и условия, ограничивающие пределы применения полученных результатов, являются неотъемлемой частью данного экспертного заключения:
      
\noindent  - результаты, полученные экспертом-техником, носят рекомендательный консультационный характер и не являются обязательными. Исполнитель высказывает своё субъективное суждение о наиболее вероятных будущих (абстрактных) расходах, их предполагаемом размере и даёт заключение в пределах своей компетенции;

\noindent - под компетенцией эксперта-техника понимают его знания и опыт в области теории и методов экспертных исследований ТС, а также круг полномочий, представленных ему законом, и вопросов, которые он может решать на основе своих специальных познаний, в компетенцию эксперта-техника входит исследование технического состояния повреждённого ТС в целях установления характера повреждений ТС, установления причины возникновения технических повреждений технологии, методов, стоимости его ремонта;

\noindent - исполнитель в рамках своих обязательств по заключённому договору об экспертном обслуживании признает свою ответственность перед заказчиком и настоящим утверждает, что экспертное заключение выполнено профессионально, тщательно и с должной заботливостью и внимаем, как это обычно принято для компетентного специалиста в области технической экспертизы ТС при ОСАГО, а полученная величина восстановительных расходов, разумна и реальна;

\noindent - исполнитель считает, что поскольку, по общему правилу, оценка доказательств является прерогативой и компетенцией органа дознания, следствия или суда, а в досудебном порядке - страховщика, постольку после проверки результатов экспертизы последним, их признания и принятия решения о выплате страхового возмещения этап возможного оспаривания достоверности исследований между заказчиком и исполнителем завершён,  соответственно, обязанности Исполнителя по договору являются надлежаще исполненными в полном объёме и от исполнителя не требуется свидетельствовать по поводу произведённого исследования перед третьими лицами;

\noindent - отдельные части настоящего экспертного исследования не могут трактоваться раздельно, а только в связи с полным текстом о проведённых расчётах;

\noindent - исходные данные, использованные исполнителем при подготовке экспертного заключения, получены из надёжных источников и считаются достоверными. Тем не менее, исполнитель не может гарантировать абсолютную точность, поэтому там, где это, возможно, делаются ссылки на источники информации;

\noindent - в процессе экспертного исследования специальная юридическая экспертиза документов, касающихся прав собственности на ТС, не проводилась;

\noindent - суждения, содержащиеся в экспертном заключении, основываются на текущей ситуации на дату аварии и в будущем могут быть подвержены изменениям;

\noindent - исполнитель не принимает на себя никакой ответственности за изменение экономических, юридических и иных факторов, которые могут возникнуть после даты исследования и повлиять на результаты технической экспертизы;

\noindent - при анализе скрытых повреждений экспертом-техником не принимается во внимание наличие или отсутствие записей о них в документах компетентных органов, в связи с отсутствием у сотрудников компетентных органов объективной возможности  идентификации таких повреждений на месте происшествия;


\noindent - выводы, содержащиеся в настоящем Заключении, могут расцениваться как достоверные только в контексте того количества информации, на основании которого они были сделаны. При поступлении дополнительной или изменённой информации данные выводы могут быть
скорректированы; 

\noindent - данное заключение составлено на основании Правил Независимой Технической Экспертизы и может применяться только при решении вопроса о выплате страхового возмещения по ОСАГО.


\section{Исследование}

Настоящее исследование проводится на основании материалов, предоставленных Заказчиком, а также на основании данных, самостоятельно полученных экспертом-техником.

\subsubsection{Объект исследования}

	\par Из предоставленных материалов   экспертом-техником установлена следующая общая информация об автомобиле, имеющая значение для дачи заключения:
 \parbox[]{10cm}{}
\begin{itemize}
	\item[ ] 
	\begin{description}
		\item[Марка, модель] -- \тс
        %	\item[Заводское обозначение модели] --  
		\item[VIN] -- \vin
		\item[Шасси] -- \шасси
		\item[Год выпуска] -- \год
		\item[Цвет ЛКП] -- \цвет
		\item[Пробег] --  \пробег\, км, считан с одометра
		\item[Дата начала эксплуатации] -- \началоэкспл
	%	\item[Двигатель] -- \двигатель
	%	\item[Объем двигателя] -- 1328 $ \text{см}^3 $
	%	\item[Свидетельство о регистрации] -- \свид
	%	\item[ПТС] --\птс
	\end{description}
\end{itemize}
%
%\subparagraph*{} Идентификационный код автомобиля (VIN)  \vin \, содержит следующую информацию о транспортном средстве, имеющую значение для 	дачи заключения (Рис. \ref{fig:vin} ):\\[3mm]
%%	
%	\noindent\parbox[]{10cm}{
%		\begin{itemize}
%			\item[ ] 
%			\begin{description}
%				\item[Дата изготовления] \hfill \датаизготовления
%				\item[Расположенние руля] \hfill Left
%				\item[Двигатель] \hfill \двигатель
%			%	\item[Объем двигателя] \hfill 1328 $ \text{см}^3 $
%				\item[КПП] \hfill МКПП
%				\item[Тип кузова] \hfill  \типкузова
%				\item[Количество дверей] \hfill 5 
%				%	\item[VDS] --
%					
%			\end{description}
%	\end{itemize}}\\
	
%\vspace{3mm}
	
%
%Описание модели:
%	\begin{figure}[H]
%		\centering
%		\includegraphics[width=0.999\linewidth]{}
%		\caption[]{{\footnotesize Комплектация автомобиля VIN \vin\, по данным\textit{ \url{https://emex.ru/catalogs/original/?screen=units\&vin=}\вин}} }
%		\label{fig:vin}
%	\end{figure}
	
\vspace{3mm}
%
%\begin{figure}[H]
%    \centering
%    \includegraphics[width=0.65\linewidth]{foto/рынок}
%    \caption[]{Диаграмма рыночной стоимость автомобиля, аналогичного исследуемому \тс\, \textsl{источник:} %\url{https://spec.drom.ru}, \url{https://automama.ru/ocenka-avto}
%    }
%    \label{fig:рыночная}
%\end{figure}
%


	\subsubsection{Осмотр транспортного средства}
	
   \osm\,  проводился осмотр повреждённого транспортного средства \tc\,государственный регистрационный знак \grz. Осмотр проводился в сухую, ясную погоду с 18-00  до 18-30 на открытой площадке   по адресу: \местоосмотра. При осмотре присутствовали:  владелец транспортного средства \тс \,\присутствовали.  Соответствие маркировочных обозначений на кузове представленного ТС записям в регистрационных документах ТС экспертом-техником установлено. Видимые изменения конструкции ТС отсутствуют. Представленное на исследование транспортное средство \тс\, регистрационный знак \грз\, имеет кузов типа <<\типкузова». Кузов ТС окрашен двухслойной   %лессирующей (с металликовым эффектом) 
   эмалью (краской)  \colr \, цвета.
   

\subsubsection{Рыночная стоимость}
Рыночная стоимость  транспортного средства, аналогичного исследуемому \тс, по данным специализированных открытых источников на момент повреждения  составляла 

\рынок \,  (\!\числопрописью{\рынок}\!) рублей.\\
Источник:  \url{https://automama.ru/ocenka-avto}, \url{https://spec.drom.ru}


                       
\subsubsection{Установлении причин возникновения повреждений транспортного средства}
% Фотоизображения места ДТП предста влены ниже на рис. \ref{ris:images/дтп1} и рис. \ref{ris:images/дтп2}.
%    
%    \begin{figure}[!h]\centering
%        \parbox[t]{0.49\textwidth}
%        {\centering
%            \includegraphics[width=.49\textwidth]{foto/дтп1}
%            \caption{\footnotesize {Фото места ДТП. На переднем плане автомобиль второго участника ДТП}}
%            \label{ris:images/дтп1}}
%        \hfil \hfil%раздвигаем боксы по горизонтали 
%        \parbox[t]{0.49\textwidth}
%        {\centering
%            \includegraphics[width=.49\textwidth]{foto/дтп2}
%            \caption{\footnotesize {Фото места ДТП. Автомобиль \тс}}
%            \label{ris:images/дтп2}}
%    \end{figure}
    
    Причины возникновения технических повреждений и возможность их отнесения к
рассматриваемому ДТП исследованы при осмотре ТС. Для определения причины возникновения повреждений, указанных в Акте осмотра ТС  №  \NomerDoc\, (Приложение № 1) экспертом-техником изучены документы, представленные Заказчиком. По предоставленным документам экспертом-техником установлена причина ДТП, установлены обстоятельства ДТП, выявлены повреждения ТС и установлены причины их образования. Проведено исследование характера выявленных повреждений, сопоставление повреждений ТС потерпевшего с повреждениями ТС иных участников ДТП в соответствии со сведениями, зафиксированными в документах о ДТП.  Проведена проверка взаимосвязанности повреждений на ТС с заявленными обстоятельствами ДТП. 

В результате проведённых исследований эксперт-техник приходит к заключению о соответствии механических повреждений, имеющихся на транспортном средстве \, \тс\, регистрационный знак \грз\, на момент осмотра заявленным обстоятельствам. 


\subsubsection{Исследование наличия, характера и объёма технических повреждений}

  Наличие, характер и объем технических повреждений транспортного средства \tc\, регистрационный знак \grz, исследованы в присутствии заинтересованных лиц,  зафиксированы в акте осмотра № \NomerDoc\,  (Приложение, <<Акт осмотра>> ),  и фотоматериалах (Приложение, <<Фототаблица>>) по принадлежности. Планируемые (предполагаемые) ремонтные воздействия для восстановления повреждённого  транспортного средства назначены экспертом-техником с учётом особенностей конструкции и рекомендаций изготовителя  транспортного средства, укрупненных показателей трудозатрат по кузовному ремонту и устранению перекосов проёмов и кузова легковых автомобилей иностранных производителей, приложение 3 к приложению к Положению Банка России от 19 сентября 2014 года № 432-П и приведены ниже в таблице \ref{tab:5}.
 
  %\pagebreak
  
  \begin{longtable}{G{3mm}|M{120mm}|G{30mm}}
      \caption[]{\footnotesize {Повреждения автомобиля, установленные при его осмотре}} 
      \label{tab:5}\\ 
      \hline 
      \hline  \toprule 
\bf  {\footnotesize  n/n}  &\bf {\small Наименование  детали с описанием повреждения} & \bf {\small Изображение} \\   \hline\hline  \toprule \endhead 
%%%%___________________________________________________________________    
%\пов{Наименование детали- описание повреждения }{example-image}
\пов{Капот - сложная  деформация панели и каркаса  детали на площади более 80\% поверхности}{example-image}
%\пов{}{example-image}
%\пов{}{example-image}
%\пов{}{example-image}
%\пов{}{example-image}
%\пов{}{example-image}
%\пов{}{example-image}
%\пов{}{example-image}
%\пов{}{example-image}
%\пов{}{example-image}
\end{longtable}\setcounter{rownum}{0}
  %\end{longtable}\setcounter{rownum}{0}
  
  
\subsubsection{Определение стоимости восстановительных расходов}

 В соответствии с существующей экспертной методикой размер расходов на восстановительный ремонт определяется исходя из стоимости ремонтных работ (работ по восстановлению, в том числе окраске, контролю, диагностике и регулировке, сопутствующих работ), стоимости используемых в процессе восстановления транспортного средства деталей (узлов, агрегатов) и материалов взамен повреждённых. Расчёт размера расходов (в рублях) на восстановительный ремонт производится по формуле (\ref{eq:cr}): 
      
\begin{equation}\label{eq:cr}
C_{\text{вр}}  =\sum{C_{ip}}= \sum\left({T_{ij}}\cdot {C_{i\text{нч}}}\right) + \sum{C_{ip^{\text{\,\,\,руб}}}} , \,\,\,\text{где:} 
\end{equation}
%\vspace{2mm}
\begin{itemize}
	\item[ ]$ C_{ip} $ -- стоимость работ i-го вида: $C_\text {зам} $, $ C_\text{восст} $, $ C_\text{рег} $, $C_\text{контр} $, $ C_\text{антикор} $, $ C_\text{зч} $, $ C_\text{ом} $,$ C_\text{соп} $, $ C_\text{вм} $, руб;
	\item[ ]$ T_{ij} $ -- трудоёмкость j-й операции(комплекса) по i-му виду работ, руб;
	\item[ ]$ C_{i\text{нч}} $ -- стоимость нормо-часа по i-му виду работ, руб;
	\item[ ]$ C_{ip^\text{\,\,руб}} $ -- стоимость работ $ C_{ip} $, принятая непосредственно в денежном выражении, руб.
\end{itemize}

\par При определении стоимости восстановительного ремонта АМТС с учётом износа под износом следует понимать количественную меру физического старения АМТС и его элементов, достигнутого в результате эксплуатации, т.е. эксплуатационный износ.
%
Расчёт износа производится в  соответствии с Положением Банка России от «19» сентября 2014 года № 432-П «О единой методике определения размера расходов на восстановительный ремонт в отношении повреждённого транспортного средства» [3].
Износ комплектующих изделий (деталей, узлов, агрегатов) рассчитывается по следующей формуле (\ref{eq:I}):
%
%
%
\begin{equation}\label{eq:I}
\text{И}_{\text{ки}} 
= 100\cdot\left( 1-e^ {-\left( \Delta_{T} \cdot T_{\text{КИ}} + \Delta_{L} \cdot L_{\text{КИ}} \right)}\right), \,\,\,\,\text{где:}   
\end{equation}
%
\begin{itemize}
	\item[ ]$ \text{И}_{\text{ки}} $ -- износ комплектующего изделия (детали, узла, агрегата) (процентов); 
	\item[ ]$ e $ -- основание натуральных логарифмов (e =  2,72);
	\item[ ]$ \Delta_{T}$ --  срок эксплуатации комплектующего изделия (детали, узла, агрегата) (лет);
	\item[ ]$ T_{\text{КИ}} $ -- стоимость работ $ C_{ip} $, принятая непосредственно в денежном выражении, руб;
	\item[ ]$ \Delta_{L} $ -- коэффициент, учитывающий влияние на износ комплектующего (детали, узла, агрегата) величины пробега транспортного средства с этим комплектующим изделием;
	\item[ ]$ L_{\text{КИ}} $ -- пробег транспортного средства на дату дорожно-транспортного происшествия (тысяч километров).  
\end{itemize}
\vspace{5mm}
\par Значения коэффициентов $ \Delta_{T}$  и $ \Delta_{L} $  для различных категорий и марок транспортных средств приведены в п. 5. сп. лит~[3]. При этом, на комплектующие изделия (детали, узлы, агрегаты), которые находятся в заведомо худшем состоянии, чем общее состояние транспортного средства в целом, и его основные части, вследствие влияния факторов, не учтённых при расчёте износа (например, проведение ремонта с нарушением технологии, не устранение значительных повреждений лакокрасочного покрытия), может быть начислен дополнительный индивидуальный износ. 

Износ шины транспортного средства рассчитывается по следующей формуле (\ref{eq:sh}):
\begin{equation}\label{eq:sh}
\text{И}_{\text{ш}} = \frac{\text{Н}_{\text{н}}-\text{Н}_{\text{ф}}}{\text{Н}_{\text{н}}-\text{Н}_{\text{доп}}} \cdot{100}\%,  \,\,\,\,\text{где:} 
\end{equation}
%
\begin{itemize}
	\item[ ] $ \text{И}_{\text{ш}} $ -- износ шины, \%;
	\item[ ] $ \text{Н}_{\text{н}} $ -- высота рисунка протектора новой шины, мм;
	\item[ ] $\text{Н}_{\text{ф}} $ -- фактическая высота рисунка протектора шины, мм;
	\item[ ] $ \text{Н}_{\text{доп}} $ --минимально допустимая высота рисунка протектора шины в соответствии с требованиями законодательства Российской Федерации, мм.
\end{itemize}
%
\vspace{5mm}
\relax
%\renewcommand\baselinestretch{1}\small\normalsize
%
Износ шины дополнительно увеличивается для шин с возрастом от 3 до 5 лет - на 15 процентов, свыше 5 лет - на 25 процентов.

                                                 
\subsubsection{Данные для расчёта}

\noindent Объект экспертизы:  транспортное средство \tc\,
регистрационный знак \грз;\\ 
VIN: \вин;\\
Пробег:    \пробег км (установлен по показаниям одометра);\\
Год выпуска:     \год;\\ 
Дата ввода в эксплуатацию:  \началоэкспл;\\
Дата ДТП:  \датадтп;\\
Перечень ремонтных воздействий представлен в таблице \ref{tab:осм}.

\subsubsection{Ремонтные воздействия, необходимые для устранения повреждений}

\setcounter{rownum}{0}

\begin{longtable}{G{3mm}|M{130mm}|G{5mm}|G{5mm}|G{5mm}}
	\caption[]{Таблица ремонтных воздействий, необходимых для устранения повреждений ТС \тс, полученных в заявленных обстоятельствах}
	\label{tab:осм}\\
	\hline  \hline   \toprule 
	\bf  {\footnotesize  n/n}  &\bf {\small Наименование  детали и описание повреждения} & \bf {\small E} & \bf {\small I} & \bf {\small L}\\\hline \hline \toprule  \endhead 
	
	
%%%%______________________________________%%%%%%%%%%%%
%%%%%%%%%   ОПИСАНИЕ ПОВРЕЖДЕНИЙ   
%\\ps{ деталь - повреждение }{E}{I}{L} 
\акт{Крыло переднее правое }{}{\7}{\7}
\акт{Бампер передний }{\7 }{ }{\7 }
\акт{Кронштейн бампера переднего правый }{\7 }{ }{ }
\акт{Диск колеса правого переднего семиспицевый }{\7 }{ }{ }
\акт{Фара правая }{\7 }{ }{ }
\акт{Облицовка крыла правого переднего }{\7 }{ }{ }
\акт{Спойлер бампера переднего правая часть }{\7 }{ }{ }
\акт{Пленка  защитная антигравийная бампера переднего и крыла правого }{\7 }{ }{ }


\end{longtable}\setcounter{rownum}{0} 
	
	\textit{E - заменить деталь, I - ремонтировать, L - окрасить}
	%
	
	%\subsubsection{ Расчёт}
	
	%
	\renewcommand\baselinestretch{1.2}\small\normalsize

\subsubsection{ Расчёт}
    
\indent Полный расчёт стоимости восстановительных расходов на ремонт ТС с учётом износа в соответствии с правилами обязательного страхования гражданской ответственности владельцев транспортных средств выполнен в  лицензированном для решения задач в рамках ОСАГО программном комплексе   SilverDAT myClaim и приведён в Калькуляции № \NomerDoc.
 Расчёт износа произведён программой  SilverDAT myClaim и представлен  в калькуляции расчёта затрат № \NomerDoc.
 
\indent Итоговые результаты расчёта  стоимости восстановительных расходов ТС \тс\, \грз\, представлены ниже:\\
  
  
\begin{figure}[H]
        	\centering
        	\includegraphics[width=0.95\linewidth]{example-image}
    %    		\caption{}
    %    		\label{fig:screenshot001}
        \end{figure}
  
    %
    \begin{figure}[H]
    	\centering
    	\includegraphics[width=0.95\linewidth]{example-image}
%    		\caption{}
%    		\label{fig:screenshot001}
    \end{figure}
    \begin{figure}[H]
    	\centering
    	\includegraphics[width=0.95\linewidth]{example-image}
%    		\caption{}
%    		\label{fig:screenshot002}
    \end{figure}
    \medskip
    \renewcommand\baselinestretch{1.2}\small\normalsize

%
%
%
\subparagraph{}Стоимость одного нормо-часа работ определена в соответствии с пунктом 3.8.1 Единой методики [3] путём применения электронных баз данных стоимостной информации.
Трудоёмкость работ по разборке/сборке/замене  соответствует трудоёмкостям работ, рекомендованным заводом изготовителем ТС. Трудоёмкости окрасочных работ приняты согласно рекомендаций Единой методики, п.3.7.1. в соответствии с технологией  AZT (\url{http://www.schwacke.ru/down/azt _reparaturlackierung_ru.pdf}). Расчёт размера расходов на материалы произведён  согласно пункту 3.7.2 Приложения к Единой методике [3]. Артикулы запасных частей определены с помощью программы SilverDAT и электронных  каталогов запасных частей \url{emex.ru}, \url{partsouq.com}.
Стоимость запасных частей определена в соответствии с пунктом 3.6.3 Единой методики путём применения электронных баз данных стоимостной информации (по утверждённому справочнику: \url{http://prices.autoins.ru/priceAutoParts/repair_parts.html} ).
% 
\subparagraph{}Таким образом,  наиболее вероятная стоимость ремонта транспортного средства \tc\, регистрационный знак \грз, получившего повреждения в результате дорожно-транспортного происшествия  \датадтп\, составляет \итог \, (\!\числопрописью{\итог}\!) руб.,  размер затрат на восстановительный ремонт ТС с учётом износа составляет \итогизнос \,(\!\числопрописью{\итогизнос}\!) руб.
%      
%\subfile{../рынокОСАГО}
%
\subfile{../sections/утсОСАГО}
\subfile{../sections/годныеОСАГО}
%
%\section{В ы в о д ы}
%
%
%1) Наличие, характер и объем (степень) технических повреждений, причинённых ТС, определены при осмотре и зафиксированы в Акте осмотра № \NomerDoc\,  фототаблице повреждений и таблице \ref{tab:5}, являющимися неотъемлемой частью настоящего экспертного заключения.\\[3mm]
%    
%2) Направление, расположение и характер повреждений определены путём сопоставления полученных повреждений, изучения административных материалов по рассматриваемому событию, и  являются  следствиями рассматриваемого ДТП (события).\\[3mm]
%    
%3) Технология и объем необходимых ремонтных воздействий зафиксированы в калькуляции № \NomerDoc\, по определению стоимости восстановительного ремонта транспортного средства \tc\, VIN  \vin. \\[3mm]
%    
%4)  Стоимость восстановительного ремонта  транспортного средства \tc\, регистрационный знак \грз,\, \, получившего механические повреждения в результате дорожно-транспортного происшествия, имевшего место \датадтп\, составляет \итог\, (\числопрописью{\итог})  руб.\\[3mm]
%    
%5) Размер затрат на проведение восстановительного ремонта с учётом износа (восстановительные расходы) транспортного средства \tc\, регистрационный знак \grz\, составляет  \итогизнос \, (\числопрописью{\итогизнос}) руб.\\[3mm]
%    
%    6) Стоимость годных остатков ТС \тс\, регистрационный знак \грз\, оставляет  $ 10\,560$ (Десять тысяч пятьсот шестьдесят) рублей.
%    
%    6) Величина утраты товарной стоимости транспортного средства \тс\,  регистрационный знак \грз\, составляет  
