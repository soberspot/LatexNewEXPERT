ГОСТ Р 27.102-2021 «Надежность в технике. Надежность объекта. Термины и определения» \cite{271022021:gost} устанавливает следующие термины и определения:



• 36 отказ – событие, заключающееся в нарушении работоспособного состояния объекта;

• 37 дефект - каждое отдельное несоответствие объекта требованиям, установленным документацией;

• 54 конструктивный отказ –  отказ, возникший по причине, связанной с несовершенством или нарушением установленных правил и (или) норм проектирования и конструирования;

• 55 производственный отказ - отказ, возникший по причине, связанной с несовершенством или нарушением установленного процесса изготовления или ремонта, выполняемого на ремонтном предприятии; 

• 56 эксплуатационный отказ –  отказ, возникший по причине, связанной с нарушением установленных правил и (или) условий эксплуатации.


Дефект и (или) повреждение могут служить причиной возникновения частичного или полного отказа объекта. Наличие дефекта и (или) повреждения приводит объект в неисправное состояние.
